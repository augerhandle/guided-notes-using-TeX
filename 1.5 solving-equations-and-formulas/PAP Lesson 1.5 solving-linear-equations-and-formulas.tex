\documentclass[fleqn,letterpaper,12pt,printwatermark=false]{memoir}
% memoir commands to define the text block geometry
\setulmarginsandblock{0.5in}{*}{*}
\setlrmarginsandblock{0.5in}{*}{*}
% put "extra" vertical space at the bottom of a page
\raggedbottom 

\usepackage{amsmath}
\usepackage{etoolbox} % for \ifblank etc
\usepackage{xparse} % for NewDocumentCommand et al.
\usepackage{enumitem}
\usepackage{transparent} % for \transparent, which I use in the watermark
\usepackage[slantedGreek]{mathpazo} \usepackage{helvet} % use Palatino et al.
\usepackage{booktabs} % prettier tables

\usepackage[]{xwatermark}
\newwatermark*[
    allpages,
    color=red!30,angle=45,
    scale=4,
    xpos=-10, ypos=0
]{%
    \transparent{0.4}dhasan example%
}

\usepackage{dashundergaps} % for \gap
\dashundergapssetup{
    teacher-mode=true, % set to true to show answers 
    gap-format=underline,
    teacher-gap-format=underline,
    gap-font={\sffamily},
    gap-numbers=true,
    gap-widen=true,
    gap-extend-percent=150, % note: making this too big might create errors
    gap-number-format=\,\textsuperscript{\normalfont(\thegapnumber)},
}

\usepackage{tcolorbox}
\tcbuselibrary{skins}
\tcbuselibrary{raster}

\usepackage{graphicx}
\graphicspath{ {../images/} }
\begin{document} 

\newcommand{\myClassName}{Pre-AP Algebra 2}
\newcommand{\myUnitNumber}{1}
\newcommand{\myUnitTitle}{Introduction to Functions}
\newcommand{\myLessonNumber}{5}
\newcommand{\myLessonTitle}{Solving Linear Equations and Formulas}


\copypagestyle{myPagestyle}{empty}


\newcommand{\myFooterSize}{\footnotesize}
\makeoddfoot{myPagestyle}{\myFooterSize\myClassName}{\myFooterSize\thepage\,of\,\pageref*{xwmlastpage}}{\myFooterSize\myLessonTitle}
\makeevenfoot{myPagestyle}{\myFooterSize\myLessonTitle}{\myFooterSize{\thepage{}~of~\pageref*{xwmlastpage}}}{\myFooterSize\myClassName}

%
% A command to change the appearance of the cognitive verb 
% in the objectives.
%
\newcommand{\myCognitiveVerb}[1]{\textcolor{blue}{\textbf{#1}}}

%
% Font styling commands (so I can change them in a single place)
%
\NewDocumentCommand{\myUnitLessonNumberFont}{}{\sffamily\bfseries\HUGE}
\NewDocumentCommand{\myUnitTitleFont}{}{\sffamily\large}
\NewDocumentCommand{\myLessonTitleFont}{}{\sffamily\bfseries\huge}
\NewDocumentCommand{\myHeadingFont}{}{\sffamily\bfseries\Large}


%
% #1 is the fill-in text
%
\NewDocumentCommand{\myFillInBlank}{m}{%
    \,%
    \gap[u]{#1}%
    \,%
}


% Definition for the LESSON HEADER + OBJECTIVES
%
\newenvironment{myNotesHeader}{
    \begin{flushleft}
        {\myUnitLessonNumberFont{\myUnitNumber.\myLessonNumber}}
        \hfill\;\;
        \begin{minipage}[b]{0.75\textwidth}
            \begin{flushright}
                {\myUnitTitleFont{}Unit \myUnitNumber\,\,\,\myUnitTitle}\\ \vspace{0.75em}
                {\myLessonTitleFont\myLessonTitle}
            \end{flushright}
        \end{minipage}
        \hrule
    \end{flushleft}
    \noindent{\myHeadingFont Objectives:}
    \begin{enumerate}[label=\arabic*)]
}{
    \end{enumerate}
}


% Definitions related to the VOCABULARY TABLE
%
\newenvironment{myVocabulary}{
    {\noindent{\myHeadingFont Vocabulary:}}\vspace{1em}

    \begin{tabular}{ll}
        \toprule
            \emph{word} & \emph{meaning} \\ 
        \midrule
}{
    \bottomrule
    \end{tabular}
    \vspace{1em}
}

\newcommand{\myVocabularyWord}[2]{%
{\textcolor{blue}{\textbf{#1}}} & #2 \\
}


% Definitions related to an INTRODUCTION
%
\newenvironment{myLesson}{
    {\noindent{\myHeadingFont Lesson:}}\vspace{1em}

    \begin{adjustwidth}{2em}{0pt}
    \begin{itemize}
}{
    \end{itemize}
    \end{adjustwidth}
    \vspace{1em}
}


% Definition related to KEY CONCEPTS
%
% #1 : the key concept (which appears as a tcolorbox title)
%
\NewDocumentEnvironment{myKeyConcepts}{ O{Key Concepts:} }{
    \begin{tcolorbox}[
        title=#1, fonttitle=\myHeadingFont,
        coltitle=black, 
        colbacktitle=black!25!yellow, 
        colframe=black!50!yellow,
        colback=white!70!yellow,
        boxrule=2pt, 
        ]
}{
    \end{tcolorbox}
}


% Definition related to EXAMPLES

% #1 Optional example number 
% #2 A statement of the example problem.
% #3 How much empty vertical space to leave for the example box.
%
\NewDocumentCommand{\myExample}{omm}{%
    \begin{tcolorbox}[
        enhanced,
        sharp corners, 
        colback=white,
        boxrule=0pt,
        borderline={0.5pt}{0pt}{black,dashed},
        ]
        {\myHeadingFont Example\IfValueT{#1}{{ #1}}:}
        #2
        \tcblower
        \vspace{#3}
    \end{tcolorbox}
}

% #1 Optional example number 
% #2 A statement of the example problem.
%
\NewDocumentEnvironment{myExampleForTikzGraphs}{om}{%
    \begin{tcolorbox}[
        enhanced,
        sharp corners, 
        colback=white,
        boxrule=0pt,
        borderline={0.5pt}{0pt}{black,dashed},
        ]
        {\myHeadingFont Example\IfValueT{#1}{{ #1}}:}
        #2
        \tcblower
        \begin{center}
}
% user would insert 
% \begin{tikzpicture}\begin{axis}...\end{axis}\end{tikzpicture}
{
        \end{center}
    \end{tcolorbox}
}


% Definitions related to PROBLEMS

% A counter to number the problems in the guided notes.
\newcounter{MyProblemCounter}
\setcounter{MyProblemCounter}{1}
\newcommand{\useMyProblemCounter}{\theMyProblemCounter\stepcounter{MyProblemCounter}}

% an environment for two adjacent problems
%
% #1 : directions for all the problems
% #2 : vertical height of the problem boxes
% #3 : details for problem 1
% #4 : details for problem 2
%
\newenvironment{myProblems2}[4]{%
    \noindent
    {\myHeadingFont Practice:}\hspace{0.5em}#1\nopagebreak%
    \begin{tcbraster}[%
        raster equal height,%
        raster columns=2,%
        raster column skip=0.5mm,%
        raster row skip=0.5mm,%
        raster every box/.style={%
            enhanced,%
            sharp corners,%
            colback=white,%
            coltitle=black, colbacktitle=black!10!white,%
            boxrule=0pt, borderline={0.5pt}{0pt}{black},%
            title={\texttt\useMyProblemCounter},%
            },%
        ]%
        \begin{tcolorbox}[attach boxed title to top left]
            #3
            \tcblower\vspace{#2}
        \end{tcolorbox}
        \begin{tcolorbox}[attach boxed title to top right]
            #4
            \tcblower
        \end{tcolorbox}%
}{%
    \end{tcbraster}
}

% an environment for 4 adjacent problems
%
% #1 : directions for all the problems
% #2 : vertical height of the problem boxes
% #3 : details for problem 1
% #4 : details for problem 2
% #5 : details for problem 3
% #6 : details for problem 4
%
\newenvironment{myProblems4}[6]{%
    \noindent
    \textbf{\myHeadingFont Practice:}\hspace{0.5em}#1\nopagebreak%
    \begin{tcbraster}[%
        raster equal height,%
        raster columns=2,%
        raster column skip=0.5mm,%
        raster row skip=0.5mm,%
        raster every box/.style={%
            enhanced,%
            sharp corners,%
            colback=white,%
            coltitle=black, colbacktitle=black!10!white,%
            boxrule=0pt, borderline={0.5pt}{0pt}{black},%
            title={\texttt\useMyProblemCounter},%
            },%
        ]%
        \begin{tcolorbox}[attach boxed title to top left]
            #3
            \tcblower\vspace{#2}
        \end{tcolorbox}
        \begin{tcolorbox}[attach boxed title to top right]
            #4
            \tcblower
        \end{tcolorbox}%
        \begin{tcolorbox}[attach boxed title to bottom left]
            #5
            \tcblower
        \end{tcolorbox}%
        \begin{tcolorbox}[attach boxed title to bottom right]
            #6
            \tcblower
        \end{tcolorbox}%
}{%
    \end{tcbraster}
}
\pagestyle{myPagestyle}

\checkandfixthelayout
\setlist{labelindent=\parindent,leftmargin=*,itemsep=0.025em,label=$\circ$}

% ---------------------HEADER------------------------------
\begin{myNotesHeader}
    \item \myCognitiveVerb{solve} one-step linear equations
    \item \myCognitiveVerb{solve} multi-step linear equations
    \item \myCognitiveVerb{solve} formulas for one variable given the value of all the other variables
    \item \myCognitiveVerb{solve} formulas for one variable in terms of the other variables
\end{myNotesHeader}

\begin{myVocabulary}
    \myVocabularyWord{equation}
    {
        algebraic expressions connected with an $=$ sign 
    }
    \myVocabularyWord{formula}
    {
        an equation with more than one variable
    }
    \myVocabularyWord{inverse operation}
    {
        two operations that do \emph{opposite} things
        (They \emph{undo} each other.)
    }
    \myVocabularyWord{solve}
    {
        find the value of a variable that makes an equation a true statement
    }
\end{myVocabulary}

% ---------------------LESSON 2------------------------------
\begin{myLesson}[][]
    The key to success when solving equations is to get the unknown variable \emph{by itself}
    on one side of the equation.
    \begin{myLessonBox}
        It is not really important whether the variable is on the left or right.
        In this class, I will assume that you understand that
        $x=55$ means the exact same thing as $55=x$.
    \end{myLessonBox}
\end{myLesson}

% ---------------------CONCEPT 1------------------------------
\begin{myKeyConcepts}[To solve one-step equations\dots]
    Follow these steps:
    \setlist{labelindent=\parindent,}
    \begin{enumerate}
        \item \myEmph{Identify} the one operation in the equation.
        \item \myEmph{Find} the corresponding inverse operation.
        \item \myEmph{Apply} the inverse operation to \emph{both sides of the equation}.
        \item \myEmph{Solve} for the variable. 
    \end{enumerate}
\end{myKeyConcepts}

One-step equations are crazy-simple.
We will not be spending much time on them.
Here are two examples.

\myExample{
    Solve this equation:
    \(
        p - 6 = 4
    \)
}{1.25in}

\myExample{
    Solve this equation:
    \(
        -3p = 5
    \)
}{1.25in}

% ---------------------CONCEPT 2------------------------------
\begin{myKeyConcepts}[To solve multi-step equations\dots]
    Follow these steps:
    \setlist{labelindent=\parindent,}
    \begin{enumerate}
        \item \myEmph{Distribute} wherever it is possible.
        \item \myEmph{Combine} like terms on each side of the equation.
        \item \myEmph{Move} all terms with the variable to one side of the equation
        and all numbers to the other side using addition and subtraction as inverse operations.
        \item \myEmph{Combine} like terms again. This will give you a one-step equation.
        \item \myEmph{Solve} the one-step equation using multiplication and division as inverse operations.
    \end{enumerate}
\end{myKeyConcepts}

\myExample{
    Solve this equation:
    \(
        2(x+5) = -11
    \)
}{1.75in}

\myExample{
    Solve this equation:
    \(
        -13 = 5 + 4x -6x
    \)
}{2in}

\myExample{
    Solve this equation:
    \(
        3x + 20 = x-8
    \)
}{2in}

\myExample{
    Solve this equation:
    \(
        2(3-x) + 3 - 4x = 8 + 9x - 2(x-1)
    \)
}{3.75in}

% ---------------------LESSON 2------------------------------
\begin{myLesson}[][]
    A formula is just a fancy kind of equation that has more than one variable.
    For example, here is the quadratic formula:
    \[
        x = 
        \frac{
            - b 
            \pm
            b^2 - 4ac
        }
        {
            2a
        }
    \]
    where there are four variables $x$, $a$, $b$, $c$.
    If I tell you the values of three of those variables,
    in theory, you should be able to rewrite the equation 
    in order to solve for the remaining variable.
    We use the words \emph{solve} for this, as in 
    ``Solve that formula for $a$.''
\end{myLesson}

% ---------------------CONCEPT 3------------------------------
\begin{myKeyConcepts}[To solve a formula for one variable given all the other variables\dots]
    Follow these steps:
    \setlist{labelindent=\parindent,}
    \begin{enumerate}
        \item \myEmph{Substitute} the values of the other variables into the formula.
        This will give you an equation with only one variable left.
        \item \myEmph{Simplify} that equation, which will often involve combining like terms.
        \item \myEmph{Solve} that equation. This means rewriting the equation to get that variable by itself.
    \end{enumerate}
\end{myKeyConcepts}

\myExample{
    The formula for the perimeter, $P$, of a $h \times w$ rectangle is
    \[
        P = 2h + 2w
    \]
    Solve this formula for $h$ given that $P=256$ and $w=64$.
}{2.5in}

\myExample{
    The formula for the area, $A$, of a triangle is
    \[
        A = \frac{1}{2}bh
    \]
    where $b$ is the base and $h$ is the height of the triangle.
    A triangle has an area of 40 square feet 
    and a base of 10 feet.
    Solve this formula for the height, $h$, given those values for $b$ and $A$.
}{2in}

% ---------------------LESSON 3------------------------------
\begin{myLesson}[][]
    In science,
    we often want to solve a formula for one of the variables
    \myEmph{without} knowing the value of the other variables.
    This freaks algebra students out. 
    It shouldn't. 
    Here's why.

    Suppose I give you a formula like $F = 2a + 3b$.
    And then I tell you to solve it for $b$ given $F=1000$ and $a=10$.
    You could do that using what you just learned above.
    But then I change my mind and say $F=2000$ and $a=25$.
    You could do it again doing almose the same thing, just with different numbers.
    But then I change my mind again and say $F=900$ and $a=2.1$.
    You could do it yet again. And again. And again.
    
    Eventually you'll get tired of this game\footnote{
        Don't believe me? Come see me sometime later in the year, and I'll give you 
        an assignment of solving $F = ma$ for $a$ one hundred times for the different values of $F$ and $m$ like these:
        \begin{align*}
            100 &= 2a\\
            101 &= 2.5a\\
            103.2 &= 6.8a\\
            121 &= 52a\\
            1234 &= 89a
        \end{align*}
    },
    since each time I change my mind, you'd end up doing the exact same thing 
    just with different numbers.
    How many times to you have to do it, before you ask, ``Isn't there a better way?''
    There is a better way: \myEmph{solve the formula first}, before you plug in numbers.
\end{myLesson}

% ---------------------CONCEPT 3------------------------------
\begin{myKeyConcepts}[To solve a formula for one variable in terms of all the other variables\dots]
    \setlist{labelindent=\parindent,}
    \begin{enumerate}
        \item \myEmph{Rewrite} the equation so that you get the variable by itself.
        This process is virtually identical to how you solve the equation,
        except when you apply inverse operations, you will be writing algebra instead of doing arithmetic.
    \end{enumerate}
\end{myKeyConcepts}

These examples illustrate this idea.

\myExample{
    Solve the perimeter equation for $h$:
    \[
        P = 2h + 2w
    \]
}{2.75in}

\myExample{
    Solve this formula for $b$: 
    \[
        Q = 5a^2 + 2b -c
    \]
}{2in}

\myExample{
    Einstein's relativity equation is, $E = mc^2$.
    Solve this formula for $m$.
}{2in}

% \begin{myProblems2}%
%     {Factor the following monomials into prime factors.}%
%     {2in}%
%     %
%     {\( 32x^2 \)}
%     {\( 8 x^3y^2z \)}
% \end{myProblems2}
  


\end{document}