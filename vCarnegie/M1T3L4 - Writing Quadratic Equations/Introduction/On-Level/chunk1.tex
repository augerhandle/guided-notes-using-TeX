\section{The General Idea}

Given 
\begin{itemize}[nosep]
    \item a point (called the \gap{focus}), and 
    \item a line (called the \gap{directrix}),
\end{itemize}
the set of all points that are the \gap{same} \gap{distance} 
from the focus (point) and directrix (line) is a \gap{parabola}.

\begin{center}
\begin{myTikzpictureGrid}{0.5} [-1]{15}[-1]{15}
        % Here's the directrix
        \draw[very thick, dashed, black] (-2,1) -- (17,1); \node[] at (17.4,1.1) {\small $\mathscr{D}$};
        %
        % Here are lines increasingly distant from the directrix
        \draw[thick, dotted, black] (-2,3.05) -- (16,3.05); \node[] at (16.1,3.05) {\tiny 2};
        \draw[thick, dotted, black] (-2,4.05) -- (16,4.05); \node[] at (16.1,4.1) {\tiny 3};
        \draw[thick, dotted, black] (-2,5.05) -- (16,5.05); \node[] at (16.1,5.1) {\tiny 4};
        \draw[thick, dotted, black] (-2,6.05) -- (16,6.05); \node[] at (16.1,6.1) {\tiny 5};
        \draw[thick, dotted, black] (-2,7.05) -- (16,7.05); \node[] at (16.1,7.1) {\tiny 6};
        \draw[thick, dotted, black] (-2,8.05) -- (16,8.05); \node[] at (16.1,8.1) {\tiny 7};
        \draw[thick, dotted, black] (-2,9.05) -- (16,9.05); \node[] at (16.1,9.1) {\tiny 8};
        %
        % Here's the focus
        \draw[color=black,thick,fill=white] (7,7) circle [radius=0.35];
        \draw[color=black,thick,fill=black,opacity=1.0] (7,7) circle [radius=0.15];
        \node[] at (6.5,7.5) {\tiny $F$};
        %
        % Here are circles increasingly distant from the focus
        \draw[color=black,dotted,thick,fill=white,fill opacity=0.](7,7) circle [radius=2]; \node[] at (7,9) {\tiny 2};
        \draw[color=black,dotted,thick,fill=white,fill opacity=0.](7,7) circle [radius=3]; \node[] at (7,10) {\tiny 3};
        \draw[color=black,dotted,thick,fill=white,fill opacity=0.](7,7) circle [radius=4]; \node[] at (7,11) {\tiny 4};
        \draw[color=black,dotted,thick,fill=white,fill opacity=0.](7,7) circle [radius=5]; \node[] at (7,12) {\tiny 5};
        \draw[color=black,dotted,thick,fill=white,fill opacity=0.](7,7) circle [radius=6]; \node[] at (7,13) {\tiny 6};
        \draw[color=black,dotted,thick,fill=white,fill opacity=0.](7,7) circle [radius=7]; \node[] at (7,14) {\tiny 7};
        \draw[color=black,dotted,thick,fill=white,fill opacity=0.](7,7) circle [radius=8]; \node[] at (7,15) {\tiny 8};
        % \whenTEACHER{
            \draw[color=black,thick,fill=black] (7,4) circle [radius=0.15];
            \draw[color=black,thick,fill=black] (10.55,5.05) circle [radius=0.15];
            \draw[color=black,thick,fill=black] (3.55,5.05) circle [radius=0.15];
            \draw[color=black,thick,fill=black] (11.95,6.1) circle [radius=0.15];
            \draw[color=black,thick,fill=black] (2,6.1) circle [radius=0.15];
            \draw[color=black,thick,fill=black] (12.95,7) circle [radius=0.15];
            \draw[color=black,thick,fill=black] (0.95,7) circle [radius=0.15];
            \draw[color=black,thick,fill=black] (13.95,8) circle [radius=0.15];
            \draw[color=black,thick,fill=black] (0.,8) circle [radius=0.15];
            \draw[color=black,thick,fill=black] (14.75,9) circle [radius=0.15];
            \draw[color=black,thick,fill=black] (-0.75,9) circle [radius=0.15];
            %
            \draw[thick, solid, black] (7,7) -- (10.55,5.05);
            \node[black] at (9.1,6.5) {4};
            \draw[thick, solid, black] (10.55,5.05) -- (10.55,1);
            \draw[ thick, solid, black] (10.1,1.4) -- (10.1,1);
            \draw[ thick, solid, black] (10.1,1.4) -- (10.55,1.4);
            \node[black] at (11.1,2.5) {4};
            %
            \tkzFct[ solid, ultra thick, samples=500, domain=-10:15,]{((x-7)**2)/12 + 4}
        % }
\end{myTikzpictureGrid} 
\end{center}

