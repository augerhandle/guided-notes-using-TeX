\section{The Geometry of Parabolas}

The geometry of parabolas has six ``pieces''.
\begin{enumerate}[nosep]
    \item the \gap{focus}, $F$, which is a \gap{point},
    \item the \gap{directrix} $\mathscr{D}$, which is a \gap{line},
    \item the \gap{vertex}, $F$, which is another \gap{point},
    \item the \gap{axis} \gap{of} \gap{symmetry} (AOS), which is a \gap{line},
    \item a parameter \gap{$p$}, which is a \gap{distance},
    \item and the parabola itself.
\end{enumerate}

\subsection{Vertical Parabolas} 

\begin{minipage}{0.49\textwidth}
\begin{myTikzpictureGrid}{0.45} [-1]{15}[-1]{15}
    \whenTEACHER{
        % Here's the AOS 
        \draw[very thick, dashed, red] (7,-2) -- (7,16);
        \node[red] at (6,16) {\tiny AOS};
        % Here's the focus
        \draw[color=red,thick,fill=white] (7,11) circle [radius=0.35];
        \draw[color=red,thick,fill=red,opacity=1.0] (7,11) circle [radius=0.15];
        \node[red] at (5.5,11) {\tiny $F$ $(7,11)$};
        %
        % Here's the vertex
        \draw[color=red,thick,fill=red] (7,7) circle [radius=0.25];
        \node[red] at (5.5,7) {\tiny $V$ $(7,7)$};
        %
        % Here's the directrix
        \draw[ultra thick, solid, red] (-2,3) -- (12,3); 
        \node[red] at (14,3.1) {\small $\mathscr{D}$: $y=3$};
        %
        % Here are the annotations for p
        \draw[to-to,thick, solid, red] (7.75,7.05) -- (7.75,11); 
        \node[red] at (8.25,9) {\tiny $p=4$};
        \draw[to-to,thick, solid, red] (7.75,3) -- (7.75,6.95); 
        \node[red] at (8.25,5) {\tiny $p=4$};
        %
        % Here is the parabola
        \tkzFct[ dotted, very thick, samples=500, domain=-10:15,]{((x-7)**2)/12 + 7}
    }
\end{myTikzpictureGrid}
\end{minipage}
%
\begin{minipage}{0.49\textwidth}
    \tiny
    \whenTEACHER{
        \begin{enumerate}[fullwidth,]
            \item Start by drawing the focus $F$ and directrix $\mathscr{D}$
                \begin{itemize}[nosep]
                    \item Say, ``The focus is a POINT.''
                    \item Say, ``The directrix is a HORIZONTAL LINE.''
                    \item Say, ``The eq of a horiz line is \fbox{$y=\dots$}.''
                \end{itemize}
            \item Draw the vertex $V$.
                \begin{itemize}[nosep]
                    \item Say, ``The vertex is HALFWAY between $F$ \& $\mathscr{D}$.''
                    \item B/c parabola pts are the SAME dist from $F$ and $\mathscr{D}$.
                \end{itemize}
            \item Show the 2 $p$'s on the graph. 
            \item Draw the AOS 
                \begin{itemize}[nosep]
                    \item Say, ``perpendicular to the directrix''
                    \item Say, ``through the vertex \& focus''
                \end{itemize}
            \item Sketch in a parabola.
                \begin{itemize}[nosep]
                    \item Say and write, ``start at the vertex''
                    \item Say and write, ``bend AWAY from the directrix''
                    \item Say and write, ``WRAP around the focus''
                \end{itemize}
        \end{enumerate}
    }
\end{minipage}

\subsection{Horizontal Parabolas}


\begin{minipage}{0.49\textwidth}
    \begin{myTikzpictureGrid}{0.45} [-1]{15}[-1]{15}
        \whenTEACHER{
            % Here's the AOS 
            \draw[very thick, dashed, red] (-1,8) -- (14,8);
            \node[red] at (15,8) {\tiny AOS};
            % Here's the focus
            \draw[color=red,thick,fill=white] (8,8) circle [radius=0.35];
            \draw[color=red,thick,fill=red,opacity=1.0] (8,8) circle [radius=0.15];
            \node[red] at (8,8.75) {\tiny $F$ $(8,8)$};
            %
            % Here's the vertex
            \draw[color=red,thick,fill=red] (5,8) circle [radius=0.25];
            \node[red] at (5,8.75) {\tiny $V$ $(5,8)$};
            %
            % Here's the directrix
            \draw[ultra thick, solid, red] (2,1) -- (2,14); 
            \node[red] at (2,0.5) {\small $\mathscr{D}$: $x=2$};
            %
            % Here are the annotations for p
            \draw[to-to,thick, solid, red] (2,7.25) -- (4.95,7.25); 
            \node[red] at (3.5,6.75) {\tiny $p=3$};
            \draw[to-to,thick, solid, red] (5.05,7.25) -- (8,7.25); 
            \node[red] at (6.5,6.75) {\tiny $p=3$};
            % %
            % % Here is the parabola
            \tkzFct[ dotted, very thick, samples=500, domain=4:15,]{sqrt(12*x - 60) + 8}
            \tkzFct[ dotted, very thick, samples=500, domain=4:15,]{-sqrt(12*x - 60) + 8}
        }
    \end{myTikzpictureGrid}
    \end{minipage}
    %
    \begin{minipage}{0.49\textwidth}
        \tiny
        \whenTEACHER{
            \begin{enumerate}[fullwidth,]
                \item Start by drawing the focus $F$ and directrix $\mathscr{D}$
                    \begin{itemize}[nosep]
                        \item Say, ``The focus is a point.''
                        \item Say, ``The directrix is a VERTICAL LINE.''
                        \item Say, ``The eq of a vert line is \fbox{$x=\dots$}.''
                    \end{itemize}
                \item Draw the vertex $V$.
                    \begin{itemize}[nosep]
                        \item Say, ``The vertex is HALFWAY between $F$ \& $\mathscr{D}$.''
                        \item B/c parabola pts are the SAME dist from $F$ and $\mathscr{D}$.
                    \end{itemize}
                \item Show the 2 $p$'s on the graph. 
                \item Draw the AOS 
                    \begin{itemize}[nosep]
                        \item Say, ``perpendicular to the directrix''
                        \item Say, ``through the vertex \& focus''
                    \end{itemize}
                \item Sketch in a parabola.
                    \begin{itemize}[nosep]
                        \item Say and write, ``start at the vertex''
                        \item Say and write, ``bend AWAY from the directrix''
                        \item Say and write, ``WRAP around the focus''
                    \end{itemize}
            \end{enumerate}
        }
    \end{minipage}
    