\section{Finding Vertical Asymptotes and Holes}

\begin{myConceptSteps}{To find the vertical asymptotes and holes of a rational function\dots}
    \myStep{factor}{%
        \gap{Completely} \gap{factor} the numerator and denominator.
    }
    \myStep{simplify}{%
        Cancel pairs of \gap{matching} factors in the numerator/denominator.
        \begin{itemize}[nosep]
            \item The \gap{simplified} \gap{function} is the function 
                without the cancelled factors.
            \item Make sure to \gap{remember} which factors you cancelled.
        \end{itemize}
    }
    \myStep{holes}{%
        Holes come from \gap{cancelled} factors.
        For each cancelled factor $(x-h)$,
        \begin{itemize}[nosep]
            \item Evaluate the function at $h$: \gap{$f(h)$}.
            \item There is a hole located at \gap{$(h, f(h))$}.
            \item It lies \gap{on} the graph of the simplified function.
        \end{itemize}
    }
    \myStep{vertical asymptotes}{%
        Verical asymptotes come from factors in the \myEmph{denominator} 
        that were \gap{not} \gap{cancelled}.
        For each remaining factor $(x-h)$ in the simplified function,
        \begin{itemize}[nosep]
            \item The asymptote is the \gap{vertical} \gap{line} 
                $x = h$.
            \item The graph of the simplified function \gap{approaches} 
                but \gap{never} reaches the line.
        \end{itemize}
    }
\end{myConceptSteps}

