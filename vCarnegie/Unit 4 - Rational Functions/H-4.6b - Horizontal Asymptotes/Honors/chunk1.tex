\section{Introduction}

\subsection{All Things Great and Small}

\begin{minipage}{0.5\textwidth}
    \ttfamily\small
    \renewcommand{\arraystretch}{2}
    \begin{tabular}{c||lllllll}
        $x$ 
        & 2 & 1   & 0.1 & 0.01 & 0.001 & 0.000001 \\
        $\frac{1}{x} $
        & 0.5 & 1 & 10  & 100 & 1000   & 1,000,000\\
    \end{tabular}
\end{minipage}
\begin{minipage}{0.45\textwidth}
    When the \myEmph{denominator} gets \gap{small}, 
    the fraction gets \gap{large}.
\end{minipage}

\begin{minipage}{0.5\textwidth}
\ttfamily\small
\renewcommand{\arraystretch}{2}
\begin{tabular}{c||lllllll}
    $x$ 
    & 0.5 & 1 & 10  & 100 & 1000   & 1,000,000\\
    $\frac{1}{x} $
    & 2 & 1   & 0.1 & 0.01 & 0.001 & 0.000001 \\
\end{tabular}
\end{minipage}
\begin{minipage}{0.45\textwidth}
    When the \myEmph{denominator} gets \gap{large}, 
    the fraction gets \gap{small}.
\end{minipage}


\subsection{End Behaviors of Rational Functions}
\gap{Horizontal} \gap{asymptotes} of rational functions 
can be thought of as the far left and right \gap{end} \gap{behavior} of the functions.

\vspace{-1\onelineskip}
\begin{tcolorbox}[center,width=6in,colback=white,colframe=white,]
    \hfill
    \begin{minipage}{0.25\textwidth}
        \large
        $
        \frac
            {6x^3 - x^2 + x - 30}
            {2x^3 - x^2 + 20}
        $
    \end{minipage}
    \hfill
    \begin{minipage}{0.6\textwidth}
        \begin{myTikzpictureGrid}{0.175} {10}{6}
            \tkzFct[ solid, gray, ultra thick, samples=100, domain=-10:-2.7,] 
                { (6*x**3 - x**2 + x - 30)  /  (2*x**3 - x**2 + 20) }
            \tkzFct[ solid, gray, ultra thick, samples=100, domain=-1.6:10,] 
                { (6*x**3 - x**2 + x - 30)  /  (2*x**3 - x**2 + 20) }
            \draw [dashed, black, very thick] (-12,3) -- (12,3);
    \end{myTikzpictureGrid}
    \end{minipage}
    \hfill{\,}
\end{tcolorbox}

Exactly what happens depends on the \gap{degree} of the polynomials 
and on their \gap{leading} \gap{coefficients}.