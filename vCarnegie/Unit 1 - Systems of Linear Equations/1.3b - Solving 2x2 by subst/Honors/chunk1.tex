
Last time, 
we solved systems like

\hfil
\begin{minipage}{0.3\textwidth}
    \begin{align*}
            y &= 2x - 4 \\
            y &= -x + 2
    \end{align*}
\end{minipage}
\begin{minipage}{0.3\textwidth}
    \begin{align*} 
        x = 2y - 4 \\
        3x + 3y =  2
    \end{align*}
\end{minipage}
\hfil

Today's systems are more \gap{complicated}:
\begin{align*} 
    &5x + 10y = 15 \\
    &3x + 8y =  7
\end{align*}

\begin{myConceptSteps}{To solve a system by substitution\dots}
    \myStep{first variable}{
        Solve one of the equations for either \gap{$x$} or \gap{$y$}.
        The result is usually an \gap{expression}.
    }
    \myStep{substitute}{Substitute that into the \gap{other} equation.}
    \myStep{second variable}{
        Solve that equation for the \gap{second} variable
        (a \gap{number}).
    }
    \myStep{substitute}{
        Substitute that into the expression for the {first} variable
        (a \gap{number}).
    }
\end{myConceptSteps}

\vspace{2\onelineskip} 
\begin{tcolorbox}[center,width=5.5in]
    In Step-1,
    make things easy for yourself  
    by picking the \gap{easiest} equation to solve.
\end{tcolorbox}
