\section{Writing a system of 2-variable inequalities from graphs}

\begin{myConceptSteps}{To write the system of inequalities given the graph of its solution\dots}
  [Follow these steps for \gap{each} \gap{boundary} \gap{line} in the graph.]
  \myStep{equations}{%
    Write an equation for the line ($y=mx+b$, \,\,$y=k$, \,\,$x=k$).
  }
  \myStep{inequalities}{%
    Write the inequality for the line based on shading.
    \begin{itemize}[fullwidth]
      \item Use {\Large $>$} or {\Large $\ge$} if shading is \gap{above} the line
        (or to the \gap{right} of vertical lines).
      \item Use {\Large $<$} or {\Large $\le$} if shading is \gap{below} the line
        (or to the \gap{left} of vertical lines).
    \end{itemize}
  }
  \myStep{equals or not}[?]{
    \begin{itemize}[fullwidth]
      \item Use {\Large $\ge$} or {\Large $\le$} for \gap{solid} lines.
      \item Use {\Large $>$} or {\Large $<$} for \gap{dashed} lines.
    \end{itemize}
  }
\end{myConceptSteps}



\myWideProblemWithContent[%
  Write the system of inequalities 
  whose solution is shown in this graph.]
{
  \begin{myTikzpictureGrid}{0.5} [-6]{5}{5} [dotted]
    \tkzVLine[color=black,style=solid,ultra thick]{4}
    \tkzFct[ solid, ultra thick, samples=100, domain=-10:10,]{x+3}
    \tkzFct[ loosely dashed, ultra thick, samples=100, domain=-10:10,]{3}
    \tkzFct[ loosely dashed, ultra thick, samples=100, domain=-10:10,]{-2}
    %
    % I'd like to just shade between the left line (a) and right (vertical), but
    % we can't use a VLine for shading. Evidently just functions.
    % So I split it into two pieces and shaded accordingly.
    %
    \tkzDrawAreafg[between= a and c,color=black!75,domain = -5:0]
    \tkzDrawAreafg[between= b and c,color=black!75,domain = 0:4]
  \end{myTikzpictureGrid}
}
