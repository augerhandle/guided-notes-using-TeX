\section{Simple Systems of Equations}

The systems in this lesson will always be written like this:

\begin{minipage}{0.3\textwidth}
    \begin{align*}
        y = -2x + 3 \\
        y = x-3
    \end{align*}    
\end{minipage}
\begin{minipage}{0.3\textwidth}
    \begin{align*}
        y = -2x + 3 \\
        -x+y = -3
    \end{align*}    
\end{minipage}
\begin{minipage}{0.3\textwidth}
    \begin{align*}
        2x + y = 3 \\
        x = y+3
    \end{align*}    
\end{minipage}

\begin{tcolorbox}[center,width=4in]
    These $2\times2$ systems are ``simple'' since one (or both) equations are already solved
    for one of the variables.
    They start with  
    \gap{$y =$} or \gap{$x=$}.
\end{tcolorbox}
We could solve these by \gap{graphing}.

\begin{center}
    \begin{myTikzpictureGrid}{0.5} [-5]{5}[-5]{5}
        \tkzFct[ solid, ultra thick, samples=100, domain=-10:10,]{-2*x + 3}
        % \tkzText(13,8){$x - y = 0$}
        \tkzFct[ solid, ultra thick, samples=100, domain=-10:10,]{x-3}
        % \tkzText(-13,7.25){$x + 2y = 6$}
        \end{myTikzpictureGrid}
\end{center}
%
But today we solve them 
\gap{algebraically}.