\dummypart{2}{Polynomials}
\myLesson{Factoring Polynomials with GCFs}[4]

\begin{myObjectives}
    \myObjective{completely factor}{a monomial}
    \myObjective{find}{the GCF of a polynomial}
    \myObjective{factor}{polynomials by removing the GCF}
\end{myObjectives}

\begin{myVocabulary}
    \myDefinition{coefficient}{the \gap{number} in front of a variable}
    \myDefinition{factors}{things that are \gap{multiplied} together}
    \myDefinition{common factor}{a factor of \gap{several} monomial terms}
    \myDefinition{greatest common factor}{the \gap{largest} common factor}
    \myDefinition{GCF}{\gap{greatest} \gap{common} \gap{factor}}
\end{myVocabulary}

% \begin{tcolorbox}[center,width=5in]
%     When we {\itshape factor something}, 
%     we \gap{rewrite} it as several pieces (the ``factors'')
%     \gap{multiplied} together.
% \end{tcolorbox}


% \section{Quick Polynomial Division Using Synethetic Division}

% % \begin{myCenteredBox}[width=5.25in,]
%     \gap{Synthetic} division is an alternative to \gap{long} division.
%     Use it when the {\bfseries\itshape divisor} is a \gap{linear} \gap{binomial}. 
%     \begin{itemize}[nosep]
%         \item It is fast.
%         \item It is short.
%     \end{itemize}
% % \end{myCenteredBox}


\begin{myConceptSteps}{~To divide polynomials using synthetic division\dots}
    \myStep{setup}{%
        \begin{itemize}[nosep,leftmargin=*]
            \item Draw an L-shaped frame for two rows of numbers.
            \item Write the coefficients of the {\bfseries\scshape Dividend}.
                \begin{itemize}[nosep,leftmargin=*]
                    \item[$\circ$] in \gap{descending} exponent order
                    \item[$\circ$] with \gap{zeros} for missing terms
                \end{itemize}
            \item Write $a$ from the {\bfseries\scshape Divisor}. 
                \begin{itemize}[nosep,leftmargin=*]
                    \item[$\circ$] $a$ is the \gap{opposite} of what you see.
                \end{itemize}
        \end{itemize}
    }
    \myStep{start}{Bring the \gap{leading} coefficient down.}
    \myStep{multiply}{Multiply $a$ times the number at the bottom.}
        \begin{itemize}[nosep,leftmargin=*]
            \item Put it under the next coefficient.
        \end{itemize} 
    \myStep{add}{Add the numbers in the right column.}
        \begin{itemize}[nosep,leftmargin=*]
            \item Put the sum at the bottom.
        \end{itemize} 
    \myStep{repeat}{Repeat until done. (Goto step 3.)}
    \myStep{identify}{Identify the {\bfseries\scshape Quotient} and {\bfseries\scshape Remainder}.}
    \myStep{result}{%
        The result is 
        {
            $ 
                \text{\bfseries\scshape Quotient} 
                + 
                \frac{\text{\bfseries\scshape Remainder}}{\text{\bfseries\scshape Divisor}} 
            $
        }
    }
\end{myConceptSteps}

\myProblems[Divide these polynomials.]
    {
        $
            (x^3 + 18x^2 + 71x -74) 
            \div
            (x+8)
        $
    }
    {
        \Large
        $\frac
            {t^3 - 49t -8} 
            {t - 7}
        $
    }
    {2.5in}

\begin{myConceptSteps}{
    To solve a \gap{quadratic} equation using the quadratic formula\dots
}
\myStep{standard form}{Write the equation as \gap{$ax^2 + bx + c = 0$}.}
\myStep{coefficients}{Write down the values of \gap{$a$}, \gap{$b$}, \gap{$c$}. }
\myStep{discriminant}{Calculate the discriminant as \gap{$ D = b^2 - 4ac $}.}
\myStep{quadratic formula}{
    The quadratic formula is:
    \begin{tcolorbox}[center,width=1.5in,colback=white,]
        $\displaystyle x = \frac{-b \pm \sqrt{D}}{2a}$
    \end{tcolorbox}
    {\normalsize 
    which is really \gap{two} solutions:
    \begin{align*} 
        x &= \frac 
        {-b \bm{+} \myRoot{b^2 - 4ac}}
        {2a}
        &
        x &= \frac 
        {-b \bm{-} \myRoot{b^2 - 4ac}}
        {2a}
    \end{align*}
    }
}
\myStep{substitute}{Substitute $a$, $b$, $c$, and $D$ into this formula.
Usually you will get \gap{two} solutions.
\begin{itemize}
    \item When $D =$ \gap{$0$}, there is only \gap{one} \gap{solution}.
    \item When $D$ is \gap{negative}, there are \gap{no} \gap{real} \gap{solutions}.
\end{itemize}
}
\myStep{evaluate}{\gap{Simplify} the right-hand-side of the equations.}
\end{myConceptSteps}
\newpage

\section{The Factor Theorem}

Start with the Division Algorithm.

% \vspace{5in}
\vfill 

What this means is

{
    \hfill 
    \begin{tcolorbox}[width=2.5in,height=5em,valign=center,nobeforeafter]
        \centering
        {\gap{$P(a)=0$}}
    \end{tcolorbox}
    \hfill 
    \begin{tcolorbox}[
        width=4em,height=5em,valign=center,
        colback=white, colframe=white,
        nobeforeafter
        ]
        \centering
        \huge
        $\bm{\Leftrightarrow}$
    \end{tcolorbox}
    \hfill 
    \begin{tcolorbox}[width=2.5in,height=5em,valign=center,nobeforeafter]
        \centering
        $(x-a)$ is a {\bfseries\itshape factor} of $P(x)$
    \end{tcolorbox}
    \hfill{}
}

or

{
    \hfill 
    \begin{tcolorbox}[width=2.5in,valign=center,nobeforeafter]
        \centering
        {\large $P(x) \div (x-a) $}\\[1\baselineskip]
        \gap{remainder}{$=0$}
        % \gap{remainder} of\\[1\baselineskip] 
        % {\large $P(x) \div (x-a) $} \\[1\baselineskip] 
        % is zero
    \end{tcolorbox}
    \hfill 
    \begin{tcolorbox}[
        width=4em,height=5em,valign=center,
        colback=white, colframe=white,
        nobeforeafter
        ]
        \centering
        \huge
        $\bm{\Leftrightarrow}$
    \end{tcolorbox}
    \hfill 
    \begin{tcolorbox}[width=2.5in,height=5em,valign=center,nobeforeafter]
        \centering
        $(x-a)$ is a {\bfseries\itshape factor} of $P(x)$
    \end{tcolorbox}
    \hfill{}
}

