% Unit circle
% Author: Supreme Aryal
% A unit circle with cosine and sine values for some
% common angles.
\documentclass{article}
\usepackage{tikz}
\usepackage{amsmath}
%%%<
\usepackage{verbatim}
% \usepackage[active,tightpage]{preview}
% \PreviewEnvironment{tikzpicture}
% \setlength\PreviewBorder{5pt}%
%%%>


\usepackage[top=1in,bottom=1in,right=1in,left=1in]{geometry}
\begin{document}
\thispagestyle{empty}

\newcommand{\myRadius}{2in}
\newcommand{\myRadiusIncrement}{0.175in}
\newcommand{\myInnerRadius}{\myRadius-\myRadiusIncrement}
\newcommand{\myOuterRadius}{\myRadius+\myRadiusIncrement}
\newcommand{\myFarOuterRadius}{\myRadius+1in}

\newcommand{\mySquareEdge}{3.4cm}
\newcommand{\mySquareInset}{0.75cm}

\newcommand{\mySquare}[1]{
    \begin{tikzpicture}[scale=#1,cap=round,>=latex]
        \draw[black,solid,ultra thick] (-\mySquareEdge,\mySquareEdge) -- (\mySquareEdge,\mySquareEdge);
        \draw[black,solid,ultra thick] (-\mySquareEdge,\mySquareEdge) -- (-\mySquareEdge,-\mySquareEdge);
        \draw[black,solid,ultra thick] (\mySquareEdge,\mySquareEdge) -- (\mySquareEdge,-\mySquareEdge);
        \draw[black,solid,ultra thick] (-\mySquareEdge,-\mySquareEdge) -- (\mySquareEdge,-\mySquareEdge);

        \draw (-\mySquareEdge+\mySquareInset,\mySquareEdge-\mySquareInset) 
            node[fill=white,inner sep=0pt,minimum size=3em] {\centering\Huge\sffamily a};
        \draw (-\mySquareEdge+\mySquareInset,-\mySquareEdge+\mySquareInset) 
            node[fill=white,inner sep=0pt,minimum size=2em] {\Huge\sffamily d};
        \draw (\mySquareEdge-\mySquareInset,-\mySquareEdge+\mySquareInset) 
            node[fill=white,inner sep=0pt,minimum size=2em] {\Huge\sffamily c};
        \draw (\mySquareEdge-\mySquareInset,\mySquareEdge-\mySquareInset) 
            node[fill=white,inner sep=0pt,minimum size=2em] {\Huge\sffamily b};

        \draw[solid,fill=black] (0cm,0cm) circle(0.05in);
        \foreach \x in {45,135,...,315} {
            % lines from center to point
            \draw[black,dotted] (0cm,0cm) -- (\x:\myInnerRadius);
        }
        % \draw[] (0cm,0cm) circle(\myInnerRadius);
        % \draw[] (0cm,0cm) circle(\myRadius);


    \end{tikzpicture}
}

\ttfamily
\vfill 
\begin{center}
    \renewcommand{\arraystretch}{4}
    \begin{tabular}{cc}
        \mySquare{1.0} & \mySquare{1.0} \\ 
        \mySquare{1.0} & \mySquare{1.0} \\ 
        \mySquare{1.0} & \mySquare{1.0} \\ 
    \end{tabular} 
\end{center}
\vfill 
\end{document}