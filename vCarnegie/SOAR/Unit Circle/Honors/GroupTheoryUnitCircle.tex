% Unit circle
% Author: Supreme Aryal
% A unit circle with cosine and sine values for some
% common angles.
\documentclass[landscape]{article}
\usepackage{tikz}
%%%<
\usepackage{verbatim}
\usepackage[active,tightpage]{preview}
\PreviewEnvironment{tikzpicture}
\setlength\PreviewBorder{5pt}%
%%%>

\begin{comment}
:Title: Unit circle

A unit circle with cosine and sine values for some common angles.
\end{comment}

\usepackage[top=1in,bottom=1in,right=1in,left=1in]{geometry}
\begin{document}

    \newcommand{\myRadius}{1cm}
    \newcommand{\myInnerRadius}{0.95\myRadius}
    \newcommand{\myOuterRadius}{1.1\myRadius}

    \begin{tikzpicture}[scale=5.0,cap=round,>=latex]

        \foreach \x in {0} {
                % lines from center to point
                \draw[black,dotted] (0cm,0cm) -- (\x:\myRadius);
                \draw[black,solid,ultra thick,line cap=butt] (\x:\myInnerRadius) -- (\x:\myRadius);
                % draw each angle in degrees
                \draw (\x:1.1cm) node[fill=white,inner sep=0pt,minimum size=2em] {\Large $0^\circ \atop 360^\circ$};
        }

        \foreach \x in {15,30,...,345} {
                % draw each angle in degrees
                \draw (\x:\myOuterRadius) node[fill=white,inner sep=0pt,minimum size=2em] {\small$\x^\circ$};
                % lines from center to point
                \draw[gray,dotted] (0cm,0cm) -- (\x:\myRadius);
                \draw[black,solid,thick,line cap=butt] (\x:\myInnerRadius) -- (\x:\myRadius);
        }

        \foreach \x in {0,45,...,360} {
                % lines from center to point
                \draw[gray,solid] (0cm,0cm) -- (\x:\myRadius);
                \draw[black,solid,line width = 3pt,line cap=butt] (\x:\myInnerRadius) -- (\x:\myRadius);
        }

        % draw the unit circle
        \draw[] (0cm,0cm) circle(\myInnerRadius);
        \draw[] (0cm,0cm) circle(\myRadius);
    \end{tikzpicture}
\end{document}