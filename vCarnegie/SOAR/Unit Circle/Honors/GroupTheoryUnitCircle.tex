% Unit circle
% Author: Supreme Aryal
% A unit circle with cosine and sine values for some
% common angles.
\documentclass{article}
\usepackage{tikz}
\usepackage{amsmath}
%%%<
\usepackage{verbatim}
% \usepackage[active,tightpage]{preview}
% \PreviewEnvironment{tikzpicture}
% \setlength\PreviewBorder{5pt}%
%%%>


\usepackage[top=1in,bottom=1in,right=1in,left=1in]{geometry}
\begin{document}

\newcommand{\myRadius}{2in}
\newcommand{\myRadiusIncrement}{0.175in}
\newcommand{\myInnerRadius}{\myRadius-\myRadiusIncrement}
\newcommand{\myOuterRadius}{\myRadius+\myRadiusIncrement}
\newcommand{\myFarOuterRadius}{\myRadius+1in}

\ttfamily
\vfill 
\begin{center}
    \begin{tikzpicture}[scale=1.0,cap=round,>=latex]

        \foreach \x in {0} {
                % lines from center to point
                \draw[black,dotted] (0cm,0cm) -- (\x:\myRadius);
                \draw[black,solid,ultra thick,line cap=butt] (\x:\myInnerRadius) -- (\x:\myRadius);
                % draw each angle in degrees
                \draw (\x:\myOuterRadius) node[fill=white,inner sep=0pt,minimum size=2em] {\Large $\text{0}^\circ \atop \text{360}^\circ$};
        }

        \foreach \x in {0,45,...,360} {
                % lines from center to point
                \draw[gray,solid] (0cm,0cm) -- (\x:\myRadius);
                \draw[black,solid,line width = 3pt,line cap=butt] (\x:\myInnerRadius) -- (\x:\myRadius);
                \draw[black,solid,line cap=butt] (\x:\myRadius) -- (\x:\myFarOuterRadius);
        }

        \foreach \x in {15,30,...,345} {
                % draw each angle in degrees
                \draw (\x:\myOuterRadius) node[fill=white,inner sep=0pt,minimum size=2em] {\normalsize$\text{\x}^\circ$};
                % lines from center to point
                \draw[gray,dotted] (0cm,0cm) -- (\x:\myRadius);
                \draw[black,solid,thick,line cap=butt] (\x:\myInnerRadius) -- (\x:\myRadius);
        }

        % draw the unit circle
        \draw[] (0cm,0cm) circle(\myInnerRadius);
        \draw[] (0cm,0cm) circle(\myRadius);
    \end{tikzpicture}
\end{center}
\vfill 
\end{document}