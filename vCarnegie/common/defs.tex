% ---------------------------------------------------------------------------
% A "strut" command
%
% I can use this command to squeeze some NEGATIVE vertical space into text 
% without taking up any horizontal space.
%
% #1 descent distance below the baseline
% ---------------------------------------------------------------------------
\newcommand{\myDropStrutLength}{-13ex}
\NewDocumentCommand{\myDropStrut}{O{\myDropStrutLength}}
{%
    \rule[#1]{0pt}{0pt}%
}

% ---------------------------------------------------------------------------
% A warning box similar to that in the tcolorbox documentation.
% This is code written by Thomas F. Sturm.
% See: https://tex.stackexchange.com/questions/307432/tcolorbox-manual-tip-box-how-to-clone
%
% I made some minor changes to what shows up in that post.
% ---------------------------------------------------------------------------
\newtcolorbox{myWarningBox}[1][]{enhanced,
  before skip=2mm, after skip=3mm,
  boxrule=0.4pt,
  left=5mm, right=2mm, top=1mm, bottom=1mm,
  colback=yellow!20, colframe=yellow!20!black,
  sharp corners,rounded corners=southeast,arc is angular,arc=3mm,
  underlay={%
    \path[fill=tcbcolback!80!black] 
        ([yshift=3mm]interior.south east)--++(-0.4,-0.1)--++(0.1,-0.2);
    \path[draw=tcbcolframe,shorten <=-0.05mm,shorten >=-0.05mm] 
        ([yshift=3mm]interior.south east)--++(-0.4,-0.1)--++(0.1,-0.2);
    \path[fill=yellow,draw=none] 
        (interior.south west) rectangle node[black]{\huge\bfseries !} ([xshift=4mm]interior.north west);
    },
  drop fuzzy shadow,#1}


% ---------------------------------------------------------------------------
% I'm SO tired of writing these two explicitly!
% ---------------------------------------------------------------------------
\newcommand{\myEmph}[1]{{\bfseries\itshape#1}}

% ---------------------------------------------------------------------------
% Draw a circle around the argument
% ---------------------------------------------------------------------------
\newcommand*\mycirc[1]{%
  \begin{tikzpicture}
    \node[draw,circle,inner sep=2pt] {#1};
  \end{tikzpicture}
}


% ---------------------------------------------------------------------------
% For writing TI-84 instructions it's useful to have a visually
% distinct way to format the keys.
% ---------------------------------------------------------------------------
\newcommand{\myKey}[1]{{\ttfamily [#1]}}

\newcommand{\myDesmos}{{\scshape Desmos~}}
\newcommand{\myTi}{{\scshape TI-84~}}

% % ---------------------------------------------------------------------------
% % Teachers vs. Students ("tstu")
% % ---------------------------------------------------------------------------
% \newif\iftstu

% % Typically, I will set one of these ONCE at the top of the doc.
% \newcommand{\forTEACHER}{%
%     \tstutrue%
%     \dashundergapssetup{teacher-mode=true,}%
% }
% \newcommand{\forSTUDENT}{%
%     \tstufalse%
%     \dashundergapssetup{teacher-mode=false,}%
% }

% % These will conditionally generate output (the argument) based on 
% % whether this document is "for" HONORS or ONLEVEL according to the previous commands.
% %
% % #1 the content 
% % #2 text color
% \NewDocumentCommand{\whenTEACHER}{ O{red} m }{\iftstu{\ECFAugie\color{#1}#2}\else{}\fi}
% \newcommand{\whenSTUDENT}[1]{\iftstu{}\else{#1}\fi}



% ---------------------------------------------------------------------------
% command for creating a 1- or 2-question warmup problem
%
% #1 date
% #2 optional general directions
%
% ---------------------------------------------------------------------------
\NewDocumentEnvironment{myWarmupProblems}{ 
        O{
            \begin{itemize}
                \item Show {\itshape all} your work.
                \item Put a \fbox{box} around your answer.
            \end{itemize}
            \vspace{0.5em}
        } 
    }
{
    \newpage
    {
        \sffamily
        \begin{center}
            {
                \huge\bfseries%
                \whenHONORS{Honors~}Algebra 2 Warmup
            }%
            \\[0.5em]
            {\Large\itshape\printdate}
        \end{center}
        \noindent{\Large#1}
    }
    \large
}{
}


% ---------------------------------------------------------------------------
% A version of \sqrt 
% ---------------------------------------------------------------------------
%
% #1 index of the root
% #2 uproot amount
% #3 radicand
%
\NewDocumentCommand{\myRoot}{ o O{2} m  }{%
    \IfNoValueTF{#1}{\sqrt{#3\,}}{\sqrt[\uproot{#2}#1]{#3\,}}
}


% ---------------------------------------------------------------------------
% changes to part/chapter/section
% I am piggybacking Units and Lessons on Latex Parts and Chapters, respectively. 
% ---------------------------------------------------------------------------

% Here are some of the macros that capture "my settings".
\newcommand{\myPartName}{Unit}      % what I'll replace "Unit" with.
\newcommand{\myChapterName}{Lesson} % what I'll replace "Chapter" with.
\newcommand{\myLessonSuffix}{}      % a suffix after lesson (chapter) numbers (eg, 'a' in Lesson 5.2a)

% Redefine the Latex part and chapter names to use "my settings".
\renewcommand{\partname}{\myPartName}       % "Unit 2" instead of "Part II"
\renewcommand{\thepart}{\arabic{part}}
\renewcommand{\chaptername}{\myChapterName} % "Lesson 2" instead of "Chapter 2"

% A \chapter-based command to generate a lesson.
% This is a thin wrapper around \chapter that allows me
% to explicitly specify the lesson number. 
%
% #1 lesson name
% #2 optional lesson number (eg, the '2' in Lesson 5.2)
% #3 optional lesson suffix (eg, the 'a' in Lesson 5.2a)
%
\NewDocumentCommand{\myLesson}{ m O{1} O{} }
{
    \setcounter{chapter}{#2-1}
    \renewcommand{\myLessonSuffix}{#3}
    \chapter{#1}
}

% remove chapter numbers from section numbering
\makeatletter
\renewcommand\thesection{\@arabic\c@section}
\makeatother

% simulate a part without typesetting one
% 
% #1 - the part (unit) number
% #2 - the part (unit) title
\newcommand{\dummypart}[2]{%
    \setcounter{part}{#1}
    \partmark{#2}
}

% ---------------------------------------------------------------------------
% These are "annotations", the foundation of the blocked-in-text that I use.
%
% I use the term "annotations" to capture the common
% infrastructure I use to define Objectives, Vobabulary & Concepts.
% ---------------------------------------------------------------------------

% font and styling commands for
% Objectives, Voculary, Key Concepts, etc...
\newcommand{\myAnnotationStyling}{\bfseries\normalsize}

% #1 : name of the kind of annotation (Objectives, ...)
% #2 : title text to go with the annotation
% #3 : extra tcolorbox options
%
\NewDocumentEnvironment{myAnnotate}{ m m O{}}{
    \begin{tcolorbox}[
        colbacktitle=blue!10!white,
        colback=white,
        coltitle=black,
        fonttitle={\myAnnotationStyling},
        title={#1:~},
        after title={\normalfont\itshape#2},
        #3,
        ]
}{
    \end{tcolorbox}
}

% #1 : name of the kind of annotation 
% #2 : title 
%
\NewDocumentEnvironment{myTabularAnnotate}{ m m }{
    \begin{myAnnotate}{#1}{#2}
    \begin{tabular}{r|l}
}{
    \end{tabular}
    \end{myAnnotate}
}
% #1 - column 1 text
% #2 - column 2 text
%
\NewDocumentCommand{\myRow}{mm}{{\bfseries\itshape \textcolor{blue}{#1}}&#2\\[1ex]}

% #1 : name of the kind of annotation 
% #2 : title 
% #3 : text before the list starts
%
\NewDocumentEnvironment{myListAnnotate}{ m m o }{
    \begin{myAnnotate}{#1}{#2}
    \IfValueT{#3}{#3}
    \begin{enumerate}[itemsep=0pt,fullwidth,]
}{
    \end{enumerate}
    \end{myAnnotate}
}
% #1 - column 1 text
% #2 - column 2 text
%
\NewDocumentCommand{\myItem}{mm}{\item{\bfseries\itshape \textcolor{blue}{#1}} #2}



% Read an environment variable.
% 1: name of the variable 
% 2: optional command to "insert" it into so that the command 
%    evaluates to the env var's value
%
% If option 2 is not present, the \getenv command will just 
% echo the env var value into the output file.
%
% See: https://tex.stackexchange.com/questions/62010/can-i-access-system-environment-variables-from-latex-for-instance-home
%
\newcommand{\getenv}[2][]{%
  \CatchFileEdef{\temp}{"|kpsewhich --var-value #2"}{\endlinechar=-1}%
  \if\relax\detokenize{#1}\relax\temp\else\let#1\temp\fi}


% a macro for typesetting buttons (eg, TI-84)
% 1: text 
%
\NewDocumentCommand{\myLog}{ O{10} }{
    \log_{{}_{#1}}
}