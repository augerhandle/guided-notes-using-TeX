\section{Polynomial Long Division}

\begin{myConcept}{~To divide polynomials by long division \dots}
    \begin{center}
        \renewcommand{\arraystretch}{1.25}
        \begin{tabular}{m{0.5\textwidth}|p{0.45\textwidth}}
            & 
            {\Large $ \frac{2x^2  -1}{x-3} $}
            \rule{0in}{1\baselineskip}
            \\[0.5\onelineskip] \hline
            {\bfseries\itshape Step 1.}
            Write the {\scshape Dividend} and {\scshape Divisor}
            as a long division problem.
            Write both polynomials in \myEmph{descending order}.
            Fill in \myEmph{zeros} for missing terms. 
            & 
            \rule{0in}{2\baselineskip}
            \\ \hline
            {\bfseries\itshape Step 2.}
            Divide the \myEmph{leading term} in the {\scshape Dividend} 
            by the \myEmph{leading term} in the {\scshape Divisor}.
            Write this above the {\scshape Dividend}.
            & 
            \rule{0in}{4\baselineskip}
            \\ \hline
            {\bfseries\itshape Step 3.}
            Multiply the {\scshape Divisor} by this expression (distribute).
            Write the new terms \myEmph{lined up} under the {\scshape Dividend}.
            & 
            \rule{0in}{4\baselineskip}
            \\ \hline
            {\bfseries\itshape Step 4.}
            \myEmph{Subtract} and then \myEmph{bring down} the 
            rest of the previous line to form a new polynomial.
            & 
            \rule{0in}{7\baselineskip}
            \\ \hline
            {\bfseries\itshape Step 5.}
            \myEmph{Repeat} with the new polynomial.
            & 
            \rule{0in}{9\baselineskip}
            \\ \hline
            {\bfseries\itshape Step 6.}
            Stop when there are no terms left in the polynomial 
            that you haven't already worked with.
            The {\scshape Remainder} is the polynomial (or number)
            that you are left with.
            The {\scshape Quotient} is the polynomial above the {\scshape Dividend}.
            Write the {\scshape Remainder} as a fraction over the {\scshape Divisor}.
            & 
            \\ 
        \end{tabular}
    \end{center}
\end{myConcept}

