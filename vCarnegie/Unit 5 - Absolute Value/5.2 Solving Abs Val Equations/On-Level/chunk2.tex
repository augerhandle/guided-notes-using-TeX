\section{How to solve simple absolute value equations}

\begin{myConceptSteps}{To solve simple absolute value equations\dots}
    \myStep{isolate}{Get the absolute value \gap{by} \gap{itself}.}
    \myStep{check}{%
        If the expression equals a \gap{negative}, 
        then \gap{no} \gap{solution}.
        \whenTEACHER{
            \tiny
            \begin{itemize}[nosep]
                \item Because distance is never negative.
            \end{itemize}
        }
    }
    \myStep{split}{%
        Write equations for the \gap{positive} and \gap{negative} cases.
        \begin{itemize}[nosep]
            \item {\itshape No more absolute value!}
        \end{itemize}
    }
    \myStep{solve}{Solve each of the \gap{simpler} equations.}
\end{myConceptSteps}
\whenTEACHER{\small
    To teach the isolate step in examples,
    I suggest using an analogy with linear 
    equations. For example, for $5|3z-4| = 15$, ask,
    ``How would you get x by itself in $5x = 15$?'' This really helps.
    \vspace{-2\onelineskip}
}\par