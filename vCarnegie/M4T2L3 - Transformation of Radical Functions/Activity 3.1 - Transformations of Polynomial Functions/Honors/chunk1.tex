\begin{tcolorbox}[center,colback=white,width=6in,]
    \centering
    {
    \large
    \vspace{-1\baselineskip}
    \begin{align*}
    y &= \bm{A} f\left( \text{\rule{0pt}{0.9em}}\bm{B}(x-\bm{C})\right) + \bm{D} \\ 
    y &= \bm{A} \myRoot{ \bm{B}(x-\bm{C})} + \bm{D}\\ 
    y &= \bm{A} \myRoot[3]{ \bm{B}(x-\bm{C})} + \bm{D}
    \end{align*}
    }
    \tcblower 
    % \small
    \begin{itemize}[fullwidth]
        \item {$\bm{C}$ is the \gap{opposite} of what you see}
        \item $\bm{A}$ and $\bm{D}$ are ``outside'': they transform \gap{$y$} (\gap{vertical} changes)
        \item $\bm{B}$ and $\bm{C}$ are ``inside'': they transform \gap{$x$} (\gap{horizontal} changes)
    \end{itemize}
\end{tcolorbox}

\myProblems[Find the values of $A$, $B$, $C$, and $D$ for these functions.]
{
    $g(x) = -2 f\left(\rule{0pt}{0.925em} \frac{1}{2}(x-6)\right) + 7$
    \myAnswer{\tiny\hfill A=-2, B=$\frac{1}{2}$, C=6, D=7}
}
{
    $g(x) =  \myRoot{(x+2)} - 1$
    \myAnswer{\tiny\hfill A=1, B=1, C=-2, D=-1}
}
{0.45in}

\myProblems
{
    $g(x) =  -\myRoot[3]{(x-6)} $
    \myAnswer{\tiny\hfill A=-1, B=1, C=6, D=0}
}
{
    $g(x) =  \frac{1}{3}\myRoot{2x} + 4 $
    \myAnswer{\tiny\hfill A=$\frac{1}{3}$, B=2, C=0, D=4}
}
{0.45in}
