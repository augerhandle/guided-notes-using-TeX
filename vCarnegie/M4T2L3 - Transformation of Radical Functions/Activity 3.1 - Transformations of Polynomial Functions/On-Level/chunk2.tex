
\begin{center}
    \renewcommand{\arraystretch}{2}
    \begin{tabular}{|>{\raggedright}p{1in}|>{\raggedright\arraybackslash}p{1.5in}|l|>{\raggedright\arraybackslash}p{0.9in}|}
        \hline
        \multicolumn{4}{|c|}{
            \begin{tabular}{c}
                \myEmph{\Large Transformations} \\
                % {\itshape\normalsize What happens when we transform $f(x)$?} \\
            \end{tabular}
            } \\
        \hline
        {\itshape When the transformed function is\dots} 
            & {\itshape and the value is\dots} 
            & {\itshape then the transformation \dots}  
            & {\itshape example}\\
        \hline\hline
        \multirow{4}*{$y = \bm{A} f(x)$} 
            & $\bm{A} = 1$ \myAnswer{\tiny\hfill A is missing}
            & {does \gap{nothing}}
            &\multirow{4}*{
                \begin{tabular}{l}
                    \myAnswer{\tiny $y=5f(x)$} \\ 
                    \myAnswer{\tiny $A=5$} \\
                    \myAnswer{\tiny v. stretch by 5}\\
                \end{tabular}
                }\\ 
            \cline{2-3}
            & $|\bm{A}| > 1$       
            & vert. \gap{stretches} by $|\bm{A}|$ 
            &\\
            \cline{2-3}
            & $ 0 < |\bm{A}| < 1$ 
            & vert. \gap{compresses} by $|\bm{A}|$ 
            &\\
            \cline{2-3}
            & $ \bm{A} < 0$ \myAnswer{\tiny\hfill A is NEG}
            & vert. \gap{reflects} across \gap{$x$-axis}
            &\\
        \hline\hline
        %
        \multirow{4}*{$y = f( \bm{B} x)$} 
            & $\bm{B} = 1$ \myAnswer{\tiny\hfill B is missing}
            & {does \gap{nothing}}
            &\multirow{4}*{
                \begin{tabular}{l}
                    \myAnswer{\tiny $y=\myRoot{\frac{1}{3}x}$} \\ 
                    \myAnswer{\tiny $B=\frac{1}{3}$} \\
                    \myAnswer{\tiny h. stretch by 3}\\
                \end{tabular}
                }\\ 
            \cline{2-3}
            & $|\bm{B}| > 1$       
            & horiz. \gap{stretches} by $\frac{1}{|\bm{B}|}$  
            & \\
            \cline{2-3}
            & $ 0 < |\bm{B}| < 1$ 
            & horiz. \gap{compresses} by $\frac{1}{|\bm{B}|}$ 
            & \\
            \cline{2-3}
            & $ \bm{B} < 0$    \myAnswer{\tiny\hfill B is NEG}   
            & \gap{reflects} across \gap{$y$-axis} 
            & \\
        \hline\hline
        %
        \multirow{4}*{$y = f(x - \bm{C})$}
            & $\bm{C} = 0$ \myAnswer{\tiny\hfill C is missing}
            & {does \gap{nothing}}
            &\multirow{3}*{
                \begin{tabular}{l}
                    \myAnswer{\tiny $y = \myRoot[3]{x-6}$}\\
                    \myAnswer{\tiny C=6}\\
                    \myAnswer{\tiny right by 6}\\
                \end{tabular}
                }\\
            \cline{2-3}
            & $\bm{C}>0$ \myAnswer{\tiny\hfill C is POS}
            &  horiz. \gap{shifts} \gap{right} by $\bm{C}$ 
            & \\
            \cline{2-3}
            & $\bm{C}<0$ \myAnswer{\tiny\hfill C is NEG}
            &  horiz. \gap{shifts} \gap{left} by $|\bm{C}|$ 
            & \\
            \cline{2-3}
        \hline\hline
        %
        \multirow{3}*{$y = f(x) + \bm{D}$}
            & $\bm{D} = 0$ \myAnswer{\tiny\hfill D is missing}
            & {does \gap{nothing}}
            &\multirow{3}*{
                \begin{tabular}{l}
                    \myAnswer{\tiny $y=f(x) - 4$}\\
                    \myAnswer{\tiny D=-4}\\
                    \myAnswer{\tiny down by 4}\\
                \end{tabular}
                }\\ 
            \cline{2-3}
            & $\bm{D}>0$ \myAnswer{\tiny\hfill D is POS}
            &  vert. \gap{shifts} \gap{up} by $\bm{D}$ 
            & \\
            \cline{2-3}
            & $\bm{D}<0$ \myAnswer{\tiny\hfill D is NEG}
            &  vert. \gap{shifts} \gap{down} by $|\bm{D}|$ 
            & \\
            \cline{2-3} 
        \hline
    \end{tabular}
\end{center}


\begin{tcolorbox}[center,colback=white,width=6in,]
    Remember that
    $\bm{C}$ is the \gap{opposite} of what you see in the function.
\end{tcolorbox}


