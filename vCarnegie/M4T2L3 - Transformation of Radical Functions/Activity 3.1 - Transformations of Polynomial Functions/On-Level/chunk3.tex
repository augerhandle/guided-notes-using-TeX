\myProblemsWithContent[ Given these values of $A$, $B$, $C$, $D$, write equation of the transformed version of the \myEmph{square root} function.]
{
    \vspace{1\baselineskip}
    \small $A=1$,\,\,\,\, $B=1$,\,\,\,\, $C=-5$,\,\,\,\, $D=-2$
    \tcblower
    $y = $ \gap{$\myRoot{x+5}-2$}
}
{
    \vspace{1\baselineskip}
    \small $A=\frac{1}{3}$,\,\,\,\, $B=1$,\,\,\,\, $C=3$,\,\,\,\, $D=12$
    \tcblower
    $y = $ \gap{$\frac{1}{3}\myRoot{x-3}+12$}
    \vspace{0.2in}
}

\myProblemsWithContent[Given these transformed functions, find the values of $A$, $B$, $C$, $D$.]
{
    $y = \myRoot[3]{x-4} + 6$
    \tcblower
    $A=$\gap{1}, $B=$\gap{1}, $C=$\gap{4}, $D=$\gap{6}
}
{
    $y = 2 f(x)$
    \tcblower
    $A=$\gap{2}, $B=$\gap{1}, $C=$\gap{0}, $D=$\gap{0}
}

\myProblemsWithContent[Given these transformed functions, describe the transformations.]
{
    $y = 5\myRoot{x}-1 $
    \tcblower
    {
        \noindent
        \myAnswer{%
            \tiny\noindent%
            vertical stretch by 5 {b/c A=5}\\
            vert. shift down by 1 {b/c D=-1 (opposite of what you see!)}
        }
    }
    \vspace{1in}
}
{
    $y = \frac{1}{2} f(x+1)$
    \tcblower
    {
        \noindent
        \myAnswer{%
            \tiny\noindent%
            vertical compress by $\frac{1}{2}$ {b/c A=1/2}\\
            horiz. shift left by 1 {b/c C=-1 (opposite of what you see!)}
        }
    }
    \vspace{1in}
}
