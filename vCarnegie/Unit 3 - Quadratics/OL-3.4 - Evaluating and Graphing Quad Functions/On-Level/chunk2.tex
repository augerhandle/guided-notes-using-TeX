\section{Graphing}


\begin{myConceptSteps}{To sketch the graph of a quadratic function\dots}
    \myStep{Enter}{Type the function into {\scshape Desmos} or use the {\scshape TI-84} \myKey{Y=} key.}
    \myStep{Copy}{%
        Look at the graph. Copy it onto paper and sketch the following.
        \begin{itemize}
            \item Neatly \gap{sketch} the \gap{parabola}.
            \item Put \gap{big} \gap{dot} at each \gap{$x$-intercept}.
            \item Put a \gap{big} \gap{dot} at the \gap{vertex}.
            \item Draw the \gap{axis} \gap{of} \gap{symmetry} (AOS) as a \gap{vertical} line through the \gap{vertex}.
        \end{itemize}
    }
\end{myConceptSteps}

\whenTEACHER{
    \footnotesize
    Teacher note: 
    The point of sketching 
    \begin{itemize}[nosep]
        \item $x$-intercepts 
        \item vertex 
        \item axis of symmetry
    \end{itemize}
    Is just to familiarize the students with the vocabulary. 
    We will spend more time in a later lesson 
    on on the "details" (equations, coordinates).
}

\myProblemsWithContent[Sketch the graphs of these quadratic functions.]
{
    $f(x) = x^2 - 3x - 6$\\
    \begin{myTikzpictureGrid}{0.4} {10}{10}
        % \tkzText(13,8){$x - y = 0$}
        \whenTEACHER{
            \tkzFct[ solid, red, ultra thick, samples=100, domain=-10:10,]
            { x**2 - 3*x - 6 }
            \draw [loosely dashed,line width=1mm,red] (1.5,-10) -- (1.5,10);
            \draw [fill,red] (1.5,-8.25) circle (4mm);
            \draw [fill,red] (-1.37,0) circle (4mm);
            \draw [fill,red] (4.37,0) circle (4mm);
        }
    \end{myTikzpictureGrid}
}
{
    $f(x) = -x^2 - 10x - 17 $\\
    \begin{myTikzpictureGrid}{0.4} {10}{10}
        % \tkzText(13,8){$x - y = 0$}
        \whenTEACHER{
            \tkzFct[ solid, red, ultra thick, samples=100, domain=-10:10,]
            { -x**2 - 10*x - 17 }
            \draw [loosely dashed,line width=1mm,red] (-5,-10) -- (-5,10);
            \draw [fill,red] (-5,8) circle (4mm);
            \draw [fill,red] (-2.17,0) circle (4mm);
            \draw [fill,red] (-7.83,0) circle (4mm);
        }
    \end{myTikzpictureGrid}
}
