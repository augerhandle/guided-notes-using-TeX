\newpage
\myWideProblemWithContent[%
    Analyze the graphs of $f(x)$ (the original function) 
    and $g(x)$ (the transformed function).
    For each reference point on $f$, 
    the corresponding points on $g$ are shown.
    \begin{itemize}[nosep]
        \item Describe all the transformations.
        \item Write the equation for $g(x)$ in terms of $f(x)$.
    \end{itemize}
]
{
    \begin{center}
        \begin{myTikzpictureGrid}{0.275} {9}{9} [solid]
            \tkzFct[ solid, thick, samples=100, domain=-4:4,]{x**3}
            \myDot{(0,0)} \node at (1,-0.5) {\tiny $(0,0)$};
            \myDot{(1,1)} \node at (2.3,1.25) {\tiny $(1,1)$};
            \myDot{(2,8)} \node at (3.3,7.75) {\tiny $(2,8)$};
            \node[fill=white] at (2,-7.9) {\normalsize $y=f(x)$};
        \end{myTikzpictureGrid}
        \hfill 
        \raisebox{2in}{transformations:}
        \hfill
        \begin{myTikzpictureGrid}{0.275} {9}{9} [solid]
            \tkzFct[ solid, thick, samples=100, domain=-4:4,]{(x+2)**3-3}
            \myDot{(-2,-3)} \node at (0,-3.75) {\tiny $(-2,-3)$};
            \myDot{(-1,-2)} \node at (1.5,-1.75) {\tiny $(-1,-2)$};
            \myDot{(0,5)} \node at (1.5,4.75) {\tiny $(0,5)$};
            \node[fill=white] at (0,-7.9) {\normalsize $y=g(x)$};
        \end{myTikzpictureGrid}
    \end{center}
    \hfill $g(x) = $ \gap{$f(x+2) - 3$} \hfill{\,}
}
\vfill
\myWideProblemWithContent
{
    \begin{center}
        \begin{myTikzpictureGrid}{0.275} {9}{9} [solid]
            \tkzFct[ solid, thick, samples=100, domain=-4:4,]{x**3}
            \myDot{(0,0)} \node at (1,-0.5) {\tiny $(0,0)$};
            \myDot{(1,1)} \node at (2.3,1.25) {\tiny $(1,1)$};
            \myDot{(2,8)} \node at (3.3,7.75) {\tiny $(2,8)$};
            \node[fill=white] at (2,-7.9) {\normalsize $y=f(x)$};
        \end{myTikzpictureGrid}
        \hfill 
        \raisebox{2in}{transformations:}
        \hfill
        \begin{myTikzpictureGrid}{0.275} {9}{9} [solid]
            \tkzFct[ solid, thick, samples=100, domain=-4:6,]{-(x-3)**3+ 4}
            \myDot{(3,4)} \node at (4,5) {\tiny $(3,4)$};
            \myDot{(4,3)} \node at (6,3) {\tiny $(4,3)$};
            \myDot{(5,-4)} \node at (7,-3.5) {\tiny $(5,-4)$};
            \node[fill=white] at (-2,-7.9) {\normalsize $y=g(x)$};
        \end{myTikzpictureGrid}
    \end{center}
    \hfill $g(x) = $ \gap{$-f(x-3) +4$}\hfill{\,}
}
\vfill
\myWideProblemWithContent
{
    \begin{center}
        \begin{myTikzpictureGrid}{0.275} {9}{9} [solid]
            \tkzFct[ solid, thick, samples=100, domain=-4:4,]{x**3}
            \myDot{(0,0)} \node at (1,-0.5) {\tiny $(0,0)$};
            \myDot{(1,1)} \node at (2.3,1.25) {\tiny $(1,1)$};
            \myDot{(2,8)} \node at (3.3,7.75) {\tiny $(2,8)$};
            \node[fill=white] at (2,-7.9) {\normalsize $y=f(x)$};
        \end{myTikzpictureGrid}
        \hfill 
        \raisebox{2in}{transformations:}
        \hfill
        \begin{myTikzpictureGrid}{0.275} {9}{9} [solid]
            \tkzFct[ solid, thick, samples=100, domain=-4:9,]{0.25*(x-5)**3}
            \myDot{(5,0)} \node at (5.5,-1) {\tiny $(5,0)$};
            \myDot{(6,0.25)} \node at (8.5,0.75) {\tiny $(6,0.25)$};
            \myDot{(7,2)} \node at (8.5,2.5) {\tiny $(7,2)$};
            \node[fill=white] at (2,-7.9) {\normalsize $y=f(x)$};
        \end{myTikzpictureGrid}
    \end{center}
    \hfill $g(x) = $ \gap{$0.25f(x-5)$}\hfill{\,}
}
