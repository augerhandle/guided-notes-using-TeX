\section{Function Notation}

\myCenteredBox{
    {\bfseries\itshape Function notation} tells you two things:
    \begin{enumerate}
        \item{the function's \gap{name} (often \gap{$f$})}
        \item{how to \gap{calculate} the \gap{output} if you know the \gap{input}}
    \end{enumerate}
}

\myProblemsWithContent[%
    What is the expression for the function that has the name specified?]
{
    {\small
        Function name: $\gamma$. \\
        Expression: \gap{$4x^3 -2x^2 + \pi$}
    }
    \begin{itemize}
        \item{$ \alpha(x) = \myRoot{5x - 6}$}
        \item{$ \beta(x) = 3x^2 + x - 8$}
        \item{$ \delta(x) = 10 - 3x$}
        \item{$ \gamma(x) = 4x^3 -2x^2 + \pi$}
    \end{itemize}
}
{
    {\small
        Function name: $Q$. \\
        Expression: \gap{$\myRoot[3]{x-2}$}
    }
    \begin{itemize}
        \item{$ f(x) = x^3 + 5x$}
        \item{$ g(x) = 5x^2$}
        \item{$ Q(x) = \myRoot[3]{x-2}$}
        \item{$ \text{\ttfamily foo}(x) = 9x - 1$}
    \end{itemize}
}


\myCenteredBox[width=5.75in, colback=white, sharp corners, boxrule=0.5pt, ]{
    \small
    There is a style of programming called {\itshape functional programming.}
    It uses fuctions for almost everything.
    Here's an example of the function notation used in {\scshape Scala}.
    \\[1\baselineskip]
    %
    {
        \sffamily
        def {\bfseries foo}( x : Double ) : Double = \{\\
            \phantom{xxxx}9*x - 1.0;\\
        \}    
    }
}
