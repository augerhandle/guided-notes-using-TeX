\section{Evaluating a Function From Its Expression}

\begin{myConceptSteps}{To \gap{evaluate} a function given its expression\dots} 
    \myStep{substitute}{replace the \gap{variable} with the input}
    \myStep{simplify}{simplify the expression (Remember \gap{PEMDAS}.)}
\end{myConceptSteps}

\subsection{Evaluating functions given a number as input}

\myProblemsWithContent[Evaluate these functions {\bfseries without} a calculator.]
    {
        function: $f(x) = -2x^3 -x^2 - x - 5$ \\
        input: $-5$
        \vspace{1in}\\
        $f(-5) =$ \gap{225} % -2*-125 -25 + 5 - 5 = 250-25 = 225
    }
    {
        function: $s(x) = x(3x^2 - 1)$ \\
        input: $-2$
        \vspace{1in}\\
        $s(-2) =$ \gap{14} % -2(3.4-1) = -2(12-1) = -2.11 = -22
    }



\myCenteredBox[width=6in]{
    When you do use a calculator to evaluate a function,
    {\bfseries\itshape always, always}
    use \gap{parentheses} when you enter the input value.
}


\myCenteredBox{
    {\bfseries\itshape Warning when you use calculators:}
    I can't see work on a calculator. 
    If you make a mistake, you'll get the problem {\itshape wrong}.
    So, {\bfseries\itshape do it twice when you use a calculator!}
}



\subsection{Evaluating functions given an expression as input}

Sometimes we evaluate functions by passing \gap{expressions} (not numbers) as inputs.


\myProblemsWithContent[Evaluate these functions with the given expressions as input.]
    {
        function: $f(x) = 5x - 3$ \\
        input: $z+1$
        \vspace{1in}\\
        $f(z+1) =$ \gap{$15z+2$} % 5(z+1) - 3 = 15z + 5 - 3 = 15z + 2
    }
    {
        function: $R(x) = x^2 - x$ \\
        input: $-2t$
        \vspace{1in}\\
        $s(2t) =$ \gap{$4t^2 + 2t$} % (-2t)^2 - (-2t) = 4t^2 + 2t
    }
