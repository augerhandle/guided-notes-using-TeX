\begin{myConceptSteps}{To solve linear equations\dots}[%
    \gap{Rewrite} the equation (multiple times?) to get the variable by itself.
    Keep the $=$ {\bfseries\itshape aligned} for each rewrite.
    (Draw a vertical \gap{line} through all the $=$ signs.)
    ]
    \myStep{like terms}{\gap{Combine} like terms on both sides of the equation.}
    \myStep{variables}{\gap{Move} variable terms to one side of the equation.}
    \myStep{numbers}{\gap{Move} numbers to the other side of the equation.}
    \myStep{like terms}{\gap{Combine} like terms again.}
    \myStep{divide}{Divide both sides by the variable coefficient to get the answer.}
    \myStep{check}{Plug in your answer and see if you get a \gap{true} equation.}
\end{myConceptSteps}

\myProblems[Solve these equations]
{
    \centering 
    $x + 8 = 10$
}
{
    \centering
    $3x = -27$
}
{1.5in}

\myProblems
{
    \centering 
    $2x + 5 = 25$
}
{
    \centering
    $75 = 3x - 15$
}
{2in}

\myProblems
{
    \centering 
    $3x + 5 = 25 + 2x$
}
{
    \centering
    $5x - 7 = 14 + 2x$
}
{2.25in}

\noindent
By the end of the period tomorrow,
you will remember how to solve linear equations that look like this:
\[
    x - 10 + 4x + 3 = 5x + 6 - 3x +8
\]