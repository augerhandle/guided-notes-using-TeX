\begin{myConceptSteps}{To solve linear equations\dots}
    \myStep{distribute}{If necessary, distribute and combine like terms.}
    \myStep{move}{Get variables and numbers on \gap{opposite} sides of the equation.}
    \myStep{balance}{%
        Keep the $=$ symbols {\bfseries\itshape aligned}. 
        (Draw a \gap{vertical} line.)
        }
    \myStep{divide}{Divide both sides by the variable coefficient to get the answer.}
    \myStep{check}{Plug in your solution and see if you get a \gap{true} equation.}
\end{myConceptSteps}

\myProblems[Solve these equations]
{
    \centering 
    $x + 8 = 10$
}
{
    \centering
    $3x = -27$
}
{1.5in}

\myProblems
{
    \centering 
    $2x + 5 = 25$
}
{
    \centering
    $75 = 3x - 15$
}
{2in}

\myProblems
{
    \centering 
    $3x + 5 = 25 + 2x$
}
{
    \centering
    $5x - 7 = 14 + 2x$
}
{2.25in}

% \noindent
% By the end of the period tomorrow,
% you will remember how to solve linear equations that look like this:
% \[
%     x - 10 + 4x + 3 = 5x + 6 - 3x +8
% \]