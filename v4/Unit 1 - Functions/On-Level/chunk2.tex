\section{Points on a Graph}

If the relation is given as a \gap{graph}, 
it has an \gap{infinite} number of points.
You can find the coordinates of those points.

\begin{myConceptSteps}{~to find the $y$-value of a point on a graph given the $x$-value\dots}
    \myStep{vertical line}{Draw a \gap{vertical line} at the $x$-value. }
    \myStep{intersect}{Draw a dot where the line crosses the graph.}
    \myStep{horizontal line}{Draw a \gap{horizontal line} through the dot.}
    \myStep{$y$-coordinate}{\gap{Read} the $y$-coordinate off the $y$-axis.}
    \myStep{write}{\gap{Write} both coordinates of the dot as an $(x,y)$ pair.}
\end{myConceptSteps}

\myProblemsWithContent[Find the point on the graph given its $x$-coordinate.]
{
    \begin{minipage}{0.49\textwidth}
        \centering
        $x$-coordinate: \texttt{-2}
    \end{minipage}
    %
    \hfil 
    %
    \begin{minipage}{0.49\textwidth}
        \centering
        \begin{myTikzpictureGrid}{0.45} [-5]{2}[-3]{5}
            \tkzFct[ solid, ultra thick, samples=100, domain=-5:2,]{x**3 + 3*x**2}
        \end{myTikzpictureGrid}
    \end{minipage}
}
{
    \begin{minipage}{0.49\textwidth}
        \centering
        $x$-coordinate: \texttt{4}\\
    \end{minipage}
    %
    \hfil 
    %
    \begin{minipage}{0.49\textwidth}
        \centering
        \begin{myTikzpictureGrid}{0.4} [-2]{6}[-4]{4}
            \tkzFct[ solid, ultra thick, samples=100, domain=-3:7,]{-(x-4)**3 - 3*(x-4)**2 + 1}
        \end{myTikzpictureGrid}
    \end{minipage}
}
