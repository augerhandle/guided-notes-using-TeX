\section{Absolute Values of Numbers}

A number and its negative are the \gap{same} \gap{distance} from the origin.
\begin{tcbraster}[
    raster columns=2,
    raster before skip = 1em, raster after skip = 1em,
    ]
    \begin{tcolorbox}[colback=white,]
        \vspace{2\baselineskip}
        \begin{tikzpicture}[scale=0.75]
            \myDrawNumberlineCircle{0}{white}
            \myDrawNumberlineCircle{2}{black}
            \myDrawNumberlineCircle{-2}{black}
            \myDrawNumberline{5}
        \end{tikzpicture}
        \begin{align*}
            |2| &= 2\\
            |-2| &= 2
        \end{align*}
    \end{tcolorbox}
    \begin{tcolorbox}[colback=white,]
        \vspace{2\baselineskip}
        \begin{tikzpicture}[scale=0.75]
            \myDrawNumberlineCircle{0}{white}
            \myDrawNumberlineCircle{4}{black}
            \myDrawNumberlineCircle{-4}{black}
            \myDrawNumberline{5}
        \end{tikzpicture}
        \begin{align*}
            |4| &= 4\\
            |-4| &= 4
        \end{align*}
    \end{tcolorbox}
\end{tcbraster}

So\dots

\begin{myConcept}{To find the absolute value of a number\dots}
    \begin{itemize}
        \item \gap{Drop} the negative sign (if any) in front of the number.
        \item The result will \gap{never} be negative.
    \end{itemize}
    \begin{tcolorbox}[center,width=5.5in,]
        \centering
        This only works with numbers, \gap{not} with \gap{variables}.
    \end{tcolorbox}
\end{myConcept}

\vspace{1\baselineskip}

\myProblemsWithContent[Find the following absolute values.]
{
    $|12|=$   \gap{$12$}
}
{
    $|-8|=$   \gap{$8$}
}

\vspace{2\baselineskip}

\myProblemsWithContent
{
    $|\frac{28}{257}|=$   \gap{$\frac{28}{257}$}
}
{
    $|-\frac{8}{3}|=$   \gap{$\frac{8}{3}$}
}

\vspace{2\baselineskip}

\myProblemsWithContent
{
    $|3.14159|=$   \gap{$3.14159$}
}
{
    $|-\myRoot{3}|=$   \gap{$\myRoot{3}$}
}
