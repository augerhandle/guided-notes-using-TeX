\section{Introduction}

We will study transformations that convert the absolute value parent function into something else.

\begin{tcbraster}[
    raster equal height,
    raster left skip = 1in, raster right skip = 1in, 
    raster column skip = 0.5in,
    raster before skip = 1.5\baselineskip, raster after skip = 1.5\baselineskip,
    ]
    \begin{tcolorbox}[]
        \centering
        {\bfseries\itshape parent function}\\[0.5\baselineskip]
        \large
        $f(x) = |x|$
    \end{tcolorbox}
    \begin{tcolorbox}[]
        \centering
        {\bfseries\itshape transformed function}\\[0.5\baselineskip]
        \large
        $g(x) = \bm{a} | x-\bm{h} | + \bm{k}$
    \end{tcolorbox}
\end{tcbraster}

\begin{tcbraster}[]
\end{tcbraster}


The constants {$\bm{a}$}, {$\bm{h}$}, and {$\bm{k}$} result in 
different kinds of {\bfseries\itshape transformations} 
of the parent function:
\begin{itemize}[nosep]
    \item vertical shifts
    \item horizontal shifts
    \item vertical stretches and compressions
    \item reflections across the $x$-axis
\end{itemize}