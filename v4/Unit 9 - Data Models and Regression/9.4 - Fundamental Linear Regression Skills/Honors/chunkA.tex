\small
\begin{minipage}[t]{0.5\textwidth}
\begin{myConcept}{To find the line of best fit using {\scshape Desmos}\dots}
    \begin{enumerate}[fullwidth,nosep]
        \item Open {\scshape Desmos} on your phone or laptop.
        \item Enter the $(x,y)$ data.
            \begin{itemize}
                \item Click 
                    \raisebox{-0.5em}{\includegraphics[width=0.25in]{plus-button.png}}
                    in the upper left corner.
                \item Select \gap{``table''}.
                \item Enter the \gap{$x$} and \gap{$y$}.
            \end{itemize}
        \item Set up the window.
            \begin{itemize}
                \item Click 
                    \raisebox{-0.5em}{\includegraphics[width=0.3in]{wrench-tool.png}} 
                    in the upper right corner.
                \item \gap{\ttfamily X-Axis}: enter $x$ upper/lower bounds.
                \item \gap{\ttfamily Y-Axis}: enter $y$ upper/lower bounds.
            \end{itemize}
        \item You should see dots for your data.
        \item Do the linear regression.
            \begin{itemize}
                \item Tap the line \myEmph{below} the table.
                \item Enter \gap{$y_1 \thicksim m x_1 + b$}
                    %\quad (\fbox{$\thicksim$} on {\ttfamily ABC} keypad.) 
            \end{itemize}
        \item {\scshape Desmos} shows you the values of
            \begin{itemize}
                \item \gap{$m$} and \gap{$b$} %(slope and $y$-intercept of the line of best fit)
                \item The \gap{graph} of $y=mx+b$.
                \item \gap{$r$} (the correlation coefficient)
            \end{itemize}
    \end{enumerate}
\end{myConcept}
\end{minipage}
%
\begin{minipage}[t]{0.5\textwidth}
    \begin{myConcept}{To find the line of best fit using a {\scshape TI-84}\dots}
        \begin{enumerate}[fullwidth,nosep]
            \item Enable display of $r$.
                \begin{itemize}
                    \item Select 
                        \fbox{\ttfamily 2ND}
                        \fbox{\ttfamily 0}\,\,{\footnotesize\ttfamily (CATALOG)}
                    \item Select 
                        \fbox{$x^{-1}$}\,\,{\footnotesize (jump down the the Ds)}
                    \item Select 
                        \fbox{\ttfamily DiagnosticOn}
                        \fbox{\ttfamily ENTER} 
                        \fbox{\ttfamily ENTER}
                \end{itemize}
            \item Enable scatterplots.
                \begin{itemize}
                    \item Select 
                        \fbox{\ttfamily 2ND}
                        \raisebox{0.2em}{\fbox{\ttfamily y=}}\,\,{\footnotesize\ttfamily (STAT PLOT)}
                    \item Select 
                        \fbox{\ttfamily Plot1}
                        \fbox{\ttfamily On}
                    \item Make sure that 
                        \begin{itemize} 
                            \item {\ttfamily Type} is the scatterplot.
                            \item $x$ and $y$ are in $L_1$ and $L_2$ 
                        \end{itemize}
                \end{itemize}
            \item Enter your data.
                \begin{itemize}
                    \item Select 
                        \fbox{\ttfamily stat} 
                        \fbox{\ttfamily EDIT}
                    \item Put $x, y$ data into $L_1, L_2$ columns.
                    \item Use cursor keys to switch columns.
                    \item To clear columns, move to the header\\ and select 
                        \fbox{\ttfamily CLEAR} 
                        \fbox{\ttfamily ENTER}
                \end{itemize}
            \item Set up the plot window.
                \begin{itemize}
                    \item Select 
                        \fbox{\ttfamily ZOOM}
                        \fbox{\ttfamily 9}\,\,{\footnotesize\ttfamily (ZoomStat)}
                \end{itemize}
            % \item Double check your data by looking at the scatterplot.
            %     \begin{itemize}
            %         \item Select 
            %             \fbox{\ttfamily GRAPH} 
            %         \item If you want to inspect the coordinates of each point, 
            %             select 
            %             \fbox{\ttfamily TRACE} and use the cursor to move from point to point.
            %     \end{itemize}
            \item Perform linear regression.
                \begin{itemize}
                    \item Select 
                        \fbox{\ttfamily STAT}
                        \fbox{\ttfamily CALC}
                        \fbox{\ttfamily 4}\,\,{(\small\ttfamily LinReg(ax+b))}
                    \item Select 
                        \fbox{\ttfamily Calculate} 
                        \fbox{\ttfamily ENTER}
                \end{itemize}
        \end{enumerate}
    \end{myConcept}
    \end{minipage}
    
    