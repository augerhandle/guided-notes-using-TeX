\section{Getting Started (Day 1)}

\myCenteredBox[width=5in,colback=white,sharp corners,]{
    Today you will 
    \begin{itemize}[nosep,label=\checkmark]
        \item break into teams,
        \item build catapults,
        \item test them with \mymm{}s, and 
        \item practice taking measurements.
    \end{itemize}
}


\subsection{Groups}

You will work in groups of 4 people. 
Everyone must have a ``role''.
Record that information here.

\myCenteredBox[colback=\myFillinColor]{
    \centering
    \renewcommand{\arraystretch}{1.5}
    \begin{tabular}{l|p{2.75in}|p{3in}}
        \toprule
        {\itshape role} & {\itshape what?} & {\itshape who?} \\
        \toprule
        {\bfseries\itshape launcher} & hold and launch the catapult. & \\
        \midrule
        {\bfseries\itshape timer}    & measure \mymm~flight times&\\
        \midrule
        {\bfseries\itshape measurer} & measure how far the \mymm{}s go&\\
        \midrule
        {\bfseries\itshape recorder} & write down all measurements&\\
        \bottomrule
    \end{tabular}
}

\subsection{Catapult Construction}

Each team will build a catapult using big popsicle sticks and rubber bands.
Check them off the steps as you go.
Ask for help if you need it.
\myCenteredBox[colback=\myFillinColor]{
    \begin{itemize}[label={\Large$\Box$}]
        \item Wind a rubber band several times around one end of a stick 
            so that the rubber band stays in place.
            Your \mymm{}s will rest against this rubber band when you launch.
        \item Put a second stick next to the first one. Wrap a rubber band around the other end of them
            to hold them tightly together. 
        \item Insert a pencil between the two sticks. 
            You might want to use more rubber bands so that the pencil 
            does not easily move back and forth.
            The gap created by the pencil is what makes the catapult ``spring''.
    \end{itemize}
}

Give your catapult a {name}.
\myCenteredBox[colback=\myFillinColor]{
    \centering
    {\itshape name:} \fbox{\phantom{\LARGE Xxxxxx Xxxxxx-Xxxxxx}}
}

\subsection{Dry Runs}

Now try out your catapult. 
Find a clear space, and try launching an \mymm. 
This will give you a feel for how it works and how much space you need.

Here is the general idea.
\begin{itemize}[nosep]
    \item You will {\scshape Launch} your \mymm{}s from your catapult.
    \item The candy will fly across the room following a parabolic trajectory.
    \item The candy will {\scshape Land} at a point ``downrange''.
    \item You will measure the downrange distance.
    \item You will also measure the ``flight time''.
\end{itemize}

\begin{center}
\includegraphics[width=3in]{../launch-and-landing.jpg}
\end{center}

Practice timing and measuring. 
Launch {\bfseries\itshape at least 3} \mymm{}s. 
\begin{itemize}[nosep]
    \item The {\bfseries\itshape timer} person should use a phone to measure 
        the {\bfseries\itshape flight time} (in seconds).
    \item The {\bfseries\itshape measurer} should use a yardstick to measure 
        the {\bfseries\itshape downrange distance} (in centimeters).
    \item The {\bfseries\itshape recorder} should write down everything.
    \item Everyone should eventually fill in their own packets.
\end{itemize}

\myCenteredBox[colback=\myFillinColor]{
    \centering
    \renewcommand{\arraystretch}{1.5}
    \begin{tabular}{c|p{2in}|p{2in}}
        \toprule
        {\itshape dry run \#} & {\itshape fight time (sec)} & {\itshape downrange distance (cm)} \\
        \toprule
        1 & & \\
        \midrule
        2 & & \\
        \midrule
        3 & & \\
        \bottomrule
    \end{tabular}
}

Are your times and distances all approximately the same? 
\myCenteredBox[colback=\myFillinColor]{
    \centering 
    \qquad
    {\scshape Yes}
    \qquad
    {\scshape No}
    \qquad
    (circle one)
}

If not, write a short explanation of what was different and 
what you think explains the difference.%
\footnote{%
    For example:%
    ``The second dry run was messed up since we didn't 
    start timing right when the catapault launched.''
}

\myCenteredBox[colback=\myFillinColor]{
    \vspace{1em}
    \underline{\hspace{\textwidth}}\\[0.5\baselineskip]
    \underline{\hspace{\textwidth}}\\[0.5\baselineskip]
    \underline{\hspace{\textwidth}}\\[0.5\baselineskip]
    \underline{\hspace{\textwidth}}\\[0.5\baselineskip]
    \underline{\hspace{\textwidth}}\\[0.5\baselineskip]
    \underline{\hspace{\textwidth}}\\
}
