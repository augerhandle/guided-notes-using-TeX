% What we are doing today is \gap{completing} \gap{the} \gap{square}.
% Is is {\bfseries\itshape very similar} to what we did in {\bfseries Lesson 4.2a}, 
% with one important difference:
% \begin{itemize}[topsep=-0.5\baselineskip]
%     \item This time, there is a coefficient \gap{in} \gap{front} of the $x^2$.
% \end{itemize}

\vspace{0.75\baselineskip}

% \begin{minipage}{0.65\textwidth}
\begin{myConceptSteps}{To rewrite a quadratic function into vertex form by completing the square\dots}
    \myStep{pseudo-standard form}{Write the function as \gap{$x^2 + bx + c$}.}
    \myStep{middle term}{%
        \begin{itemize}
            \item Write down \gap{$\frac{1}{2}b$} (might be negative).
            \item Then write down that value \gap{squared}.
        \end{itemize}
        }
    \myStep{add/subtract}{%
        \gap{Add} and \gap{subtract} the result
        \gap{after} \gap{bx}.
    }
    \myStep{group}{%
        \begin{itemize}
            \item Group the first \gap{three} terms in \gap{parentheses}.
            \item Then rewrite that group as \gap{$(x+\Box)^2$}, 
                where $\Box$ is \gap{$\frac{1}{2}b$}
        \end{itemize}
        }
    \myStep{simplify}{Combine the last two terms into a single number.}
\end{myConceptSteps}
% \end{minipage}

\par