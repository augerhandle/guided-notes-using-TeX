\section*{Part 2 (the effects of $\bm{h}$)}

{
\small
For each of the following functions,
\begin{enumerate}[nosep]
    \item Sketch the {\bfseries\itshape graph} of the new function. (Draw it on top of the parent function.)
    \item Find the following attributes of the transformed function.
    \begin{itemize}[nosep]
        \item the $(x,y)$ coordinates of the {\bfseries\itshape vertex}
        \item the equation of the {\bfseries\itshape axis of symmetry} (AOS)
        \item the {\bfseries\itshape domain} and {\bfseries\itshape range}
    \end{itemize}
    \item Circle one of: {\bfseries\itshape max} or {\bfseries\itshape min}.
    \item Circle one of: {\bfseries\itshape opens up} or {\bfseries\itshape opens down}.
    \item Describe {\bfseries\itshape what changed} in going from the parent function to the transformed function.
        (Hint: What happens to the location of the vertex?)
\end{enumerate}
}
\vspace{\baselineskip}

\begin{tcbposter}[
        poster = {
            rows=7,
            columns=2,
            height=10cm,
            spacing=1mm,
            % showframe,
            },
        boxes = {colbacktitle=black!15,coltitle=black,colback=white,sharp corners}
    ]
\posterbox[]{name=A,column=1,row=1,span=1}{
    \Large
    $y = (x-5)^2$
    \hfill 
    \large
        ($\bm{h}=5$)
    } 
\posterbox[adjusted title=What changed?] {name=Z,column=1,span=2,row=6,rowspan=2}{}
\posterbox[adjusted title=graph] {name=B,row=2,between=A and Z}{
    \centering
    \begin{myTikzpictureGrid}{0.2} {10}{10}
        % \tkzText(13,8){$x - y = 0$}
        \tkzFct[ solid, ultra thick, samples=100, domain=-10:10,]{x**2}
        % \tkzText(-13,7.25){$x + 2y = 6$}
    \end{myTikzpictureGrid}
} 
\posterbox[adjusted title=attributes] {row=2,column=2,between=A and Z}{
    vertex: \gap{$(5,0)$} 

    AOS: \gap{$x=5$}

    domain: \gap{all real numbers}
    
    range: \gap{$y \ge 0$}\\[1\baselineskip]
    circle one: MAX or MIN \\[1\baselineskip]
    circle one: opens UP or DOWN
}
\end{tcbposter}




\begin{tcbposter}[
    poster = {
        rows=7,
        columns=2,
        height=10cm,
        spacing=1mm,
        % showframe,
        },
    boxes = {colbacktitle=black!15,coltitle=black,colback=white,sharp corners}
]
\posterbox[]{name=A,column=1,row=1,span=1}{
    \Large
    $y = (x+3)^2$
    \hfill 
    \large
        ($\bm{h}=-3$)
} 
\posterbox[adjusted title=What changed?] {name=Z,column=1,span=2,row=6,rowspan=2}{}
\posterbox[adjusted title=graph] {name=B,row=2,between=A and Z}{
\centering
\begin{myTikzpictureGrid}{0.2} {10}{10}
    % \tkzText(13,8){$x - y = 0$}
    \tkzFct[ solid, ultra thick, samples=100, domain=-10:10,]{x**2}
    % \tkzText(-13,7.25){$x + 2y = 6$}
\end{myTikzpictureGrid}
} 
\posterbox[adjusted title=attributes] {row=2,column=2,between=A and Z}{
vertex: \gap{$(-3,0)$} 

AOS: \gap{$x=-3$}

domain: \gap{all real numbers}

range: \gap{$y \ge 0$}\\[1\baselineskip]
circle one: MAX or MIN \\[1\baselineskip]
circle one: opens UP or DOWN
}
\end{tcbposter}



\vspace{1.5\baselineskip}

\begin{tcbposter}[
    poster = {
        rows=7,
        columns=2,
        height=10cm,
        spacing=1mm,
        % showframe,
        },
    boxes = {colbacktitle=black!15,coltitle=black,colback=white,sharp corners}
]
\posterbox[]{name=A,column=1,row=1,span=1}{
    \Large
    $y = (x-7)^2$
    \hfill 
    \large
        ($\bm{h}=7$)
} 
\posterbox[adjusted title=What changed?] {name=Z,column=1,span=2,row=6,rowspan=2}{}
\posterbox[adjusted title=graph] {name=B,row=2,between=A and Z}{
\centering
\begin{myTikzpictureGrid}{0.2} {10}{10}
    % \tkzText(13,8){$x - y = 0$}
    \tkzFct[ solid, ultra thick, samples=100, domain=-10:10,]{x**2}
    % \tkzText(-13,7.25){$x + 2y = 6$}
\end{myTikzpictureGrid}
} 
\posterbox[adjusted title=attributes] {row=2,column=2,between=A and Z}{
vertex: \gap{$(7,0)$} 

AOS: \gap{$x=7$}

domain: \gap{all real numbers}

range: \gap{$y \ge 0$}\\[1\baselineskip]
circle one: MAX or MIN \\[1\baselineskip]
circle one: opens UP or DOWN
}
\end{tcbposter}



What is the {\bfseries\itshape effect} of $\bm{h}$ on the parent function?

\rule{\textwidth}{0.15mm}

\rule{\textwidth}{0.15mm}

\rule{\textwidth}{0.15mm}

\rule{\textwidth}{0.15mm}

\rule{\textwidth}{0.15mm}

\vfill 

The {\bfseries\itshape left/right} movement of the parent function is 
\gap{horizontal} \gap{translation}. 
The \gap{shape} of the curve does not change.