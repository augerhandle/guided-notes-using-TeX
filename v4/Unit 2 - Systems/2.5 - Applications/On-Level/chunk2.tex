\section{Solving Systems with a TI-84}

You can solve a \gap{system of equations} using a TI-84.

\begin{myConceptSteps}{To solve a system of equations using a TI-84\dots}
  [
    Follow these steps.
    {
      \itshape
      We had another handout with detailed instructions 
      for using the TI-84 for these kinds of problems.
    }
  ]
  \myStep{align}{%
    Rewrite the equations: variables on the \gap{left}, constants on the \gap{right}.
  }
  \myStep{augmented matrix}{%
      Write the \gap{augmented} \gap{matrix} for the system.
      }
  \myStep{data entry}{%
      Enter the matrix entries into the \gap{TI-84}.
      }
  \myStep{RREF}{%
      Use the \gap{rref} function to convert to 
      {\bfseries\itshape reduced row echelon form}:
      {
        \footnotesize
        \[
          \begin{bmatrix}[ccc|c]
              \boldsymbol{1} & 0 & 0 & \blacksquare \\
              0 & \boldsymbol{1} & 0 & \blacksquare \\
              0 & 0 & \boldsymbol{1} & \blacksquare
          \end{bmatrix}
        \]
      }
      \vspace{-1em}
    }
  \myStep{solution}{%
    Get the variable values from the rightmost column.
    Make sure to recognize \gap{no solution} and \gap{infinite solutions} cases.
}
\end{myConceptSteps}

\myWideProblemWithContent
{
  Solve the system of equations in problem \#1.
  \vspace{1.25in}
  \begin{center}
    \begin{tabular}{|m{3.25in}|m{3.25in}|}
        \hline
        % \underline{\bfseries\itshape variables:} & \underline{\bfseries\itshape system of equations:}  \\
        % \scalebox{5}{\fontsize{32pt}{0pt}\selectfont \phantom{\textbf{I}}} & \phantom{X} \\
        % \hline
        \underline{\bfseries\itshape augmented matrix:} & \underline{\bfseries\itshape RREF matrix:} \\
        \scalebox{6}{\fontsize{32pt}{0pt}\selectfont \phantom{\textbf{I}}} & \phantom{X} \\
        \hline
    \end{tabular}
\end{center}
%
Each kind of coupon costs\dots\\[0.4em]
\hfill 
red coupon: \$\gap{0.50}
\hfill 
green coupon: \$\gap{0.35}
\hfill 
blue coupon: \$\gap{0.25}
\hfill 
}[\normalsize]

