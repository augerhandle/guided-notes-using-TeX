\section{Solving Systems with a TI-84}

Once you have the word problem 
written as a \gap{system of equations},
you can solve that system using a TI-84.

\begin{myConceptSteps}{To solve a system of equations using a TI-84\dots}
  [
    Follow these steps.
    {
      \itshape
      We had another handout with detailed instructions 
      for using the TI-84 for these kinds of problems.
    }
  ]
  \myStep{augmented matrix}{%
      Write the \gap{augmented} \gap{matrix} corresponding to the system.
      }
  \myStep{data entry}{%
      Enter the augmented matrix entries into the TI-84.
      }
  \myStep{RREF}{%
      Use the {\ttfamily rref} function to convert to \gap{reduced} row echelon form,
      which looks like this:
      \[
        \begin{bmatrix}[ccc|c]
            \boldsymbol{1} & 0 & 0 & \blacksquare \\
            0 & \boldsymbol{1} & 0 & \blacksquare \\
            0 & 0 & \boldsymbol{1} & \blacksquare
        \end{bmatrix}
    \]
    }
  \myStep{solution}{%
    Get the variable values from the rightmost column.
    Make sure to recognize \gap{no solution} and \gap{infinite solutions} cases.
}
\end{myConceptSteps}

\myWideProblem
{
  Solve the system of equations in problem \#1.
}[\normalsize]{3.75in}[
  \footnotesize
  The reduced row echelon form looks like
  \(
    \begin{bmatrix}[cc|c]
        \boldsymbol{1} & 0  & \blacksquare \\
        0 & \boldsymbol{1}  & \blacksquare 
    \end{bmatrix}
\),
 because there are only two equations in this problem.
]