\noindent 
You already know how to use \gap{elimination} to solve systems like these

%
\vspace{1\baselineskip}
\noindent
\hfil 
\begin{minipage}{0.2\textwidth}
    \centering
    \sysdelim..
    \systeme{
        2x+3y=-8,
        -2x+4y=-6
    }
\end{minipage}
%
\hfil
% 
\begin{minipage}{0.2\textwidth}
    \centering
    \sysdelim..
    \systeme{
        2x+3y=10,
        2x-4y=-4
    }
\end{minipage}
\vspace{1\baselineskip}

\noindent
But what do you do with a system like this?
\begin{center}
    \centering
    \sysdelim..
    \systeme{
        3x-2y=7,
        x-3y=-14
    }
\end{center}

\myCenteredBox[width=5in]{
    Sometimes,
    you have to multiply \gap{one} of the equations 
    by a \gap{number} 
    to get {\bfseries\itshape negative coefficients} to line up.
}

\begin{myConceptSteps}{To solve systems by multiplying\dots}
    \myStep{opposites}{
        Multiply one of the equations by a \gap{number}
        so that the coefficients for one of the variables are \gap{negatives}.
        }
    \myStep{first variable}{
        Add the equations. This will \gap{eliminate} a variable.
        Solve for the \gap{other} variable.}
    \myStep{second variable}{
        Substitute the value of the \gap{first} variable 
        into one of the equations.
        Solve for the \gap{other} variable.}
\end{myConceptSteps}


\myWideProblem[%
    Solve these systems by elimination. 
    ]
{
    \begin{center}
        \sysdelim..
        \systeme{
            x-3y=-14,
            3x-2y=7
        }
        \end{center}
}{2.75in}

\myWideProblem
{
    \begin{center}
        \sysdelim..
        \systeme{
            4x+10y=-10,
            -5x+5y=-5
        }
    \end{center}
}{3.5in}