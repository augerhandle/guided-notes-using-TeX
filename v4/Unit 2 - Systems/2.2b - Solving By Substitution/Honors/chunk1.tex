\section{Solving Not-So-Simple Systems Algebraically}

\myCenteredBox[width=5.5in]{
    The process today is {\bfseries\itshape exactly} \gap{the same} 
    from yesterday. 
    The only difference is that the first step is more 
    \gap{complicated}.
}

\begin{myConceptSteps}{To solve a simple system by substitution\dots}
    \myStep{first variable}{
        Solve one of the equations for either \gap{$x$} or \gap{$y$}.
        You will usually have to \gap{divide} by a coefficient.
        The result is usually an \gap{expression}.
    }
    \myStep{substitute}{Substitute that expression into the \gap{other} equation.}
    \myStep{second variable}{
        Solve that equation for the \gap{second} variable.
        The result will be a \gap{number}.
    }
    \myStep{substitute}{
        Substitute that number into the expression for the other variable.
        The result will be a \gap{number}.
    }
\end{myConceptSteps}

\myWideProblem[Solve this simple system by substitution.]
{
    \vspace{-\baselineskip}
    \begin{align*}
        3x - 8y &= 3 \\
        8x + 5y &= 8
    \end{align*}
}{3.2in}


\myWideProblem
{
    \vspace{-\baselineskip}
    \begin{align*}
        4x - 8y &= 16\\
        5x - 2y &= 20
    \end{align*}
}{3.2in}

