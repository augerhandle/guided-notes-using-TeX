\section{Writing a system of 2-variable inequalities from graphs}

\begin{myConceptSteps}{To write the system of inequalities given the graph of its solution\dots}
  [Follow these steps for \gap{each line} in the graph.]
  \myStep{equations}{%
    Look at the line, and write its equation.
  }
  \myStep{inequalities}{%
    Write the inequality for the line based on shading.
    \begin{itemize}[fullwidth]
      \item Use {\Large $>$} or {\Large $\ge$} if the shading is \gap{above} the line.
      \item Use {\Large $<$} or {\Large $\le$} if the shading is \gap{below} the line.
      \item {\itshape For vertical lines,}
      \begin{itemize}
        \item Use {\Large $>$} or {\Large $\ge$} if the shading is to the \gap{right}.
        \item Use {\Large $<$} or {\Large $\le$} if the shading is to the \gap{left}.
      \end{itemize}
    \end{itemize}
  }
  \myStep{equals or not}{
    \begin{itemize}[fullwidth]
      \item Use {\Large $\ge$} or {\Large $\le$} for \gap{solid} lines.
      \item Use {\Large $>$} or {\Large $<$} for \gap{dashed} lines.
    \end{itemize}
  }
\end{myConceptSteps}



\myWideProblemWithContent[%
  Write the system of inequalities 
  whose solution is shown in this graph.]
{
  \begin{myTikzpictureGrid}{0.75} [-6]{5}{5} [dotted]
    \tkzFct[ solid, ultra thick, samples=100, domain=-10:10,]{x-1}
    \tkzFct[ solid, ultra thick, samples=100, domain=-10:10,]{x+3}
    \tkzFct[ loosely dashed, ultra thick, samples=100, domain=-10:10,]{3}
    \tkzFct[ loosely dashed, ultra thick, samples=100, domain=-10:10,]{-2}
    \tkzDrawAreafg[between= b and d,color=gray,domain = -5:-1]
    \tkzDrawAreafg[between= b and a,color=gray,domain = -1:0]
    \tkzDrawAreafg[between= c and a,color=gray,domain = 0:4]
  \end{myTikzpictureGrid}
}
