\section{On the Parent Function}

Consider the cube root \gap{parent} function.
The equation of its graph is
    \begin{myEqBox}
        \begin{align}
            y = \myRoot[3]{x}
            \text{\whenTEACHER{\quad we talked about this parent function in Lesson 6.3a}}
            \label{eq:parent}
        \end{align}
    \end{myEqBox}
which is true \myEmph{only} for points that are \gap{on the curve}
\whenTEACHER{and false everywhere else}.

Here's what that means\dots

Consider a generic point on the curve, \gap{$(x_{p}, y_{p})$}
\whenTEACHER{(``p'' for parent)}.
\begin{center}
    \begin{myTikzpictureGrid}{0.5} {10}{4}
        \tkzFct[ solid, ultra thick, samples=200, domain=0:10,]{(x)**(1.0/3.0)}
        \tkzFct[ solid, ultra thick, samples=200, domain=-10:0,]{-(-x)**(1.0/3.0)}
        \whenTEACHER{
            \draw[color=red,thick,fill=white] (2,1.26) circle (0.25 cm);
            \tkzText(0,1.5){$(x_{p}, y_{p})$}
        }
    \end{myTikzpictureGrid}
\end{center}

Because $(x_{p}, y_{p})$ is any point \gap{on the curve}, 

\begin{myEqBox}
    \begin{align}
        y_p = \myRoot[3]{x_p}
        \label{eq:original}
    \end{align}
\end{myEqBox}

\whenTEACHER{which is not true for points off the curve.}