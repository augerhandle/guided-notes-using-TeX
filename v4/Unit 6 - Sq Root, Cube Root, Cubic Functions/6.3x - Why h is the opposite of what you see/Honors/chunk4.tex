\section{The Translated Equation}

We're used to equations 
written as $y=\dots$.
So let's get that $y_t$ \gap{by} \gap{itself}.

\begin{myEqBox}
    \begin{align}
        y_t - k = \myRoot[3]{x_t - h} 
        \text{\whenTEACHER{\quad repeat of equation~\ref{eq:repeat2}}}
        \label{eq:repeat3}
    \end{align}
\end{myEqBox}

\begin{myEqBox}
    \begin{align}
        y_t = \myRoot[3]{x_t - h} + k
        \label{eq:repeat4}
    \end{align}
\end{myEqBox}

How did we get from equation~\ref{eq:repeat3} to equation~\ref{eq:repeat4}? \hrulefill 

\hrulefill 

We're used to writing it like this \whenTEACHER{(no subscripts)}

\begin{myEqBox}
    \begin{align}
        y = \myRoot[3]{x - h} + k
    \end{align}
\end{myEqBox}

This is the equation of all the \gap{translated} points.
Here is its graph.
\vspace{-1\onelineskip}
\begin{center}
    \begin{myTikzpictureGrid}{0.5} {10}{4}
        \tkzFct[ solid, thin, samples=200, domain=0:10,]{(x)**(1.0/3.0)}
        \tkzFct[ solid, thin, samples=200, domain=-10:0,]{-(-x)**(1.0/3.0)}
        %
        \tkzFct[ solid, ultra thick, samples=200, domain=5:12,]{(x-5)**(1.0/3.0)+ 2}
        \tkzFct[ solid, ultra thick, samples=200, domain=-10:5,]{-(-x+5)**(1.0/3.0) + 2}
        \whenTEACHER{
            \tkzText(0,1.5){$(x_{o}, y_{o})$}
            \draw[->, very thick] (2,1.26) -- (7,1.26);
            \tkzText(4.5,0.75){$h$}
            \draw[color=red,thick,fill=white] (2,1.26) circle (0.25 cm);
            \draw[color=red,thick,fill=white] (7,3.26) circle (0.25 cm);
            \draw[->, very thick] (7,1.26) -- (7,3.26);
            \tkzText(7.5,2.5){$k$}
            \tkzText(7,4.2){$(x_{n}, y_{n})$}
        }
    \end{myTikzpictureGrid}
\end{center}

So the equation for the graph of the \gap{translated} cube root function is 
\begin{tcolorbox}[center,width=2in,colback=blue!10,]
    \vspace{-1.25\onelineskip}
    \begin{align*}
        y = \myRoot[3]{x-h} + k
    \end{align*}
\end{tcolorbox}

\begin{itemize}
    \item The minus sign means that $h$ is the \gap{opposite of what you see}. 
    \item And there is \gap{no} \gap{minus} \gap{sign} in front of $k$.
\end{itemize}

\begin{tcolorbox}[center,colback=blue!10,]
    \centering
    So \myEmph{this} is why \gap{only} $h$ is the \gap{opposite of what you see}.
\end{tcolorbox}

\begin{center}
    \LARGE
    \sffamily
    \bfseries
    fin
\end{center}