\section{Translating $(x_p,y_p)$}

Now suppose we \gap{translate} the generic point to a new location.
\vspace{-1\onelineskip}
\begin{center} 
    \renewcommand{\arraystretch}{2}
    \begin{tabular}{|p{2in}|p{2in}|}
        \hline
        shift horizontally by \gap{$h$} & $x_p \Longrightarrow x_p + $ \gap{$h$} \\
        \hline
        shift vertically by \gap{$k$}   & $y_p \Longrightarrow y_p + $ \gap{$k$} \\
        \hline
    \end{tabular}
\end{center}

This creates a \gap{new} point, \gap{$(x_t,y_t)$} \whenTEACHER{(``t'' for translated)}.

\begin{center}
    \begin{myTikzpictureGrid}{0.5} {10}{4}
        \tkzFct[ solid, ultra thick, samples=200, domain=0:10,]{(x)**(1.0/3.0)}
        \tkzFct[ solid, ultra thick, samples=200, domain=-10:0,]{-(-x)**(1.0/3.0)}
        \whenTEACHER{
            \tkzText(0,1.5){$(x_{p}, y_{p})$}
            \draw[->, very thick] (2,1.26) -- (7,1.26);
            \tkzText(4.5,0.75){$h$}
            \draw[color=red,thick,fill=white] (2,1.26) circle (0.25 cm);
            \draw[color=red,thick,fill=white] (7,3.26) circle (0.25 cm);
            \draw[->, very thick] (7,1.26) -- (7,3.26);
            \tkzText(7.5,2.5){$k$}
            \tkzText(7,4.2){$(x_{t}, y_{t})$}
        }
    \end{myTikzpictureGrid}
\end{center}

Here is how the \gap{parent} and \gap{translated} coordinates are related.
\begin{myEqBox}
    \begin{align*}
        x_t &= x_p + h \\
        y_t &= y_p + k \myTagThis
        \label{eq:newtranslate}
    \end{align*}
\end{myEqBox}

\vfill 
Is the translated point on the curve of the parent function? \hrulefill

\hrulefill 

\hrulefill 

\vfill 

Because the translated point is \gap{not} \gap{on the curve}, 
\gap{
    $
    y_t \neq \myRoot[3]{x_t}
    $
}

\vfill 

\begin{tcolorbox}[center,width=4in,colback=blue!10,]
    \centering
    So what \myEmph{do} we know about the \gap{translated} point, \gap{$(x_t,y_t)$}?
\end{tcolorbox}

\vfill
\newpage 