% \needspace{2.5in}
\section{Shifting the Parent Function}

\begin{tcolorbox}[center,width=5in,]
    When you shift the parent function, 
    start by shifting its \gap{end} \gap{point}.
\end{tcolorbox}

\myWideProblemWithContent[%
    Use a graphing calculator to graph these functions.\vspace{-0.75\baselineskip}
    \begin{itemize}[nosep]
        \item Sketch the graphs of $f$ and $g$.
        \item What are the coordinates of $g$'s end point?
        \item What are  $g$'s domain and range?
        \item What happened going from $f$ to $g$?
    \end{itemize}
    ]
{
    $f(x) = \myRoot{x}$ \hspace{1in} $g(x) = \myRoot{x} - 4$
    \tcblower
    \begin{minipage}{0.32\textwidth}
        \begin{myTikzpictureGrid}{0.4} [-2]{10}{5}
            \whenTEACHER{
                \tkzText(9,4){$f$}
                \tkzFct[ solid, thin, samples=100, domain=0:10,]{sqrt(x)}
                \draw[black,thick,fill=black] (0,0) circle (0.2 cm);
                \tkzText(9,-2){$g$}
                \tkzFct[ solid, ultra thick, samples=100, domain=0:10,]{sqrt(x)-4}
                \draw[black,thick,fill=black] (0,-4) circle (0.2 cm);
            }
            % \tkzText(-13,7.25){$x + 2y = 6$}
        \end{myTikzpictureGrid}
    \end{minipage}
    \begin{minipage}{0.67\textwidth}
        \begin{itemize}[fullwidth]
            \item The coordinates of $g$'s end point are \gap{$(0,-4)$}.
            \item The domain of $g$ is \gap{$x >= 0$}.
            \item The range of $g$ is \gap{$y >= -4$}.
        \end{itemize}
    \end{minipage}
}
