\section{The Transformations Due To $a$, $h$, $k$}

\begin{tcbraster}[
    raster equal height,
    raster left skip = 1in, raster right skip = 1in, 
    raster column skip = 0.5in,
    raster before skip = 1\baselineskip, raster after skip = 1\baselineskip,
    ]
    \begin{tcolorbox}[]
        \centering
        {\bfseries\itshape parent function}\\[0.5\baselineskip]
        \large
        $f(x) = \myRoot{x}$
    \end{tcolorbox}
    \begin{tcolorbox}[]
        \centering
        {\bfseries\itshape transformed function}\\[0.5\baselineskip]
        \large
        $g(x) = \bm{a} \myRoot{x-\bm{h}} + \bm{k}$
    \end{tcolorbox}
\end{tcbraster}

\begin{tcbraster}[]
\end{tcbraster}


The constants {$\bm{a}$}, {$\bm{h}$}, and {$\bm{k}$} result in 
different kinds of {\bfseries\itshape transformations} 
of the parent function:
\vspace{-1\baselineskip}
\begin{center}
    \renewcommand{\arraystretch}{1.5}
    \newcommand{\myTempColumnWidth}{3.5in}
    \begin{tabular}{c|p{\myTempColumnWidth}}
        {\itshape constant} & {\itshape what it does} \\
        \toprule 
        $\bm{a}$ 
            & 
            \begin{minipage}[t]{\myTempColumnWidth}
                \begin{itemize}[nosep,fullwidth]
                    \item[] \gap{reflection} across the $x$-axis 
                \end{itemize}
            \end{minipage}
            \\ 
            & 
            \begin{minipage}[t]{\myTempColumnWidth}
                \begin{itemize}[nosep,fullwidth]
                    \item[] \gap{stretch} or \gap{compress} by $|\bm{a}|$
                \end{itemize}
            \end{minipage}
            \\[1em]
        \toprule
        $\bm{h}$ 
            & 
            \begin{minipage}[t]{\myTempColumnWidth}
                \begin{itemize}[nosep,fullwidth]
                    \item[] shift \gap{left} or \gap{right} by $|\bm{h}|$
                \end{itemize}
            \end{minipage}
            \\
        \toprule
        $\bm{k}$ 
            & 
            \begin{minipage}[t]{\myTempColumnWidth}
                \begin{itemize}[nosep,fullwidth]
                    \item[] shift \gap{up} or \gap{down} by $|\bm{k}|$
                \end{itemize}
            \end{minipage}
        \\
        \bottomrule
    \end{tabular}
\end{center}


Remember that $\bm{h}$ is the \gap{opposite} 
of what you see in the formula for $g(x)$.



\myProblems[%
    Find $a$, $h$, $k$, and 
    describe the transformations that convert the parent function
    into the following functions.
]
{
    $g(x) = \myRoot{x+4} - 5$
}
{
    $g(x) = -3\myRoot{x-7} + 1$
}{1.5in}


\myProblemsWithContent[%
    Given these transformations that convert the parent function
    into the following functions,
    find $a$, $h$, $k$, and write the formula for $g(x)$.
]
{
    \begin{itemize}[nosep,fullwidth]
        \item reflection across the $x$-axis 
        \item shift down by 2
    \end{itemize}
    \tcblower 
    \large
    $a$=\gap{-1}, $h$=\gap{0}, $k$=\gap{-2}\\[1\baselineskip] 
    $g(x) =$ \gap{$-\myRoot{x}-2$}
}
{
    \begin{itemize}[nosep,fullwidth]
        \item stretch by 2 
        \item shift right by 4 
    \end{itemize}
    \tcblower 
    \large
    $a$=\gap{2}, $h$=\gap{4}, $k$=\gap{0}\\[1\baselineskip] 
    $g(x) =$ \gap{$2\myRoot{x-4}$}
}
