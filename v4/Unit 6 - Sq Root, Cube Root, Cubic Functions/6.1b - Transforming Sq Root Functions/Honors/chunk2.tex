\section{The Graph of Transformed Absolute Value Functions}

\begin{myConceptSteps}{To quickly sketch the graph of a transformed square root function\dots}
    \myStep{vertex}{%
        Shift the \gap{end} \gap{point} by $\bm{h}$ and $\bm{k}$. 
        Put a \gap{dot} at $(\bm{h},\bm{k})$.
    }
    \myStep{arc}{%
        Draw the ``arc'' of a square root function starting at the end point.
        \begin{itemize}
            \item arch up if $\bm{a}$ is \gap{positive}
            \item arch down if $\bm{a}$ is \gap{negative}
        \end{itemize}
    }
\end{myConceptSteps}

\begin{tcolorbox}[center,width=6in,colback=white,after skip=-11pt,]
    Those steps \myEmph{ignore} stretching and compressing. 
    To graph the effect of those, you'd really need to use a graphing calculator 
    or plot a lot of points in a table.
\end{tcolorbox}


\myProblemsWithContent[Quickly sketch the graphs of these functions.]
{
    $g(x) = \myRoot{x - 3} -1$
    \tcblower 
    \centering
    ${a} =$ \gap{$1$},\quad ${h} =$ \gap{$3$},\quad ${k} =$ \gap{$-1$}\\[\baselineskip]
    \begin{myTikzpictureGrid}{0.35} [-1]{10}{4}
        \whenTEACHER{
            \tkzFct[ solid, ultra thick, samples=900, domain=-10:10,]{sqrt(x-3)-1}
        }
    \end{myTikzpictureGrid}
}
{
    $g(x) = -\myRoot{x+3} +2 $
    \tcblower 
    \centering
    ${a} =$ \gap{$-1$},\quad ${h} =$ \gap{$-3$},\quad ${k} =$ \gap{$2$}\\[\baselineskip]
    \begin{myTikzpictureGrid}{0.35} [-4]{6}{4}
        \whenTEACHER{
            \tkzFct[ solid, ultra thick, samples=500, domain=-10:10,]{-sqrt(x+3)+2}
        }
    \end{myTikzpictureGrid}
}
