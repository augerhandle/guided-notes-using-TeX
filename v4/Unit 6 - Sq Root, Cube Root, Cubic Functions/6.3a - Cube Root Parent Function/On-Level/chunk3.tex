% \needspace{2.5in}
\section{Shifting the Parent Function}

\begin{tcolorbox}[center,width=6.75in,]
    \centering
    When you shift the parent function, 
    start by shifting its \gap{inflection} \gap{point}.
\end{tcolorbox}

\myWideProblemWithContent[%
    Use a graphing calculator to graph these functions.\vspace{-0.75\baselineskip}
    \begin{itemize}[nosep]
        \item Quickly sketch the graphs of $f$ and $g$.
        \item What are the coordinates of $g$'s inflection point?
        \item What are  $g$'s domain and range?
        \item What happened going from $f$ to $g$?
    \end{itemize}
    ]
{
    $f(x) = \myRoot[3]{x}$ \hspace{1in} $g(x) = \myRoot[3]{x} - 4$
    \tcblower
    \begin{minipage}{0.5\textwidth}
        \begin{myTikzpictureGrid}{0.4} {9}[-7]{3}
            \whenTEACHER{
                % parent function, f
                \tkzFct[ solid, thin, samples=1000, domain=0:10,]{x**(1.0/3.0)}
                \draw[black,thin,fill=red] (0,0) circle (0.2 cm);
                \tkzFct[ solid, thin, samples=1000, domain=-10:0,]{-(-x)**(1.0/3.0)}
                \tkzText(4.5,2.5){$f$}
                % transformed function, g
                \tkzFct[ solid, ultra thick, samples=1000, domain=0:10,]{x**(1.0/3.0) - 4}
                \tkzFct[ solid, ultra thick, samples=1000, domain=-10:0,]{-(-x)**(1.0/3.0) - 4}
                \draw[black,ultra thick,fill=red] (0,-4) circle (0.2 cm);
                \tkzText(4.5,-3.0){$g$}
            }
        \end{myTikzpictureGrid}
    \end{minipage}
    \begin{minipage}{0.5\textwidth}
        \begin{itemize}[fullwidth]
            \item $g$'s inflection point: \gap{$(0,-4)$}.
            \item $g$'s domain: \gap{all real numbers}.
            \item $g$'s range: \gap{all real numbers}.
            \item $g$ shifted \gap{down} by \gap{$4$}
        \end{itemize}
    \end{minipage}
}
