\vfill 

We are building up a \myEmph{toolbox} of techniques to solve different kinds of equations.
\vspace{1\onelineskip}

\begin{tcbraster}[
    raster equal height, 
    raster columns = 2,
    raster column skip = 0.5in,
]
    \begin{tcolorbox}[colback=white,boxrule=0.5pt,]
        \raggedright
        {\small solving \myEmph{linear} equations \phantom{xxxxxxx}}
        \begin{center}
            \small
            $5x +3 = 13$
            \quad{\large$\Rightarrow$}\quad
            $x =$ \gap{$2$}
        \end{center}
    \end{tcolorbox}
    \begin{tcolorbox}[colback=white,boxrule=0.5pt,]
        \raggedright
        {\small solving \myEmph{quadratic} equations}
        \begin{center}
            \small
            $x^2 + 5x + 6 = 0$
            \quad{\large$\Rightarrow$}\quad
            $x =$ \gap{$-2, -3$}
        \end{center}
    \end{tcolorbox}
    \begin{tcolorbox}[colback=white,boxrule=0.5pt,]
        \raggedright
        {\small solving \myEmph{absolute value} equations}
        \begin{center}
            \small
            $|4x -2 | = 10$
            \quad{\large$\Rightarrow$}\quad
            $x =$ \gap{$-2, 3$}
        \end{center}
    \end{tcolorbox}
    \begin{tcolorbox}[colback=white,boxrule=0.5pt,]
        \raggedright
        {\small solving \myEmph{square root} equations}
        \begin{center}
            \small
            $\myRoot{6x+1} = 4$
            \quad{\large$\Rightarrow$}\quad
            $x =$ \gap{$\frac{5}{2}$}
        \end{center}
    \end{tcolorbox}
    \begin{tcolorbox}[colback=white,boxrule=0.5pt,]
        \raggedright
        {\small solving \myEmph{cube root} equations}
        \begin{center}
            \small
            $\myRoot[3]{5x-8} = 3$
            \quad{\large$\Rightarrow$}\quad
            $x =$ \gap{$7$}
        \end{center}
    \end{tcolorbox}
    \begin{tcolorbox}[colback=white,boxrule=0.5pt,]
        \raggedright
        {\small Now, solving \myEmph{cubic} equations}
        \begin{center}
            \small
            $(5x-8)^3 = -125$
            % 5x-8 = -5
            % 5x = 3 
            % x = 3/5
            \quad{\large$\Rightarrow$}\quad
            $x =$ \gap{$\frac{3}{5}$}
        \end{center}
    \end{tcolorbox}
\end{tcbraster}

\vfill

These \gap{algebraic} ways to solve equations will give you \gap{exact} solutions.
There is also a \gap{geometric} way. We will use {\scshape Desmos}, but this can 
be done on {\scshape TI-84}s, also.

\vfill 

\begin{myConceptSteps}{To solve equations by graphing\dots}
    \myStep{left}{Graph the \gap{left} side of the equation.}
    \myStep{right}{Graph the \gap{right} side of the equation.}
    \myStep{intersections}{Find all points where the two graphs \gap{intersect}.}
    \myStep{$x$}{
        The solution is the \gap{$x$} \gap{coordinates} of each of the intersection points.
        \begin{itemize}[nosep]
            \item[$\circ$] If you are using {\scshape Desmos}, you can click on the points to find their coordinates.
        \end{itemize}
    }
\end{myConceptSteps}


\vfill 