% A counter to number the problems in the guided notes.
\newcounter{MyProblemCounter}
\setcounter{MyProblemCounter}{1}
\newcommand{\useMyProblemCounter}{\theMyProblemCounter\stepcounter{MyProblemCounter}}


\newcommand{\myProblemFont}{\bfseries\itshape}



% ---------------------------------------------------------------------------
% These are the commands I use to format the problems in the guided notes
% that have an empty space where I will write during class.
% ---------------------------------------------------------------------------

% A single problem that takes half the page.
%
% #1 : optional directions for the problem(s)
% #2 : details for problem 1
% #3 : optional font style for box titles
% #4 : vertical height of the problem boxes
% #5 : optional text at the bottom
%
\NewDocumentCommand{\myProblem}{ o m O{\large} m O{} }{%
    \IfValueT{#1}{\vspace{1\parskip}\noindent#1\nopagebreak}%
    \begin{tcbraster}[%
        raster equal height,%
        raster columns=2,%
        raster column skip=0.5mm,%
        raster row skip=0.5mm,%
        ]%
        % This is the first problem.
        \begin{tcolorbox}[%
            enhanced,%
            sharp corners,%
            colback=white,%
            coltitle=black, colbacktitle=black!10!white,%
            boxrule=0pt, borderline={0.5pt}{0pt}{black},%
            title={\texttt{\useMyProblemCounter}},%
            attach boxed title to top left%
            ]
            #3#2
            \tcblower\vspace{#4}#5
        \end{tcolorbox}
        %
        % There IS no second problem. So make it empty space.
        \begin{tcolorbox}[colback=white, colframe=white,]\end{tcolorbox}%
    \end{tcbraster}
}

% A single problem that takes the full width of the page.
%
% #1 : optional directions for the problem(s)
% #2 : details for problem 1
% #3 : optional font style for box titles
% #4 : vertical height of the problem boxes
% #5 : optional text at the bottom
%
\NewDocumentCommand{\myWideProblem}{ o m O{\large} m O{} }{%
    \IfValueT{#1}{\vspace{1\parskip}\noindent#1\nopagebreak}%
    \begin{tcbraster}[%
        raster equal height,%
        raster columns=1,%
        raster column skip=0.5mm,%
        raster row skip=0.5mm,%
        ]%
        % This is the first problem.
        \begin{tcolorbox}[%
            enhanced,%
            sharp corners,%
            colback=white,%
            coltitle=black, colbacktitle=black!10!white,%
            boxrule=0pt, borderline={0.5pt}{0pt}{black},%
            title={\texttt{\useMyProblemCounter}},%
            attach boxed title to top left%
            ]
            #3#2
            \tcblower\vspace{#4}#5
        \end{tcolorbox}
    \end{tcbraster}
}

% Two problems next to each other.
%
% #1 : optional directions for the problem(s)
% #2 : details for problem 1
% #3 : details for problem 2
% #4 : optional font style for box titles
% #5 : vertical height of the problem boxes
% #6 : optional text at the bottom of problem 1
% #7 : optional text at the bottom of problem 2
%
\NewDocumentCommand{\myProblems}{ o m m O{\large} m O{} O{#6} }{%
    \IfValueT{#1}{\vspace{1\parskip}\noindent#1\nopagebreak}%
    \begin{tcbraster}[%
        raster equal height,%
        raster columns=2,%
        raster column skip=0.5mm,%
        raster row skip=0.5mm,%
        ]%
        % This is the first problem.
        \begin{tcolorbox}[%
            enhanced,%
            sharp corners,%
            colback=white,%
            coltitle=black, colbacktitle=black!10!white,%
            boxrule=0pt, borderline={0.5pt}{0pt}{black},%
            title={\texttt{\useMyProblemCounter}},%
            attach boxed title to top left%
            ]
            #4#2
            \tcblower\vspace{#5}#6
        \end{tcolorbox}
        % This is the second problem. 
        \begin{tcolorbox}[%
            enhanced,%
            sharp corners,%
            colback=white,%
            coltitle=black, colbacktitle=black!10!white,%
            boxrule=0pt, borderline={0.5pt}{0pt}{black},%
            title={\texttt{\useMyProblemCounter}},%
            attach boxed title to top left%
            ]
            #4#3
            \tcblower\vspace{#5}#7
        \end{tcolorbox}
    \end{tcbraster}
}

% ---------------------------------------------------------------------------
% These are the commands I use to format the problems in the guided notes
% that have an partially filled space where I will also write during class.
%
% I use the term "with content" to refer to this partially filled space.
% ---------------------------------------------------------------------------

% A single problem that takes half the page.
%
% #1 : optional directions
% #2 : the problem contents
% #3 : optional font style for the content
%
\NewDocumentCommand{\myProblemWithContent}{ o m O{\large} }{%
    \IfValueT{#1}{\vspace{1\parskip}\noindent#1\nopagebreak}%
    \begin{tcbraster}[%
        raster equal height,%
        raster columns=2,%
        raster column skip=0.5mm,%
        raster row skip=0.5mm,%
        ]%
        % This is the first problem.
        \begin{tcolorbox}[%
            enhanced,%
            sharp corners,%
            colback=white,%
            coltitle=black, colbacktitle=black!10!white,%
            boxrule=0pt, borderline={0.5pt}{0pt}{black},%
            title={\texttt{\useMyProblemCounter}},%
            attach boxed title to top left%
            ]
            #3#2
        \end{tcolorbox}
        %
        % There IS no second problem. So make it empty space.
        \begin{tcolorbox}[colback=white, colframe=white,]\end{tcolorbox}%
    \end{tcbraster}
}


% Two problems that that sit next to each other.
%
% #1 : optional directions
% #2 : the 1st problem contents
% #3 : the 2nd problem contents
% #4 : optional font style for the content
%
\NewDocumentCommand{\myProblemsWithContent}{ o m m O{\large} }{%
    \IfValueT{#1}{\vspace{1\parskip}\noindent#1\nopagebreak}%
    \begin{tcbraster}[%
        raster equal height,%
        raster columns=2,%
        raster column skip=0.5mm,%
        raster row skip=0.5mm,%
        ]%
        % This is the first problem.
        \begin{tcolorbox}[%
            enhanced,%
            sharp corners,%
            colback=white,%
            coltitle=black, colbacktitle=black!10!white,%
            boxrule=0pt, borderline={0.5pt}{0pt}{black},%
            title={\texttt{\useMyProblemCounter}},%
            attach boxed title to top left%
            ]
            #4#2
        \end{tcolorbox}
        % This is the second problem.
        \begin{tcolorbox}[%
            enhanced,%
            sharp corners,%
            colback=white,%
            coltitle=black, colbacktitle=black!10!white,%
            boxrule=0pt, borderline={0.5pt}{0pt}{black},%
            title={\texttt{\useMyProblemCounter}},%
            attach boxed title to top left%
            ]
            #4#3
        \end{tcolorbox}
    \end{tcbraster}
}


% A wide problem that can have any latex code in the content area of the box.
%
% #1 : optional directions 
% #2 : the problem
% #3 : optional font style for the content
%
\NewDocumentCommand{\myWideProblemWithContent}{ o m O{\large} }{%
    \IfValueT{#1}{\vspace{1\parskip}\noindent#1\nopagebreak}%
    \begin{tcbraster}[%
        raster equal height,%
        raster columns=1,%
        raster column skip=0.5mm,%
        raster row skip=0.5mm,%
        ]  
        \begin{tcolorbox}[%
            enhanced,%
            sharp corners,%
            colback=white,%
            coltitle=black, colbacktitle=black!10!white,%
            boxrule=0pt, borderline={0.5pt}{0pt}{black},%
            title={\texttt{\useMyProblemCounter}},%
            attach boxed title to top left%
            ]%
            #3#2
        \end{tcolorbox}
    \end{tcbraster}
}
