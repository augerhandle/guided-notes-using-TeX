% ---------------------------------------------------------------------------
% Honors vs. On-Level ("hol")
% ---------------------------------------------------------------------------
\newif\ifhol

% Typically, I will set one of these ONCE at the top of the doc.
\newcommand{\forHONORS}{\holtrue}
\newcommand{\forONLEVEL}{\holfalse}

% These will conditionally generate output (the argument) based on 
% whether this document is "for" HONORS or ONLEVEL according to the previous commands.
\newcommand{\whenHONORS}[1]  {\ifhol{#1}\else{}  \fi}
\newcommand{\whenONLEVEL}[1] {\ifhol{}\else{#1}\fi}


% ---------------------------------------------------------------------------
% A centered tcolorbox
% ---------------------------------------------------------------------------
%
% #1 - optional options to pass to tcolorbox
% #2 - the contents to put in the box
%
\NewDocumentEnvironment{myCenteredBox}{ O{} m }{%
    \begin{center}
        \begin{tcolorbox}[#1]#2\end{tcolorbox}
    \end{center}
}

% ---------------------------------------------------------------------------
% changes to part/chapter/section
% ---------------------------------------------------------------------------

% This is the name of the course that these lessons are for.
% Once upon a time, I had this as a package option, but I'm 
% not using .sty files for this year. (Maybe next year.)
\newcommand{\lsnCourseName}{\whenHONORS{Honors~}Algebra 2}

% Instead of "Part I" we want "Unit 1"
\renewcommand{\thepart}{\arabic{part}}
\renewcommand{\partname}{Unit} 

% Instead of "Chapter 1" we want "Lesson 1"
\renewcommand{\chaptername}{Lesson}

% remove chapter numbers from section numbering
\makeatletter
\renewcommand\thesection{\@arabic\c@section}
\makeatother

% simulate a part without typesetting one
% 
% #1 - the part (unit) number
% #2 - the part (unit) title
\newcommand{\dummypart}[2]{%
    \setcounter{part}{#1}
    \partmark{#2}
}

% ---------------------------------------------------------------------------
% These are "annotations", the foundation of the blocked-in-text that I use.
%
% I use the term "annotations" to capture the common
% infrastructure I use to define Objectives, Vobabulary & Concepts.
% ---------------------------------------------------------------------------

% font and styling commands for
% Objectives, Voculary, Key Concepts, etc...
\newcommand{\myAnnotationStyling}{\bfseries\large}

% #1 : name of the kind of annotation (Objectives, ...)
% #2 : title text to go with the annotation
% #3 : extra tcolorbox options
%
\NewDocumentEnvironment{myAnnotate}{ m m O{}}{
    \begin{tcolorbox}[
        colbacktitle=blue!10!white,
        colback=white,
        coltitle=black,
        fonttitle={\myAnnotationStyling},
        title={#1: },
        after title={\normalfont\itshape#2},
        #3,
        ]
}{
    \end{tcolorbox}
}

% #1 : name of the kind of annotation 
% #2 : title 
%
\NewDocumentEnvironment{myTabularAnnotate}{ m m }{
    \begin{myAnnotate}{#1}{#2}
    \begin{tabular}{r|l}
}{
    \end{tabular}
    \end{myAnnotate}
}
% #1 - column 1 text
% #2 - column 2 text
%
\NewDocumentCommand{\myRow}{mm}{{\bfseries\itshape \textcolor{blue}{#1}}&#2\\[1ex]}

% #1 : name of the kind of annotation 
% #2 : title 
% #3 : text before the list starts
%
\NewDocumentEnvironment{myListAnnotate}{ m m o }{
    \begin{myAnnotate}{#1}{#2}
    \IfValueT{#3}{#3}
    \begin{enumerate}[itemsep=0pt,]
}{
    \end{enumerate}
    \end{myAnnotate}
}
% #1 - column 1 text
% #2 - column 2 text
%
\NewDocumentCommand{\myItem}{mm}{\item{\bfseries\itshape \textcolor{blue}{#1}} #2}
