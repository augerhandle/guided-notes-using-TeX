
% ---------------------------------------------------------------------------
% For writing TI-84 instructions it's useful to have a visually
% distinct way to format the keys.
% ---------------------------------------------------------------------------
\newcommand{\myKey}[1]{{\ttfamily [#1]}}


% ---------------------------------------------------------------------------
% Honors vs. On-Level ("hol")
% ---------------------------------------------------------------------------
\newif\ifhol

% Typically, I will set one of these ONCE at the top of the doc.
\newcommand{\forHONORS}{\holtrue}
\newcommand{\forONLEVEL}{\holfalse}

% These will conditionally generate output (the argument) based on 
% whether this document is "for" HONORS or ONLEVEL according to the previous commands.
\newcommand{\whenHONORS}[1]  {\ifhol{#1}\else{}  \fi}
\newcommand{\whenONLEVEL}[1] {\ifhol{}\else{#1}\fi}


% ---------------------------------------------------------------------------
% Teachers vs. Students ("tstu")
% ---------------------------------------------------------------------------
\newif\iftstu

% Typically, I will set one of these ONCE at the top of the doc.
\newcommand{\forTEACHER}{%
    \tstutrue%
    \dashundergapssetup{teacher-mode=true,}%
}
\newcommand{\forSTUDENT}{%
    \tstufalse%
    \dashundergapssetup{teacher-mode=false,}%
}

% These will conditionally generate output (the argument) based on 
% whether this document is "for" HONORS or ONLEVEL according to the previous commands.
\newcommand{\whenTEACHER}[1]  {\iftstu{#1}\else{}  \fi}
\newcommand{\whenSTUDENT}[1] {\iftstu{}\else{#1}\fi}



% ---------------------------------------------------------------------------
% command for creating a 1- or 2-question warmup problem
%
% #1 date
% #2 optional general directions
%
% ---------------------------------------------------------------------------
\NewDocumentEnvironment{myWarmupProblems}{ 
        O{
            \begin{itemize}
                \item Show {\itshape all} your work.
                \item Put a \fbox{box} around your answer.
            \end{itemize}
            \vspace{0.5em}
        } 
    }
{
    \newpage
    {
        \sffamily
        \begin{center}
            {
                \huge\bfseries%
                \whenHONORS{Honors~}Algebra 2 Warmup
            }%
            \\[0.5em]
            {\Large\itshape\printdate}
        \end{center}
        \noindent{\Large#1}
    }
    \large
}{
}


% ---------------------------------------------------------------------------
% A centered tcolorbox
% ---------------------------------------------------------------------------
%
% #1 - optional options to pass to tcolorbox
% #2 - the contents to put in the box
%
\NewDocumentEnvironment{myCenteredBox}{ O{} m }{%
    \begin{center}
        \begin{tcolorbox}[#1]#2\end{tcolorbox}
    \end{center}
}

% ---------------------------------------------------------------------------
% Split the output into two (roughly) equally sized side-by-side minipages.
% ---------------------------------------------------------------------------
%
% #1 - content of first minipage
% #2 - content of second minipage
%
\NewDocumentCommand{\myTwoMinipages}{mm}{
    \begin{minipage}{0.49\textwidth}
        #1
    \end{minipage}
    % \hfill 
    \begin{minipage}{0.49\textwidth}
        #2
    \end{minipage}
}


% ---------------------------------------------------------------------------
% A version of \sqrt 
% ---------------------------------------------------------------------------
%
% #1 index of the root
% #2 uproot amount
% #3 radicand
%
\NewDocumentCommand{\myRoot}{ o O{2} m  }{%
    \IfNoValueTF{#1}{\sqrt{#3\,}}{\sqrt[\uproot{#2}#1]{#3\,}}
}


% ---------------------------------------------------------------------------
% changes to part/chapter/section
% ---------------------------------------------------------------------------

% This is the name of the course that these lessons are for.
% Once upon a time, I had this as a package option, but I'm 
% not using .sty files for this year. (Maybe next year.)
\newcommand{\lsnCourseName}{\whenHONORS{Honors~}Algebra 2}

% Instead of "Part I" we want "Unit 1"
\renewcommand{\thepart}{\arabic{part}}
\renewcommand{\partname}{Unit} 

% Instead of "Chapter 1" we want "Lesson 1"
\renewcommand{\chaptername}{Lesson}

% lesson suffix (eg, the 'a' in Lesson 5.2a)
\newcommand{\myLessonSuffix}{}

% A \chapter-based command to generate a lesson.
% This is a thin wrapper around \chapter that allows me
% to explicitly specify the lesson number. 
%
% #1 lesson name
% #2 optional lesson number (eg, the '2' in Lesson 5.2)
% #3 optional lesson suffix (eg, the 'a' in Lesson 5.2a)
%
\NewDocumentCommand{\myLesson}{ m O{1} O{} }
{
    \setcounter{chapter}{#2-1}
    \renewcommand{\myLessonSuffix}{#3}
    \chapter{#1}
}

% remove chapter numbers from section numbering
\makeatletter
\renewcommand\thesection{\@arabic\c@section}
\makeatother

% simulate a part without typesetting one
% 
% #1 - the part (unit) number
% #2 - the part (unit) title
\newcommand{\dummypart}[2]{%
    \setcounter{part}{#1}
    \partmark{#2}
}

% ---------------------------------------------------------------------------
% These are "annotations", the foundation of the blocked-in-text that I use.
%
% I use the term "annotations" to capture the common
% infrastructure I use to define Objectives, Vobabulary & Concepts.
% ---------------------------------------------------------------------------

% font and styling commands for
% Objectives, Voculary, Key Concepts, etc...
\newcommand{\myAnnotationStyling}{\bfseries\large}

% #1 : name of the kind of annotation (Objectives, ...)
% #2 : title text to go with the annotation
% #3 : extra tcolorbox options
%
\NewDocumentEnvironment{myAnnotate}{ m m O{}}{
    \begin{tcolorbox}[
        colbacktitle=blue!10!white,
        colback=white,
        coltitle=black,
        fonttitle={\myAnnotationStyling},
        title={#1: },
        after title={\normalfont\itshape#2},
        #3,
        ]
}{
    \end{tcolorbox}
}

% #1 : name of the kind of annotation 
% #2 : title 
%
\NewDocumentEnvironment{myTabularAnnotate}{ m m }{
    \begin{myAnnotate}{#1}{#2}
    \begin{tabular}{r|l}
}{
    \end{tabular}
    \end{myAnnotate}
}
% #1 - column 1 text
% #2 - column 2 text
%
\NewDocumentCommand{\myRow}{mm}{{\bfseries\itshape \textcolor{blue}{#1}}&#2\\[1ex]}

% #1 : name of the kind of annotation 
% #2 : title 
% #3 : text before the list starts
%
\NewDocumentEnvironment{myListAnnotate}{ m m o }{
    \begin{myAnnotate}{#1}{#2}
    \IfValueT{#3}{#3}
    \begin{enumerate}[itemsep=0pt,fullwidth,]
}{
    \end{enumerate}
    \end{myAnnotate}
}
% #1 - column 1 text
% #2 - column 2 text
%
\NewDocumentCommand{\myItem}{mm}{\item{\bfseries\itshape \textcolor{blue}{#1}} #2}


% ---------------------------------------------------------------------------
% A table for systems of equations word problems.
%
% #1 scale factor
% ---------------------------------------------------------------------------
\NewDocumentCommand{\mySystemTable}{O{8}}{
    \begin{center}
        \begin{tabular}{|m{5em}|m{2in}|m{2in}|m{1.5in}|}
            \hline
            {\bfseries\itshape variables} & {\bfseries\itshape equations} & {\bfseries\itshape system} & {\bfseries\itshape augmented matrix} \\
            \hline\hline
            \scalebox{#1}{\fontsize{32pt}{0pt}\selectfont \phantom{\textbf{I}}} & \phantom{X} & \phantom{X} & \phantom{X}\\
            \hline
        \end{tabular}
    \end{center}
}

\NewDocumentCommand{\myBetterSystemTable}{O{4}}{
    \begin{center}
        \begin{tabular}{|m{3.25in}|m{3.25in}|}
            \hline
            \underline{\bfseries\itshape variables:} & \underline{\bfseries\itshape system of equations:}  \\
            \scalebox{#1}{\fontsize{32pt}{0pt}\selectfont \phantom{\textbf{I}}} & \phantom{X} \\
            \hline
            \underline{\bfseries\itshape augmented matrix:} & \underline{\bfseries\itshape RREF matrix:} \\
            \scalebox{#1}{\fontsize{32pt}{0pt}\selectfont \phantom{\textbf{I}}} & \phantom{X} \\
            \hline
        \end{tabular}
    \end{center}
}