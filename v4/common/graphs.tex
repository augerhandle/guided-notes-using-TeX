% ---------------------------------------------------------------------------
% x-y graphs using Tkz
% ---------------------------------------------------------------------------

% a simple "symmetric" x-y graph
%
% #1 scale of the full graph
% #2 max/min x
% #3 max/min y (optional, if absent it defaults to the x argument)
% #4 optional function expressed compatibly with how Tkz functions are written (eg, "\x^2")
%
\NewDocumentCommand{\mySymmetricGraph}{ O{0.5} m o }{
    \begin{tikzpicture}[
        scale=#1,
        xaxe style/.style = { very thick, arrows={-{Straight Barb}}, },                 
        yaxe style/.style = { very thick, arrows={-{Straight Barb}}, },                 
        ]
        \tkzInit[
            xmax=#2, xmin=-#2, xstep=1,
            ymax=\IfValueTF{#3}{#3}{#2}, ymin=-\IfValueTF{#3}{#3}{#2}, ystep=1,
            ]
        \tkzGrid[ sub, subxstep=1, subystep=1, ]
        \tkzDrawX[label={$x$},color=black, right=0.2em,]
        \tkzDrawY[label={$y$},color=black, above=0.2em,]
        % \tkzLabelX[orig=false,]
        % \tkzLabelY[orig=false,]
        % \tkzFct[{-(},solid,very thick,color=black,samples=50,domain =-6:6.5]{\x}%
    \end{tikzpicture}
}

% a simple "symmetric" x-y graph
%
% #1 optional scale of the full graph
% #2 function expressed compatibly with how Tkz functions are written (eg, "x**2")
% #3 xmax
% #4 optional xmin
% #5 ymax 
% #6 optional ymin 
% #7 optional domain max    NOTE: if this is used, #6 and #7 MUST be specified
% #8 optional domain min    NOTE: if this is used, #6        MUST be specified
%
\NewDocumentCommand{\myGraphWithFunction}{ O{0.5} m mo mo oo }{
    \begin{tikzpicture}[
        scale=#1,
        xaxe style/.style = { thick, arrows={-{Straight Barb}}, },                 
        yaxe style/.style = { thick, arrows={-{Straight Barb}}, },                 
        ]
        \tkzInit[
            xmax=#3,
            xmin=\IfValueTF{#4}{#4}{-#3},
            xstep=1,
            ymax=#5, 
            ymin=\IfValueTF{#6}{#6}{-#5},
            ystep=1,
            ]
        \tkzGrid[ sub, subxstep=1, subystep=1, ]
        \tkzDrawX[label={$x$},color=black, right=0.2em,]
        \tkzDrawY[label={$y$},color=black, above=0.2em,]
        % \tkzLabelX[orig=false,] \tkzLabelY[orig=false,]
        \tkzFct[
            solid, ultra thick, %color=black,
            samples=100,
            domain = \IfValueTF{#4}{#4}{-#3} : #3,
            ]
            {#2}% This is the function to be graphed.
    \end{tikzpicture}
}

% a simple number line
%
% #1 scale of the numberline
% #2 max x
% #3 min x (optional, if absent it defaults to negative of the max)
% #4 label on the x-axis (optional, if absent there's no label)
%
\NewDocumentCommand{\myNumberLine}{ O{0.4} m o O{{}} }{
    \begin{tikzpicture}[
        scale=#1,
        xaxe style/.style = { very thick, arrows={-{Straight Barb}}, },                 
        ]
        \tkzInit[ xmax=#2, xmin=\IfValueTF{#3}{#3}{-#2}, xstep=1, ]
        \tkzDrawX[
            % right=12pt,
            label=#4,
            noticks=false, % Yes, I want ticks!
            tickup=1em, tickdn=1em,
            right space=0.05cm, left space=0.05cm, % extend the line beyond the last ticks
            ]
        % \tkzLabelX[ 
        %     % orig=false,
        %     step=#4,
        %     font=\small,
        %     below left=2pt,
        %  ]
    \end{tikzpicture}
}