\usepackage{xparse}
\usepackage{blindtext}
\usepackage{enumitem}
\usepackage{graphicx}
\usepackage{makecell}

\usepackage{amsmath,mathtools,amssymb,amsthm}
% See https://texblog.net/latex-archive/maths/amsmath-matrix/ 
% for an explanation of this extention of the amsmath matrix commands.
% It's a way to enable "augmented matrices" using a new optional argument:
%
% \begin{pmatrix}[cc|c]
%     1 & 2 & 3\\
%     4 & 5 & 9
%   \end{pmatrix}
%
\makeatletter
\renewcommand*\env@matrix[1][*\c@MaxMatrixCols c]{%
  \hskip -\arraycolsep
  \let\@ifnextchar\new@ifnextchar
  \array{#1}}
\makeatother

\usepackage{bm} % bold math package

\usepackage{booktabs}
\usepackage{multirow}
\usepackage{hyperref}
\usepackage{systeme}

\usepackage[most]{tcolorbox}
    \tcbuselibrary{skins}
    \tcbuselibrary{raster}
    \tcbuselibrary{skins}
    \tcbuselibrary{poster}
\usepackage{tikz}
    \usetikzlibrary{arrows.meta}
\usepackage{tkz-base}
\usepackage{tkz-fct}    
\usepackage{pgfplots}
    \pgfplotsset{compat=newest}

% for inserting blanks that the students fill in
\usepackage{dashundergaps} % for \gap
\dashundergapssetup{
    teacher-mode=false, % set to true to show answers 
    gap-format=underline,
    teacher-gap-format=underline,
    gap-font={\ECFAugie\MTversion{augie}\color{black}},
    gap-numbers=false,
    gap-widen=true,
    gap-extend-percent=100, % note: making this too big might create errors
    gap-number-format=\,\textsuperscript{\normalfont(\thegapnumber)},
}

\usepackage{emerald}
\usepackage[subdued]{mathastext}% no italic for Augie anyhow
    \MTDeclareVersion[n]{lmvtt}{T1}{lmvtt}{m}{n}
    \MTfamily{augie}
    \Mathastext[augie]
