\section{Multiplying rational expressions}

\begin{myConceptSteps}{To {\bfseries\itshape multiply} two rational expressions\dots}
    \myStep{factor}{%
        Factor the numerators and denominators.
    }
    \myStep{multiply}{%
        Multiply the numerators. 
        Multiply the denominators.
        \begin{itemize}[nosep]
            \item [$\circ$] Write the factors next to each other. \gap{Do not} distribute.
            \item [$\circ$] \gap{Do not} cancel!
        \end{itemize}
    }
    \myStep{domain}{%
        Exclude \gap{zeros} of \myEmph{all} the factors 
        in the denominator.
        \begin{itemize}
            \item those that cancel (\gap{holes})
            \item those that don't cancel (\gap{vertical asymptotes})
        \end{itemize}
    }
    \myStep{simplify}{%
        Cancel all common factors. This is the {\itshape simplified function}.
    }
    \myStep{result}{%
        The result has \gap{two} parts.
        \begin{itemize}
            \item the \gap{simplified} function, and 
            \item the \gap{domain}        
        \end{itemize}
    }
\end{myConceptSteps}

\myProblems[Multiply these rational expressions.]
    {
        $
        \cfrac
        {4(x-2)}
        {(x+3)}
        \cdot
        \cfrac
        {(x+3)}
        {2(x+2)}
        $
    }
    {
        $
        \cfrac
        {6(x+2)(x-2)}
        {(x+3)}
        \cdot
        \cfrac
        {(x+3)(x+1)}
        {3(x+2)}
        $
    }
{3in}

\myWideProblem 
{
    $
    \cfrac
    {x^2 + 7x +12}
    {x-2}
    \cdot
    \cfrac
    {x+1}
    {x^2 - 2x - 15}
    $
}
{3.25in}
