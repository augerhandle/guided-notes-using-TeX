\section{Dividing rational expressions}


\begin{myConceptSteps}{To {\bfseries\itshape divide} two rational expressions\dots}
    \myStep{factor}{%
        Factor the numerators and denominators.
    }
    \myStep{domain}{%
        Find the domain as 
        \begin{itemize}[nosep]
            \item the \gap{zeros} of the first \gap{denominator}, 
            \item the \gap{zeros} of the second \gap{denominator}, and
            \item the \gap{zeros} of the second \gap{numerator}. 
        \end{itemize}
    }
    \myStep{flip and multiply}{%
        Multiply the first expression by the \gap{reciprocal} of the second.
        \begin{itemize}
            \item Write the factors next to each other. \gap{Do not} distribute.
        \end{itemize}
    }
    \myStep{simplify}{%
        Cancel all common factors.
    }
    \myStep{result}{%
        The result has \gap{two} parts.
        \begin{itemize}
            \item the \gap{simplified} function, and 
            \item the \gap{domain}        
        \end{itemize}
    }
\end{myConceptSteps}

\begin{tcolorbox}[center,colback=white,width=5in,]
    The domain for division problems is \gap{ tricky}
    because there is division happening in \gap{three} places.
    So be careful!
\end{tcolorbox}


\myProblems[Divide these rational expressions.]
        {
            $
            \cfrac
            {(x+1)}
            {6(x+2)}
            \div
            \cfrac
            {(x-3)}
            {3(x+2)}
            $
        }
        {
            $
            \cfrac
            {5(x+1)(x-3)}
            {(x+4)}
            \div
            \cfrac
            {5(x+1)}
            {(x+3)(x+4)}
            $
        }
{3in}

\myWideProblem
{
    $
    \cfrac
    {2x-6}
    {x+1}
    \div
    \cfrac
    {x^2-x-6}
    {x-4}
    $
}
{3in}