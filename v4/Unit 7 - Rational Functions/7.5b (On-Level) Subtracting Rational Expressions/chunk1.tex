In the last lesson, we added rational expressions. 
We simplified expressions that looked like this.
\[
    \frac
    {(x-2)}
    {(x+3)}
    +
    \frac
    {3x}
    {(x+3)(x+5)}
\]
%
Today we will subtract like this:
\[
    \frac
    {(x-2)}
    {(x+3)}
    \bm{-}
    \frac
    {3x}
    {(x+3)(x+5)}
\]


\begin{tcolorbox}[center,colback=white,width=7in,]
    {\bfseries\itshape Subtraction} 
    is mostly the same as addition.
    The key idea is to combine the fractions 
    over a common \gap{denominator} (LCD).\par
    \vspace{1em}
    But for subtraction, 
    you need to remember to \gap{distribute} the negative.
    This is easy to forget!
\end{tcolorbox}

\begin{myConceptSteps}{To {\bfseries\itshape subtract} two rational expressions\dots}
    \myStep{common denominator}{
        If the fractions have \gap{different} denominators\dots
        \begin{itemize}[nosep]
            \item Factor the denominators.
            \item Find the \gap{LCD} of the two rational expressions.
            \item \gap{Multiply} the numerators and denominators 
            by factors needed to make their denominators equal to the \gap{LCD}.
        \end{itemize}
    }
    \myStep{subtract}{%
        Subtract the \gap{numerators}.
        \begin{itemize}[nosep]
            \item{\bfseries\itshape Do not} subtract the denominators.
            The denominator becomes the \gap{LCD}. 
        \end{itemize}
    }
    \myStep{simplify}{%
        Simplify the expression.
        \begin{itemize}[nosep]
            \item Remember to \gap{distribute} the negative. 
        \end{itemize}
    }
\end{myConceptSteps}


\myWideProblem
    {
        $
        \frac{2x}{x^2+2x-8}
        -
        \frac{1}{2}
        $
    }
    {6.5in}