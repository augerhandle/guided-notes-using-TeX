% A command to format a bold right arrow that I use to indicate
% "and then you do this...".
%
% #1 spacing on left/right of the arrow 
%
\NewDocumentCommand{\myAndThen}{ O{0.25em} }{%
    \hspace{#1}% 
    {\Large$\bm{\Rightarrow}$}%
    \hspace{#1}
}

\myWideProblemWithContent[%
    Find the attributes of these rational functions.
    ]
{
    $ f(x) =\cfrac {2x+4} {x^2 - 4} $
\tcblower
    %
    %
    \myEmph{Step 1. $y$--intercept}\\
        \whenTEACHER{
            \small
            $ x = 0 $ 
            \myAndThen
            $ y = f(0) = \frac{4}{-4} = \fbox{-1} $
        }
        \\[2\onelineskip]
    \myEmph{Step 2. horizontal\whenHONORS{ or slope} asymptote}\\
        \whenTEACHER{
            \small
            $n = 1$, $d = 2$ 
            \myAndThen 
            $n<d$ 
            \myAndThen 
            HA @ $\fbox{y=0}$
        }
        \\[2\onelineskip]
    \myEmph{Step 3. factor \& simplify}\\
        \whenTEACHER{
            \small
            $ f_{\scalebox{0.5}{factored}}(x) = \frac {2(x+2)} {(x+2)(x-2)} $ 
            \myAndThen[1em]
            $ 
                f_{\scalebox{0.5}{simplified}}(x) 
                = \frac {2\cancel{(x+2)}} {\cancel{(x+2)}(x-2)}  
                = \frac {2} {(x-2)} 
            $ 
            \myAndThen[1em]
            canc. factors: $(x+2)$
        }
        \\[2\onelineskip]
    \myEmph{Step 4. holes}\\
        \whenTEACHER{
            \small
            \cancel{$(x+2)$} for $f_{\scalebox{0.5}{simplified}}(x)$
            \myAndThen 
            hole @ $\fbox{x=-2}$
        }
        \\[2\onelineskip]
    \myEmph{Step 5. vertical asymptotes}\\
        \whenTEACHER{
            \small
            $f_{\scalebox{0.5}{simplified}}(x)$ denom $=0$   
            \myAndThen 
            VA @ $\fbox{x=2}$        
        }
        \\[2\onelineskip]
    \myEmph{Step 6. domain}\\
        \whenTEACHER{
            \small 
            holes \& VA 
            \myAndThen
            \fbox{$x \ne 2, -2$}
        }
        \\[2\onelineskip]
    \myEmph{Step 7. $x$--intercepts}\\
        \whenTEACHER{
            \small
            $ f_{simp}(x) $  numer $=0$
            \myAndThen 
            \fbox{no $y$-intercept}
        }
        \\[2\onelineskip]
}
