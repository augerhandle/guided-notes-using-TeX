% A command to format a bold right arrow that I use to indicate
% "and then you do this...".
%
% #1 spacing on left/right of the arrow 
%
\NewDocumentCommand{\myAndThen}{ O{0.25em} }{%
    \hspace{#1}% 
    {\Large$\bm{\Rightarrow}$}%
    \hspace{#1}
}

\myWideProblemWithContent[%
    Find the attributes of these rational functions.
    ]
{
    $ f(x) =\cfrac {x^3-x^2-12x} {x^2-3x-4} $
\tcblower
    %
    %
    \myEmph{Step 1. $y$--intercept}\\
        \whenTEACHER{
            \small
            $ x = 0 $ 
            \myAndThen
            $ y = f(0) = \frac{0}{-4} = \fbox{0} $
        }
        \\[2\onelineskip]
    \myEmph{Step 2. horizontal\whenHONORS{ or slope} asymptote}\\
        \whenTEACHER{
            \small
            $n = 3$, $d = 2$ 
            \myAndThen 
            $n=d+1$ 
            \myAndThen 
            long div quotient: $x+2$ 
            \myAndThen 
            slant asym: \fbox{$y=x+2$}
        }
        \\[10\onelineskip]
    \myEmph{Step 3. factor \& simplify}\\
        \whenTEACHER{
            \small
            $ f_{\scalebox{0.5}{factored}}(x) = \frac {x(x+3)(x-4)} {(x+1)(x-4)} $ 
            \myAndThen[1em]
            $ 
                f_{\scalebox{0.5}{simplified}}(x) 
                = \frac {x(x+3)\cancel{(x-4)}} {(x+1)\cancel{(x-4)}}  
                = \frac {x(x+3)} {(x+1)} 
            $ 
            \myAndThen[1em]
            canc. factors: $(x-4)$
        }
        \\[2\onelineskip]
    \myEmph{Step 4. holes}\\
        \whenTEACHER{
            \small
            \cancel{$(x-4)$} for $f_{\scalebox{0.5}{simplified}}(x)$
            \myAndThen 
            hole @ $\fbox{x=4}$
        }
        \\[2\onelineskip]
    \myEmph{Step 5. vertical asymptotes}\\
        \whenTEACHER{
            \small
            $f_{\scalebox{0.5}{simplified}}(x)$ denom $=0$   
            \myAndThen 
            VA @ $\fbox{x=-1}$        
        }
        \\[2\onelineskip]
    \myEmph{Step 6. domain}\\
        \whenTEACHER{
            \small 
            holes \& VA 
            \myAndThen
            \fbox{$x \ne 4, -1$}
        }
        \\[2\onelineskip]
    \myEmph{Step 7. $x$--intercepts}\\
        \whenTEACHER{
            \small
            $ f_{simp}(x) $  numer $=0$
            \myAndThen 
            $x(x+3)=0$ 
            \myAndThen
            \fbox{$x = 0, -3$}
        }
        \\[2\onelineskip]
}
