\section{Classifying Polynomials}

Polynomials are also {\bfseries\itshape classified} by \gap{degree},
the largest \gap{exponent} on the variable.

\begin{center}
    \large\renewcommand{\arraystretch}{1}
    \begin{tabular}{c|ll}
        \gap{constant} 
            & a polynomial of degree {\bfseries\itshape 0}
            & $9$ \\
        \gap{linear} 
            & a polynomial of degree {\bfseries\itshape 1}
            & $5x + 3$ \\
        \gap{quadratic} 
            & a polynomial of degree {\bfseries\itshape 2}
            & $x^2 -6x +9$ \\
        \gap{cubic} 
            & a polynomial of degree {\bfseries\itshape 3} 
            & $x^3 - 27$ \\
        \gap{quartic} 
            & a polynomial of degree {\bfseries\itshape 4} 
            & $5x^4 - x^3 - 27$ \\
        \gap{quintic} 
            & a polynomial of degree {\bfseries\itshape 5} 
            & $x^5 + x^2 +2x -7$ \\
    \end{tabular}    
\end{center}


{
Polynomials \gap{never} have 
\begin{center}
\renewcommand{\arraystretch}{1.25}
\begin{tabular}{l|c}
    variables with \gap{negative} exponents    & {\large $x^{-2}$}          \\
    variables in the \gap{denominator}         & {\normalsize $\frac{1}{x^2}$}   \\
    variables with \gap{non-integer} exponents & {\large $x^{2/3}$} \\
    variables under a \gap{radical}            & {\large $\myRoot{x}$}      \\
\end{tabular}
\end{center}
}
