We will \gap{multiply} polynomials using the \gap{box method}.
This is useful for
\begin{itemize}[topsep=0in]
    \item multiplying \gap{long} polynomials,
    \item collecting \gap{like} terms, and 
    \item \gap{factoring} (which we will do later).
\end{itemize}
\vspace{1\baselineskip}

\begin{myConceptSteps}{To multiply polynomials using the {\bfseries\itshape box method}\dots}
    \myStep{table}{%
        Create a table with a bunch of \gap{boxes} in it.
        \begin{itemize}[fullwidth]
            \item Write the polynomials along the 
            \gap{top} and \gap{left} side of a rectangle.
            \item Make a \gap{row} and \gap{column} for each term.
        \end{itemize}
    }
    \myStep{boxes}{Fill each {\bfseries\itshape row/column} \gap{box}
        with the \gap{product} of two monomials.
    }
    \myStep{add}{\gap{Add} all the boxes together, 
        \gap{combining} like terms.
        }
\end{myConceptSteps}


\myProblems[Multiply the following polynomials.]
    {
        $ (2x-5) (3x-4) $
    }
    {
        $ (3x-5) (4x+2) $
    }
    {1.75in}


\myProblems
    {
        $ (2x-5) (x^2 + 2x + 3) $
    }
    {
        $ (3x-2) (3x^2 + 2x + 4) $
    }
    {3in}

\myCenteredBox[width=4.5in]{
    Remember that squaring something means 
    multiplying it \gap{times} \gap{itself} twice:
    \Large
    \[ (2x+1)^2 = (2x+1)(2x+1 )\]
}

\myProblems
    {
        $ (x-3y)^2 $
    }
    {
        $ (4x+y)^2 $
    }
    {2.5in}
