\section{Like Terms in the Box Method}

When you use the \gap{box method},
if you 
\begin{itemize}[topsep=0in]
    \item organize {\bfseries\itshape single-variable} polynomials by \gap{decreasing} exponents, and 
    \item put \gap{zero coefficients} for missing terms,
\end{itemize}
then like terms are always along a \gap{diagonal}.

\myProblems
    {
        $ (2x+1)  (x^2 + 3x -5) $
    }
    {
        $ (x-2)  (3x^2 - 4x -2) $
    }
    {2in}


\myWideProblem
    {
        $ (x^2+2x+3)  (x^3-3x-5) $
    }
    {3in}
