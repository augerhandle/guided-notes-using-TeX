When you factor a polynomial,
your \gap{first} step should {\bfseries\itshape always} be 
to remove a \gap{GCF} (if possible).
In this lesson, 
we assume you have already done this and that you are 
focusing on factoring a quadratic trinomial.

\begin{myConceptSteps}{To factor a quadratic trinomial\dots}[%
    \begin{center}
        % \large
        \renewcommand{\arraystretch}{1.5}
        \setlength{\fboxsep}{4pt}
        \setlength{\fboxrule}{0.9pt}
        \begin{tabular}{r|c|c|}
            {} & {\normalsize\itshape GCF} & {\normalsize\itshape GCF} \\
            \hline
            {\normalsize\itshape GCF} & \fbox{\,$a$\,}$x^2$ & \gap{?} $x$ \\
            \hline
            {\normalsize\itshape GCF} & \gap{?} $x$ & \fbox{\,$c$\,} \\
            \hline
        \end{tabular}
    \end{center}
    Follow these steps:]
    \myStep{descending order}{%
        Rewrite as
        \gap{$ax^2+bx+c$}.
    }
    \myStep{constants}{Write down $a$, $b$, $c$. Calculate \gap{$ac$}.}
    \myStep{factors of $ac$}{Find \gap{factors} of $ac$ that add to $b$.}
    \myStep{boxes}{Fill in the $a$ and $c$ boxes. Put the factors in the two blanks (any order).}
    \myStep{GCF \#1}{%
        Find the GCF of the \gap{top} \gap{row}. Write it outside on the left.
    }
    \myStep{GCF \#2,3,4}{%
        Find the other GCFs using the area model: 
        $ \text{\itshape height} \times \text{\itshape width} = \text{\itshape area}$.
    }
    \myStep{result}{%
        The result is the product of 
        \begin{itemize}[nosep]
            \item the \gap{left} {\itshape GCF} column and 
            \item the \gap{top} {\itshape GCF} row.
        \end{itemize}
    }
\end{myConceptSteps}



\myProblems[Factor these quadratic trinomials.]
    {
        $ x^2 + 7x + 12 $
    }
    {
        $ z^2 +z -56 $
    }
    {3in}

\myProblems
    {
        $ 21x + 4x^2 - 18 $
    }
    {
        $ 11r-20+4r^2 $
    }
    {4in}

