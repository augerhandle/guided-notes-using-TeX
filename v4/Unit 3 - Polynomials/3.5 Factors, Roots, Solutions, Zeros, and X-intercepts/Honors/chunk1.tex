


\begin{myConcept}{The following four things are all related. Given one, you can find the others.}
\begin{tcbraster}[
    % raster before skip=2\baselineskip,
    % raster after skip=2\baselineskip,
    raster columns=2, raster rows=2,
    raster left skip=1.25in, raster right skip=1.25in,
    raster column skip = 0.75in, raster row skip = 0.25in,
    raster equal height, 
    valign=center,
]
    \begin{tcolorbox}
        \begin{center}
            {\bfseries\itshape linear factors}\\[0.25em]
            {\large $(x-a)$}\\[0.25em]
            of $f(x)$
        \end{center}
    \end{tcolorbox}
    \begin{tcolorbox}
        \begin{center}
            {\bfseries\itshape $x$-intercepts}\\
            of $f(x)$
        \end{center}
    \end{tcolorbox}
    \begin{tcolorbox}
        \begin{center}
            {\bfseries\itshape roots/solutions}\\
            of\\
            $f(x)=0$
        \end{center}
    \end{tcolorbox}
    \begin{tcolorbox}
        \begin{center}
            {\bfseries\itshape zeros}\\ 
            of $f(x)$
        \end{center}
    \end{tcolorbox}
\end{tcbraster}

\begin{itemize}[fullwidth]
    \item If \gap{$(x-a)$} is a linear factor, then $a$ is the \gap{$x$-coordinate} of an $x$ intercept.
    \item \gap{Zeros} and \gap{roots}/\gap{solutions} have the same value.
    \begin{itemize}
        \item[$\circ$] the \gap{$x$-coordinates} of the $x$ intercepts
    \end{itemize}
\end{itemize}
\end{myConcept}
