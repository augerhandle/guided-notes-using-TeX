\section{The Division Algorithm}

If $P(x)$ and $d(x)$ are two polynomials (with $d(x) \ne 0$),
the {\bfseries\itshape Division Algorithm} states:

\myCenteredBox[width=3.5in,valign=center]{
    \Large
    \vspace{-\baselineskip}
    \[ P(x) = Q(x) d(x) + R(x) \]
}

There is another way to write this.

\myCenteredBox[width=3.5in,valign=center]{
    \Large
    % \vspace{-\baselineskip}
    \[ \frac{P(x)}{d(x)} = Q(x) + \frac{R(x)}{d(x)} \]
}

which I like to write as 

{
    \Large
    \begin{center}
    $ 
    \frac{\text{\scshape Dividend}}{\text{\scshape Divisor}} 
    =
    \text{\scshape Quotient} 
        + 
        \frac{\text{\scshape Remainder}}{\text{\scshape Divisor}} 
    $
    \end{center}
}



\newpage




\section{The Remainder Theorem}

Start with the Division Algorithm.

\vspace{2.5in}

What this means is:

\myCenteredBox[]{
    The remainder of 
    $ \frac{P(x)}{(x-a)} $,
    is equal to $P(a)$.
    (You can evaluate $P$ at $x=a$ by doing synthetic division and using $R$ as the answer!)
}



\section{The Factor Theorem}

Start with the Division Algorithm.

\vspace{2.5in}

What this means is:

\myCenteredBox[]{
    If you get a remainder of \gap{zero} 
    when you divide 
    $ \frac{P(x)}{(x-a)} $,
    then 
    $(x-a)$ 
    is a {\bfseries\itshape factor} of $P(x)$.
}
