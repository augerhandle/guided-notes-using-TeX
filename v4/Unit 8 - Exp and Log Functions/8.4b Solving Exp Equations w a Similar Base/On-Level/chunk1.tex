    Exponential equations are easy to solve then they have the 
    \gap{same} \gap{base}.
    \begin{tcolorbox}[center,colback=white,width=5.5in]
        \begin{itemize}
            \item You cancel the base\dots 
            \item \dots \myEmph{only if} 
            you can get 
            \gap{one} exponential expression with the \gap{same} \gap{base}
            on each side.
        \end{itemize}
    \end{tcolorbox}

    Sometimes you can \gap{convert} the bases to be the same. 
    For example,

    \vspace{-1\onelineskip}
    \begin{center} 
        \large
        \renewcommand{\arraystretch}{1.25}
        \begin{tabular}{c|c}
            \toprule
            {\itshape original equation} & {\itshape converted equation} \\
            \midrule 
            $ 2^{3x-1} = 1 $   &   $2^{3x-1} = 2^0$ \\
            $ 10^{x+1} = \frac{1}{10} $   &   $10^{x+1} = 10^{-1}$ \\
            $ 5^{7x+13} = 25^x $   &   $5^{7x+13} = 5^{2x}$ \\
            \bottomrule
        \end{tabular}
    \end{center}

