Most calculators have \myEmph{common logs} (base 10) and \myEmph{natural logs} (base $e$). 
But what if the base is something else?

\begin{tcolorbox}[center,colback=white,width=5in,]
    \small
    On the {\scshape TI-84} calculator, you can calculate $log_b(A)$ 
    for any base $b$:
    \begin{itemize}[nosep]
        \item Press \fbox{\ttfamily MATH}.
        \item Scroll down to option \fbox{\ttfamily A} which is \fbox{\ttfamily logBASE}.
        \item Press \fbox{\ttfamily ENTER}.
        \item Enter the $b$, and then enter $A$.
        \item Press \fbox{\ttfamily ENTER}.
    \end{itemize}
\end{tcolorbox}

\myProblems[Solve these exponential equations.]
{
    $3^{(2a-1)} -4 = 21$
}
{
    $2\cdot 5^{(3b-8)} + 4 = 82$
}
{5in}
[\whenTEACHER{
    \tiny
    $a=\frac{log_3{25}+1}{2}\approx1.965$ 
    \quad 
    (using {\ttfamily logBASE} on {\scshape TI-84})
}]
[\whenTEACHER{
    \tiny$b=\frac{log_5{39} + 8}{3}\approx3.425$ 
    \quad 
    (using {\ttfamily logBASE} on {\scshape TI-84}
}]


% \myWideProblem
% {
%     $5 \cdot 8^{(4c+10)} - 6 = 49$
% }
% {2in}
% [\whenTEACHER{
%     \tiny
%     $x=\frac{log_8{11}-10}{4}\approx-2.212$ 
%     \quad 
%     (using {\ttfamily logBASE} on {\scshape TI-84})
% }]
