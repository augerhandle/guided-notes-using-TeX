\documentclass[fleqn,letterpaper,12pt,printwatermark=true]{memoir}
% memoir commands to define the text block geometry
\setulmarginsandblock{0.5in}{*}{*}
\setlrmarginsandblock{0.5in}{*}{*}
% put "extra" vertical space at the bottom of a page
\raggedbottom 

\usepackage{amsmath}
\usepackage{etoolbox} % for \ifblank etc
\usepackage{xparse} % for NewDocumentCommand et al.
\usepackage{enumitem}
\usepackage{transparent} % for \transparent, which I use in the watermark
\usepackage[slantedGreek]{mathpazo} \usepackage{helvet} % use Palatino et al.
\usepackage{booktabs} % prettier tables

\usepackage[]{xwatermark}
\newwatermark*[
    allpages,
    color=red!30,angle=45,
    scale=4,
    xpos=-10, ypos=0
]{%
    \transparent{0.4}dhasan example%
}

\usepackage{dashundergaps} % for \gap
\dashundergapssetup{
    teacher-mode=true, % set to true to show answers 
    gap-format=underline,
    teacher-gap-format=underline,
    gap-font={\sffamily},
    gap-numbers=true,
    gap-widen=true,
    gap-extend-percent=150, % note: making this too big might create errors
    gap-number-format=\,\textsuperscript{\normalfont(\thegapnumber)},
}

\usepackage{tcolorbox}
\tcbuselibrary{skins}
\tcbuselibrary{raster}

\usepackage{graphicx}
\graphicspath{ {../images/} }

% ---------------------DOCUMENT------------------------------
\begin{document}

\newcommand{\myClassName}{Pre-AP Algebra 2}
\newcommand{\myUnitNumber}{1}
\newcommand{\myUnitTitle}{Introduction to Functions}
\newcommand{\myLessonNumber}{8}
\newcommand{\myLessonTitle}{Composing Functions}


\copypagestyle{myPagestyle}{empty}


\newcommand{\myFooterSize}{\footnotesize}
\makeoddfoot{myPagestyle}{\myFooterSize\myClassName}{\myFooterSize\thepage\,of\,\pageref*{xwmlastpage}}{\myFooterSize\myLessonTitle}
\makeevenfoot{myPagestyle}{\myFooterSize\myLessonTitle}{\myFooterSize{\thepage{}~of~\pageref*{xwmlastpage}}}{\myFooterSize\myClassName}

%
% A command to change the appearance of the cognitive verb 
% in the objectives.
%
\newcommand{\myCognitiveVerb}[1]{\textcolor{blue}{\textbf{#1}}}

%
% Font styling commands (so I can change them in a single place)
%
\NewDocumentCommand{\myUnitLessonNumberFont}{}{\sffamily\bfseries\HUGE}
\NewDocumentCommand{\myUnitTitleFont}{}{\sffamily\large}
\NewDocumentCommand{\myLessonTitleFont}{}{\sffamily\bfseries\huge}
\NewDocumentCommand{\myHeadingFont}{}{\sffamily\bfseries\Large}


%
% #1 is the fill-in text
%
\NewDocumentCommand{\myFillInBlank}{m}{%
    \,%
    \gap[u]{#1}%
    \,%
}


% Definition for the LESSON HEADER + OBJECTIVES
%
\newenvironment{myNotesHeader}{
    \begin{flushleft}
        {\myUnitLessonNumberFont{\myUnitNumber.\myLessonNumber}}
        \hfill\;\;
        \begin{minipage}[b]{0.75\textwidth}
            \begin{flushright}
                {\myUnitTitleFont{}Unit \myUnitNumber\,\,\,\myUnitTitle}\\ \vspace{0.75em}
                {\myLessonTitleFont\myLessonTitle}
            \end{flushright}
        \end{minipage}
        \hrule
    \end{flushleft}
    \noindent{\myHeadingFont Objectives:}
    \begin{enumerate}[label=\arabic*)]
}{
    \end{enumerate}
}


% Definitions related to the VOCABULARY TABLE
%
\newenvironment{myVocabulary}{
    {\noindent{\myHeadingFont Vocabulary:}}\vspace{1em}

    \begin{tabular}{ll}
        \toprule
            \emph{word} & \emph{meaning} \\ 
        \midrule
}{
    \bottomrule
    \end{tabular}
    \vspace{1em}
}

\newcommand{\myVocabularyWord}[2]{%
{\textcolor{blue}{\textbf{#1}}} & #2 \\
}


% Definitions related to an INTRODUCTION
%
\newenvironment{myLesson}{
    {\noindent{\myHeadingFont Lesson:}}\vspace{1em}

    \begin{adjustwidth}{2em}{0pt}
    \begin{itemize}
}{
    \end{itemize}
    \end{adjustwidth}
    \vspace{1em}
}


% Definition related to KEY CONCEPTS
%
% #1 : the key concept (which appears as a tcolorbox title)
%
\NewDocumentEnvironment{myKeyConcepts}{ O{Key Concepts:} }{
    \begin{tcolorbox}[
        title=#1, fonttitle=\myHeadingFont,
        coltitle=black, 
        colbacktitle=black!25!yellow, 
        colframe=black!50!yellow,
        colback=white!70!yellow,
        boxrule=2pt, 
        ]
}{
    \end{tcolorbox}
}


% Definition related to EXAMPLES

% #1 Optional example number 
% #2 A statement of the example problem.
% #3 How much empty vertical space to leave for the example box.
%
\NewDocumentCommand{\myExample}{omm}{%
    \begin{tcolorbox}[
        enhanced,
        sharp corners, 
        colback=white,
        boxrule=0pt,
        borderline={0.5pt}{0pt}{black,dashed},
        ]
        {\myHeadingFont Example\IfValueT{#1}{{ #1}}:}
        #2
        \tcblower
        \vspace{#3}
    \end{tcolorbox}
}

% #1 Optional example number 
% #2 A statement of the example problem.
%
\NewDocumentEnvironment{myExampleForTikzGraphs}{om}{%
    \begin{tcolorbox}[
        enhanced,
        sharp corners, 
        colback=white,
        boxrule=0pt,
        borderline={0.5pt}{0pt}{black,dashed},
        ]
        {\myHeadingFont Example\IfValueT{#1}{{ #1}}:}
        #2
        \tcblower
        \begin{center}
}
% user would insert 
% \begin{tikzpicture}\begin{axis}...\end{axis}\end{tikzpicture}
{
        \end{center}
    \end{tcolorbox}
}


% Definitions related to PROBLEMS

% A counter to number the problems in the guided notes.
\newcounter{MyProblemCounter}
\setcounter{MyProblemCounter}{1}
\newcommand{\useMyProblemCounter}{\theMyProblemCounter\stepcounter{MyProblemCounter}}

% an environment for two adjacent problems
%
% #1 : directions for all the problems
% #2 : vertical height of the problem boxes
% #3 : details for problem 1
% #4 : details for problem 2
%
\newenvironment{myProblems2}[4]{%
    \noindent
    {\myHeadingFont Practice:}\hspace{0.5em}#1\nopagebreak%
    \begin{tcbraster}[%
        raster equal height,%
        raster columns=2,%
        raster column skip=0.5mm,%
        raster row skip=0.5mm,%
        raster every box/.style={%
            enhanced,%
            sharp corners,%
            colback=white,%
            coltitle=black, colbacktitle=black!10!white,%
            boxrule=0pt, borderline={0.5pt}{0pt}{black},%
            title={\texttt\useMyProblemCounter},%
            },%
        ]%
        \begin{tcolorbox}[attach boxed title to top left]
            #3
            \tcblower\vspace{#2}
        \end{tcolorbox}
        \begin{tcolorbox}[attach boxed title to top right]
            #4
            \tcblower
        \end{tcolorbox}%
}{%
    \end{tcbraster}
}

% an environment for 4 adjacent problems
%
% #1 : directions for all the problems
% #2 : vertical height of the problem boxes
% #3 : details for problem 1
% #4 : details for problem 2
% #5 : details for problem 3
% #6 : details for problem 4
%
\newenvironment{myProblems4}[6]{%
    \noindent
    \textbf{\myHeadingFont Practice:}\hspace{0.5em}#1\nopagebreak%
    \begin{tcbraster}[%
        raster equal height,%
        raster columns=2,%
        raster column skip=0.5mm,%
        raster row skip=0.5mm,%
        raster every box/.style={%
            enhanced,%
            sharp corners,%
            colback=white,%
            coltitle=black, colbacktitle=black!10!white,%
            boxrule=0pt, borderline={0.5pt}{0pt}{black},%
            title={\texttt\useMyProblemCounter},%
            },%
        ]%
        \begin{tcolorbox}[attach boxed title to top left]
            #3
            \tcblower\vspace{#2}
        \end{tcolorbox}
        \begin{tcolorbox}[attach boxed title to top right]
            #4
            \tcblower
        \end{tcolorbox}%
        \begin{tcolorbox}[attach boxed title to bottom left]
            #5
            \tcblower
        \end{tcolorbox}%
        \begin{tcolorbox}[attach boxed title to bottom right]
            #6
            \tcblower
        \end{tcolorbox}%
}{%
    \end{tcbraster}
}
\pagestyle{myPagestyle}

\checkandfixthelayout
\setlist{labelindent=\parindent,leftmargin=*,itemsep=0.025em,label=$\circ$}

% ---------------------HEADER------------------------------
\begin{myNotesHeader}
    \item \myCognitiveVerb{evaluate} a composite fuction with a numbers
    \item \myCognitiveVerb{compose} two functions
\end{myNotesHeader}

\begin{myVocabulary}
    \myVocabularyWord{compose}
        {
            construct something from smaller pieces
        }
        \myVocabularyWord{composing functions}
        {
            construct a function from smaller functions
        }
        \myVocabularyWord{composite function}
        {
            a function composed from smaller functions
        }
\end{myVocabulary}

% ---------------------LESSON 1------------------------------
\begin{myLesson}[Functions as Meat Grinders]
    A function can be thought of as a \emph{meat grinder}.\footnote{
        My sincere apologies to you vegetarians and vegans.
    }
    You put chunks of beef into a meat grinder (the \emph{input}),
    and ground hamburger comes out (the \emph{output}). 

    \begin{center}
    \includegraphics[width=2in]{meat-grinder.png} 
    \end{center}

    Similarly,
    in math a function has \emph{inputs} and \emph{outputs}.
    We usually use $x$ for the inputs and $y$ as the outputs.
    You put $x$'s into a function, $f(x)$,
    and $y$'s come out.
    Think of the $x$'s as the chunks of beef
    and the $y$'s as the ground hamburger.
    \[
        x
        \Longrightarrow
        f
        \Longrightarrow
        y
    \]

    You could ``daisy chain'' two meat grinders---when 
    you take the output of one function and put it into
    another one.
    The two griders act as a  
    \emph{single, bigger meat grinder} which takes chunks of beef
    as an input and creates 
    {\bfseries\itshape ultra-ground beef} as an output.

    \begin{center}
        \fbox{
            \includegraphics[width=1.5in]{meat-grinder.png} 
            \hspace{0.5in}
            \includegraphics[width=1.5in]{meat-grinder.png} 
        }
    \end{center}
    which we could also draw like this
    \begin{center}
        \fbox{
            \includegraphics[width=1.5in]{composite-meat-grinder.png} 
        }
    \end{center}
    to emphasize that the two grinders work together.
    This is called ``composing''. 
    We have composed two grinders together.
    The single, bigger grinder in the box is called 
    a ``composite'' grinder.

    When you do this to functions,
    we say you are ``composing'' the two functions.
    The result is a \emph{single, bigger function}.
    We call is a ``composite function''.

    \begin{center}
    \vspace{2em}
    \(
        x
        \Longrightarrow
    \)
    \fbox{
        \(
            f
            \Longrightarrow
            g
        \)
    }
    \(
        \Longrightarrow
        y
    \)
    \vspace{2em}
    \end{center}
    Today you'rll learn about what 
    \fbox{
        \(
            f
            \Longrightarrow
            g
        \)
    }
    is all about.
\end{myLesson}

\begin{myLesson}[][1]
Consider the following functions.
\begin{align*}
    f(x) &= 2x + 3 \\
    g(x) &= -x^2 + 5
\end{align*}
You already know how to evaluate them using a numerical value.
For example, evaluate them at $x=1$:
\begin{align*}
    f(1) &= 2(1) + 3 = 2+3 = 5\\
    g(1) &= -(1)^2 + 5 = -1 + 5 = 4
\end{align*}
When we \emph{compose} functions, 
we evaluate one of them and take the {\bfseries\itshape output} and pass it 
into the other one.

The math notation for function composition looks like this:
$f \circ g$,
which you pronounce as ``$f$ compose with $g$''. 
But there's something confusing about this notation.
What you need to know is this.

\begin{myLessonBox}
    Beware of the order of the composed functions!

    $(f \circ g)(1)$ 
\end{myLessonBox}



\end{myLesson}
% ---------------------CONCEPT 1------------------------------
\begin{myKeyConcepts}[To evaluate a composite function $f \circ g$ at a number\dots]
    Follow these steps:
    \setlist{labelindent=\parindent,itemsep=0.4em}
    \begin{enumerate}
        \item xxx
        \item xxx
    \end{enumerate}
    The order is important here. (TODO: more detail)xxxxxxxxxxxxx
\end{myKeyConcepts}

\myExample[1]{
    Find the inverse of (TODO: fixme)
}{3in}

\myExample[2]{
    Find the inverse of (TODO:fixme)
}{3in}

% ---------------------LESSON 1------------------------------
\begin{myLesson}[2]
    foo
\end{myLesson}

% ---------------------CONCEPT 2------------------------------
\begin{myKeyConcepts}[To compose functions $f$ and $g$\dots]
    Follow these steps:
    \setlist{labelindent=\parindent,itemsep=0.4em}
    \begin{enumerate}
        \item xxx
        \item xxx
    \end{enumerate}
    The order is important here. (TODO: more detail)xxxxxxxxxxxxx
\end{myKeyConcepts}

\myExample[1]{
    Find the inverse of (TODO: fixme)
}{3in}

\myExample[2]{
    Find the inverse of (TODO:fixme)
}{3in}

% \begin{myProblems2}%
%     {Factor the following monomials into prime factors.}%
%     {2in}%
%     %
%     {\( 32x^2 \)}
%     {\( 8 x^3y^2z \)}
% \end{myProblems2}
  


\end{document}