\copypagestyle{myPagestyle}{empty}


%
% Since we're using oneside, thee is only an ODD footer (no even one)
%
\newcommand{\myFooterSize}{\footnotesize}
\makeoddfoot{myPagestyle}{{}}{\myFooterSize{\thepage{}~of~\pageref*{xwmlastpage}}}{\myFooterSize\thetitle}

%
% A command to change the appearance of the cognitive verb 
% in the objectives.
%
\newcommand{\myCognitiveVerb}[1]{\textcolor{blue}{\textbf{#1}}}

%
% Terminology explanation:
%
% "unit" refers to the Algebra 2 unit being taught.
% "section" refers to the section within the unit being taught.
% "heading" refers to the headings in this document (Objectives, Example, ...)
%
\NewDocumentCommand{\myUnitSectionNumberFont}{}{\sffamily\bfseries\HUGE}
\NewDocumentCommand{\myUnitNameFont}{}{\sffamily\large}
\NewDocumentCommand{\mySectionNameFont}{}{\sffamily\bfseries\huge}
\NewDocumentCommand{\myHeadingFont}{}{\sffamily\bfseries\Large}


%
% #1 is the fill-in text
%
\NewDocumentCommand{\myFillInBlank}{m}{%
    \,%
    \gap[u]{#1}%
    \,%
}


%
% #1 : optional unit name
% #2 : optional unit/section number
% #3 : mandatory title
%
\NewDocumentEnvironment{myNotesHeader}{oom}{
    \title{#3}
    \begin{flushleft}
        \IfValueT{#2}{{\myUnitSectionNumberFont#2}}
        \hfill\;\;
        \begin{minipage}[b]{0.75\textwidth}
            \begin{flushright}
                \IfValueT{#1}{
                    {\myUnitNameFont#1}\\ \vspace{0.75em}
                }
                {\mySectionNameFont#3}
            \end{flushright}
        \end{minipage}
        \hrule
    \end{flushleft}
    \noindent{\myHeadingFont Objectives:}
    \begin{enumerate}[label=\arabic*)]
}{
    \end{enumerate}
}

%
% set up a 2-column tabular intended to hold vocabulary words
%
\newenvironment{myVocabulary}{
    {\noindent{\myHeadingFont Vocabulary:}}\vspace{1em}

    \begin{tabular}{ll}
        \toprule
            \emph{word} & \emph{meaning} \\ 
        \midrule
}{
    \bottomrule
    \end{tabular}
}

%
% intended as a row in a 2-column tabular
%
\newcommand{\myVocabularyWord}[2]{%
{\textcolor{blue}{\textbf{#1}}} & #2 \\
}


%
% #1 : the key concept (which appears as a tcolorbox title)
%
\NewDocumentEnvironment{myKeyConcepts}{ O{Key Concepts:} }{
    \begin{tcolorbox}[
        title=#1, fonttitle=\myHeadingFont,
        coltitle=black, 
        colbacktitle=black!25!yellow, 
        colframe=black!50!yellow,
        colback=white!70!yellow,
        boxrule=2pt, 
        ]
}{
    \end{tcolorbox}
}


%
% #1 Optional example number 
% #2 A statement of the example problem.
% #3 How much empty vertical space to leave for the example box.
%
\NewDocumentCommand{\myExample}{omm}{%
    \begin{tcolorbox}[
        enhanced,
        sharp corners, 
        colback=white,
        boxrule=0pt,
        borderline={0.5pt}{0pt}{black,dashed},
        ]
        {\myHeadingFont Example\IfValueT{#1}{{ #1}}:}
        #2
        \tcblower
        \vspace{#3}
    \end{tcolorbox}
}

%
% A counter to number the problems in the guided notes.
%
\newcounter{MyProblemCounter}
\setcounter{MyProblemCounter}{1}
\newcommand{\useMyProblemCounter}{\theMyProblemCounter\stepcounter{MyProblemCounter}}

%
% #1 : directions for all the problems
% #2 : vertical height of the problem boxes
% #3 : details for problem 1
% #4 : details for problem 2
%
\newenvironment{myProblems2}[4]{%
    \noindent
    {\myHeadingFont Practice:}\hspace{0.5em}#1\nopagebreak%
    \begin{tcbraster}[%
        raster equal height,%
        raster columns=2,%
        raster column skip=0.5mm,%
        raster row skip=0.5mm,%
        raster every box/.style={%
            enhanced,%
            sharp corners,%
            colback=white,%
            coltitle=black, colbacktitle=black!10!white,%
            boxrule=0pt, borderline={0.5pt}{0pt}{black},%
            title={\texttt\useMyProblemCounter},%
            },%
        ]%
        \begin{tcolorbox}[attach boxed title to top left]
            #3
            \tcblower\vspace{#2}
        \end{tcolorbox}
        \begin{tcolorbox}[attach boxed title to top right]
            #4
            \tcblower
        \end{tcolorbox}%
}{%
    \end{tcbraster}
}

%
% #1 : directions for all the problems
% #2 : vertical height of the problem boxes
% #3 : details for problem 1
% #4 : details for problem 2
% #5 : details for problem 3
% #6 : details for problem 4
%
\newenvironment{myProblems4}[6]{%
    \noindent
    \textbf{\myHeadingFont Practice:}\hspace{0.5em}#1\nopagebreak%
    \begin{tcbraster}[%
        raster equal height,%
        raster columns=2,%
        raster column skip=0.5mm,%
        raster row skip=0.5mm,%
        raster every box/.style={%
            enhanced,%
            sharp corners,%
            colback=white,%
            coltitle=black, colbacktitle=black!10!white,%
            boxrule=0pt, borderline={0.5pt}{0pt}{black},%
            title={\texttt\useMyProblemCounter},%
            },%
        ]%
        \begin{tcolorbox}[attach boxed title to top left]
            #3
            \tcblower\vspace{#2}
        \end{tcolorbox}
        \begin{tcolorbox}[attach boxed title to top right]
            #4
            \tcblower
        \end{tcolorbox}%
        \begin{tcolorbox}[attach boxed title to bottom left]
            #5
            \tcblower
        \end{tcolorbox}%
        \begin{tcolorbox}[attach boxed title to bottom right]
            #6
            \tcblower
        \end{tcolorbox}%
}{%
    \end{tcbraster}
}