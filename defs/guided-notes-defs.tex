\copypagestyle{myPagestyle}{empty}


\newcommand{\myFooterSize}{\footnotesize}
\makeoddfoot{myPagestyle}{\myFooterSize\myClassName}{\myFooterSize\thepage\,of\,\pageref*{xwmlastpage}}{\myFooterSize\myUnitNumber.\myLessonNumber\,\,\myLessonTitle}
\makeevenfoot{myPagestyle}{\myFooterSize\myUnitNumber.\myLessonNumber\,\,\myLessonTitle}{\myFooterSize{\thepage{}~of~\pageref*{xwmlastpage}}}{\myFooterSize\myClassName}

%
% A command to change the appearance of the cognitive verb 
% in the objectives.
%
\newcommand{\myCognitiveVerb}[1]{\textcolor{blue}{\textbf{#1}}}

%
% Font styling commands (so I can change them in a single place)
%
\NewDocumentCommand{\myUnitLessonNumberFont}{}{\sffamily\bfseries\HUGE}
\NewDocumentCommand{\myUnitTitleFont}{}{\sffamily\large}
\NewDocumentCommand{\myLessonTitleFont}{}{\sffamily\bfseries\huge}
\NewDocumentCommand{\myHeadingFont}{}{\sffamily\bfseries\Large}


%
% #1 is the fill-in text
%
\NewDocumentCommand{\myFillInBlank}{m}{%
    \,%
    \gap[u]{#1}%
    \,%
}


% -----------------HEADER / OBJECTIVES -related stuff----------------
%
\newenvironment{myNotesHeader}{
    \begin{flushleft}
        {\myUnitLessonNumberFont{\myUnitNumber.\myLessonNumber}}
        \hfill\;\;
        \begin{minipage}[b]{0.75\textwidth}
            \begin{flushright}
                {\myUnitTitleFont{}Unit \myUnitNumber\,\,\,\myUnitTitle}\\ \vspace{0.75em}
                {\myLessonTitleFont\myLessonTitle}
            \end{flushright}
        \end{minipage}
        % TODO: This \hrule skims the bottom of the lowest character
        %       in the header. It's only really visible when the characters
        %       do not extend below the baseline. I probably need to redesign
        %       how this header works. For now, it's good enough. :-)
        \hrule
    \end{flushleft}
    \noindent{\myHeadingFont Objectives:}
    \begin{enumerate}[label=\arabic*)]
}{
    \end{enumerate}
}


% -----------------VOCABULARY -related stuff----------------
%

% commands to save/change/restore \arraystretch
\newcommand{\mySavedArrayStretch}{1}

% #1 new value of \arraystretch
\NewDocumentCommand{\mySaveAndChangeArrayStretch}{ m }{%
    \renewcommand{\mySavedArrayStretch}{\arraystretch}
    \renewcommand{\arraystretch}{#1}
}

\newcommand{\myRestoreArrayStretch}{%
    \renewcommand{\arraystretch}{\mySavedArrayStretch}
}

\newenvironment{myVocabulary}{
    {\noindent{\myHeadingFont Vocabulary:}}\vspace{1em}
    
    \mySaveAndChangeArrayStretch{1.25} 
    \begin{tabular}{ll}
        \toprule
            \emph{word} & \emph{meaning} \\ 
        \midrule
}{
    \bottomrule
    \end{tabular}
    \myRestoreArrayStretch
    \vspace{1em}
}

\newcommand{\myVocabularyWord}[2]{%
{\textcolor{blue}{\textbf{#1}}} & #2 \\
}


% -----------------LESSON / CONCEPT -related stuff----------------
%
\newenvironment{myLesson}{
    {\noindent{\myHeadingFont Lesson:}}\par
}{
    \vspace{1em}
}

% #1 width of the centered box
%
\NewDocumentEnvironment{myLessonBox}{ O{6in} }{
    \begin{center}
        \begin{tcolorbox}[width=#1,]
}{
        \end{tcolorbox}
    \end{center}
}


% #1 : the key concept (which appears as a tcolorbox title)
%
\NewDocumentEnvironment{myKeyConcepts}{ O{Key Concept:} }{
    \begin{samepage}
    \begin{tcolorbox}[
        title={Key Concept: #1}, fonttitle=\myHeadingFont,
        coltitle=black, 
        colbacktitle=black!25!yellow, 
        colframe=black!50!yellow,
        colback=white!70!yellow,
        boxrule=2pt, 
        ]
}{
    \end{tcolorbox}
    \end{samepage}
}

% -----------------EXAMPLE-related stuff----------------

% A counter to number the examples in the guided notes.
\newcounter{MyExampleCounter}
\setcounter{MyExampleCounter}{1}
\newcommand{\useMyExampleCounter}{\theMyExampleCounter\stepcounter{MyExampleCounter}}


% #1 Optional example number 
% #2 A statement of the example problem.
% #3 How much empty vertical space to leave for the example box.
%
\NewDocumentCommand{\myExample}{omm}{%
    \begin{tcolorbox}[
        enhanced,
        sharp corners, 
        colback=white,
        boxrule=0pt,
        borderline={0.5pt}{0pt}{black,dashed},
        ]
        {\myHeadingFont Example \IfValueTF{#1}{#1}{\useMyExampleCounter}:} #2
        \tcblower
        \vspace{#3}
    \end{tcolorbox}
}

% #1 optional star. It's presence means DO NOT CENTER
% #2 Optional example number 
% #3 A statement of the example problem.
%
\NewDocumentEnvironment{myExampleForTikzGraphs}{som}{%
    \begin{tcolorbox}[
        enhanced,
        sharp corners, 
        colback=white,
        boxrule=0pt,
        borderline={0.5pt}{0pt}{black,dashed},
        ]
        {\myHeadingFont Example \IfValueTF{#2}{#2}{\useMyExampleCounter}:} #3
        \tcblower
        \IfBooleanTF{#1}
            {}
            {\begin{center}}
}
% user would insert 
% \begin{tikzpicture}\begin{axis}...\end{axis}\end{tikzpicture}
{
        \IfBooleanTF{#1}
            {}
            {\end{center}}
    \end{tcolorbox}
}


% -----------------PROBLEM-related stuff------------------

% A counter to number the problems in the guided notes.
\newcounter{MyProblemCounter}
\setcounter{MyProblemCounter}{1}
\newcommand{\useMyProblemCounter}{\theMyProblemCounter\stepcounter{MyProblemCounter}}

% an environment for two adjacent problems
%
% #1 : directions for all the problems
% #2 : vertical height of the problem boxes
% #3 : details for problem 1
% #4 : details for problem 2
%
\newenvironment{myProblems2}[4]{%
    \noindent
    {\myHeadingFont Practice:}\hspace{0.5em}#1\nopagebreak%
    \begin{tcbraster}[%
        raster equal height,%
        raster columns=2,%
        raster column skip=0.5mm,%
        raster row skip=0.5mm,%
        raster every box/.style={%
            enhanced,%
            sharp corners,%
            colback=white,%
            coltitle=black, colbacktitle=black!10!white,%
            boxrule=0pt, borderline={0.5pt}{0pt}{black},%
            title={\texttt\useMyProblemCounter},%
            },%
        ]%
        \begin{tcolorbox}[attach boxed title to top left]
            #3
            \tcblower\vspace{#2}
        \end{tcolorbox}
        \begin{tcolorbox}[attach boxed title to top right]
            #4
            \tcblower
        \end{tcolorbox}%
}{%
    \end{tcbraster}
}

% an environment for 4 adjacent problems
%
% #1 : directions for all the problems
% #2 : vertical height of the problem boxes
% #3 : details for problem 1
% #4 : details for problem 2
% #5 : details for problem 3
% #6 : details for problem 4
%
\newenvironment{myProblems4}[6]{%
    \noindent
    \textbf{\myHeadingFont Practice:}\hspace{0.5em}#1\nopagebreak%
    \begin{tcbraster}[%
        raster equal height,%
        raster columns=2,%
        raster column skip=0.5mm,%
        raster row skip=0.5mm,%
        raster every box/.style={%
            enhanced,%
            sharp corners,%
            colback=white,%
            coltitle=black, colbacktitle=black!10!white,%
            boxrule=0pt, borderline={0.5pt}{0pt}{black},%
            title={\texttt\useMyProblemCounter},%
            },%
        ]%
        \begin{tcolorbox}[attach boxed title to top left]
            #3
            \tcblower\vspace{#2}
        \end{tcolorbox}
        \begin{tcolorbox}[attach boxed title to top right]
            #4
            \tcblower
        \end{tcolorbox}%
        \begin{tcolorbox}[attach boxed title to bottom left]
            #5
            \tcblower
        \end{tcolorbox}%
        \begin{tcolorbox}[attach boxed title to bottom right]
            #6
            \tcblower
        \end{tcolorbox}%
}{%
    \end{tcbraster}
}