\documentclass[letterpaper]{memoir}
% memoir commands to define the text block geometry
\setulmarginsandblock{1in}{*}{*}
\setlrmarginsandblock{1in}{*}{*}

\usepackage{xparse}
\usepackage{tcolorbox}
\usepackage{blindtext}
\usepackage{xwatermark}

\newcommand{\myHeadFootStyle}{\footnotesize\sffamily}
\copypagestyle{myPagestyle}{empty}
\makeoddhead{myPagestyle}
    {\myHeadFootStyle\,}
    {\myHeadFootStyle\,}
    {\myHeadFootStyle\chaptername\,\thechapter.\themyLessonCounter\,\,\myCurrentChapterTitle}
\makeevenhead{myPagestyle}
    {\myHeadFootStyle\,}
    {\myHeadFootStyle\,}
    {\myHeadFootStyle\chaptername\,\thechapter.\themyLessonCounter\,\,\myCurrentChapterTitle}
\makeoddfoot{myPagestyle}
    {\myHeadFootStyle\myCurrentBookTitle}
    {\myHeadFootStyle\thepage\,of\,\pageref*{xwmlastpage}}
    {\myHeadFootStyle\thechapter.\themyLessonCounter\,\,\myCurrentLessonTitle}
\makeevenfoot{myPagestyle}
    {\myHeadFootStyle\thechapter.\themyLessonCounter\,\,\myCurrentLessonTitle}
    {\myHeadFootStyle{\thepage{}~of~\pageref*{xwmlastpage}}}
    {\myHeadFootStyle\myCurrentBookTitle}

\begin{document}
\pagestyle{myPagestyle}
\checkandfixthelayout

%
% I am piggy-backing off of the standard memoir document
% divisions:
%
% book    -- is for the name of my class (eg, Alg2)
% part    -- is for the semester 
% chapter -- is for units
% section -- is for the lessons in a unit
%
% To do this, I need to change the default value of various
% division-related commands. That's what the following is for.

%----- CLASS ---------
\newcommand{\myCurrentBookTitle}{}
\let\myOldBook\book
\renewcommand{\book}[1]{
    \renewcommand{\myCurrentBookTitle}{#1}
    \myOldChapter{#1}
}
% I don't want roman numerals.
\makeatletter
\renewcommand*{\thebook}{\@arabic\c@book}
\makeatother
% I just want the title (not "Class 1 <title>")
\renewcommand{\printbookname}{}
\renewcommand{\booknamenum}{}
\renewcommand{\printbooknum}{}
%----- SEMESTER -------
% I don't want roman numerals.
\makeatletter
\renewcommand*{\thepart}{\@arabic\c@part}
\makeatother
%----- UNIT -----------
\renewcommand{\chaptername}{Unit}
\newcommand{\myCurrentChapterTitle}{}
\let\myOldChapter\chapter
\renewcommand{\chapter}[1]{
    \renewcommand{\myCurrentChapterTitle}{#1}
    \myOldChapter{#1}
}

%-------- LESSONS --------------
\newcounter{myLessonCounter}[chapter]
\newcommand{\myCurrentLessonTitle}{}

\newcommand{\myLesson}[1]{
    \clearpage
    \stepcounter{myLessonCounter}
    \renewcommand{\myCurrentLessonTitle}{#1}
    \noindent
    \begin{minipage}[b]{0.25\textwidth}
        {
            \HUGE\bfseries\sffamily
            \thechapter.\themyLessonCounter
        }
    \end{minipage}
    \begin{minipage}[b]{0.74\textwidth}
        \begin{flushright}
            {
                \hfill
                \Huge\bfseries\sffamily
                #1
            }
        \end{flushright}
    \end{minipage}
    \newline
    \rule{0.99\textwidth}{0.75pt}
    % \newline
}

%-------- ANNOTATIONS (in boxes) --------------
% font and styling commands for
% Objectives, Voculary, Key Concepts, etc...
\newcommand{\myAnnotationStyling}{\bfseries\large}

% #1 : name of the kind of annotation (Objectives, ...)
% #2 : title text to go with the annotation
\NewDocumentEnvironment{myAnnotate}{ m m }{
    \begin{tcolorbox}[
        colframe=black!15!yellow,
        colback=white,
        coltitle=black,
        fonttitle={\myAnnotationStyling},
        title={#1: },
        after title={\normalfont\itshape#2},
        ]
}{
    \end{tcolorbox}
}

% #1 : name of the kind of annotation 
% #2 : title 
\NewDocumentEnvironment{myTabularAnnotate}{ m m }{
    \begin{myAnnotate}{#1}{#2}
    \begin{tabular}{rl}
}{
    \end{tabular}
    \end{myAnnotate}
}
% #1 - column 1 text
% #2 - column 2 text
\NewDocumentCommand{\myRow}{mm}{{\bfseries\itshape #1}&#2\\}

% #1 : name of the kind of annotation 
% #2 : title 
\NewDocumentEnvironment{myListAnnotate}{ m m }{
    \begin{myAnnotate}{#1}{#2}
    \begin{enumerate}
}{
    \end{enumerate}
    \end{myAnnotate}
}
% #1 - column 1 text
% #2 - column 2 text
\NewDocumentCommand{\myItem}{mm}{{\item\bfseries\itshape #1} #2}



%----------- OBJECTIVES , VOCABULARY, CONCEPTS ---------------
\newenvironment{myObjectives}{
    \begin{myListAnnotate}{Objectives}{}
}{
    \end{myListAnnotate}
}
\newcommand{\myObjective}[2]{
    \myItem{#1}{#2}
}

\newenvironment{myVocabulary}{
    \begin{myTabularAnnotate}{Vocabulary}{}
}{
    \end{myTabularAnnotate}
}
\newcommand{\myDefinition}[2]{
    \myRow
{#1}{#2}
}

\NewDocumentEnvironment{myConcept}{m}{
    \begin{myAnnotate}{Key Concept}{#1}
}{
    \end{myAnnotate}
}







\book{Pre-AP Algebra 2}
\part{Fall Semester}
\chapter{Introduction to Relations and Functions}
\chapterprecis{\blindtext}

\myLesson{This is a lesson}

\begin{myObjectives}
    \myObjective{listen}{to the teacher}
    \myObjective{take}{notes in my notebook}
    \myObjective{see}{farther than anyone else}
\end{myObjectives}

\begin{myVocabulary}
    \myDefinition{vertical}{up and down}
    \myDefinition{horizontal}{left and right}
    \myDefinition{foo}{an arbitrary name}
\end{myVocabulary}

\begin{myConcept}{To do this\dots}
    then what you need to do is that.
\end{myConcept}
\begin{myConcept}{To do this\dots}
    Follow these steps.
    \begin{enumerate}
        \item foo
        \item bar
        \item baz
    \end{enumerate}
\end{myConcept}

\myLesson{This is another lesson}

\vspace{2em}
\begin{tabular}{ll}
    \toprule
    {\bfseries\itshape old division} & {\bfseries\itshape new division} \\
    \midrule
    book & class \\
    part & semester \\
    chapter & unit \\
    section & lesson \\
    subsection & concept \\
    \bottomrule
\end{tabular}
\vspace{2em}


\chapter{Another unit ggg}
\myLesson{This is another lesson g}

\blindtext


\end{document}