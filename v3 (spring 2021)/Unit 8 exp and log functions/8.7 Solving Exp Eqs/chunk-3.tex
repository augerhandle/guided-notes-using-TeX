\section{Solving using logarithms}

The inverse of an exponential is a logarithm.
We can \gap{liberate} the variable from the exponent by using this inverse.

Solve exponential equations by taking the \gap{logarithm} of both sides 
when the equation contains a single exponential term.

\begin{myConceptSteps}{~to solve exponential equations by rewriting\dots}
    \myStep{isolate}{Get the exponential term by itself.}
    \myStep{rewrite}{%
        Take the \gap{log} of both sides to {\bfseries\itshape liberate} the variable from the exponent.
        \begin{itemize}[nosep,]
            \item[$\circ$] Use $log_b$ where $b$ is the same base as the exponential.
        \end{itemize} 
    }
    \myStep{solve}{Solve for the variable (evaluating a logarithm if needed).}
\end{myConceptSteps}

\begin{my2Problems}[\normalsize]{4in}[Solve these exponential equations.]
    {
        $27 = e^{9x} + 6$
    }
    {
        $15 \cdot 10^{\frac{x}{4} + 2 = 1502}$
    }
\end{my2Problems}
