\section{Explicit and Recursive Formulas}

The terms in a sequence are often written like this:
$ a_1, a_2, a_3, a_4, \dots$
The integer subscript on each term is called the \gap{index}.
\begin{itemize}[nosep]
    \item The index of the \gap{first} term is always \gap{1}.
    \item The index of the \gap{second} term is always \gap{2}.
    \item \dots
    \item The index of the $n$th term is \gap{$n$}.
\end{itemize}

\begin{center}
\begin{tcolorbox}[width=4in]
    \centering
    {
        \large
        \begin{tabular}{r|l}
            $a_1$ & the first (initial) term \\
            $r$ & the common ratio \\
            $a_{n}$ & the current ($n$th) term \\
            $a_{n-1}$ & the previous term \\
        \end{tabular}
    }
\end{tcolorbox}
\end{center}


\begin{tcolorbox}
    \centering
    {
        \renewcommand{\arraystretch}{2}
        \begin{tabular}{|r|p{3in}|c|}
            \hline
            {\large\bfseries\itshape common ratio}
                & The ratio of any term to the previous one.
                & {\large $r = \frac{a_n}{a_{n-1}}$ }
                \\
            {\large\bfseries\itshape explicit formula}
                & Calculate a term from the {\bfseries\itshape first term} and {\bfseries\itshape common ratio}.
                & {$a_n = a_1 \cdot r^{\,n-1}$ }
                \\
            {\large\bfseries\itshape recursive formula}
                & Calculate a term from the {\bfseries\itshape previous term} and {\bfseries\itshape common ratio}.
                & 
                    $a_1 = $\underline{\phantom{99999}}
                    \quad
                    $a_n = r \cdot a_{n-1}$
                \\
            \hline
        \end{tabular}
    }
\end{tcolorbox}

\begin{myConceptSteps}{~To write explicit or recursive formulas
        for a geometric sequence from a word problem \dots
    }
    \myStep{initial}{%
        Find the initial term, \gap{$a_1$}, from the statement of the problem.
    }
    \myStep{ratio}{%
        Find the common ratio, \gap{$r$}, from the statement of the problem.
    }
    \myStep{formula}{%
        Write the formulas for \gap{$a_n$} according to the table above.
    }
\end{myConceptSteps}


\begin{myWideProblem}{3in}[%
        Write the explicit formula and the recursive formulas 
        for the sequence representing the value of the car in the $n$th year
        given the following information.
    ]
    {
        A car initially worth \$35,000 
        was worth \$29,750 in year two of ownership, 
        \$25,288 in year three,
        and \$21,495 in year four. 
    }
\end{myWideProblem}

\newpage
\begin{myWideProblem}{5in}[%
    For the following geometric sequence problem,
    \begin{enumerate}[nosep]
        \item Write the explicit formula.
        \item Write the recursive formula.
        \item Make a table of $n$ and $a_n$ values for $n=1\dots5$.
        \item Find the value of $a_n$ for $n=25$.
        \item Explain which formula you used for the last part, and why.
    \end{enumerate}
]
{
    A biochemist has an experiment studying the bacteria population. 
    At the beginning of the experiement there are six bacteria in the petrie dish.
    The number of bacteria double every hour.
    The biochemist counts the bacteria population every hour.
}
\end{myWideProblem}
