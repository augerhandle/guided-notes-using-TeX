\section{The Main Logarithm Functions}

There are infinitely many logarithmic functions, $f(x) = log_b(x)$, (one for each $b$).
We'll study three more closely.

\hfil
\fbox{
    \begin{minipage}{0.25\textwidth}
        \begin{center}
            $f(x) = log_2(x)$
            \begin{tikzpicture}[
                    scale=0.55,
                    xaxe style/.style = { thick, arrows={-{Straight Barb}}, label={}, },                 
                    yaxe style/.style = { thick, arrows={-{Straight Barb}}, label={}, },                 
                ]
                \scriptsize
                \tkzInit[
                    xmax=6, xmin=-6, xstep=2,
                    ymax=6, ymin=-6, ystep=2,
                ]
                \tkzGrid[
                    sub, subxstep=1, subystep=1,
                ]
                \tkzDrawXY[label={},color=black,]
                \tkzLabelX[orig=false,]
                % \tkzLabelY[orig=false,]
                \tkzFct[{-},solid,very thick,color=black,samples=50,domain =0.02:6]{log(\x)/log(2)}
                \tkzDefPointByFct(1)\tkzGetPoint{A}\tkzDrawPoint[size=4pt](A)
            \end{tikzpicture}
        \end{center}
        \begin{tabular}{rl}
            Asymptotes & \underline{\hspace{0.75in}}\\
            Intercepts & \underline{\hspace{0.75in}}\\
            Domain     & \underline{\hspace{0.75in}}\\
            Range      & \underline{\hspace{0.75in}}\\
        \end{tabular}
\end{minipage}
}
\hfil
\fbox{
    \begin{minipage}{0.25\textwidth}
        \begin{center}
            $f(x) = log(x)$ \quad($b=10$)
            \begin{tikzpicture}[
                    scale=0.55,
                    xaxe style/.style = { thick, arrows={-{Straight Barb}}, label={}, },                 
                    yaxe style/.style = { thick, arrows={-{Straight Barb}}, label={}, },                 
                ]
                \scriptsize
                \tkzInit[
                    xmax=6, xmin=-6, xstep=2,
                    ymax=6, ymin=-6, ystep=2,
                ]
                \tkzGrid[
                    sub, subxstep=1, subystep=1,
                ]
                \tkzDrawXY[label={},color=black,]
                \tkzLabelX[orig=false,]
                % \tkzLabelY[orig=false,]
                \tkzFct[{-},solid,very thick,color=black,samples=50,domain =0.000001:6]{log(\x)/log(10)}
                \tkzDefPointByFct(1)\tkzGetPoint{A}\tkzDrawPoint[size=4pt](A)
            \end{tikzpicture}
        \end{center}
        \begin{tabular}{rl}
            Asymptotes & \underline{\hspace{0.75in}}\\
            Intercepts & \underline{\hspace{0.75in}}\\
            Domain     & \underline{\hspace{0.75in}}\\
            Range      & \underline{\hspace{0.75in}}\\
        \end{tabular}
\end{minipage}
}
\hfil
\fbox{
    \begin{minipage}{0.25\textwidth}
        \begin{center}
            $f(x) = ln(x)$ \quad($b=e$)
            \begin{tikzpicture}[
                    scale=0.55,
                    xaxe style/.style = { thick, arrows={-{Straight Barb}}, label={}, },                 
                    yaxe style/.style = { thick, arrows={-{Straight Barb}}, label={}, },                 
                ]
                \scriptsize
                \tkzInit[
                    xmax=6, xmin=-6, xstep=2,
                    ymax=6, ymin=-6, ystep=2,
                ]
                \tkzGrid[
                    sub, subxstep=1, subystep=1,
                ]
                \tkzDrawXY[label={},color=black,]
                \tkzLabelX[orig=false,]
                % \tkzLabelY[orig=false,]
                \tkzFct[{-},solid,very thick,color=black,samples=50,domain =0.0002:6]{log(\x)}
                \tkzDefPointByFct(1)\tkzGetPoint{A}\tkzDrawPoint[size=4pt](A)
            \end{tikzpicture}
        \end{center}
        \begin{tabular}{rl}
            Asymptotes & \underline{\hspace{0.75in}}\\
            Intercepts & \underline{\hspace{0.75in}}\\
            Domain     & \underline{\hspace{0.75in}}\\
            Range      & \underline{\hspace{0.75in}}\\
        \end{tabular}
\end{minipage}
}
\hfil
