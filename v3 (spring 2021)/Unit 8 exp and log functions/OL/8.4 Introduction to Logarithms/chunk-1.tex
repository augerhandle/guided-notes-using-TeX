\section{Logarithmic Functions}

The exponential function looks like this:
{
    \large
    \[ y = b^x \]
}
Remember that you get the inverse of a function by \gap{swapping} $x$ and $y$. 
So the inverse can be written like this:
{
    \large
    \[ x = b^y \]
}

The inverse of an {\bfseries\itshape exponential function} is called 
a \gap{logarithm}.
Since they are inverses, they are \gap{mirror} images of each other:

{
    \begin{center}
        \begin{tikzpicture}[
                scale=0.65,
                xaxe style/.style = { thick, arrows={-{Straight Barb}}, label={}, },                 
                yaxe style/.style = { thick, arrows={-{Straight Barb}}, label={}, },                 
            ]
            \scriptsize
            \tkzInit[
                xmax=6, xmin=-6, xstep=2,
                ymax=6, ymin=-6, ystep=2,
            ]
            \tkzGrid[
                sub, subxstep=1, subystep=1,
            ]
            \tkzDrawXY[label={},color=black,]
            \tkzLabelX[orig=false,]
            % \tkzLabelY[orig=false,]
            \tkzFct[{-(},dashed,very thick,color=black,samples=50,domain =-6:6.5]{2**\x}
            \tkzDefPointByFct(0)\tkzGetPoint{A}\tkzDrawPoint[size=4pt](A)
            \tkzFct[{-(},solid,color=black,samples=50,domain =-6:6.5]{\x}
            \tkzFct[{-},solid,very thick,color=black,samples=50,domain =0.002:6]{log(\x)/log(2)}
            \tkzDefPointByFct(1)\tkzGetPoint{B}\tkzDrawPoint[size=4pt](B)
        \end{tikzpicture}
    \end{center}
}

