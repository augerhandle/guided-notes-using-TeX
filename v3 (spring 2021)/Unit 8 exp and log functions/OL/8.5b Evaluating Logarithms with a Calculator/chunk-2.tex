\section{Using a calculator for logs}

Calculators usually have two \gap{logarithm} buttons:
\begin{center}
    \large
    \renewcommand{\arraystretch}{2}
    \begin{tabular}{|c|l|l|}
        \hline
        {\ttfamily log} & \gap{common}  log & $log(x) \equiv log_{10}(x)$ \\ \hline
        {\ttfamily ln}  & \gap{natural} log & $ln(x) \equiv log_e(x)$ \\ \hline
        \end{tabular}
\end{center}

We can calculate the logarithm in any base using $log$ (or $ln$) on a calculator using
the {\scshape Change of Base Formula}:
\begin{myCenteredBox}[width=3in]
{
    \large
    \[
        log_b(x) = \frac{log_c(x)}{log_c(b)}
    \]
}
\end{myCenteredBox}

\begin{my2Problems}{1.5in}[%
        Evaluate these logarithms using the 
        {\scshape Change of Base Formula} and the {\ttfamily log} button on your calculator.
    ]
    {$log_2(10)$}
    {$log_{20}(450)$}
\end{my2Problems}
\begin{my2Problems}{1.5in}[%
        Evaluate these logarithms using the 
        {\scshape Change of Base Formula} and the {\ttfamily ln} button on your calculator.
    ]
    {$log_{0.9}(15.7)$}
    {$log_{16}{80000}$}
\end{my2Problems}