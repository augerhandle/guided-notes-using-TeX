\section{Solving using exponentials}

The inverse of an logarithm is a exponential.
We can \gap{free} the variable from the log by using this inverse.

Solve logarithmic equations by taking the \gap{exponential} of both sides 
when the equation contains a single logarithm.

\begin{myConceptSteps}{~to solve logarithmic equations by rewriting\dots}
    \myStep{isolate}{Get the log by itself.}
    \myStep{rewrite}{%
        Take the \gap{exponential} of both sides.
        \begin{itemize}[nosep,]
            \item[$\circ$] Use the same base as the log.
        \end{itemize} 
    }
    \myStep{solve}{Solve for the variable (evaluating a logarithm if needed).}
\end{myConceptSteps}

\begin{my2Problems}[\normalsize]{4in}[Solve these exponential equations.]
    {
        $log_5(6-x) + 1 = 3$
    }
    {
        $4\,ln(x) + 2 = 34$
    }
\end{my2Problems}
