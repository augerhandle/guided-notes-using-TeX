

\section{Composite Transformations}

A \gap{composite} transformation involves two or more transformations of the parent function.
The general forms of these transformations are:

{\centering
\begin{tcbraster}[
    raster before skip=0.25in, raster after skip=0.5in,
    raster left skip=1in, raster right skip=1in,
    raster equal height=rows,
    raster column skip=0.5in,
    raster columns=2,
    enhanced,
    size=small,
    ]
    \begin{tcolorbox}[width=3in]%
        \large\centering
        $g(x) = {\boldsymbol a} (x-{\boldsymbol h})^3 + {\boldsymbol k}  $
    \end{tcolorbox}
    \begin{tcolorbox}[width=3in]%
        \large\centering
        $   g(x) = {\boldsymbol a} \myRoot[3]{x-{\boldsymbol h}} + {\boldsymbol k}   $
    \end{tcolorbox}
\end{tcbraster}
}




\begin{taggedblock}{on-level}
    \begin{myProblem}{0.5in}[How many transformations are involved in $g(x)$?]
        {
        $g(x) = 5 \myRoot[3]{x-2} + 7$
        }
    \end{myProblem}

    \begin{my2Problems}{1in}[Find the values of $a$, $h$, and $k$ in $g(x)$.]
        {
        $g(x) =  4(x-5)^3 + 7$
        }
        {
        $g(x) = -2 \myRoot[3]{x+6} -8$
        }
    \end{my2Problems}    
\end{taggedblock}




\begin{taggedblock}{pre-AP}
    \begin{myProblem}[\large]{2in}[How many transformations are involved in $g(x)$? Find $a$, $h$, and $k$.]{
        $g(x) = -\frac{1}{2} \myRoot[3]{x+\frac{1}{4}} + \pi$
        }
    \end{myProblem}
\end{taggedblock}

\newpage

