\begin{myConceptSteps}{To solve square root equations\dots}
    \myStep{isolate}{%
        Move the radical to \gap{one side} of the equation.
        \begin{itemize}[nosep]
            \item[$\circ$] If there are two radicals, put them on \gap{opposite} 
            sides of the equation.
        \end{itemize}
        }
    \myStep{square}{%
        Square \gap{both sides} of the equation. 
        \begin{itemize}[nosep]
            \item[$\circ$] This will \tagged{pre-AP}{\gap{usually}}
            make the radicals \gap{go away}.
        \end{itemize}
        }
    \myStep{solve}{%
        Solve the equation.
    %     \begin{taggedblock}{pre-AP}
    %         \begin{itemize}[nosep]
    %             \item[$\circ$] You might have to solve a \gap{quadratic} equation.
    %             \item[$\circ$] If the equation \gap{still} has a radical, go to step \gap{1}.
    %         \end{itemize}
        % \end{taggedblock}
        }
    \myStep{check}{%
        Look for extraneous solutions.
        \begin{itemize}[nosep]
            \item[$\circ$] Anything that does not check, 
            is \gap{extraneous}. 
            {\bfseries\itshape Reject} it.
        \end{itemize}
        }
\end{myConceptSteps}

\begin{myCenteredBox}[width=4.5in, ]
\large
When solving square root equations,
some of the answers you get might be \gap{wrong}!
\end{myCenteredBox}
%
% put cartoon figure here saying
% Why?...because squaring both sides can mess with minus signs...
%
\vfill
%
So you must always \gap{check} for \gap{extraneous} solutions.