\section{Single Transformations}
Last time, 
we talked about seven kinds of transformations.
\begin{myCenteredBox}[
    colback=white,
    title={\large seven kinds of transformations},
    colbacktitle={black!10!white},
    coltitle=black,
    ]
\begin{center}
    % \small
    \renewcommand{\arraystretch}{1.2}
    \begin{tabular}{r|l||r|l}
        {\bfseries\itshape transformation} 
            & {\bfseries\itshape formula} 
            & {\bfseries\itshape example} 
            & {\bfseries\itshape description}\\
        \midrule
        {\itshape shift right/left}          
            & $g(x) = \myRoot{x-{\boldsymbol h}} $  & $g(x) = \myRoot{x-1}$ & (${\boldsymbol h}=1$) shift right by \gap{1}\\
        {}           
            &                         & $g(x) = \myRoot{x+2}$ & (${\boldsymbol h}=-2$) shift left by \gap{2}\\ 
        \midrule
        {\itshape shift up/down}             
            & $g(x) = \myRoot{x} + {\boldsymbol k}$ & $g(x) = \myRoot{x}+3$ & (${\boldsymbol k}=3$) shift up by \gap{3}\\
        {}           
            &                         & $g(x) = \myRoot{x}-4$ & (${\boldsymbol k}=-4$) shift down by \gap{4}\\
        \midrule
        {\itshape vertical stretch/compress}     
            & $g(x) = {\boldsymbol a} \myRoot{x} $  & $g(x) = 5\myRoot{x}$  & (${\boldsymbol a}=5$) stretch by a factor of \gap{5}\\
        {} 
            &                         & $g(x) = \frac{1}{6}\myRoot{x}$ & (${\boldsymbol a}=\frac{1}{6}$) compress by a factor of \gap{$\frac{1}{6}$}\\
        \midrule
        {\itshape reflection}           
            & $g(x) = {\boldsymbol a} \myRoot{x} $  & $g(x) = -\myRoot{x-1}$ & (${\boldsymbol a}=-1$) reflect across the $x$-axis\\
    \end{tabular}
\end{center}
\end{myCenteredBox}

\section{Composite Transformations}

A \gap{composite} transformation involves two or more transformations of the parent function.

\begin{taggedblock}{on-level}
    \begin{myProblem}[\normalsize]{0.5in}[How many transformations are involved in $g(x)$?]{
        $g(x) = 3 \myRoot{x} + 7$
        }
    \end{myProblem}
\end{taggedblock}
\begin{taggedblock}{pre-AP}
    \begin{myProblem}[\normalsize]{0.5in}[How many transformations are involved in $g(x)$?]{
        $g(x) = -3 \myRoot{x+4}$
        }
    \end{myProblem}
\end{taggedblock}

\vspace{2em}
The general form of a composite transformation of a square root function is:

{\centering
\begin{tcolorbox}[width=3in]
    \large%
    \[   g(x) = {\boldsymbol a} \myRoot{x-{\boldsymbol h}} + {\boldsymbol k}   \]
\end{tcolorbox}
}

\begin{my2Problems}[\normalsize]{1in}[Find the values of $a$, $h$, and $k$ in $g(x)$.]
    {
    $g(x) =  \myRoot{x-5} + 7$
    }
    {
    $g(x) = -2 \myRoot{x+6} -8$
    }
\end{my2Problems}

