\section{Single Transformations}

Last time, 
we talked about seven kinds of transformations.

\begin{center}
\begin{tabular}{r||c|l}
    {\bfseries\itshape transformation} & {\bfseries\itshape example} & {\bfseries\itshape description}\\
    \midrule
    {\itshape shift right}          & $g(x) = \myRoot{x-1}$ & shift right by 1\\
    {\itshape shift left}           & $g(x) = \myRoot{x+2}$ & shift left by 2\\
    {\itshape shift up}             & $g(x) = \myRoot{x}+3$ & shift up by 3\\
    {\itshape shift down}           & $g(x) = \myRoot{x}-4$ & shift down by 4\\
    {\itshape vertical stretch}     & $g(x) = 5\myRoot{x}$  & stretch by a factor of 5\\
    {\itshape vertical compression} & $g(x) = \frac{1}{6}\myRoot{x-1}$ & compress by a factor of $\frac{1}{6}$\\
    {\itshape reflection}           & $g(x) = -\myRoot{x-1}$ & reflect across the $x$-axis\\
\end{tabular}
\end{center}

\section{Composite Transformations}

A \gap{composite} transformation involves two or more transformations of the parent function.

\begin{myProblem}[\normalsize]{0.5in}[What two transformations are involved in $g(x)$?]{
    $g(x) = 3 \myRoot{x} + 7$
    }
\end{myProblem}

The general form of a composite transformation of a square root function is:

{\centering
\begin{tcolorbox}[width=3in]
    \large%
    \[   g(x) = {\boldsymbol a} \myRoot{x-{\boldsymbol h}} + {\boldsymbol k}   \]
\end{tcolorbox}
}

\begin{my2Problems}[\normalsize]{0.5in}[Find the values of $a$, $h$, and $k$ in $g(x)$.]
    {
    $g(x) = 3 \myRoot{x-5} + 7$
    }
    {
    $g(x) = -2 \myRoot{x+6} -8$
    }
\end{my2Problems}

\begin{minipage}{0.49\linewidth}
    \begin{center}
        \begin{tcolorbox}[title={\centering\LARGE $h$},colback=white,coltitle=black,colbacktitle=black!10!white,]
            \renewcommand{\arraystretch}{1.25}
            \large
            \begin{tabular}{r|l}
                $h>0$   &   shift \gap{right} by $h$ \\
                $h<0$   &   shift \gap{left} by $|h|$ \\
            \end{tabular}
        \end{tcolorbox}
    \end{center}
\end{minipage}
%
\hfill
%
\begin{minipage}{0.49\linewidth}
    \begin{center}
        \begin{tcolorbox}[title={\centering\LARGE $k$},colback=white,coltitle=black,colbacktitle=black!10!white,]
            \renewcommand{\arraystretch}{1.25}
            \large
            \begin{tabular}{r|l}
                $k>0$   &   shift \gap{up} by $k$ \\ 
                $k<0$   &   shift \gap{down} by $|k|$ \\
            \end{tabular}
        \end{tcolorbox}
    \end{center}
\end{minipage}

\hfill
\begin{minipage}{0.75\linewidth}
    \begin{center}
        \begin{tcolorbox}[title={\centering\LARGE $a$},colback=white,coltitle=black,colbacktitle=black!10!white,]
            \renewcommand{\arraystretch}{1.25}
            \large
            \begin{tabular}{r|l}
                $a<0$       &   \gap{reflect} across the $x$-axis \\ 
                $|a|>1$     &    \gap{stretch} by a factor of $|a|$ \\
                $0<|a|<1$   &    \gap{compress} by a factor of $|a|$ \\
            \end{tabular}
        \end{tcolorbox}
    \end{center}
\end{minipage}
\hfill
