Square root functions
and quadratic functions are \gap{almost} inverses of each other.

\begin{myConceptSteps}{\,To find the inverse of a function $f(x)$\dots}
    \myStep{original domain/range}{%
        Write down the \gap{domain} and \gap{range} of $f(x)$.
        }
    \myStep{inverse domain/range}{Swap the domain and range.
        {
            \small
            \begin{itemize}[nosep,leftmargin=0.5em,]
                \item[$\circ$] The \gap{domain} of the original function
                becomes the \gap{range} of the inverse function.
                \item[$\circ$] The \gap{range} of the original function
                becomes the \gap{domain} of the inverse function.
            \end{itemize}
        }
        }
    \myStep{$f(x) \mapsto y$}{Replace $f(x)$ by $y$.}
    \myStep{swap}{Swap $x$ and $y$ in the equation.}
    \myStep{solve}{%
        Solve for $y$ in terms of $x$.
        \begin{tcolorbox}[width=6in,]
        You might need to use the \gap{domain} and \gap{range} of $f(x)$
        to pick the ``correct'' inverse function from two possibilities.
        For example, when there is a \gap{plus or minus} in your inverse function.
        \end{tcolorbox}
        }
    \myStep{$y \mapsto f^{-1}(x)$}{Replace $y$ by $f^{-1}(x)$.}
\end{myConceptSteps}

