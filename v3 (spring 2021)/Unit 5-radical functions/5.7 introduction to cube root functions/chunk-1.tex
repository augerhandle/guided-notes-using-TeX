\section{Function families}

A {\bfseries\itshape function family} is a set of functions that look \gap{alike}. 
Now you know {\bfseries\itshape three}.

\fbox{
    \begin{minipage}{0.3\textwidth}
        \begin{center}
            {\Large\bfseries\itshape Linear Function Family}
            \begin{tikzpicture}[
                    scale=0.75,
                    xaxe style/.style = { very thick, arrows={-{Straight Barb}}, label={}, },                 
                    yaxe style/.style = { very thick, arrows={-{Straight Barb}}, label={}, },                 
                ]
                \scriptsize
                \tkzInit[
                    xmax=6, xmin=-6, xstep=2,
                    ymax=6, ymin=-6, ystep=2,
                ]
                \tkzGrid[
                    sub, subxstep=1, subystep=1,
                ]
                \tkzDrawXY[label={},color=black,]
                \tkzLabelX[orig=false,]
                % \tkzLabelY[orig=false,]
            \end{tikzpicture}

            {\bfseries\itshape Parent function: \\$f(x) = x$}

            {\bfseries\itshape Transformed functions: \\$g(x) = mx + b$}
        \end{center}
    \end{minipage}
}
\hfill 
\fbox{
    \begin{minipage}{0.3\textwidth}
        \begin{center}
            {\Large\bfseries\itshape Quadratic Function Family}
            \begin{tikzpicture}[
                    scale=0.75,
                    xaxe style/.style = { very thick, arrows={-{Straight Barb}}, label={}, },                 
                    yaxe style/.style = { very thick, arrows={-{Straight Barb}}, label={}, },                 
                ]
                \scriptsize
                \tkzInit[
                    xmax=6, xmin=-6, xstep=2,
                    ymax=6, ymin=-6, ystep=2,
                ]
                \tkzGrid[
                    sub, subxstep=1, subystep=1,
                ]
                \tkzDrawXY[label={},color=black,]
                \tkzLabelX[orig=false,]
                % \tkzLabelY[orig=false,]
            \end{tikzpicture}

            {\bfseries\itshape Parent function: \\$f(x) = x^2$}

            {\bfseries\itshape Transformed functions: \\$g(x) = a(x-h)^2 + k$}
        \end{center}
    \end{minipage}
}
\hfill 
\fbox{
    \begin{minipage}{0.3\textwidth}
        \begin{center}
            {\Large\bfseries\itshape Square Root Function Family}
            \begin{tikzpicture}[
                    scale=0.75,
                    xaxe style/.style = { very thick, arrows={-{Straight Barb}}, label={}, },                 
                    yaxe style/.style = { very thick, arrows={-{Straight Barb}}, label={}, },                 
                ]
                \scriptsize
                \tkzInit[
                    xmax=6, xmin=-6, xstep=2,
                    ymax=6, ymin=-6, ystep=2,
                ]
                \tkzGrid[
                    sub, subxstep=1, subystep=1,
                ]
                \tkzDrawXY[label={},color=black,]
                \tkzLabelX[orig=false,]
                % \tkzLabelY[orig=false,]
            \end{tikzpicture}

            {\bfseries\itshape Parent function: \\$f(x) = \myRoot{x}$}

            {\bfseries\itshape Transformed functions: \\$g(x) = a\myRoot{x-h} + k$}
        \end{center}
    \end{minipage}
}

In the next few lessons, we are studing \gap{two} new function families: \gap{cubic} and \gap{cube root} functions.

\vspace{0.25\baselineskip}
These kinda look like \gap{narro} and \gap{twisted} versions of \gap{parabolas}.


