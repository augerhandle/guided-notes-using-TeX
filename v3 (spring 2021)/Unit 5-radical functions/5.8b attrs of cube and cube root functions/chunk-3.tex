For a general transformed cubic and cube root functions,
$g(x) = a \myRoot{x-h} + k$, 
the attributes 
are in this table.

\begin{taggedblock}{on-level}
    \begin{myCenteredBox}[width=4.5in,]
        \Large
        \begin{center}
        \begin{tabular}{r||l}
            {\bfseries\itshape inflection point}     
                & (${\boldsymbol h}$, ${\boldsymbol k}$)  \\
            {\bfseries\itshape domain}               
                & all real numbers  \\
            {\bfseries\itshape range}               
                & all real numbers  \\
        \end{tabular}
        \end{center}
    \end{myCenteredBox}
\end{taggedblock}

\begin{taggedblock}{pre-AP}
    \begin{myCenteredBox}[width=6.25in,]
        \Large
        \begin{center}
        \begin{tabular}{r||l}
            {\bfseries\itshape inflection point}     
                & (${\boldsymbol h}$, ${\boldsymbol k}$)  \\
            {\bfseries\itshape domain}               
                & all real numbers  \\
            {\bfseries\itshape range}               
                & all real numbers  \\
            \midrule
            {\bfseries\itshape right end behavior}   
                & if ${\boldsymbol a>0}$\quad  as $x \rightarrow +\infty$, $f(x) \rightarrow +\infty$ \\
            
                & if ${\boldsymbol a<0}$\quad  as $x \rightarrow +\infty$, $f(x) \rightarrow -\infty$ \\
            \midrule
            {\bfseries\itshape left end behavior}    
            & if ${\boldsymbol a>0}$\quad  as $x \rightarrow -\infty$, $f(x) \rightarrow -\infty$ \\
            & if ${\boldsymbol a<0}$\quad  as $x \rightarrow -\infty$, $f(x) \rightarrow +\infty$ \\
        \end{tabular}
        \end{center}
    \end{myCenteredBox}
\end{taggedblock}

This allows us to find the attributes \gap{without} graphing the function
as long as we know \gap{$a$} and \gap{$h$} and \gap{$k$}.

\begin{taggedblock}{on-level}
    \begin{my2Problems}{3.5in}[
        For the following transformed functions,
        \vspace{-0.75\baselineskip}
        \small
        \begin{itemize}[nosep]
            \item Sketch the graph of the parent function.
            \item Sketch the graph of the transformed function.
            \item Find the $(x,y)$ coordinates of the inflection point.
            \item Find the domain and range.
        \end{itemize}
        ]
        {
            $g(x) = \myRoot[3]{x-5} +2$
        }
        {
            $g(x) = -(x-2)^3 + 5$
        }
    \end{my2Problems}
\end{taggedblock}

\begin{taggedblock}{pre-AP}
    \begin{my2Problems}{2.75in}[
        For the following transformed functions,
        \vspace{-0.75\baselineskip}
        \small
        \begin{itemize}[nosep]
            \item Sketch the graph of the parent function.
            \item Sketch the graph of the transformed function.
            \item Find these attributes:
            \begin{itemize}[nosep]
            \item the $(x,y)$ coordinates of the inflection point
            \item the domain and range
            \item the left and right end behaviors.
            \end{itemize}
        \end{itemize}
        ]
        {
            $g(x) = 2\pi \myRoot[3]{x-5} +2$
        }
        {
            $g(x) = - \frac{23}{7} (x-2)^3 + 5$
        }
    \end{my2Problems}
\end{taggedblock}



