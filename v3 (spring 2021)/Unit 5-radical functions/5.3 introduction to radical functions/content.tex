% Do I want to print the text on top of the guided notes blanks?
\dashundergapssetup{teacher-mode=false,}

% ----------------------------------------------------------------
% Title, objectives, vocabulary
% ----------------------------------------------------------------
% My class "units" are based on the LaTeX \chapter macros. But we're not doing chapters
% in this one-lesson file. So we "fake out" the relevant 
% chapter (unit) number and title.
\renewcommand{\thechapter}{5} 
\renewcommand{\myCurrentChapterTitle}{Radical Functions}

\myLesson{Square Root Functions}[3]

\begin{myObjectives}
    \myObjective{sketch}{the graph of the \gap{square root} parent function }
    %
    \begin{taggedblock}{on-level}
        \myObjective{find}{the \gap{domain} and \gap{range} of the square root parent function}
    \end{taggedblock}
    \begin{taggedblock}{pre-AP}
        \myObjective{find}{\gap{attributes} of the square root parent function }
    \end{taggedblock}
    %
    \myObjective{find}{the \gap{domain} of a \gap{general} square root function }
\end{myObjectives}
%
\begin{myVocabulary}
    \myDefinition{function family}{a \gap{group} of functions that look alike}
    \myDefinition{parent function}{the \gap{simplest} function in a \gap{family}}
    \myDefinition{domain}{all the \gap{$x$} (input) values for a function}
    \myDefinition{range}{all the \gap{$y$} (output) values for a function}
    \myDefinition{algebraically}{using \gap{equations} (not graphs)}
\end{myVocabulary}

% ----------------------------------------------------------------
% Now the real content
% ----------------------------------------------------------------

\section{Factoring Cubic Polynomials by Grouping}


\begin{myCenteredBox}[width=4.75in,]
    Sometimes you can \gap{factor} cubic polynomials by \gap{grouping} 
    using this pattern.
    {\Large
    \begin{equation*}
        a(c+d) + b(c+d) = (a+b)(c+d)
    \end{equation*}
    }
\end{myCenteredBox}

\begin{myConceptSteps}{~to factor cubic polynomials by grouping \dots}
    \myStep{groups}{Group the first two terms together and the last two terms together.}
    \myStep{GCFs}{Factor out GCFs from each group.}
    \myStep{rewrite}{Write the result as two factored polynomials.}
\end{myConceptSteps}


\begin{taggedblock}{on-level}
    \begin{my2Problems}[\large]{3.5in}[
        Factor these cubic polynomials. Hint: Always remove a GCF first (if you can).
        ]
        {
            $ 4x^3 + x^2 + 12x + 3 $
        }
        {
            $ 9x^3 - 12x^2 - 27x + 36 $
        }
    \end{my2Problems}
    \begin{myProblem}[\large]{3.5in}
        {
            $ 2r^3 - r^2 - 8r + 4 $
        }
    \end{myProblem}
\end{taggedblock}

\begin{taggedblock}{pre-AP}
    \begin{my2Problems}[\large]{3.5in}[
        Factor these cubic polynomials. Hint: Always remove a GCF first (if you can).
        ]
        {
            $ 35x^3 + 49x^2 + 25x + 35 $
        }
        {
            $ 6p^4 -8p^3 -24p^2 +32p $
        }
    \end{my2Problems}
\end{taggedblock}

\begin{my2Problems}[\normalsize]{1in}[%
    Here are equations for a transformed square root function.
    For each problem,
    \vspace{-1em}
    \begin{itemize}[nosep]
        \item Find the value of $a$ or $h$ or $k$. (You decide which one!)
        \item Describe (in words) the transformation from the parent function, $f(x)=\myRoot{x}$.
    \end{itemize}
    ]
    {
        $g(x) = 5 \myRoot{x}$
    }
    {
        $g(x) = \myRoot{x} + 3$
    }
\end{my2Problems}
\begin{my2Problems}[\normalsize]{1in}
    {
        $g(x) = \myRoot{x-7}$
    }
    {
        $g(x) = \frac{1}{3}\myRoot{x}$
    }
\end{my2Problems}



\begin{my2Problems}[\normalsize]{1in}[%
    Here are descriptions of transformations of the square root parent function.
    For each problem, write the corresponding equation of $g(x)$.
    ]
    {
        shift down by 8
    }
    {
        reflect across the $x$-axis
    }
\end{my2Problems}
\begin{my2Problems}[\normalsize]{1in}
    {
        stretch vertically be a factor of 2
    }
    {
        shift left by 6
    }
\end{my2Problems}





\section*{Finding the maximum,  minimum, domain, and range}

\begin{myConceptSteps}{
    To find the maximum and minimum of a quadratic function in standard form\dots
    }
    \myStep{$\mathbf a$}{
        Find $\mathbf a$ from the standard form equation.
    }
    \myStep{positive $a$}{
        The parabola {\bfseries\itshape opens up} if $a$ is positive.
        In that case, the vertex is a {\bfseries\itshape minimum},
        and there is no maximum.
    }
    \myStep{negative $a$}{
        The parabola {\bfseries\itshape opens down} if $a$ is negative.
        In that case, there is no minimum, and 
        the vertex is a {\bfseries\itshape maximum}.
    }
\end{myConceptSteps}

\myBlankExample{2.5in}{
    Find the maximum and minimum of this quadratic function:
    $y = 2x^2 + 4x - 3$
}

\myBlankExample{2.5in}{
    Find the maximum and minimum of this quadratic function:
    $y = -2x^2 - 4x + 3$
}


\begin{myConceptSteps}{
    To find the domain and range of a quadratic function in standard form\dots
    }
    \myStep{domain}{
        The domain of any quadratic function is {\bfseries\itshape all real numbers}.
    }
    \myStep{vertex}{
        Find coordinates of the vertex, $(x_V, y_V)$.
    }
    \myStep{positive $a$}{
        The parabola {\bfseries\itshape opens up} if $a$ is positive.
        In that case, the vertex is a {\bfseries\itshape minimum},
        and so the range is $y \geq y_V$.
    }
    \myStep{negative $a$}{
        The parabola {\bfseries\itshape opens down} if $a$ is negative.
        In that case, the vertex is a {\bfseries\itshape maximum},
        and so the range is $y \leq y_V$.
    }
\end{myConceptSteps}

\myBlankExample{2.5in}{
    Find the domain and range of this quadratic function:
    $y = 2x^2 + 4x - 3$
}

\myBlankExample{2.5in}{
    Find the domain and range of this quadratic function:
    $y = -2x^2 - 4x + 3$
}

