% Do I want to print the text on top of the guided notes blanks?
\dashundergapssetup{teacher-mode=false,}

% ----------------------------------------------------------------
% Title, objectives, vocabulary
% ----------------------------------------------------------------
% My class "units" are based on the LaTeX \chapter macros. But we're not doing chapters
% in this one-lesson file. So we "fake out" the relevant 
% chapter (unit) number and title.
\renewcommand{\thechapter}{5} 
\renewcommand{\myCurrentChapterTitle}{Radical Functions}

\myLesson{Square Root Functions}[3]

\begin{myObjectives}
    \myObjective{sketch}{the graph of the \gap{square root} parent function }
    %
    \begin{taggedblock}{on-level}
        \myObjective{find}{the \gap{domain} and \gap{range} of the square root parent function}
    \end{taggedblock}
    \begin{taggedblock}{pre-AP}
        \myObjective{find}{\gap{attributes} of the square root parent function }
    \end{taggedblock}
    %
    \myObjective{find}{the \gap{domain} of a \gap{general} square root function }
\end{myObjectives}
%
\begin{myVocabulary}
    \myDefinition{function family}{a \gap{group} of functions that look alike}
    \myDefinition{parent function}{the \gap{simplest} function in a \gap{family}}
    \myDefinition{domain}{all the \gap{$x$} (input) values for a function}
    \myDefinition{range}{all the \gap{$y$} (output) values for a function}
    \myDefinition{algebraically}{using \gap{equations} (not graphs)}
\end{myVocabulary}

% ----------------------------------------------------------------
% Now the real content
% ----------------------------------------------------------------
\section*{The Quadratic Formula}

\begin{center}
\begin{tcolorbox}[width=5.5in]
    Given a quadratic equation in standard form 
    {\LARGE
        \[
            a x^2 + bx + c = 0
        \]
    }
    the solution for $x$ can be calculated from
    {\bfseries\itshape the Quadratic Formula}:
    {\LARGE
        \[
            x = \frac
                {-b \pm \sqrt{b^2 - 4ac}}
                {2a}
        \]
    }
\end{tcolorbox}
\end{center}


\section{Solving by rewriting into exponential form}

Solve exponential equations by \gap{rewriting} them into logarithmic form 
when the equation contains a single exponential term.

\begin{myConceptSteps}{~to solve log equations by rewriting\dots}
    \myStep{isolate}{Get the log term by itself.}
    \myStep{rewrite}{Convert to \gap{exponential} form to {\bfseries\itshape free} the variable from the exponent.}
    \myStep{solve}{Solve for the variable (evaluating a logarithm if needed).}
\end{myConceptSteps}

\begin{tcbraster}[
    raster equal height,
    raster columns=2,
    raster left skip = 1in, raster right skip=1in, raster column skip=1in,
]
\begin{tcolorbox}
    \centering\large
    {\bfseries\itshape exponential form:}

    $ x = b^y$
\end{tcolorbox}
\begin{tcolorbox}
    \centering\large
    {\bfseries\itshape logarthmic form:}

    $ y = log_b(x)$
\end{tcolorbox}
\end{tcbraster}


\begin{my2Problems}[\normalsize]{2in}[Solve these exponential equations.]
    {
        $log(3-2x) = 3$
    }
    {
        $2 \, log_3(2x-1) + 5 = 11$
    }
\end{my2Problems}

For a general transformed cubic and cube root functions,
$g(x) = a \myRoot{x-h} + k$, 
the attributes 
are in this table.

\begin{taggedblock}{on-level}
    \begin{myCenteredBox}[width=4.5in,]
        \Large
        \begin{center}
        \begin{tabular}{r||l}
            {\bfseries\itshape inflection point}     
                & (${\boldsymbol h}$, ${\boldsymbol k}$)  \\
            {\bfseries\itshape domain}               
                & all real numbers  \\
            {\bfseries\itshape range}               
                & all real numbers  \\
        \end{tabular}
        \end{center}
    \end{myCenteredBox}
\end{taggedblock}

\begin{taggedblock}{pre-AP}
    \begin{myCenteredBox}[width=6.25in,]
        \Large
        \begin{center}
        \begin{tabular}{r||l}
            {\bfseries\itshape inflection point}     
                & (${\boldsymbol h}$, ${\boldsymbol k}$)  \\
            {\bfseries\itshape domain}               
                & all real numbers  \\
            {\bfseries\itshape range}               
                & all real numbers  \\
            \midrule
            {\bfseries\itshape right end behavior}   
                & if ${\boldsymbol a>0}$\quad  as $x \rightarrow +\infty$, $f(x) \rightarrow +\infty$ \\
            
                & if ${\boldsymbol a<0}$\quad  as $x \rightarrow +\infty$, $f(x) \rightarrow -\infty$ \\
            \midrule
            {\bfseries\itshape left end behavior}    
            & if ${\boldsymbol a>0}$\quad  as $x \rightarrow -\infty$, $f(x) \rightarrow -\infty$ \\
            & if ${\boldsymbol a<0}$\quad  as $x \rightarrow -\infty$, $f(x) \rightarrow +\infty$ \\
        \end{tabular}
        \end{center}
    \end{myCenteredBox}
\end{taggedblock}

This allows us to find the attributes \gap{without} graphing the function
as long as we know \gap{$a$} and \gap{$h$} and \gap{$k$}.

\begin{taggedblock}{on-level}
    \begin{my2Problems}{3.5in}[
        For the following transformed functions,
        \vspace{-0.75\baselineskip}
        \small
        \begin{itemize}[nosep]
            \item Sketch the graph of the parent function.
            \item Sketch the graph of the transformed function.
            \item Find the $(x,y)$ coordinates of the inflection point.
            \item Find the domain and range.
        \end{itemize}
        ]
        {
            $g(x) = \myRoot[3]{x-5} +2$
        }
        {
            $g(x) = -(x-2)^3 + 5$
        }
    \end{my2Problems}
\end{taggedblock}

\begin{taggedblock}{pre-AP}
    \begin{my2Problems}{2.75in}[
        For the following transformed functions,
        \vspace{-0.75\baselineskip}
        \small
        \begin{itemize}[nosep]
            \item Sketch the graph of the parent function.
            \item Sketch the graph of the transformed function.
            \item Find these attributes:
            \begin{itemize}[nosep]
            \item the $(x,y)$ coordinates of the inflection point
            \item the domain and range
            \item the left and right end behaviors.
            \end{itemize}
        \end{itemize}
        ]
        {
            $g(x) = 2\pi \myRoot[3]{x-5} +2$
        }
        {
            $g(x) = - \frac{23}{7} (x-2)^3 + 5$
        }
    \end{my2Problems}
\end{taggedblock}




