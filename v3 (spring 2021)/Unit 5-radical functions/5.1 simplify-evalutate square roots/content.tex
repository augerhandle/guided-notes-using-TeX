% Do I want to print the text on top of the guided notes blanks?
\dashundergapssetup{teacher-mode=true,}

% ----------------------------------------------------------------
% Title, objectives, vocabulary
% ----------------------------------------------------------------
% My class "units" are based on the LaTeX \chapter macros. But we're not doing chapters
% in this one-lesson file. So we "fake out" the relevant 
% chapter (unit) number and title.
\renewcommand{\thechapter}{5} 
\renewcommand{\myCurrentChapterTitle}{Radical Functions}

\myLesson{Simplifying and Evaluating Radicals}[1]

\begin{myObjectives}
    \myObjective{simplify}{radical \gap{expressions} }
    \begin{taggedblock}{on-level}
        \myObjective{evaluate}{radical \gap{functions} with numbers as inputs }
    \end{taggedblock}
    \begin{taggedblock}{pre-AP}
        \myObjective{evaluate}{radical functions with \gap{numbers} as inputs }
        \myObjective{evaluate}{radical functions with \gap{expressions} as inputs }
    \end{taggedblock}
\end{myObjectives}
%
\begin{myVocabulary}
    % \myDefinition{term}{ a bunch of things \gap{multiplied} or \gap{divided} (no add/subtract)}
    \myDefinition{square root}{
        \raisebox{0.4em}{$\sqrt{\dots}$}
        which is the \gap{same as} 
        \raisebox{0.4em}{$\sqrt[\uproot{4}2]{\dots}$}
    }
    \myDefinition{cube root}{
        \raisebox{0.4em}{$\sqrt[\uproot{4}3]{\dots}$}
    }
    \myDefinition{radical}{\gap{square} roots, \gap{cube} roots, etc\dots}
    \myDefinition{radical symbol}{the thing that looks like a check: 
        \raisebox{0.4em}{$\sqrt{\hspace{1em}}$}
    }
    \myDefinition{radicand}{what's \gap{under} the radical symbol}
    \myDefinition{index}{
        Square root index $=$ \gap{$2$} because \raisebox{0.4em}{$\sqrt[\uproot{4}2]{\dots}$}.
        Cube root index $=$ \gap{$3$} because \raisebox{0.4em}{$\sqrt[\uproot{4}3]{\dots}$}.
    }
    \myDefinition{simplify}{rewrite something so it looks \gap{simpler}}
    \myDefinition{evaluate}{substitute inputs and calculate a \gap{result}}
\end{myVocabulary}

% ----------------------------------------------------------------
% Now the real content
% ----------------------------------------------------------------
\section*{The Quadratic Formula}

\begin{center}
\begin{tcolorbox}[width=5.5in]
    Given a quadratic equation in standard form 
    {\LARGE
        \[
            a x^2 + bx + c = 0
        \]
    }
    the solution for $x$ can be calculated from
    {\bfseries\itshape the Quadratic Formula}:
    {\LARGE
        \[
            x = \frac
                {-b \pm \sqrt{b^2 - 4ac}}
                {2a}
        \]
    }
\end{tcolorbox}
\end{center}


\section{Solving by rewriting into exponential form}

Solve exponential equations by \gap{rewriting} them into logarithmic form 
when the equation contains a single exponential term.

\begin{myConceptSteps}{~to solve log equations by rewriting\dots}
    \myStep{isolate}{Get the log term by itself.}
    \myStep{rewrite}{Convert to \gap{exponential} form to {\bfseries\itshape free} the variable from the exponent.}
    \myStep{solve}{Solve for the variable (evaluating a logarithm if needed).}
\end{myConceptSteps}

\begin{tcbraster}[
    raster equal height,
    raster columns=2,
    raster left skip = 1in, raster right skip=1in, raster column skip=1in,
]
\begin{tcolorbox}
    \centering\large
    {\bfseries\itshape exponential form:}

    $ x = b^y$
\end{tcolorbox}
\begin{tcolorbox}
    \centering\large
    {\bfseries\itshape logarthmic form:}

    $ y = log_b(x)$
\end{tcolorbox}
\end{tcbraster}


\begin{my2Problems}[\normalsize]{2in}[Solve these exponential equations.]
    {
        $log(3-2x) = 3$
    }
    {
        $2 \, log_3(2x-1) + 5 = 11$
    }
\end{my2Problems}

