\section*{Introduction}

\begin{minipage}{0.49\linewidth}
This unit is about \gap{radicals} which are also called
\end{minipage}
\hfill
\begin{minipage}{0.5\linewidth}
\begin{center}
    % \large
    \flushright
    \renewcommand{\arraystretch}{1.5}
    \begin{tabular}{|r|l|}
        \toprule
        \gap{square} roots & $\sqrt{25}$, $\sqrt{23}$, $\sqrt{8x^5}$ \\
        % \midrule
        \gap{cube} roots   & $\sqrt[\uproot{2}3]{27}$, $\sqrt[\uproot{2}3]{16}$, $\sqrt[\uproot{1}3]{27x^7}$ \\
        \bottomrule
    \end{tabular}
\end{center}
\end{minipage}

\begin{minipage}{0.6\linewidth}
    Radicals are the inverses of \gap{exponents}.
    \end{minipage}
    \hfill
    \begin{minipage}{0.39\linewidth}
    \begin{center}
        \large
        \centering
        \renewcommand{\arraystretch}{1.5}
        \begin{tabular}{|ccc|}
            \toprule
            $5^2 = 25$ & {\huge$\Leftrightarrow$} & $5 = \sqrt{25}  $ \\
            % \midrule
            $2^3 =  8$ & {\huge$\Leftrightarrow$} & $2 = \sqrt[\uproot{2}3]{8}$ \\
            \bottomrule
        \end{tabular}
    \end{center}
    \end{minipage}

    There are \gap{three} parts to a radical expression.\\
    \begin{minipage}{0.5\linewidth}
        \begin{enumerate}[itemsep=0pt]
            \item index
            \item radical symbol
            \item radicand
        \end{enumerate}
    \end{minipage}
    \hfill
    \begin{minipage}{0.24\linewidth}
    \begin{center}
        \huge
        \centering
            $\displaystyle \sqrt{8x^5\,\,}  $ \\
    \end{center}
    \end{minipage}
    \hfill
    \begin{minipage}{0.24\linewidth}
    \begin{center}
        \huge
        \centering
            $\displaystyle \sqrt[\uproot{4}\leftroot{-1}3]{16x^7+x^3\,\,}  $ \\
    \end{center}
    \end{minipage}