
% \begin{myCenteredBox}[width=7in]
    The same ideas apply even if there are variables. {\bfseries\itshape But }\dots

    % \vspace{\baselineskip}
    I need to be careful when I write the problems,
    because if the variables are allowed to have negative values,
    in some cases, the answer needs to have absolute values.

    % \vspace{\baselineskip}
    We won't worry about that, so I will try to remember to always say,
    \textit{the variable values are all positive}.
% \end{myCenteredBox}

\begin{taggedblock}{on-level}
    \begin{my2Problems}{1.75in}[Simplify these square root expressions.]
        {
            $\sqrt{25xy^2z^3}$ 
        }
        {
            $\sqrt{50x^2y^3z}$ 
        }
    \end{my2Problems}
\end{taggedblock}
\begin{taggedblock}{pre-AP}
    \begin{my2Problems}{1.75in}[Simplify these square root expressions.]
        {
            $\sqrt{50xy^2z^3}$ 
        }
        {
            $\sqrt{150x^2y^3z}$ 
        }
    \end{my2Problems}
\end{taggedblock}


\begin{taggedblock}{on-level}
    \begin{my2Problems}{1.75in}[Simplify these cube root expressions.]
        {
            $\sqrt[3]{8x^2y^6z^{10}}$ 
        }
        {
            $\sqrt[3]{24x^{10}y^4z^2}$ 
        }
    \end{my2Problems}
\end{taggedblock}
\begin{taggedblock}{pre-AP}
    \begin{my2Problems}{1.75in}[Simplify these cube root expressions.]
        {
            $\sqrt[3]{24x^7y^{18}z^{20}}$ 
        }
        {
            $\sqrt[3]{500x^{18}y^{20}z^7}$ 
        }
    \end{my2Problems}
\end{taggedblock}

% \begin{myConceptWithExamples2}{foo}{2.75in}
%     {
%         Here is the first concept that we need to look at.
%         It is a very important concept.
%         Make sure you understand it well.
%         And make sure you know how the example works.
%     }{ex}
%     {second concept}{3x2}
% \end{myConceptWithExamples2}
% %
