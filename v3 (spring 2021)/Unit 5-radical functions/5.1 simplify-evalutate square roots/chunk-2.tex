\section*{Simplifying radical terms}

A {\bfseries\itshape term} is a bunch of things 
\gap{multiplied} or \gap{divided} 
(no adding/subtracting).

We will simplify \gap{square roots} and \gap{cube roots} of terms like
\begin{center}
    \large
    \renewcommand{\arraystretch}{1.75}
    \begin{tabular}{ccccc}
        $\sqrt{25}$ & \hspace{1in} & $\sqrt{50}$ & \hspace{1in} & $\sqrt{-4}$\\
        $\sqrt{9x^2}$ & \hspace{1in} & $\sqrt{18a^3}$ & \hspace{1in} & $\sqrt[3]{8}$\\
        $\sqrt[3]{16}$ & \hspace{1in} & $\sqrt[3]{27z^3}$ & \hspace{1in} & $\sqrt[3]{16c^4}$\\
    \end{tabular}
\end{center}
        
\begin{myConceptSteps}{
    To simplify a square or cube root term (with numbers and variables)\dots
}
    \myStep{factor}{%
        Factor the {\textbf\itshape radicand} using a \gap{factor tree}
    }
    \myStep{group}{%
        If possible,
        collect the factors in groups with an exponent equal to the \gap{index}.
    }
    \myStep{escape}{%
        Each group with an exponent equal to the index \gap{escapes} from under the radical.
        Write it out front without the exponent.
    }
    \myStep{simplify}{%
        Simplify the terms in the radicand and out front.
    }
\end{myConceptSteps}

\begin{myProblems2}{Simplify these square roots.}{1.5in}
    {
        $\sqrt{180}$ % 180 = 10*18 = 2*5*2*9 = 5*2^2*3^2
    }
    {
        $\sqrt{50}$ % 50 = 5*10 = 5*2*5 = 2*5^2
    }
\end{myProblems2}

\begin{myProblems2}{Simplify these cube roots.}{2in}
    {
        $\sqrt[3]{500}$ % 500 = 4*125 = 4*5*5*5 = 4*5^3
    }
    {
        $\sqrt[3]{40}$ % 40 = 5*8 = 5*2*2*2 = 5*2^3
    }
\end{myProblems2}

\begin{myProblems4}{Simplify these radicals.}{2.5in}
    {
        $\sqrt{180}$ % 9*4*5=180
    }
    {
        $\sqrt{108}$ % 4*9*3=300
    }
    {
        $\sqrt{180}$ % 9*4*5=180
    }
    {
        $\sqrt{108}$ % 4*9*3=300
    }
\end{myProblems4}

\begin{myConceptWithExamples2}{foo}{2.75in}
    {
        Here is the first concept that we need to look at.
        It is a very important concept.
        Make sure you understand it well.
        And make sure you know how the example works.
    }{ex}
    {second concept}{3x2}
\end{myConceptWithExamples2}
%
