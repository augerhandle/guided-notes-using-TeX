\section*{Simplifying radical terms}

We will simplify \gap{square roots} and \gap{cube roots} of single terms like
\begin{center}
    \large
    \renewcommand{\arraystretch}{1.75}
    \begin{tabular}{ccccc}
        $\sqrt{25}$ & \hspace{1in} & $\sqrt{50}$ & \hspace{1in} & $\sqrt{-4}$\\
        $\sqrt{9x^2}$ & \hspace{1in} & $\sqrt{18a^3}$ & \hspace{1in} & $\sqrt[3]{8}$\\
        $\sqrt[3]{16}$ & \hspace{1in} & $\sqrt[3]{27x^2z^3}$ & \hspace{1in} & $\sqrt[3]{16a^2c^{14}x^3}$\\
    \end{tabular}
\end{center}



But before we get started, you will want to remember the \gap{product property} for exponents:
\begin{myCenteredBox}[width=2in]
    \LARGE
    $ b^n b^m = b^{n+m} $
\end{myCenteredBox}

\begin{myProblems2}{Simplify these cube roots.}{1in}
    {
        $4 \cdot 3^2 \cdot 6 \cdot 3^5$ 
    }
    {
        $4 \, z^3 \, x \, y^2 \, z^5$ 
    }
\end{myProblems2}
\vfill
\begin{myConceptSteps}{
    To simplify a square or cube root term (with numbers and variables)\dots
}
    \myStep{factor}{%
        Factor the {\textbf\itshape radicand} using a \gap{factor tree}
    }
    \myStep{group}{%
        If possible,
        collect the factors in groups with an exponent equal to the \gap{index}.
    }
    \myStep{escape}{%
        Each group with an exponent equal to the index \gap{escapes} from under the radical.
        Write it out front without the exponent.
    }
    \myStep{simplify}{%
        Simplify the terms in the radicand and out front.
    }
\end{myConceptSteps}

\begin{taggedblock}{on-level}
    \begin{myProblems2}{Simplify these square roots.}{0.5in}
        {
            $\sqrt{49}$ 
        }
        {
            $\sqrt{50}$ 
        }
    \end{myProblems2}
\end{taggedblock}
\begin{taggedblock}{pre-AP}
    \begin{myProblems2}{Simplify these square roots.}{0.5in}
        {
                $\sqrt{144}$ 
            }
            {
                $\sqrt{96}$ 
            }
    \end{myProblems2}
\end{taggedblock}

\begin{myProblems2}{Simplify these square roots, {\bfseries\itshape if you can}.}{0.5in}
    {
        $\sqrt{-49}$ 
    }
    {
        $\sqrt{-50}$ 
    }
\end{myProblems2}

\begin{myCenteredBox}[width=6in]
    If the index is \gap{even} and there is a \gap{negative} sign,
    then the radical is \gap{undefined}.
    (We aren't interested in complex numbers.)
\end{myCenteredBox}




\begin{taggedblock}{on-level}
    \begin{myProblems2}{Simplify these cube roots.}{1in}
        {
            $\sqrt[3]{125}$ 
        }
        {
            $\sqrt[3]{500}$ 
        }
    \end{myProblems2}
\end{taggedblock}
\begin{taggedblock}{pre-AP}
    \begin{myProblems2}{Simplify these cube roots.}{1in}
        {
            $\sqrt[3]{125}$ 
        }
        {
            $\sqrt[3]{1500}$ 
        }
    \end{myProblems2}
\end{taggedblock}
\begin{myProblems2}{Simplify these cube roots.}{1in}
    {
        $\sqrt[3]{-125}$ 
    }
    {
        $\sqrt[3]{-1500}$ 
    }
\end{myProblems2}

\begin{myCenteredBox}[width=6in]
    If the index is \gap{odd} and there is a \gap{negative} sign,
    then move the negative sign \gap{in front}.
\end{myCenteredBox}


