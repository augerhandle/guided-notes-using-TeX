\section{Transformations of reciprocal functions}

The reciprocal parent function, $f(x) = \frac{1}{x}$, can be \gap{transformed}.

\begin{myCenteredBox}[width=3in]
    {
        \Large
        \begin{center}
            $g(x) = 
            \frac{\boldsymbol a}{(x-{\boldsymbol h})} + {\boldsymbol k}$
        \end{center}
    }
\end{myCenteredBox}

\begin{myCenteredBox}[width=5in,]
\begin{center}
\renewcommand{\arraystretch}{1.5}
\begin{tabular}{r|l}
    {\itshape when}\dots & the {\itshape transformation} is \\
    \hline\hline
    $\boldsymbol h$ is \gap{positive} & shift {\bfseries\itshape right} by $\boldsymbol h$  \\
    $\boldsymbol h$ is \gap{negative} & shift {\bfseries\itshape left} by $|\boldsymbol h|$  \\
    \hline
    $\boldsymbol k$ is \gap{positive} & shift {\bfseries\itshape up} by $\boldsymbol k$  \\
    $\boldsymbol k$ is \gap{negative} & shift {\bfseries\itshape down} by $|\boldsymbol k|$  \\
    \hline
    $|\boldsymbol a|$ is \gap{$>1$} & {\bfseries\itshape vertical stretch} by $|a|$ \\
    $|\boldsymbol a|$ is \gap{$<1$} & {\bfseries\itshape vertical compression} by $|a|$ \\
    $\boldsymbol a$ is \gap{negative} & {\bfseries\itshape reflection} across the $x$-axis \\
    \end{tabular}
\end{center}
\end{myCenteredBox}

The \gap{asymptotes} will help you quickly sketch the graph. 