% Do I want to print the text on top of the guided notes blanks?
\dashundergapssetup{teacher-mode=false,}

% ----------------------------------------------------------------
% Title, objectives, vocabulary
% ----------------------------------------------------------------
% My class "units" are based on the LaTeX \chapter macros. But we're not doing chapters
% in this one-lesson file. So we "fake out" the relevant 
% chapter (unit) number and title.
\renewcommand{\thechapter}{7} 
\renewcommand{\myCurrentChapterTitle}{Rational Functions}


\begin{taggedblock}{on-level}
    \myLesson{The Reciprocal Function}[2][a]
\end{taggedblock}
\begin{taggedblock}{pre-AP}
    \myLesson{The Reciprocal Parent Function}[2][a]
\end{taggedblock}

\begin{taggedblock}{on-level}
    \begin{myObjectives}
        \myObjective{write}{the equation of the reciprocal {\bfseries\itshape parent} function}
        \myObjective{sketch}{the graph of the reciprocal {\bfseries\itshape parent} function}
        \myObjective{find}{attributes of the reciprocal {\bfseries\itshape parent} function}
        \myObjective{describe}{the effects of $a$/$h$/$k$ on a {\bfseries\itshape transformed} reciprocal function}
    \end{myObjectives}    
\end{taggedblock}

\begin{taggedblock}{pre-AP}
    \begin{myObjectives}
        \myObjective{write}{the equation of the reciprocal parent function}
        \myObjective{sketch}{the graph of the reciprocal parent function}
        \myObjective{find}{attributes of the reciprocal parent function}
        \myObjective{write}{domain and range using several notations}
    \end{myObjectives}
\end{taggedblock}

            
\begin{myVocabulary}
    \myDefinition{function family}{a bunch of functions that look \gap{alike}}
    \myDefinition{parent function}{the \gap{simplest} function in a function family}
    \myDefinition{attribute}{an interesting \gap{fact}}
    \myDefinition{reciprocal}{flip a \gap{fraction}, $x \Longrightarrow \frac{1}{x}$}
    \myDefinition{asymptote}{a \gap{line} that a curve gets very, very \gap{close to}}
    \myDefinition{branches}{disconnected \gap{parts} of a curve}
\end{myVocabulary}
            
% ----------------------------------------------------------------
% Now the real content
% ----------------------------------------------------------------
\section{Rational functions}

\begin{myConcept}{~Some basic facts about {\bfseries\itshape rational functions}\dots}
    \begin{itemize}[nosep]
        \item \gap{rational} functions are \gap{fractions} of polynomials.\\
        {
            \phantom{
                \huge
                \(
                    f(x) = 
                    \frac{2x^5 - 3x^4 8x^3 - 6x^2 + 5x - 23}{x-2}
                \)
            }
        }
        \item Graphs of rational functions often have multiple \gap{branches}.
        \begin{itemize}[nosep]
            \item[$\circ$] possibly \gap{many} 
        \end{itemize}
        \item Rational functions have horizontal and vertical \gap{asymptotes}.
        \item The simplest rational function is the \gap{reciprocal} function.
    \end{itemize}
\end{myConcept}

\section{Factoring Cubic Polynomials by Grouping}


\begin{myCenteredBox}[width=4.75in,]
    Sometimes you can \gap{factor} cubic polynomials by \gap{grouping} 
    using this pattern.
    {\Large
    \begin{equation*}
        a(c+d) + b(c+d) = (a+b)(c+d)
    \end{equation*}
    }
\end{myCenteredBox}

\begin{myConceptSteps}{~to factor cubic polynomials by grouping \dots}
    \myStep{groups}{Group the first two terms together and the last two terms together.}
    \myStep{GCFs}{Factor out GCFs from each group.}
    \myStep{rewrite}{Write the result as two factored polynomials.}
\end{myConceptSteps}


\begin{taggedblock}{on-level}
    \begin{my2Problems}[\large]{3.5in}[
        Factor these cubic polynomials. Hint: Always remove a GCF first (if you can).
        ]
        {
            $ 4x^3 + x^2 + 12x + 3 $
        }
        {
            $ 9x^3 - 12x^2 - 27x + 36 $
        }
    \end{my2Problems}
    \begin{myProblem}[\large]{3.5in}
        {
            $ 2r^3 - r^2 - 8r + 4 $
        }
    \end{myProblem}
\end{taggedblock}

\begin{taggedblock}{pre-AP}
    \begin{my2Problems}[\large]{3.5in}[
        Factor these cubic polynomials. Hint: Always remove a GCF first (if you can).
        ]
        {
            $ 35x^3 + 49x^2 + 25x + 35 $
        }
        {
            $ 6p^4 -8p^3 -24p^2 +32p $
        }
    \end{my2Problems}
\end{taggedblock}

\begin{my2Problems}[\normalsize]{1in}[%
    Here are equations for a transformed square root function.
    For each problem,
    \vspace{-1em}
    \begin{itemize}[nosep]
        \item Find the value of $a$ or $h$ or $k$. (You decide which one!)
        \item Describe (in words) the transformation from the parent function, $f(x)=\myRoot{x}$.
    \end{itemize}
    ]
    {
        $g(x) = 5 \myRoot{x}$
    }
    {
        $g(x) = \myRoot{x} + 3$
    }
\end{my2Problems}
\begin{my2Problems}[\normalsize]{1in}
    {
        $g(x) = \myRoot{x-7}$
    }
    {
        $g(x) = \frac{1}{3}\myRoot{x}$
    }
\end{my2Problems}



\begin{my2Problems}[\normalsize]{1in}[%
    Here are descriptions of transformations of the square root parent function.
    For each problem, write the corresponding equation of $g(x)$.
    ]
    {
        shift down by 8
    }
    {
        reflect across the $x$-axis
    }
\end{my2Problems}
\begin{my2Problems}[\normalsize]{1in}
    {
        stretch vertically be a factor of 2
    }
    {
        shift left by 6
    }
\end{my2Problems}





\section*{Finding the maximum,  minimum, domain, and range}

\begin{myConceptSteps}{
    To find the maximum and minimum of a quadratic function in standard form\dots
    }
    \myStep{$\mathbf a$}{
        Find $\mathbf a$ from the standard form equation.
    }
    \myStep{positive $a$}{
        The parabola {\bfseries\itshape opens up} if $a$ is positive.
        In that case, the vertex is a {\bfseries\itshape minimum},
        and there is no maximum.
    }
    \myStep{negative $a$}{
        The parabola {\bfseries\itshape opens down} if $a$ is negative.
        In that case, there is no minimum, and 
        the vertex is a {\bfseries\itshape maximum}.
    }
\end{myConceptSteps}

\myBlankExample{2.5in}{
    Find the maximum and minimum of this quadratic function:
    $y = 2x^2 + 4x - 3$
}

\myBlankExample{2.5in}{
    Find the maximum and minimum of this quadratic function:
    $y = -2x^2 - 4x + 3$
}


\begin{myConceptSteps}{
    To find the domain and range of a quadratic function in standard form\dots
    }
    \myStep{domain}{
        The domain of any quadratic function is {\bfseries\itshape all real numbers}.
    }
    \myStep{vertex}{
        Find coordinates of the vertex, $(x_V, y_V)$.
    }
    \myStep{positive $a$}{
        The parabola {\bfseries\itshape opens up} if $a$ is positive.
        In that case, the vertex is a {\bfseries\itshape minimum},
        and so the range is $y \geq y_V$.
    }
    \myStep{negative $a$}{
        The parabola {\bfseries\itshape opens down} if $a$ is negative.
        In that case, the vertex is a {\bfseries\itshape maximum},
        and so the range is $y \leq y_V$.
    }
\end{myConceptSteps}

\myBlankExample{2.5in}{
    Find the domain and range of this quadratic function:
    $y = 2x^2 + 4x - 3$
}

\myBlankExample{2.5in}{
    Find the domain and range of this quadratic function:
    $y = -2x^2 - 4x + 3$
}

\begin{taggedblock}{on-level}

\section{Transforming reciprocal functions}

There seven kinds of transformations.
\begin{myCenteredBox}[
    colback=white,
    title={\large seven kinds of transformations},
    colbacktitle={black!10!white},
    coltitle=black,
    ]
\begin{center}
    % \large
    \renewcommand{\arraystretch}{1.6}
    \begin{tabular}{r|l||r|l}
        {\bfseries\itshape transformation} 
            & {\bfseries\itshape formula} 
            & {\bfseries\itshape example} 
            & {\bfseries\itshape description}\\
        \hline
        {\itshape shift right/left}          
            & $g(x) = \frac{1}{x-{\boldsymbol h}} $  
            & $g(x) = \frac{1}{x-1}$ 
            & (${\boldsymbol h}=1$) shift right by \gap{1}\\
        {}           
            &                         
            & $g(x) = \frac{1}{x+2}$ 
            & (${\boldsymbol h}=-2$) shift left by \gap{2}\\ 
        \hline
        {\itshape shift up/down}             
            & $g(x) = \frac{1}{x} + {\boldsymbol k}$ 
            & $g(x) = \frac{1}{x}+3$ 
            & (${\boldsymbol k}=3$) shift up by \gap{3}\\
        {}           
            &                         
            & $g(x) = \frac{1}{x}-4$ 
            & (${\boldsymbol k}=-4$) shift down by \gap{4}\\
        \hline
        {\itshape vertical stretch/compress}     
            & $g(x) = \frac{\boldsymbol a}{x} $  
            & $g(x) = \frac{5}{x}$  
            & (${\boldsymbol a}=5$) stretch by a factor of \gap{5}\\
        {} 
            &                         
            & $g(x) = \frac{1/6}{x} = \frac{1}{6x}$ 
            & (${\boldsymbol a}=\frac{1}{6}$) compress by a factor of \gap{$\frac{1}{6}$}\\
        \hline
        {\itshape reflection}           
            & $g(x) = \frac{\boldsymbol a}{x} $  
            & $g(x) = \frac{-1}{x} = -\frac{1}{x}$ 
            & (${\boldsymbol a}=-1$) reflect across the $x$-axis\\
    \end{tabular}
\end{center}
\end{myCenteredBox}

% But they can all happen at the same time. In general\dots
% \begin{myCenteredBox}[width=5.25in]
%     The equation of a \gap{transformed} reciprocal function is
%     {
%         \Large
%         \begin{center}
%             $y = 
%             \frac{\mathbf a}{(x-{\mathbf h})} + {\mathbf k}$
%         \end{center}
%     }
% \end{myCenteredBox}



\begin{my2Problems}[\Large]{1.25in}[%
    \begin{itemize}[nosep]
        \item Find $a$, $h$, $k$ for these transformed reciprocal functions.
        \item Describe the transformations.
    \end{itemize}]
    {
        $g(x) = \frac{1}{x+3}$
    }
    {
        $g(x) = \frac{1}{x} - 8$
    }
\end{my2Problems}
\begin{my2Problems}[\Large]{1.25in}
    {
        $g(x) = \frac{2}{x-9}$
    }
    {
        $g(x)   = -\,\frac{1}{x} + 5$
    }
\end{my2Problems}


% \begin{myCenteredBox}[width=4.5in,]
%     When you sketch a \gap{shifted} graph, 
%     move the \gap{asymptotes}, too.
% \end{myCenteredBox}

% \begin{my2Problems}[\Large]{2.5in}[%
%     Find $a$, $h$, $k$ for these transformed reciprocal functions,
%     Quickly sketch 
%     \vspace{-0.5em}
%     \begin{itemize}[nosep]
%         \item the parent function, $f(x) = \frac{1}{x}$,
%         \item the transformed function, and
%         \item the asymptotes.
%     \end{itemize}
%     ]
%     {
%         $g(x) = \frac{1}{x - 2}$
%     }
%     {
%         $g(x) = \frac{1}{x} + 4$
%     }
% \end{my2Problems}
% \begin{my2Problems}[\Large]{2.5in}
%     {
%         $g(x) = \frac{3}{x}$
%     }
%     {
%         $g(x) = -\,\frac{1}{x}$
%     }
% \end{my2Problems}

\end{taggedblock}

