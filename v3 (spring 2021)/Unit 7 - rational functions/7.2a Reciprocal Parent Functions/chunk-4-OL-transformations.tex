\begin{taggedblock}{on-level}

\section{Transforming reciprocal functions}

There seven kinds of transformations.
\begin{myCenteredBox}[
    colback=white,
    title={\large seven kinds of transformations},
    colbacktitle={black!10!white},
    coltitle=black,
    ]
\begin{center}
    % \large
    \renewcommand{\arraystretch}{1.6}
    \begin{tabular}{r|l||r|l}
        {\bfseries\itshape transformation} 
            & {\bfseries\itshape formula} 
            & {\bfseries\itshape example} 
            & {\bfseries\itshape description}\\
        \hline
        {\itshape shift right/left}          
            & $g(x) = \frac{1}{x-{\boldsymbol h}} $  
            & $g(x) = \frac{1}{x-1}$ 
            & (${\boldsymbol h}=1$) shift right by \gap{1}\\
        {}           
            &                         
            & $g(x) = \frac{1}{x+2}$ 
            & (${\boldsymbol h}=-2$) shift left by \gap{2}\\ 
        \hline
        {\itshape shift up/down}             
            & $g(x) = \frac{1}{x} + {\boldsymbol k}$ 
            & $g(x) = \frac{1}{x}+3$ 
            & (${\boldsymbol k}=3$) shift up by \gap{3}\\
        {}           
            &                         
            & $g(x) = \frac{1}{x}-4$ 
            & (${\boldsymbol k}=-4$) shift down by \gap{4}\\
        \hline
        {\itshape vertical stretch/compress}     
            & $g(x) = \frac{\boldsymbol a}{x} $  
            & $g(x) = \frac{5}{x}$  
            & (${\boldsymbol a}=5$) stretch by a factor of \gap{5}\\
        {} 
            &                         
            & $g(x) = \frac{1/6}{x} = \frac{1}{6x}$ 
            & (${\boldsymbol a}=\frac{1}{6}$) compress by a factor of \gap{$\frac{1}{6}$}\\
        \hline
        {\itshape reflection}           
            & $g(x) = \frac{\boldsymbol a}{x} $  
            & $g(x) = \frac{-1}{x} = -\frac{1}{x}$ 
            & (${\boldsymbol a}=-1$) reflect across the $x$-axis\\
    \end{tabular}
\end{center}
\end{myCenteredBox}

% But they can all happen at the same time. In general\dots
% \begin{myCenteredBox}[width=5.25in]
%     The equation of a \gap{transformed} reciprocal function is
%     {
%         \Large
%         \begin{center}
%             $y = 
%             \frac{\mathbf a}{(x-{\mathbf h})} + {\mathbf k}$
%         \end{center}
%     }
% \end{myCenteredBox}



\begin{my2Problems}[\Large]{1.25in}[%
    \begin{itemize}[nosep]
        \item Find $a$, $h$, $k$ for these transformed reciprocal functions.
        \item Describe the transformations.
    \end{itemize}]
    {
        $g(x) = \frac{1}{x+3}$
    }
    {
        $g(x) = \frac{1}{x} - 8$
    }
\end{my2Problems}
\begin{my2Problems}[\Large]{1.25in}
    {
        $g(x) = \frac{2}{x-9}$
    }
    {
        $g(x)   = -\,\frac{1}{x} + 5$
    }
\end{my2Problems}


% \begin{myCenteredBox}[width=4.5in,]
%     When you sketch a \gap{shifted} graph, 
%     move the \gap{asymptotes}, too.
% \end{myCenteredBox}

% \begin{my2Problems}[\Large]{2.5in}[%
%     Find $a$, $h$, $k$ for these transformed reciprocal functions,
%     Quickly sketch 
%     \vspace{-0.5em}
%     \begin{itemize}[nosep]
%         \item the parent function, $f(x) = \frac{1}{x}$,
%         \item the transformed function, and
%         \item the asymptotes.
%     \end{itemize}
%     ]
%     {
%         $g(x) = \frac{1}{x - 2}$
%     }
%     {
%         $g(x) = \frac{1}{x} + 4$
%     }
% \end{my2Problems}
% \begin{my2Problems}[\Large]{2.5in}
%     {
%         $g(x) = \frac{3}{x}$
%     }
%     {
%         $g(x) = -\,\frac{1}{x}$
%     }
% \end{my2Problems}

\end{taggedblock}
