In the last lesson, we added rational expressions. 
We simplified expressions that looked like this.
\[
    \frac
    {(x-2)}
    {(x+3)}
    +
    \frac
    {3x}
    {(x+3)(x+5)}
\]

\begin{myCenteredBox}[width=6in,]
    {\bfseries\itshape Subtraction} 
    is mostly the same as addition.
    The key idea is to combine the fractions 
    over a common \gap{denominator} (LCD).\par
    \vspace{1em}
    But for subtraction, 
    you need to remember to \gap{distribute} the negative.
    This is easy to forget!
\end{myCenteredBox}

\begin{myWideProblem}{2in}[Subtract these rational expressions.]
    {
        $
        \frac{x}{(x+1))}
        -
        \frac{2}{(x+2)}
        $
    }
\end{myWideProblem}
\begin{myWideProblem}{4in}
    {
        $
        \frac{x}{x^2+2x-8}
        -
        \frac{1}{x+4}
        $
    }
\end{myWideProblem}
