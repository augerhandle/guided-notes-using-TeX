In the last lesson, we added rational expressions. 
We simplified expressions that looked like this.
\[
    \frac
    {(x-2)}
    {(x+3)}
    +
    \frac
    {3x}
    {(x+3)(x+5)}
\]

\begin{myCenteredBox}[width=7in,]
    {\bfseries\itshape Subtraction} 
    is mostly the same as addition.
    The key idea is to combine the fractions 
    over a common \gap{denominator} (LCD).\par
    \vspace{1em}
    But for subtraction, 
    you need to remember to \gap{distribute} the negative.
    This is easy to forget!
\end{myCenteredBox}

\begin{myConceptSteps}{%
    ~to subtract two rational numbers \gap{fractions}\dots}
    \myStep{factor}{%
        Factor the denominators.
    }
    \myStep{LCD}{%
        Find the {\bfseries\itshape least common denominator} \gap{LCD}
        of the two fractions.
    }
    \myStep{multiply}{%
        Multiply \gap{both} the numerator and denominator 
        of the fractions by the value needed to make their \gap{denominators}
        equal to the \gap{LCD}.
        % \begin{itemize}[nosep]
        %     \item[$\circ$]Use {\bfseries\itshape factor trees} to find the LCD.
        %     The denominator becomes the \gap{LCD}. 
        % \end{itemize}
    }
    \myStep{subtract}{%
        Subtract the \gap{numerators}.
        \begin{itemize}[nosep]
            \item[$\circ$]{\bfseries\itshape only} the numerators 
            \item[$\circ$]{\bfseries\itshape distribute} the negative
        \end{itemize}
    }
    \myStep{simplify}{Simplify the expressions as much as you can.}
\end{myConceptSteps}



\begin{myWideProblem}{6.5in}
    {
        $
        \frac{2x}{x^2+2x-8}
        -
        \frac{1}{2}
        $
    }
\end{myWideProblem}
