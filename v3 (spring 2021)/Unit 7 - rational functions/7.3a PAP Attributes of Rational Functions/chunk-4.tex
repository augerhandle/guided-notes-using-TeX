\section{Simplifying rational functions}

\begin{myCenteredBox}[width=4in,]
    Simplified rational functions are {\bfseries\itshape not}
    the same as the original unless you {\bfseries\itshape explicitly}
    include the domain restrictions that come from the \gap{holes}.
    \vskip1em
    Find the domain \gap{before} you simplify the function.
\end{myCenteredBox}

\begin{myConceptSteps}{~to simplify a rational function\dots}
    \myStep{domain}{%
        Find the \gap{domain} of the function. 
        (Do this {\bfseries\itshape before} you simplify the function. See above.)
    }
    \myStep{canceled factors}{%
        Simplify by {\bfseries\itshape factoring} and
        \gap{canceling} common factors.
        }
    \myStep{rewrite}{%
        Rewrite the function as the \gap{simplified} function, but make sure to write the domain.
    }
\end{myConceptSteps}


\begin{my2Problems}{3.25in}[Simplify $g(x)$, and write its domain restriction.]
    {
        $g(x) = \frac{x^2 - 64}{x+8}$
    }
    {
        $g(x) = \frac{x-2}{x^2 + 6x -16}$
    }
\end{my2Problems}


