\section{Domain of rational functions}

Remember 
that the \gap{domain} of a reciprocal function $g(x) = \frac{a}{(x-h)} + k$ is \gap{$x \neq h$}.
This was because you \gap{cannot} divide by zero.


\begin{taggedblock}{on-level}
    \begin{myWideProblem}{0.5in}[Find the domain of $g(x)$.]
        {
            $g(x) = \frac{2}{x-5} + 7$
        }
    \end{myWideProblem}
    \vspace{1em}
\end{taggedblock}
\begin{taggedblock}{pre-AP}
    \begin{myWideProblem}{0.4in}[Find the domain of $g(x)$.]
        {
            $g(x) = -\,\frac{8}{x-12} + 83$
        }
    \end{myWideProblem}
    \vspace{1em}
\end{taggedblock}



To find the domain of a \gap{rational} function, we must \gap{not allow} values of $x$ that result in a zero denominator.

\begin{myConceptSteps}{~to find the {\bfseries\itshape domain} of a rational function\dots}
    \myStep{factor}{%
        Rewrite the function by \gap{factoring} the numerator and denominator.
        \begin{itemize}[nosep]
            \item[$\circ$] \gap{Do not cancel} common factors!
        \end{itemize} 
        }
    \myStep{denominator roots}{%
        Find the \gap{zeros} of the denominator.
        \begin{itemize}[nosep]
            \item[$\circ$] Where does the denominator equal \gap{zero}?
        \end{itemize} 
    }
    \myStep{domain}{The domain \gap{excludes} all the zeros.}
\end{myConceptSteps}

\begin{my2Problems}{2in}[Find the domain of $g(x)$.]
    {
        $g(x) = \frac{(x-4)(x+4)}{x-3}$
    }
    {
        $g(x) = \frac{x-2}{x^2 - 5x + 6}$
    }
\end{my2Problems}
