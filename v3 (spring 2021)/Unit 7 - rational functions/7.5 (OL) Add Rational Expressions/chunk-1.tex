\section{Adding and subtracting rational numbers}

\begin{myConceptSteps}{%
    ~to find the LCD of two rational numbers \gap{fractions}\dots}
    \myStep{factor}{%
        Find the prime factors of the denominators using a \gap{factor tree}.
    }
    \myStep{collect}{%
        Collect the factors. 
        Each factor should be in the collection according top
        the most occurences of it in any denominator.
    }
    \myStep{multiply}{%
        The LCD is the \gap{product} of everything in your collection.
    }
\end{myConceptSteps}

\begin{my2Problems}{1.5in}[Find the LCD of these fractions.]
    {
        $
        \frac{2}{3}
        $,\quad
        $
        \frac{4}{7}
        $
    }
    {
        $
        \frac{1}{6}
        $,\quad
        $
        \frac{3}{8}
        $
    }
\end{my2Problems}


\begin{myConceptSteps}{%
    ~to add or subtract two rational numbers \gap{fractions}\dots}
    \myStep{LCD}{%
        Find the {\bfseries\itshape least common denominator} \gap{LCD}
        of the two fractions.
    }
    \myStep{multiply}{%
        Multiply \gap{both} the numerator and denominator 
        of the fractions by the value needed to make their \gap{denominators}
        equal to the \gap{LCD}.
        % \begin{itemize}[nosep]
        %     \item[$\circ$]Use {\bfseries\itshape factor trees} to find the LCD.
        %     The denominator becomes the \gap{LCD}. 
        % \end{itemize}
    }
    \myStep{add/subtract}{%
        Add (or subtract) the \gap{numerators}.
        \begin{itemize}[nosep]
            \item[$\circ$]{\bfseries\itshape Do not} add the denominators.
            The denominator becomes the \gap{LCD}. 
        \end{itemize}
    }
    \myStep{simplify}{Simplify the fraction if necessary.}
\end{myConceptSteps}


\begin{my2Problems}{1.5in}[Add/subtract these rational numbers.]
    {
        $
        \frac{2}{3}  -  \frac{4}{7}
        $
    }
    {
        $
        \frac{1}{6}  +  \frac{3}{8}
        $
    }
\end{my2Problems}
