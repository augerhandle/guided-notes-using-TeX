Sometimes you can divide polynomials by \gap{cancelling} common factors.

\begin{myConceptSteps}{~To factor polynomials by factoring \dots}
    \myStep{factor}{Factor the \gap{numerator} and \gap{denominator}.}
    \myStep{cancel}{Simplify by canceling \gap{common} factors above and below.}
\end{myConceptSteps}

If you can cancel \gap{all the factors} in the divisor,
then you have done the division by cancelling factors.

\begin{my2Problems}[\large]{1.5in}[
    Divide these polynomials by cancelling all the factors in the divisor.
    ]
    {
        $\frac{(2x-1)(x-5)}{(x-5)}$
    }
    {
        $\frac{(z+1)(2z^2 - 3)}{2z^2-3}$
    }
\end{my2Problems}

\begin{my2Problems}[\large]{1in}[
    Can you cancel {\bfseries\itshape all} the factors in the divisor of these problems?
    If so, what do you get? If not, why not?
    ]
    {
        $\frac{(2x+1)(2x+2)}{2x+3}$
    }
    {
        $\frac{(3x+1)(3x+2)}{3x-1}$
    }
\end{my2Problems}
\begin{my2Problems}[\large]{1in}
    {
        $\frac{(5x^2+1)(x+2)}{(5x+1)(x+2)}$
    }
    {
        $\frac{(5x^2+1)(7x^2+2)}{(5x^2+2)(7x^2+1)}$
    }
\end{my2Problems}


\begin{my2Problems}[\large]{3.25in}[
    Divide these polynomials by factoring first and then cancelling.
    ]
    {
        $\frac{x^2 -2x - 15}{(x-5)}$
    }
    {
        $\frac{6x^2 - 13x - 5}{3x-1}$
    }
\end{my2Problems}


