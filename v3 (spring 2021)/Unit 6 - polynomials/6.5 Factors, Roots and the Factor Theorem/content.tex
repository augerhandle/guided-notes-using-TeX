% Do I want to print the text on top of the guided notes blanks?
\dashundergapssetup{teacher-mode=false,}

% ----------------------------------------------------------------
% Title, objectives, vocabulary
% ----------------------------------------------------------------
% My class "units" are based on the LaTeX \chapter macros. But we're not doing chapters
% in this one-lesson file. So we "fake out" the relevant 
% chapter (unit) number and title.
\renewcommand{\thechapter}{6} 
\renewcommand{\myCurrentChapterTitle}{Polynomial Functions}


\myLesson{The Factor Theorem}[5]

\begin{myObjectives}
    \myObjective{understand}{%
        the relationship between the following
        \begin{itemize}[nosep]
            \item \gap{factors} of $f(x)$
            \item \gap{$x$-intercepts} or \gap{zeros} of $f(x)$
            \item \gap{roots} or \gap{solutions} of the equation $f(x)=0$
        \end{itemize}
    }
    \myObjective{determine}{%
        if $(x-a)$ is a factor of a polynomial using \gap{synthetic} division
        }
\end{myObjectives}
            
\begin{myVocabulary}
    \myDefinition{$x$-intercepts}{%
        The points where a graph of a function
        crosses the \gap{$x$-axis}.
    }
    \myDefinition{zeros}{%
        The \gap{$x$}-coordinates of the $x$-intercepts of a function.
    }
    \myDefinition{roots, solutions}{%
        The $x$ values that make $f(x)=0$.
    }
\end{myVocabulary}
            
% ----------------------------------------------------------------
% Now the real content
% ----------------------------------------------------------------

\section{Factoring Cubic Polynomials by Grouping}


\begin{myCenteredBox}[width=4.75in,]
    Sometimes you can \gap{factor} cubic polynomials by \gap{grouping} 
    using this pattern.
    {\Large
    \begin{equation*}
        a(c+d) + b(c+d) = (a+b)(c+d)
    \end{equation*}
    }
\end{myCenteredBox}

\begin{myConceptSteps}{~to factor cubic polynomials by grouping \dots}
    \myStep{groups}{Group the first two terms together and the last two terms together.}
    \myStep{GCFs}{Factor out GCFs from each group.}
    \myStep{rewrite}{Write the result as two factored polynomials.}
\end{myConceptSteps}


\begin{taggedblock}{on-level}
    \begin{my2Problems}[\large]{3.5in}[
        Factor these cubic polynomials. Hint: Always remove a GCF first (if you can).
        ]
        {
            $ 4x^3 + x^2 + 12x + 3 $
        }
        {
            $ 9x^3 - 12x^2 - 27x + 36 $
        }
    \end{my2Problems}
    \begin{myProblem}[\large]{3.5in}
        {
            $ 2r^3 - r^2 - 8r + 4 $
        }
    \end{myProblem}
\end{taggedblock}

\begin{taggedblock}{pre-AP}
    \begin{my2Problems}[\large]{3.5in}[
        Factor these cubic polynomials. Hint: Always remove a GCF first (if you can).
        ]
        {
            $ 35x^3 + 49x^2 + 25x + 35 $
        }
        {
            $ 6p^4 -8p^3 -24p^2 +32p $
        }
    \end{my2Problems}
\end{taggedblock}

\begin{my2Problems}[\normalsize]{1in}[%
    Here are equations for a transformed square root function.
    For each problem,
    \vspace{-1em}
    \begin{itemize}[nosep]
        \item Find the value of $a$ or $h$ or $k$. (You decide which one!)
        \item Describe (in words) the transformation from the parent function, $f(x)=\myRoot{x}$.
    \end{itemize}
    ]
    {
        $g(x) = 5 \myRoot{x}$
    }
    {
        $g(x) = \myRoot{x} + 3$
    }
\end{my2Problems}
\begin{my2Problems}[\normalsize]{1in}
    {
        $g(x) = \myRoot{x-7}$
    }
    {
        $g(x) = \frac{1}{3}\myRoot{x}$
    }
\end{my2Problems}



\begin{my2Problems}[\normalsize]{1in}[%
    Here are descriptions of transformations of the square root parent function.
    For each problem, write the corresponding equation of $g(x)$.
    ]
    {
        shift down by 8
    }
    {
        reflect across the $x$-axis
    }
\end{my2Problems}
\begin{my2Problems}[\normalsize]{1in}
    {
        stretch vertically be a factor of 2
    }
    {
        shift left by 6
    }
\end{my2Problems}





