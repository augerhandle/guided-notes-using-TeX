% Do I want to print the text on top of the guided notes blanks?
\dashundergapssetup{teacher-mode=false,}

% ----------------------------------------------------------------
% Title, objectives, vocabulary
% ----------------------------------------------------------------
% My class "units" are based on the LaTeX \chapter macros. But we're not doing chapters
% in this one-lesson file. So we "fake out" the relevant 
% chapter (unit) number and title.
\renewcommand{\thechapter}{6} 
\renewcommand{\myCurrentChapterTitle}{Polynomial Functions}


\myLesson{The Factor Theorem}[5]

\begin{myObjectives}
    \myObjective{understand}{%
        the relationship between the following
        \begin{itemize}[nosep]
            \item \gap{factors} of $f(x)$
            \item \gap{$x$-intercepts} or \gap{zeros} of $f(x)$
            \item \gap{roots} or \gap{solutions} of the equation $f(x)=0$
        \end{itemize}
    }
    \myObjective{determine}{%
        if $(x-a)$ is a factor of a polynomial using \gap{synthetic} division
        }
\end{myObjectives}
            
\begin{myVocabulary}
    \myDefinition{$x$-intercepts}{%
        The points where a graph of a function
        crosses the \gap{$x$-axis}.
    }
    \myDefinition{zeros}{%
        The \gap{$x$}-coordinates of the $x$-intercepts of a function.
    }
    \myDefinition{roots, solutions}{%
        The $x$ values that make $f(x)=0$.
    }
\end{myVocabulary}
            
% ----------------------------------------------------------------
% Now the real content
% ----------------------------------------------------------------
\section*{The Quadratic Formula}

\begin{center}
\begin{tcolorbox}[width=5.5in]
    Given a quadratic equation in standard form 
    {\LARGE
        \[
            a x^2 + bx + c = 0
        \]
    }
    the solution for $x$ can be calculated from
    {\bfseries\itshape the Quadratic Formula}:
    {\LARGE
        \[
            x = \frac
                {-b \pm \sqrt{b^2 - 4ac}}
                {2a}
        \]
    }
\end{tcolorbox}
\end{center}


\section{Solving by rewriting into exponential form}

Solve exponential equations by \gap{rewriting} them into logarithmic form 
when the equation contains a single exponential term.

\begin{myConceptSteps}{~to solve log equations by rewriting\dots}
    \myStep{isolate}{Get the log term by itself.}
    \myStep{rewrite}{Convert to \gap{exponential} form to {\bfseries\itshape free} the variable from the exponent.}
    \myStep{solve}{Solve for the variable (evaluating a logarithm if needed).}
\end{myConceptSteps}

\begin{tcbraster}[
    raster equal height,
    raster columns=2,
    raster left skip = 1in, raster right skip=1in, raster column skip=1in,
]
\begin{tcolorbox}
    \centering\large
    {\bfseries\itshape exponential form:}

    $ x = b^y$
\end{tcolorbox}
\begin{tcolorbox}
    \centering\large
    {\bfseries\itshape logarthmic form:}

    $ y = log_b(x)$
\end{tcolorbox}
\end{tcbraster}


\begin{my2Problems}[\normalsize]{2in}[Solve these exponential equations.]
    {
        $log(3-2x) = 3$
    }
    {
        $2 \, log_3(2x-1) + 5 = 11$
    }
\end{my2Problems}

