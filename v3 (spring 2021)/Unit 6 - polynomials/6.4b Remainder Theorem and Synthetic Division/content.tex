% Do I want to print the text on top of the guided notes blanks?
\dashundergapssetup{teacher-mode=false,}

% ----------------------------------------------------------------
% Title, objectives, vocabulary
% ----------------------------------------------------------------
% My class "units" are based on the LaTeX \chapter macros. But we're not doing chapters
% in this one-lesson file. So we "fake out" the relevant 
% chapter (unit) number and title.
\renewcommand{\thechapter}{6} 
\renewcommand{\myCurrentChapterTitle}{Polynomial Functions}


\myLesson{The Remainder Theorem and Synthetic Division}[4][b]

\begin{taggedblock}{on-level}
    \begin{myObjectives}
        \myObjective{evaluate}{polynomials using \gap{synthetic} division}
    \end{myObjectives}
    \end{taggedblock}
\begin{taggedblock}{pre-AP}
    \begin{myObjectives}
        \myObjective{derive}{the \gap{Remainder} Theorem given the \gap{Division} Algorithm}
        \myObjective{evaluate}{polynomials using \gap{synthetic} division}
    \end{myObjectives}
\end{taggedblock}
            
    
\begin{taggedblock}{on-level}
    \begin{myVocabulary}
        \myDefinition{synthetic}{imitation, made for special situations}
    \end{myVocabulary}
\end{taggedblock}
\begin{taggedblock}{pre-AP}
    \begin{myVocabulary}
        \myDefinition{synthetic}{imitation, made for special situations}
        \myDefinition{derive}{to start with one thing and show that another thing is true}
    \end{myVocabulary}
\end{taggedblock}
            
% ----------------------------------------------------------------
% Now the real content
% ----------------------------------------------------------------
\begin{taggedblock}{pre-AP}

\section*{Thinking about boundaries in one-dimension}

You now how to sketch the graphs of simple inequalities like this.

\begin{center}
\begin{minipage}{0.4\textwidth}
    \begin{itemize}[itemsep=0in]
        \item $x <    c$
        \item $x \leq c$
    \end{itemize}
\end{minipage}
\begin{minipage}{0.4\textwidth}
    \begin{itemize}[itemsep=0in]
        \item $x >    c$
        \item $x \geq c$
    \end{itemize}
\end{minipage}
\end{center}

\noindent
You do two things:
\begin{enumerate}
    \item Draw a circle at the point $x=c$.
    \item Shade to the left or right of the circle.
\end{enumerate}

\begin{center}
    \begin{tcolorbox}[width=6in]
        \begin{itemize}
            \item The point at $x=c$ in these problems is called the \gap{boundary}.
            \item The boundary might or might not be part of the solution.
                \begin{itemize}
                    \item[$\circ$] \gap{part} when the circle is \gap{filled-in}
                    \item[$\circ$] \gap{not part} when the circle is \gap{empty}
                \end{itemize}
            \item The boundary \gap{partitions} the universe (our number line) into two parts:
                \begin{itemize}
                    \item[$\circ$] a greater-than part (to the \gap{right} of the boundary)
                    \item[$\circ$] a less-than part (to the \gap{left} of the boundary)
                \end{itemize}
            \item We shade the part that satisfies the the inequality.
        \end{itemize}
    \end{tcolorbox}
\end{center}


\myBlankExample{2in}{
    Sketch the graph of this inequality on a number line:
    \[
        x > 3
    \]
}

\end{taggedblock}
\begin{taggedblock}{on-level}

\section*{Solving linear inequalities with two variables}

In Algebra 1, 
you solved inequalities that looked like this:

\begin{center}
\begin{minipage}{0.4\textwidth}
    \begin{itemize}[itemsep=0in]
        \item $y <    mx + b$
        \item $y \leq mx + b$
    \end{itemize}
\end{minipage}
\begin{minipage}{0.4\textwidth}
    \begin{itemize}[itemsep=0in]
        \item $y >    mx + b$
        \item $y \geq mx + b$
    \end{itemize}
\end{minipage}
\end{center}


\begin{myConceptSteps}{
    To solve a linear inequality with {\bfseries\itshape two variables}\dots
}
    \myStep{boundary}{
        Decide if the line $y = mx+b$ is 
        \gap{dashed} or \gap{solid}
        \begin{center}
            \fbox{\Large
            \begin{tabular}{l|c|l}
                {\bfseries\itshape dashed} & $>$, $<$       & no equals \\
                {\bfseries\itshape solid}  & $\geq$, $\leq$ & equals included \\
            \end{tabular}
            }
        \end{center}
        }
    \myStep{graph}{
        Sketch the graph of the \gap{boundary} as dashed or solid.
    }
    \myStep{shade}{
        Shade \gap{above} or \gap{below} the line.
        \begin{center}
            \fbox{\Large
            \begin{tabular}{l|c|l}
                {\bfseries\itshape above} & $>$, $\geq$ &greater... \\
                {\bfseries\itshape below} & $<$, $\leq$ &less... \\
            \end{tabular}
            }
        \end{center}
    }
\end{myConceptSteps}

\myBlankExample{2in}{
    Sketch the solution of this linear inequality:
    \[
        y < 2x - 2
    \]
}


\end{taggedblock}
            
\section{Solving by rewriting into exponential form}

Solve exponential equations by \gap{rewriting} them into logarithmic form 
when the equation contains a single exponential term.

\begin{myConceptSteps}{~to solve log equations by rewriting\dots}
    \myStep{isolate}{Get the log term by itself.}
    \myStep{rewrite}{Convert to \gap{exponential} form to {\bfseries\itshape free} the variable from the exponent.}
    \myStep{solve}{Solve for the variable (evaluating a logarithm if needed).}
\end{myConceptSteps}

\begin{tcbraster}[
    raster equal height,
    raster columns=2,
    raster left skip = 1in, raster right skip=1in, raster column skip=1in,
]
\begin{tcolorbox}
    \centering\large
    {\bfseries\itshape exponential form:}

    $ x = b^y$
\end{tcolorbox}
\begin{tcolorbox}
    \centering\large
    {\bfseries\itshape logarthmic form:}

    $ y = log_b(x)$
\end{tcolorbox}
\end{tcbraster}


\begin{my2Problems}[\normalsize]{2in}[Solve these exponential equations.]
    {
        $log(3-2x) = 3$
    }
    {
        $2 \, log_3(2x-1) + 5 = 11$
    }
\end{my2Problems}

