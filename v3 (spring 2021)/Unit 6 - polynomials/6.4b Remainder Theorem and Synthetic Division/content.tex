% Do I want to print the text on top of the guided notes blanks?
\dashundergapssetup{teacher-mode=false,}

% ----------------------------------------------------------------
% Title, objectives, vocabulary
% ----------------------------------------------------------------
% My class "units" are based on the LaTeX \chapter macros. But we're not doing chapters
% in this one-lesson file. So we "fake out" the relevant 
% chapter (unit) number and title.
\renewcommand{\thechapter}{6} 
\renewcommand{\myCurrentChapterTitle}{Polynomial Functions}


\myLesson{The Remainder Theorem and Synthetic Division}[4][b]

\begin{taggedblock}{on-level}
    \begin{myObjectives}
        \myObjective{evaluate}{polynomials using \gap{synthetic} division}
    \end{myObjectives}
    \end{taggedblock}
\begin{taggedblock}{pre-AP}
    \begin{myObjectives}
        \myObjective{derive}{the \gap{Remainder} Theorem given the \gap{Division} Algorithm}
        \myObjective{evaluate}{polynomials using \gap{synthetic} division}
    \end{myObjectives}
\end{taggedblock}
            
    
\begin{taggedblock}{on-level}
    \begin{myVocabulary}
        \myDefinition{synthetic}{imitation, made for special situations}
    \end{myVocabulary}
\end{taggedblock}
\begin{taggedblock}{pre-AP}
    \begin{myVocabulary}
        \myDefinition{synthetic}{imitation, made for special situations}
        \myDefinition{derive}{to start with one thing and show that another thing is true}
    \end{myVocabulary}
\end{taggedblock}
            
% ----------------------------------------------------------------
% Now the real content
% ----------------------------------------------------------------
\begin{taggedblock}{pre-AP}

\section*{Thinking about boundaries in one-dimension}

You now how to sketch the graphs of simple inequalities like this.

\begin{center}
\begin{minipage}{0.4\textwidth}
    \begin{itemize}[itemsep=0in]
        \item $x <    c$
        \item $x \leq c$
    \end{itemize}
\end{minipage}
\begin{minipage}{0.4\textwidth}
    \begin{itemize}[itemsep=0in]
        \item $x >    c$
        \item $x \geq c$
    \end{itemize}
\end{minipage}
\end{center}

\noindent
You do two things:
\begin{enumerate}
    \item Draw a circle at the point $x=c$.
    \item Shade to the left or right of the circle.
\end{enumerate}

\begin{center}
    \begin{tcolorbox}[width=6in]
        \begin{itemize}
            \item The point at $x=c$ in these problems is called the \gap{boundary}.
            \item The boundary might or might not be part of the solution.
                \begin{itemize}
                    \item[$\circ$] \gap{part} when the circle is \gap{filled-in}
                    \item[$\circ$] \gap{not part} when the circle is \gap{empty}
                \end{itemize}
            \item The boundary \gap{partitions} the universe (our number line) into two parts:
                \begin{itemize}
                    \item[$\circ$] a greater-than part (to the \gap{right} of the boundary)
                    \item[$\circ$] a less-than part (to the \gap{left} of the boundary)
                \end{itemize}
            \item We shade the part that satisfies the the inequality.
        \end{itemize}
    \end{tcolorbox}
\end{center}


\myBlankExample{2in}{
    Sketch the graph of this inequality on a number line:
    \[
        x > 3
    \]
}

\end{taggedblock}
\begin{taggedblock}{on-level}

\section{The Remainder Theorem}

For any polynomial $f(x)$,
the \gap{Remainder} Theorem says:
%
\begin{myCenteredBox}[width=4.75in]
    To evaluate $f(a)$, calculate the \gap{remainder} of 
    \large
    \begin{equation*}
        \frac{f(x)}{x-a}
    \end{equation*}
\end{myCenteredBox}

\end{taggedblock}
\begin{my2Problems}[\normalsize]{1in}[%
    Here are equations for a transformed square root function.
    For each problem,
    \vspace{-1em}
    \begin{itemize}[nosep]
        \item Find the value of $a$ or $h$ or $k$. (You decide which one!)
        \item Describe (in words) the transformation from the parent function, $f(x)=\myRoot{x}$.
    \end{itemize}
    ]
    {
        $g(x) = 5 \myRoot{x}$
    }
    {
        $g(x) = \myRoot{x} + 3$
    }
\end{my2Problems}
\begin{my2Problems}[\normalsize]{1in}
    {
        $g(x) = \myRoot{x-7}$
    }
    {
        $g(x) = \frac{1}{3}\myRoot{x}$
    }
\end{my2Problems}



\begin{my2Problems}[\normalsize]{1in}[%
    Here are descriptions of transformations of the square root parent function.
    For each problem, write the corresponding equation of $g(x)$.
    ]
    {
        shift down by 8
    }
    {
        reflect across the $x$-axis
    }
\end{my2Problems}
\begin{my2Problems}[\normalsize]{1in}
    {
        stretch vertically be a factor of 2
    }
    {
        shift left by 6
    }
\end{my2Problems}





