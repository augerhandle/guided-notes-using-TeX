\begin{taggedblock}{pre-AP}


\section{The Division Algorithm}

If $f(x)$ and $d(x)$ are two polynomials
(and $d(x) \neq 0$),
then there are two other polynomials 
$Q(x)$ and $R(x)$ 
so that
{\large
\begin{myCenteredBox}[width=2.5in]
    \begin{equation*}
        \frac{f(x)}{d(x)} = Q(x) + \frac{R(x)}{D(x)}
    \end{equation*}
\end{myCenteredBox}
}
\begin{minipage}{0.45\textwidth}
    \vspace{-1em}
    \begin{itemize}[nosep]
        \item $f(x)$ is the \gap{quotient}.
        \item $d(x)$ is the \gap{divisor}.
    \end{itemize}
\end{minipage}
\begin{minipage}{0.45\textwidth}
    \vspace{-1em}
    \begin{itemize}[nosep]
        \item $Q(x)$ is the \gap{quotient}.
        \item $R(x)$ is the \gap{remainder}.
    \end{itemize}
\end{minipage}


\begin{myWideProblem}{1.5in}[%
    Write the Division Algorithm for $f$ and $d$.
    (This is how we've been doing division of polynomials!)
    ]
    {
        $f(x) = x^3 + 2x^2 - 43x  -25 $, \qquad
        $d(x) = (x-6) $
    }
\end{myWideProblem}





\section{The Remainder Theorem}

We can rewrite the division algorithm if we \gap{multiply} 
by the \gap{divisor}:
\begin{align*}
    &\frac{f(x)}{d(x)}      = Q(x) + \frac{R(x)}{d(x)} \\
\end{align*}
\vspace{1in}

What happens if we let the \gap{divisor} be \gap{$(x-a)$}?
\vspace{1in}

And now
what happens if we let \gap{$x=a$}?
\vspace{1in}

We have just \gap{derived} the \gap{Remainder} Theorem:

\begin{myCenteredBox}[width=4.75in]
    To evaluate $f(a)$, calculate the \gap{remainder} of 
    \large
    \begin{equation*}
        \frac{f(x)}{x-a}
    \end{equation*}
\end{myCenteredBox}


\end{taggedblock}