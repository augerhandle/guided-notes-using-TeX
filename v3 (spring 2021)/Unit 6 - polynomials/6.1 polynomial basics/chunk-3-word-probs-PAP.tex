\begin{taggedblock}{pre-AP}



\section{Polynomial Word Problems}

To help \gap{organize} your thoughts it helps to solve hard/complex problems 
(e.g., word problems) using a little bit of \gap{structure}.

\begin{myConceptSteps}{~To solve complex word problems\dots}
    \myStep{GIVEN}{%
        Write down all the information you are given.
        (``What I know.'')
        }
    \myStep{TO DO}{%
        Rewrite the problem(s) you are solving.
        (``What I am supposed to do?'')
        }
    \myStep{SOLUTION}{%
        Solve the problem(s), and clearly show your solution(s).
        (``Here is how I solved the problem, and here is my answer.'')
        }
\end{myConceptSteps}

\begin{myCenteredBox}[width=4.25in,before skip=-1.5em,]
    If you can, you should draw a \gap{diagram}.
    This may be in the {\bfseries\itshape GIVEN} or the {\bfseries\itshape SOLUTION}.
\end{myCenteredBox}



\vspace{-2em}
\begin{myWideProblem}{4.5in}[Solve this word problem.]
        {
            Suppose that the two equal legs of a isoceles right triangle 
            have a length given by {\large $x+3$}.
            Write an expression for the area of the triangle.
            Write another expression for the length of the hypotenuse of the triangle.    
        }
\end{myWideProblem}


\end{taggedblock}
