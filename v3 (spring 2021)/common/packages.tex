\usepackage[T1]{fontenc}
\usepackage{xparse}
\usepackage{blindtext}
\usepackage{enumitem}
\usepackage{graphicx}
\usepackage{amsmath,mathtools,amssymb}
\usepackage{booktabs}
\usepackage{colortbl}
\usepackage{hhline}
\usepackage{systeme}
\usepackage{changepage}
\usepackage{multirow}
\usepackage{hyperref}

\usepackage{txfonts}
\normalfont % Just in case ...
\usepackage[T1]{fontenc} % To use T1 encoding fonts

\usepackage{tcolorbox}
    \tcbuselibrary{skins}
    \tcbuselibrary{raster}
    \tcbuselibrary{skins}
\usepackage{tikz}
    \usetikzlibrary{arrows.meta}
\usepackage{tkz-base}
\usepackage{tkz-fct}    
\usepackage{pgfplots}
    \pgfplotsset{compat=newest}

\usepackage{tagging}

% For leaving blanks in guided notes
\usepackage{dashundergaps} % for \gap
\dashundergapssetup{
    teacher-mode=true, % set to true to show answers 
    gap-format=underline,
    teacher-gap-format=dot,
    gap-font={\ECFAugie\MTversion{augie}\color{black}},
    gap-numbers=false,
    gap-widen=true,
    gap-extend-percent=150, % note: making this too big might create errors
    gap-number-format=\,\textsuperscript{\normalfont(\thegapnumber)},
}

\usepackage{emerald}

\usepackage[subdued]{mathastext}% no italic for Augie anyhow
    \MTDeclareVersion[n]{lmvtt}{T1}{lmvtt}{m}{n}
    \MTfamily{augie}
    \Mathastext[augie]
