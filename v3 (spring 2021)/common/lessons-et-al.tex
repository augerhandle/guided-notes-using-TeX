%-------- LESSONS --------------

\newcounter{myLessonCounter}[chapter] 
\newcommand{\myCurrentLessonTitle}{}
\newcommand{\myFullLessonNumber}{}  % for example 2.3b

% #1 - lesson title
% #2 - optional lesson number (default increments from previous)
% #3 - optional lesson number suffix (eg, 'a' or 'b')
%
\NewDocumentCommand{\myLesson}{ m o O{} }{
    \cleartorecto
    \renewcommand{\myCurrentLessonTitle}{#1}
    \IfValueTF{#2}
        {\setcounter{myLessonCounter}{#2}}
        {\stepcounter{myLessonCounter}}
    \renewcommand{\myFullLessonNumber}{\thechapter.\themyLessonCounter#3}
    \noindent
    \begin{minipage}[b]{0.25\textwidth}
        {
            \HUGE\bfseries\sffamily\myFullLessonNumber
        }
    \end{minipage}
    \begin{minipage}[b]{0.74\textwidth}
        \begin{flushright}
            {\sffamily\chaptername\,\thechapter\,\,\myCurrentChapterTitle}\\
            \vspace{0.75\baselineskip}
            {\Huge\bfseries\sffamily#1}
        \end{flushright}
    \end{minipage}
    
    \noindent\rule{\textwidth}{1.25pt}
    \par
}

%-------- ANNOTATIONS (in boxes) --------------
%
% I use the term "annotations" to capture the common
% infrastructure I use to define Objectives, Vobabulary & Concepts.
%

% font and styling commands for
% Objectives, Voculary, Key Concepts, etc...
\newcommand{\myAnnotationStyling}{\bfseries\large}

% #1 : name of the kind of annotation (Objectives, ...)
% #2 : title text to go with the annotation
% #3 : extra tcolorbox options
%
\NewDocumentEnvironment{myAnnotate}{ m m O{}}{
    \begin{tcolorbox}[
        colbacktitle=blue!10!white,
        colback=white,
        coltitle=black,
        fonttitle={\myAnnotationStyling},
        title={#1: },
        after title={\normalfont\itshape#2},
        #3,
        ]
}{
    \end{tcolorbox}
}

% #1 : name of the kind of annotation 
% #2 : title 
%
\NewDocumentEnvironment{myTabularAnnotate}{ m m }{
    \begin{myAnnotate}{#1}{#2}
    \begin{tabular}{r|l}
}{
    \end{tabular}
    \end{myAnnotate}
}
% #1 - column 1 text
% #2 - column 2 text
%
\NewDocumentCommand{\myRow}{mm}{{\bfseries\itshape \textcolor{blue}{#1}}&#2\\[1ex]}

% #1 : name of the kind of annotation 
% #2 : title 
% #3 : text before the list starts
%
\NewDocumentEnvironment{myListAnnotate}{ m m o }{
    \begin{myAnnotate}{#1}{#2}
    \IfValueT{#3}{#3}
    \begin{enumerate}[itemsep=0pt,]
}{
    \end{enumerate}
    \end{myAnnotate}
}
% #1 - column 1 text
% #2 - column 2 text
%
\NewDocumentCommand{\myItem}{mm}{\item{\bfseries\itshape \textcolor{blue}{#1}} #2}



%----------- OBJECTIVES , VOCABULARY, CONCEPTS ---------------
\newenvironment{myObjectives}{
    \begin{myListAnnotate}{Objectives}{}
}{
    \end{myListAnnotate}
}
\newcommand{\myObjective}[2]{
    \myItem{#1}{#2}
}

\newenvironment{myVocabulary}{
    \begin{myTabularAnnotate}{Vocabulary}{}
}{
    \end{myTabularAnnotate}
}
\newcommand{\myDefinition}[2]{
    \myRow{#1}{#2}
}

\NewDocumentEnvironment{myConcept}{m}{
    \begin{myAnnotate}{Key Concept}{\,#1}
}{
    \end{myAnnotate}
}

\NewDocumentEnvironment{myConceptSteps}{m}{
    \begin{myListAnnotate}{Key Concept}{#1}[Follow these steps.]
}{
    \end{myListAnnotate}
}
\newcommand{\myStep}[2]{
    \myItem{#1.}{#2}
}



\newcounter{myCweRowCounter}

% #1 width of column 1
% #2 text in column 1
% #3 text in column 2
\newcommand{\myCweRow}[3]{
    \begin{tcolorbox}[%
        equal height group = \arabic{myCweRowCounter},%
        toprule=0pt, leftrule=0pt,%
        rightrule=1pt, bottomrule=1pt,%
        title=row1col1, notitle,%
        sharp corners,%
        colback=white,%
        width=#1,%
        ]%
        #2%
    \end{tcolorbox}%
    \begin{tcolorbox}[%
        equal height group = \arabic{myCweRowCounter},%
        toprule=0pt,leftrule=0pt,rightrule=0pt,%
        bottomrule=1pt,%
        title=row1col1, notitle,%
        sharp corners,%
        colback=white,%
        ]%
        #3%
    \end{tcolorbox}%
    \stepcounter{myCweRowCounter}%
}

% #1 width of column 1
% #2 text in column 1
% #3 text in column 2
\newcommand{\myCweBottomRow}[3]{
    \begin{tcolorbox}[%
        equal height group = \arabic{myCweRowCounter},%
        bottomrule=0pt, leftrule=0pt, toprule=0pt,%
        rightrule=1pt,%
        title=row1col1, notitle,%
        sharp corners,%
        colback=white,%
        width=#1,%
        ]%
        #2%
    \end{tcolorbox}%
    \begin{tcolorbox}[%
        equal height group = \arabic{myCweRowCounter},%
        bottomrule=0pt, leftrule=0pt, toprule=0pt, rightrule=0pt,%
        title=row1col1, notitle,%
        sharp corners,%
        colback=white,%
        ]%
        #3%
    \end{tcolorbox}%
    \stepcounter{myCweRowCounter}%
}

\newcolumntype{z}[2]{>{\columncolor{#1}\raggedleft\arraybackslash}p{#2}}
% #1 title (eg, "Key Concepts")
% #2 description of the concepts
% #3 width of concept column
% #4 row 1 concept
% #5 row 1 example
% #6 row 2 concept
% #7 row 2 example
%
\NewDocumentEnvironment{myConceptWithExamples2}{ O{Key Concepts} mm mmmm}{
    \begin{myAnnotate}{\hspace{5mm}#1}{\,#2}[%
        left=0em,right=0em,%
        top=0em,bottom=0em,%
        boxsep=0em,%
        toptitle=1mm, bottomtitle=1mm, % since we set boxsep to zero
        ]%
    \noindent
    \begin{tabular}{>{\columncolor[gray]{0.95}}c!{\vrule width 1pt}c}
        \begin{tcolorbox}[%
            width=#3,
            leftrule=0em,rightrule=0em,toprule=0em,bottomrule=0em,
            colback=black!5, colframe=white,
            ]%
            #4
        \end{tcolorbox}
        &
        \begin{tcolorbox}[%
            width=\linewidth-#3-10mm,
            leftrule=0em,rightrule=0em,toprule=0em,bottomrule=0em,
            colback=white, colframe=white,
            ]%
            #5
        \end{tcolorbox}
        \\
        \hhline{=:=}
        \begin{tcolorbox}[%
            width=#3,
            leftrule=0em,rightrule=0em,toprule=0em,bottomrule=0em,
            colback=black!5, colframe=white,
            ]%
            #6
        \end{tcolorbox}
        &
        \begin{tcolorbox}[%
            width=\linewidth-#3-10mm,
            leftrule=0em,rightrule=0em,toprule=0em,bottomrule=0em,
            colback=white, colframe=white,
            ]%
            #7
        \end{tcolorbox}
        \\
    \end{tabular}
    }{
    \end{myAnnotate}
}
