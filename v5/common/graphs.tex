% ---------------------------------------------------------------------------
% number lines using Tkz
% ---------------------------------------------------------------------------

% a simple number line
%
% #1 scale of the numberline
% #2 max x
% #3 min x (optional, if absent it defaults to negative of the max)
% #4 label on the x-axis (optional, if absent there's no label)
%
\NewDocumentCommand{\myNumberLine}{ O{0.4} m o O{{}} }{
    \begin{tikzpicture}[
        scale=#1,
        xaxe style/.style = { very thick, arrows={-{Straight Barb}}, },                 
        ]
        \tkzInit[ xmax=#2, xmin=\IfValueTF{#3}{#3}{-#2}, xstep=1, ]
        \tkzDrawX[
            % right=12pt,
            label=#4,
            noticks=false, % Yes, I want ticks!
            tickup=1em, tickdn=1em,
            right space=0.05cm, left space=0.05cm, % extend the line beyond the last ticks
            ]
        % \tkzLabelX[ 
        %     % orig=false,
        %     step=#4,
        %     font=\small,
        %     below left=2pt,
        %  ]
    \end{tikzpicture}
}


% a numberline 
% (Must be inside a tikzpicture.)
%
% #1 lower number line limit 
% #2 upper number line limit
% #3 label font commands
%
\NewDocumentCommand{\myDrawNumberline}{ o m O{\ttfamily\tiny} }
{
        %
        % Draw the line.
        \draw (\IfValueTF{#1}{#1}{-#2},0) -- (#2,0);
        %
        % Draw tickmarks and labels.
        \foreach \i in {\IfValueTF{#1}{#1}{-#2},...,#2} {
            \draw (\i,0.1) -- + (0,-0.2); 
            \draw[node font=#3] (\i,-0.2) node[below]{\i};
        }
        %
}


% numberline labels
% (Must be inside a tikzpicture.)
%
% #1 list of x-coordinates for plain labels
% #3 plain label font commands
%
\NewDocumentCommand{\myDrawNumberlineLabels}{ m O{\ttfamily\tiny} }
{
    \foreach \i in {#1} {
        \draw[node font=#2,] (\i,-0.1) node[below] {\i}; 
        % \draw (-9,-5) node{x};
    }
}


% numberline point and label
% (Must be inside a tikzpicture.)
%
% #1 x-coordinates for the point
% #2 point circle fill color
%
\NewDocumentCommand{\myDrawNumberlineCircle}{ m m }
{
        \draw[black,thick,fill=#2] (#1,0) circle (0.175 cm);
}


% numberline point and label
% (Must be inside a tikzpicture.)
%
% #1 x-coordinates for the point
% #2 point circle fill color
%
\NewDocumentCommand{\myDrawNumberlineCircleAndLabel}{ m m }
{
        \draw[black,thick,fill=#2] (#1,0) circle (0.175 cm);
        \draw[node font=\sffamily\bfseries] (#1,-0.2) node[below]{#1};
}



% ---------------------------------------------------------------------------
% x-y graphs using Tkz
% ---------------------------------------------------------------------------

% a tikzpicture wrapper environment that uses Tkz to draw an x-y grid
%
% #1 tikzpicture scale
%
% #2 optional xmin (defaults to negative of xmax)
% #3 xmax
% #4 optional ymin (defaults to negative of ymax)
% #5 ymax
% #6 drawing style of x/y axes (solid, dashed, etc...)
%
\NewDocumentEnvironment{myTikzpictureGrid}{ m omom O{solid} }{
    \begin{tikzpicture}[
        scale=#1,
        xaxe style/.style = { #6, very thick, arrows={-{Straight Barb}}, },                 
        yaxe style/.style = { #6, very thick, arrows={-{Straight Barb}}, },                 
        ]
        \tkzInit[
            xmin={\IfValueTF{#2}{#2}{-#3}}, xmax=#3, xstep=1,
            ymin={\IfValueTF{#4}{#4}{-#5}}, ymax=#5, ystep=1,
            ]
        \tkzGrid[ sub, subxstep=1, subystep=1, ]
        \tkzDrawX[label={$x$},color=black, right=0.2em,]
        \tkzDrawY[label={$y$},color=black, above=0.2em,]
        % \tkzLabelX[orig=false,] \tkzLabelY[orig=false,]
}{
    \end{tikzpicture}
}


\NewDocumentCommand{\myDrawPointA}{ O{circle} O{4pt}} {
    \tkzDrawPoint[shape=#1, size=#2,](A)
}

% A command to draw a grid with dots at the grid points 
% with the objective of manually constructing slope fields.
%
% #1 width of resizebox wrapped around it all 
% #2 dy/dx equation
% #3 upper limits of x & y (symmetric so also is the lower limit of x & y)
%
\NewDocumentCommand{\mySlopeFieldGrid}{ O{2in} mm }{
    \resizebox{#1}{!}
    {
        \centering
        \begin{myTikzpictureGrid}{0.45} {#3}{#3}[solid]
            \foreach \x in {-#3,...,#3}
            \foreach \y in {-#3,...,#3}
            {
                \fill[black!60] (\x,\y) circle (0.11cm);
            }       
        \end{myTikzpictureGrid}
    }
    \\[0.5\onelineskip]
    \centering
    $ \frac{ dy }{ dx } = #2$
    }
