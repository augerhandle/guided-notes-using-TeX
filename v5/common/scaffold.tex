%
%-----------------------------------------------------------------------------------------
% A command to insert the "scaffolding" for the reverse box method.
%
% #1 : variable
% #2 : first factor in the answer 
% #3 : second factor in the answer
% #4 : exponent on the variable in the upper-left 
% #5 : exponent on the variable in the upper-right and lower-left
%-----------------------------------------------------------------------------------------
\NewDocumentCommand{\myReverseBoxScaffold}{O{x} mm O{2} O{\text{\:}}}{
    \small
    %
    \mycirc{1}\underline{\myEmph{Rewrite:}}
        \fbox{ \phantom{$aaax^m + bbbx^n + ccc$} }\\[0.1\onelineskip]
    %
    \mycirc{2}\underline{\myEmph{Constants:}}
    \begin{minipage}{0.75\textwidth}
        \begin{align*}
            a = \text{\fbox{\phantom{999}}} \quad
            b &= \text{\fbox{\phantom{999}}} \quad
            c = \text{\fbox{\phantom{999}}} \\
            %
            a\cdot c &= \text{\fbox{\phantom{999}}}
        \end{align*}
    \end{minipage}
    %
    \mycirc{3}\underline{\myEmph{T-chart:}}
    \hfill
    \begin{tabular}[t]{c|c}
        \fbox{\phantom{999}} factors & sum (look for \fbox{\phantom{99}}) \\
        \hline\hline 
        & \\ 
        & \\
        & \\
        & \\
    \end{tabular}\\[0.1\onelineskip]
    %
    \mycirc{4}\mycirc{5}\underline{\myEmph{reverse box:}}
    {
        \renewcommand{\arraystretch}{1.25}
        \begin{tabular}[t]{r||r|r|}
            \cellcolor{lightgray}&&\\
            \hline\hline
            \phantom{GCF~~~}& \phantom{999}$#1^#4$& \phantom{999}$#1^#5$\\
            \hline
                            & $#1^#5$             & \\
            \hline
        \end{tabular}
    }\\[0.1\onelineskip]
    %
    \mycirc{6}\underline{\myEmph{result:}} 
    {
        \setlength\fboxrule{1.5pt}
        \fbox{(\whenSTUDENT{\phantom{#2}}\whenTEACHER{#2})~(\whenSTUDENT{\phantom{#3}}\whenTEACHER{#3})}
    }
}

%
%-----------------------------------------------------------------------------------------
% A command to insert the "scaffolding" for the reverse box method.
%
% #1 : exponent on the variable in the upper-left 
% #2 : exponent on the variable in the upper-right and lower-left
% #3 : variable symbol
%-----------------------------------------------------------------------------------------
\NewDocumentCommand{\myFourTermCubicReverseBoxScaffold}{mm O{x}}{
    \small
    \mycirc{1}\underline{\myEmph{Descending:}} \fbox{ \phantom{$ax^m + bx^n + c$} }\\[0.25\onelineskip]
    %
    \mycirc{2}\underline{\myEmph{Constants:}} 
    \begin{tabular}{llll}
        $a = $ & \fbox{\phantom{999}} & $b = $ & \fbox{\phantom{999}}\\
        $c = $ & \fbox{\phantom{999}}&  $d = $ & \fbox{\phantom{999}}\\ 
    \end{tabular}\\[0.25\onelineskip]
    %
    \mycirc{3}\mycirc{4}\underline{\myEmph{reverse box:}}%\\[0.1\onelineskip]
    {
        \renewcommand{\arraystretch}{1.25}
        \begin{tabular}[t]{r||r|r|}
            \cellcolor{lightgray}&&\\
            \hline\hline
            \phantom{GCF~~~}& \phantom{999}$#3^3$& \phantom{999}$#3^2$\\
            \hline
                            & $#3$             & \\
            \hline
        \end{tabular}
    }\\[0.25\onelineskip]
    %
    \mycirc{5}\underline{\myEmph{result:}} 
    {
        \setlength\fboxrule{1.5pt}
        \fbox{(\whenSTUDENT{\hspace{0.75in}}\whenTEACHER{#1})~(\whenSTUDENT{\hspace{0.75in}}\whenTEACHER{#2})}
    }
}
%
%-----------------------------------------------------------------------------------------
% Synthetic Division scaffolding...
%
% First, some machinery for allowing loop controlled generation of 
% table columns. I confess I don't understand this!
%-----------------------------------------------------------------------------------------
% For an explanation of the first couple macros, see:
% https://tex.stackexchange.com/questions/489747/loops-on-columns-inside-tabular-environment
% (I did change some names by adding "my@" in front of a few things.)
\makeatletter
\newcounter{my@col}
\newcommand\my@colloop@add[2]
  {%
    \expandafter\g@addto@macro\expandafter\colloop@\expandafter
      {\colloop@@{#1}}%
    \ifnum#2>\value{my@col}
      \g@addto@macro\colloop@{&}%
    \fi
  }
\newcommand\colloop[2]
  {%
    \noalign
      {%
        \setcounter{my@col}{0}%
        \gdef\colloop@{}%
        \gdef\colloop@@##1{#2}%
        \loop\ifnum#1>\value{my@col}
          \stepcounter{my@col}%
          \expandafter\my@colloop@add\expandafter{\the\value{my@col}}{#1}%
        \repeat
      }%
    \colloop@
  }
\makeatletter

% Print a box for numbers used in synth. div. scaffolding
%
\newcommand{\myNumberBox}{\fbox{\phantom{N}}}

% Synthetic division scaffolding
% #1 : the number of coefficients
% #2 : text size
% #3 : multiple of \onelineskip below the frame
\NewDocumentCommand{\mySyntheticDivisionScaffold}{m O{\large} O{0}}
{
    {
    \setlength\arrayrulewidth{2pt}
    % set the text size
    #2
    % FIRST: THE TOP TWO ROWS.
    % 1. The left-most column of the top two rows (left of the frame).
    \renewcommand{\arraystretch}{1.2}
    \begin{tabular}{c}
        \myNumberBox \\ 
        \phantom{\myNumberBox} \\
    \end{tabular}
    % 2. The next column of the top two rows (first inside the frame).
    %    (The first coefficient above an empty space.)
    \begin{tabular}{|c}
        \myNumberBox \\ 
        \phantom{\myNumberBox} \\ \hline
    \end{tabular}
    % 3. The rest of the columns (hence the arg minus 1).
    \hspace{-0.65em}
    \begin{tabular}[]{cccc}
        \colloop{\numexpr#1-1\relax}{\myNumberBox} \\ % #1 in the text would print the "index"
        \colloop{\numexpr#1-1\relax}{\myNumberBox} \\ \hline
    \end{tabular}
    
    % SECOND: THE BOTTOM ROW.
    % The line below the frame. (Shave a little off the baselineskip.)
    \vspace{#3\onelineskip}
    % 1. An empty cell that aligns with the top-left cell outside the frame.
    \begin{tabular}{c}
        \phantom{\myNumberBox} \\
    \end{tabular}
    % 2. The first cell below the frame (aligned w/ the first coefficient).
    \begin{tabular}{c}
        \myNumberBox \\
    \end{tabular}
    % 3. The rest of the columns (hence the arg minus 1).
    \hspace{-0.5em}
    \begin{tabular}[]{cccc}
        \colloop{\numexpr#1-1\relax}{\myNumberBox} \\
    \end{tabular}

    \vspace{1.5\onelineskip}
    \begin{tabular}{ll}
      {\scshape Quotient:}  & \framebox[1.5in]{\phantom{A}} \\
      {\scshape Remainder:} & \framebox[0.5in]{\phantom{A}} \\ 
    \end{tabular}
    }
}

%-----------------------------------------------------------------------------------------
% Scaffolding for difference of two squares
%-----------------------------------------------------------------------------------------
\newcommand{\myDifferenceOfSquaresScaffold}{
  $A =$~\framebox[0.5in]{\phantom{A}}
  \vspace{0.25\onelineskip}

  $B =$~\framebox[0.5in]{\phantom{A}}
  \vspace{1\onelineskip}
  
  $ 
    A^2 - B^2 = \left(A + B\right) \left( A - B \right)
  $
  \vspace{1\onelineskip}
  
  \underline{\myEmph{result:}}
  {
    \setlength\fboxrule{1.5pt}
    $ 
      \text{
              \framebox[2.1in]{
                \,
                (\phantom{Z\hspace{0.6in}Z})~
                (\phantom{Z\hspace{0.6in}Z})
                \,
              }   
            }
    $
  }
}