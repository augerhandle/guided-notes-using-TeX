%
% ------------------------------------------------------------------------
%
% Synthetic Division scaffolding...
%
% First, some machinery for allowing loop controlled generation of 
% table columns. I confess I don't understand this!
%
% For an explanation, see:
% https://tex.stackexchange.com/questions/489747/loops-on-columns-inside-tabular-environment
% (I did change some names by adding "my@" in front of a few things.)
\makeatletter
\newcounter{my@col}
\newcommand\my@colloop@add[2]
  {%
    \expandafter\g@addto@macro\expandafter\colloop@\expandafter
      {\colloop@@{#1}}%
    \ifnum#2>\value{my@col}
      \g@addto@macro\colloop@{&}%
    \fi
  }
\newcommand\colloop[2]
  {%
    \noalign
      {%
        \setcounter{my@col}{0}%
        \gdef\colloop@{}%
        \gdef\colloop@@##1{#2}%
        \loop\ifnum#1>\value{my@col}
          \stepcounter{my@col}%
          \expandafter\my@colloop@add\expandafter{\the\value{my@col}}{#1}%
        \repeat
      }%
    \colloop@
  }
\makeatletter

% Print a box for numbers used in synth. div. scaffolding
%
\newcommand{\myNumberBox}{\fbox{\phantom{N}}}

% Synthetic division scaffolding
% #1 : the number of coefficients
% #2 : text size
% #3 : multiple of \onelineskip below the frame
\NewDocumentCommand{\mySyntheticDivisionScaffold}{m O{\large} O{0}}
{
    {
    \setlength\arrayrulewidth{2pt}
    % set the text size
    #2
    % FIRST: THE TOP TWO ROWS.
    % 1. The left-most column of the top two rows (left of the frame).
    \renewcommand{\arraystretch}{1.2}
    \begin{tabular}{c}
        \myNumberBox \\ 
        \phantom{\myNumberBox} \\
    \end{tabular}
    % 2. The next column of the top two rows (first inside the frame).
    %    (The first coefficient above an empty space.)
    \begin{tabular}{|c}
        \myNumberBox \\ 
        \phantom{\myNumberBox} \\ \hline
    \end{tabular}
    % 3. The rest of the columns (hence the arg minus 1).
    \hspace{-0.65em}
    \begin{tabular}[]{cccc}
        \colloop{\numexpr#1-1\relax}{\myNumberBox} \\ % #1 in the text would print the "index"
        \colloop{\numexpr#1-1\relax}{\myNumberBox} \\ \hline
    \end{tabular}
    
    % SECOND: THE BOTTOM ROW.
    % The line below the frame. (Shave a little off the baselineskip.)
    \vspace{#3\onelineskip}
    % 1. An empty cell that aligns with the top-left cell outside the frame.
    \begin{tabular}{c}
        \phantom{\myNumberBox} \\
    \end{tabular}
    % 2. The first cell below the frame (aligned w/ the first coefficient).
    \begin{tabular}{c}
        \myNumberBox \\
    \end{tabular}
    % 3. The rest of the columns (hence the arg minus 1).
    \hspace{-0.5em}
    \begin{tabular}[]{cccc}
        \colloop{\numexpr#1-1\relax}{\myNumberBox} \\
    \end{tabular}

    \vspace{1.5\onelineskip}
    \begin{tabular}{ll}
      {\scshape Quotient:}  & \framebox[1.5in]{\phantom{A}} \\
      {\scshape Remainder:} & \framebox[0.5in]{\phantom{A}} \\ 
    \end{tabular}
    }
}

