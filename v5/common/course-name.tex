% ---------------------------------------------------------------------------
% These hold the name of the course (with and without modifiers)
% ---------------------------------------------------------------------------
\newcommand{\myCourseName}{undefined}     % no modifier in front
\newcommand{\myFullCourseName}{undefined} % modifier + course-name

\newcommand{\myModifyTheCourseName}{
    \ifADVANCED 
        \ifAP
            \renewcommand*{\myFullCourseName}{AP~\myCourseName}
        \else 
            % The only other possibility (for now) is Honors. Modify accordingly.
            \renewcommand*{\myFullCourseName}{Honors~\myCourseName}
        \fi
    \else 
        \renewcommand*{\myFullCourseName}{\myCourseName}
    \fi
}

% ---------------------------------------------------------------------------
% Class name modifiers.
% This is how I am distinguishing Honors, AP and On-Level.
%
% The basic idea is that there is either On-Level or there is some kind 
% of "advanced" modifier. In my case, the modifier might be "Honors" 
% or "AP". 
% ---------------------------------------------------------------------------
\newif\ifADVANCED % At this point, we don't know which kind of advanced.
                  % If this is not true, we assume all modifiers are irrelevant 
                  % and that it's an On-Level class.
\newif\ifHONORS % Here are the "advanced" modifiers. They're intended to be mutually exclusive.
\newif\ifAP

% By default, the modifiers are all false (ie, On-Level).
\ADVANCEDfalse\HONORSfalse\APfalse

% ---------------------------------------------------------------------------
% What (modified) course are we talking about?
% ---------------------------------------------------------------------------
% These commands declare which modifier(s) to use. 
% Typically, I will use one of these ONCE at the top of a lesson document.
% They ensure that the newifs are all set "consistently".
\NewDocumentCommand{\forONLEVEL}{O{Algebra 2}}{
    \ADVANCEDfalse\HONORSfalse\APfalse 
    \renewcommand*{\myCourseName}{#1}
    \myModifyTheCourseName
    }

\NewDocumentCommand{\forHONORS}{O{Algebra 2}}{
    \ADVANCEDtrue\HONORStrue\APfalse
    \renewcommand*{\myCourseName}{#1}
    \myModifyTheCourseName
    }

\NewDocumentCommand{\forAP}{O{Algebra 2}}{
    \ADVANCEDtrue\HONORSfalse\APtrue
    \renewcommand*{\myCourseName}{#1}
    \myModifyTheCourseName
    }

% 
% These will conditionally generate output (the argument) based on 
% whichever modifier(s) are set.
\newcommand{\whenHONORS}[1]  {\ifHONORS{#1}\else{}  \fi}
\newcommand{\whenAP}[1]      {\ifAP{#1}\else{}  \fi}
\newcommand{\whenONLEVEL}[1] {\ifADVANCED{}\else{#1}\fi}