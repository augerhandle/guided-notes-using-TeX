\makeatletter
% I grabbed the following code verbatim out of the LaTeX exam class,
% exam.cls, which is on CTAN.
%
%--------------------------------------------------------------------
% BEGIN SNIP
%                            \fillwithlines


% \fillwithlines takes one argument, which is either a length or \fill
% or \stretch{number}, and it fills that much vertical space with
% horizontal lines that run the length of the current line.  That is,
% they extend from the current left margin (which depends on whether
% we're in a question, part, subpart, or subsubpart) to the right
% margin.
%
% The distance between the lines is \linefillheight, whose default value
% is set with the command
%
% \setlength\linefillheight{.25in}
%
% This value can be changed by giving a new \setlength command.
%
% The thickness of the lines is \linefillthickness, whose default value
% is set with the command
%
% \setlength\linefillthickness{.1pt}
%
% This value can be changed by giving a new \setlength command.
%
% As of version 2.503, 2016/03/25, the lines drawn by the
% \fillwithlines command will be drawn in color if the user has given
% the command
%
%      \colorfillwithlines.
%
% The actual drawing of the lines is now done by the command
% \do@fillwithlines, after the \fillwithlines command decides whether
% they will be in color.  The default color is set by the command
%
%      \definecolor{FillWithLinesColor}{gray}{0.8}
%
% and the color can be changed by giving a new \definecolor command.
% You can return to black lines by giving the command
%
%      \nocolorfillwithlines

\newlength\linefillheight
\newlength\linefillthickness
\setlength\linefillheight{.25in}
\setlength\linefillthickness{0.1pt}


\newif\if@colorfillwithlines
\@colorfillwithlinesfalse
\def\colorfillwithlines{%
  \@ifundefined{definecolor}
  {%
    \ClassError{exam}{%
      You must load the color package with the command\MessageBreak
      \space\space\protect\usepackage{color}\MessageBreak
      in order to use the command \protect\colorfillwithlines
      \MessageBreak
      }{%
      This command makes use of the package color.sty,\MessageBreak
      and so you have to load color.sty before your\MessageBreak
      \protect\begin{document} command.\MessageBreak
      }%
  }%
  {%
    \definecolor{FillWithLinesColor}{gray}{0.8}
    \@colorfillwithlinestrue
  }%
}% \colorfillwithlines
\def\nocolorfillwithlines{\@colorfillwithlinesfalse}

\newcommand\fillwithlines[1]{%
  \if@colorfillwithlines
    \color@begingroup
      \color{FillWithLinesColor}%
      \do@fillwithlines{#1}%
    \color@endgroup
  \else
    \do@fillwithlines{#1}%
  \fi
}% \fillwithlines

\newcommand\linefill{\leavevmode
    \leaders\hrule height \linefillthickness \hfill\kern\z@}

% \do@fillwithlines is called only by \fillwithlines
\def\do@fillwithlines#1{%
  \begingroup
  \ifhmode
    \par
  \fi
  \hrule height \z@
  \nobreak
  \setbox0=\hbox to \hsize{\hskip \@totalleftmargin
          \vrule height \linefillheight depth \z@ width \z@
          \linefill}%
  % We use \cleaders (rather than \leaders) so that a given
  % vertical space will always produce the same number of lines
  % no matter where on the page it happens to start:
  \cleaders \copy0 \vskip #1 \hbox{}%
  \endgroup
}% \do@fillwithlines

% END SNIP
%--------------------------------------------------------------------
\makeatother
