\begin{myConceptSteps}{
    To solve a \gap{quadratic} equation using the quadratic formula\dots
}
\myStep{standard form}{Write the equation as \gap{$ax^2 + bx + c = 0$}.}
\myStep{coefficients}{Write down the values of \gap{$a$}, \gap{$b$}, \gap{$c$}. }
\myStep{discriminant}{Calculate the discriminant as \gap{$ D = b^2 - 4ac $}.}
\myStep{quadratic formula}{
    The quadratic formula is:
    \begin{tcolorbox}[center,width=1.5in,colback=white,]
        $\displaystyle x = \frac{-b \pm \sqrt{D}}{2a}$
    \end{tcolorbox}
    {\normalsize 
    which is really \gap{two} solutions:
    \begin{align*} 
        x &= \frac 
        {-b \bm{+} \myRoot{b^2 - 4ac}}
        {2a}
        &
        x &= \frac 
        {-b \bm{-} \myRoot{b^2 - 4ac}}
        {2a}
    \end{align*}
    }
}
\myStep{substitute}{Substitute $a$, $b$, $c$, and $D$ into this formula.
Usually you will get \gap{two} solutions.
\begin{itemize}
    \item When $D =$ \gap{$0$}, there is only \gap{one} \gap{solution}.
    \item When $D$ is \gap{negative}, there are \gap{no} \gap{real} \gap{solutions}.
\end{itemize}
}
\myStep{evaluate}{\gap{Simplify} the right-hand-side of the equations.}
\end{myConceptSteps}