\section{Definitions}

\begin{itemize}
    \item An \myEmph{arithmetic sequence} has a constant \gap{difference}, $d$,
    between the terms.

    \item A \myEmph{geometric sequence} has a constant \gap{ratio}, $r$, 
    between the terms.

    \item What kind of sequence is this?
    
    \begin{tikzpicture}[auto, bend left=30]
        \dashundergapssetup{
            gap-widen=false,
        }
    
        \node (1) at (0,0)  [myShape,draw] {2};
        \node (2) at (2,0)  [myShape,draw] {5};
        \node (3) at (4,0)  [myShape,draw] {8};
        \node (4) at (6,0)  [myShape,draw] {11};
        \node (5) at (8,0)  [myShape,draw] {14};
        \node (6) at (10,0) [myShape,draw] {17};
        \node (dots) at (11.5,0) {\huge\dots};
    
    \end{tikzpicture}  

    It is an \gap{arithmetic} sequence.

    \item Find the difference for this arithmetic sequence.
    \item 
    \begin{tikzpicture}[auto, bend left=30]
        \dashundergapssetup{
            gap-widen=false,
        }

        \node (1) at (0,0)  [myShape,draw] {17};
        \node (2) at (2,0)  [myShape,draw] {13};
        \node (3) at (4,0)  [myShape,draw] {9};
        \node (4) at (6,0)  [myShape,draw] {5};
        \node (5) at (8,0)  [myShape,draw] {1};
        \node (6) at (10,0) [myShape,draw] {-3};
        \node (dots) at (11.5,0) {\huge\dots};

    \end{tikzpicture}  

    The difference is $d = $\gap{$-4$}.


    \item What kind of sequence is this?
    
    \begin{tikzpicture}[auto, bend left=30]
        \dashundergapssetup{
            gap-widen=false,
        }
    
        \node (1) at (0,0)  [myShape,draw] {2};
        \node (2) at (2,0)  [myShape,draw] {4};
        \node (3) at (4,0)  [myShape,draw] {8};
        \node (4) at (6,0)  [myShape,draw] {16};
        \node (5) at (8,0)  [myShape,draw] {32};
        \node (6) at (10,0) [myShape,draw] {64};
        \node (dots) at (11.5,0) {\huge\dots};
    
    \end{tikzpicture}  

    It is a \gap{geometric} sequence.

    \item Find the ratio for this geometric sequence.
    \item 
    \begin{tikzpicture}[auto, bend left=30]
        \dashundergapssetup{
            gap-widen=false,
        }

        \node (1) at (0,0)  [myShape,draw] {40};
        \node (2) at (2,0)  [myShape,draw] {20};
        \node (3) at (4,0)  [myShape,draw] {10};
        \node (4) at (6,0)  [myShape,draw] {5};
        \node (5) at (8,0)  [myShape,draw] {2.5};
        \node (6) at (10,0) [myShape,draw] {1.25};
        \node (dots) at (11.5,0) {\huge\dots};

    \end{tikzpicture}  

    The ratio is $r = $\gap{$0.5$}.

    \item For an \myEmph{arithmetic sequence}, if the difference ($d$) is positive, then the terms \gap{increase}.
    \item For an \myEmph{arithmetic sequence}, if the difference ($d$) is negative, then the terms \gap{decrease}.
    
    \item For a \myEmph{geometric sequence}, if the ratio is $r>1$, then the terms quickly \gap{increase}.
    \item For a \myEmph{geometric sequence}, if the ratio is $-1<r<1$, then the terms \gap{get closer to zero}.

    \item Write the recursive and explicit formulas for arithmetic and geometric sequences.
    \begin{itemize}
        \item Arithmetic Sequences: ($a$:first term, $d$:difference)
            \begin{itemize}
                \item {\scshape recursive formula}: \gap{$x_n = x_{n-1} + d$}
                \item {\scshape explicit formula}: \gap{$x_n = a + d(n-1)$}
            \end{itemize}
        \item Geometric Sequences: ($a$:first term, $r$:ratio)
        \item \begin{itemize}
            \item {\scshape recursive formula}: \gap{$x_n = x_{n-1}\times r$}
            \item {\scshape explicit formula}: \gap{$x_n = a\times r^{n-1}$}
    \end{itemize}
    \end{itemize}
\end{itemize}