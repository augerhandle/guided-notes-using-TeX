\section{Pay It Forward}

\begin{itemize}
    \item The number of people (from the video about Trevor) is a geometric sequence.
    \item Fill in the missing terms.
    
    \begin{tikzpicture}[auto, bend left=30]
        \dashundergapssetup{
            gap-widen=false,
        }
    
        \node (1) at (0,0)  [myShape,draw] {1};
        \node (2) at (2,0)  [myShape,draw] {3};
        \node (3) at (4,0)  [myShape,draw] {9};
        \node (4) at (6,0)  [myShape,draw] {\tiny\gap{27}};
        \node (5) at (8,0)  [myShape,draw] {\tiny\gap{81}};
        \node (6) at (10,0) [myShape,draw] {\tiny\gap{243}};
        \node (dots) at (11.5,0) {\huge\dots};
    
        \draw [-Stealth] (1.north east) to node  {\tiny$\times 3$} (2.north west);
        \draw [-Stealth] (2.north east) to node  {\tiny$\times 3$} (3.north west);
        \draw [-Stealth] (3.north east) to node  {\tiny$\times 3$} (4.north west);
        \draw [-Stealth] (4.north east) to node  {\tiny$\times 3$} (5.north west);
        \draw [-Stealth] (5.north east) to node  {\tiny$\times 3$} (6.north west);
    \end{tikzpicture}  

    \item For this sequence, the first term is $a=$\gap{1}, and the ratio is $r=$\gap{3}.
    
    \item The explicit formula for a geometric sequence is $x_n = a r^{n-1}$. 
    So for Trevor's sequence, the explicit formula is
    \[ x_n = \text{\gap{$3^{n-1}$}} \]

    \item So Trevor's sequence looks like this.
    
    \begin{tikzpicture}[auto, bend left=30]
        \dashundergapssetup{
            gap-widen=false,
        }
    
        \node (1) at (0,0)  [myShape,draw] {$3^0$};
        \node (2) at (2,0)  [myShape,draw] {$3^1$};
        \node (3) at (4,0)  [myShape,draw] {$3^2$};
        \node (4) at (6,0)  [myShape,draw] {$3^3$};
        \node (5) at (8,0)  [myShape,draw] {$3^4$};
        \node (6) at (10,0) [myShape,draw] {$3^5$};
        \node (dots) at (11.5,0) {\huge\dots};
    
        \draw [-Stealth] (1.north east) to node  {\tiny$\times 3$} (2.north west);
        \draw [-Stealth] (2.north east) to node  {\tiny$\times 3$} (3.north west);
        \draw [-Stealth] (3.north east) to node  {\tiny$\times 3$} (4.north west);
        \draw [-Stealth] (4.north east) to node  {\tiny$\times 3$} (5.north west);
        \draw [-Stealth] (5.north east) to node  {\tiny$\times 3$} (6.north west);
    \end{tikzpicture}  

    \item What is the next term in that sequence? \gap{$3^6$}.
    
    \item A name for this sequence is ``powers of \gap{3}''.
\end{itemize}
