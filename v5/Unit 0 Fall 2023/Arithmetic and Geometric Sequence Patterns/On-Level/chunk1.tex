\section{Simple Sequences}

\begin{itemize}
    \item A \myEmph{sequence} is a chain of \gap{numbers} (or other objects) that follow a \gap{pattern}.
    \item The individual elements (numbers) in a sequence are called \gap{terms}.
    \item Fill in the missing terms.
    
    \begin{tikzpicture}[auto, bend left=30]
        \dashundergapssetup{
            gap-widen=false,
        }
    
        \node (1) at (0,0)  [myShape,draw] {3};
        \node (2) at (2,0)  [myShape,draw] {6};
        \node (3) at (4,0)  [myShape,draw] {9};
        \node (4) at (6,0)  [myShape,draw] {12};
        \node (5) at (8,0)  [myShape,draw] {15};
        \node (6) at (10,0) [myShape,draw] {\tiny\gap{18}};
        \node (7) at (12,0) [myShape,draw] {\tiny\gap{21}};
        \node (dots) at (13.5,0) {\huge\dots};
    
        \draw [-Stealth] (1.north east) to node  {\tiny$+3$} (2.north west);
        \draw [-Stealth] (2.north east) to node  {\tiny$+3$} (3.north west);
        \draw [-Stealth] (3.north east) to node  {\tiny$+3$} (4.north west);
        \draw [-Stealth] (4.north east) to node  {\tiny$+3$} (5.north west);
        \draw [-Stealth] (5.north east) to node  {\tiny$+3$} (6.north west);
        \draw [-Stealth] (6.north east) to node  {\tiny$+3$} (7.north west);
    \end{tikzpicture}    
    
    \begin{tikzpicture}[auto, bend left=30]
        \dashundergapssetup{
            gap-widen=false,
        }
    
        \node (1) at (0,0)  [myShape,draw] {1};
        \node (2) at (2,0)  [myShape,draw] {2};
        \node (3) at (4,0)  [myShape,draw] {4};
        \node (4) at (6,0)  [myShape,draw] {8};
        \node (5) at (8,0)  [myShape,draw] {16};
        \node (6) at (10,0) [myShape,draw] {\tiny\gap{32}};
        \node (7) at (12,0) [myShape,draw] {\tiny\gap{64}};
        \node (dots) at (13.5,0) {\huge\dots};
    
        \draw [-Stealth] (1.north east) to node  {\tiny$\times 2$} (2.north west);
        \draw [-Stealth] (2.north east) to node  {\tiny$\times 2$} (3.north west);
        \draw [-Stealth] (3.north east) to node  {\tiny$\times 2$} (4.north west);
        \draw [-Stealth] (4.north east) to node  {\tiny$\times 2$} (5.north west);
        \draw [-Stealth] (5.north east) to node  {\tiny$\times 2$} (6.north west);
        \draw [-Stealth] (6.north east) to node  {\tiny$\times 2$} (7.north west);
    \end{tikzpicture}    

    \item The dots (\dots) at the end mean a sequence goes on \gap{forever}. 
    
    \item We often write each term using a \myEmph{variable} 
    and a small number. Fill in the missing information.
        
    \begin{tikzpicture}[auto, bend left=30]
        \dashundergapssetup{
            gap-widen=false,
        }

        % lower subscript 
        \fontdimen16\textfont2=5pt
        \fontdimen17\textfont2=5pt

        \node (1) at (0,0)  [myShape,draw] {$x_1$};
        \node (2) at (2,0)  [myShape,draw] {$x_2$};
        \node (3) at (4,0)  [myShape,draw] {$x_3$};
        \node (4) at (6,0)  [myShape,draw] {$x_4$};
        \node (5) at (8,0)  [myShape,draw] {$x_5$};
        \node (6) at (10,0) [myShape,draw] {\gap{$x_6$}};
        \node (7) at (12,0) [myShape,draw] {\gap{$x_7$}};
        \node (dots) at (13.5,0) {\huge\dots};
    \end{tikzpicture}    

    \item The small numbers after the $x$ are called \gap{subscripts}.
\end{itemize}
