\newpage 

\myProblemsWithContent[Sketch the graphs of these functions.]
{
    $g(x) = (x - 3)^3 -1$
    \tcblower 
    \centering
    ${a} =$ \gap{$1$},\quad ${h} =$ \gap{$3$},\quad ${k} =$ \gap{$-1$}\\[\baselineskip]
    \begin{myTikzpictureGrid}{0.35} {8}[-4]{2}
        \whenTEACHER{
            \tkzFct[ solid, ultra thick, samples=2000, domain=-10:10,]{(x-3)**(1.0/3.0) - 1}
            \draw[color=red,thick,fill=red] (3,-1) circle (0.15 cm);
            \tkzFct[ solid, ultra thick, samples=2000, domain=-10:3,]{-(-x+3)**(1.0/3.0) - 1}
        }
    \end{myTikzpictureGrid} 
}
{
    $g(x) = -(x+4)^3 +2 $
    \tcblower 
    \centering
    ${a} =$ \gap{$-1$},\quad ${h} =$ \gap{$-4$},\quad ${k} =$ \gap{$2$}\\[\baselineskip]
    \begin{myTikzpictureGrid}{0.35} {8}[-2]{4}
        \whenTEACHER{
            \tkzFct[ solid, ultra thick, samples=2000, domain=-10:10,]{-(x+4)**(1.0/3.0) + 2}
            \draw[color=red,thick,fill=red] (-4,2) circle (0.15 cm);
            \tkzFct[ solid, ultra thick, samples=2000, domain=-10:3,]{(-x-4)**(1.0/3.0) + 2}
        }
    \end{myTikzpictureGrid}
}





\myProblemsWithContent[%
    Given these transformations,
    find $a$, $h$, $k$, and write the formula for $g(x)$.
]
{
    \begin{itemize}[nosep,fullwidth]
        \item reflection across the $x$-axis 
        \item shift down by 2
    \end{itemize}
    \tcblower 
    $a$=\gap{-1}, $h$=\gap{0}, $k$=\gap{-2}\\[1\baselineskip] 
    $g(x) =$ \gap{$-\myRoot[3]{x}-2$}
}
{
    \begin{itemize}[nosep,fullwidth]
        \item stretch by 2 
        \item shift right by 4 
    \end{itemize}
    \tcblower 
    $a$=\gap{2}, $h$=\gap{4}, $k$=\gap{0}\\[1\baselineskip] 
    $g(x) =$ \gap{$2\myRoot[3]{x-4}$}
}




\myWideProblemWithContent[%
    Sketch the graphs of these functions and then fill in the blanks.
    ]
{
    $g(x) = (x - 3)^3 + 2$
    \hfill 
    ${a} =$ \gap{$1$},\quad ${h} =$ \gap{$3$},\quad ${k} =$ \gap{$2$}
    \tcblower 
    \begin{minipage}{0.5\textwidth}
        \centering
        \begin{myTikzpictureGrid}{0.35} {9}[-3]{5}
            \whenTEACHER{
                \tkzFct[ solid, ultra thick, samples=2000, domain=-10:10,]{(x-3)**(1.0/3.0) + 2}
                \draw[color=red,thick,fill=red] (3,2) circle (0.15 cm);
                \tkzFct[ solid, ultra thick, samples=2000, domain=-10:3,]{-(-x+3)**(1.0/3.0) + 2}
            }
            \end{myTikzpictureGrid}
        \end{minipage}
    \begin{minipage}{0.5\textwidth}
        \vspace{0.5\baselineskip}
        inflection point:~\dotfill\gap{$(3,2)$}\\[0.5\baselineskip]
        domain:~\dotfill\gap{all real numbers}\\[0.5\baselineskip]
        range:~\dotfill\gap{all real numbers}\\[0.5\baselineskip]
        % left end behavior~\dotfill\gap{as} \gap{$x\rightarrow 3$}, \gap{$y\rightarrow -4$}\\[0.5\baselineskip]
        % right end behavior~\dotfill\gap{as} \gap{$x\rightarrow+\infty$}, \gap{$y\rightarrow+\infty$}
    \end{minipage}
    \vspace{0.25\baselineskip}
}

\myWideProblemWithContent
{
    $g(x) =  -(x +1)^3 - 3$
    \hfill 
    ${a} =$ \gap{$-1$},\quad ${h} =$ \gap{$-1$},\quad ${k} =$ \gap{$-3$}
    \tcblower 
    \begin{minipage}{0.5\textwidth}
        \centering
        \begin{myTikzpictureGrid}{0.35} {9}[-5]{3}
            \whenTEACHER{
                \tkzFct[ solid, ultra thick, samples=2000, domain=-10:10,]{(x+1)**(1.0/3.0) - 3}
                \draw[color=red,thick,fill=red] (-1,-3) circle (0.15 cm);
                \tkzFct[ solid, ultra thick, samples=2000, domain=-10:3,]{-(-x-1)**(1.0/3.0) - 3}
            }
        \end{myTikzpictureGrid}
    \end{minipage}
    \begin{minipage}{0.5\textwidth}
        \vspace{0.5\baselineskip}
        inflection point:~\dotfill\gap{$(-1,-3)$}\\[0.5\baselineskip]
        domain:~\dotfill\gap{all real numbers}\\[0.5\baselineskip]
        range:~\dotfill\gap{all real numbers}\\[0.5\baselineskip]
        % left end behavior~\dotfill\gap{as} \gap{$x\rightarrow-1$}, \gap{$y\rightarrow3$}\\[0.5\baselineskip]
        % right end behavior~\dotfill\gap{as} \gap{$x\rightarrow+\infty$}, \gap{$y\rightarrow-\infty$}
    \end{minipage}
    \vspace{0.25\baselineskip}
}

