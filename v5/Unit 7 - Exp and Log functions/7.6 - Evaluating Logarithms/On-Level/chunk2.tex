\section{Frequently Used Logarithms}

\newcommand\mylog[1]{\mathop{{}^{#1}\mathrm{log_{\,#1}}}}

\begin{center}
    \large
    \renewcommand{\arraystretch}{1.5}
    \setlength{\tabcolsep}{12pt}
    \begin{tabular}{cccc}
        \toprule
        notation & base & name & definition \\
        \midrule
        {\ttfamily log} & 10  & \gap{common}  log & $log(x) \equiv log_{10}(x)$ \\ 
        {\ttfamily ln}  & $e$ & \gap{natural} log & $ln(x) \equiv log_e(x)$ \\ 
        \bottomrule
        \end{tabular}
\end{center}
These are buttons on most calculators: \myKey{log} \myKey{ln}.

\whenTEACHER{
    \tiny
    Make sure to tell the kids about the difference between how these 
    keys work on the TI-84 and iPhone calculators. \\[0.5\onelineskip]
    In particular, 
    on the TI, you enter \myKey{log} button first and then the number, 
    but on the iPhone calculator, you enter the number first and then the \myKey{log} button.
}

\whenTEACHER{
    \tiny
    I also point out that on some calculators, the \myKey{ln} button 
    looks so much like the word ``in'' that some people don't thing there's 
    a natural log button on their calculator.
}

\myProblems[Evaluate these logarithms using a calculator. Round your answers to three digits after the decimal point.]
{
    $log(3)$ \myAnswer{\hfill\tiny 0.477}
}
{
    $ln(0.25)$ \myAnswer{\hfill\tiny -1.386}
}
{0.25in}

