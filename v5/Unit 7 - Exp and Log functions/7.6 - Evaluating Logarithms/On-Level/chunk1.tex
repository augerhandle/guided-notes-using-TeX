\section{Evaluating Simple Logarithms}

\begin{minipage}{0.725\textwidth}
\begin{myConceptSteps}{To evaluate logs without a calculator\dots}[]
    \myStep{log form}{Write logarithmic form by putting $= x$ at the end. }
    \myStep{exp form}{Convert to \gap{exponential} form.}
    \myStep{result}{Solve for \gap{$x$}. }
\end{myConceptSteps}
\end{minipage}
\begin{minipage}{0.25\textwidth}
    \whenTEACHER{
        \tiny
        $log_4(64) = $?\\[0.25\onelineskip]
        1. log form: $log_4(64)=x$ \\
        \hspace*{2ex} $b=4$, $A=64$, $x=x$\\
        2. exp form: $4^x=64$\\
        3. use cheat sheet: $4^x=4^3$\\
        \hspace*{2ex} cancel the base\\
        \hspace*{2ex} \fbox{$x=3$}
    }
\end{minipage}

\whenTEACHER{
    \tiny 
    As that example (on the right) shows, 
    after converting to exponential form, there's an exponential equation to solve.
    \\[0.5\onelineskip]
    TELL THEM: ``You know how to solve these -- get the same base.'' 
    \\[0.5\onelineskip] 
    Encourage them to use the cheat sheet, JUST as we did when solving exp equations. 
    That's what this is, y'all!
}

\myProblems[Evaluate these logarithms. Show your work, or explain how you got the answer.]
{
    $log_2(32)$ \myAnswer{\hfill\tiny $5$}
}
{
    $log_4\left(\frac{1}{256}\right)$   \myAnswer{\hfill\tiny $-4$}
}
{0.75in}

