\section{Solving Simple Logarithm Equations with Inverse Functions}

\begin{tcolorbox}[center,width=3.75in,colback=white,]
    The ``complicated'' log equations we will solve have \gap{more} \gap{than} \gap{one} logarithm.
\end{tcolorbox}

\begin{myConceptSteps}{To solve complicated log equations with inverse functions\dots}
    \myStep{log properties}{Use \gap{log} \gap{properties} to rewrite the equation as a simple log equation.}
    \myStep{base}{Find the base $b$ of the logarithm.}
    \myStep{$b^{(\dots)} = b^{(\dots)}$}{
        Take the exponential base $b$ of \gap{both} \gap{sides} of the equation.
        \begin{itemize}[nosep] 
            \item The left and right sides will be \gap{exponents}.
        \end{itemize}
    }
    \myStep{solve}{Solve for the variable. Wait until the very end to use the calculator.}
\end{myConceptSteps}

\myProblems
{
    Find the \myEmph{exact} solution of this equation.\\[0.5\onelineskip]
    $ -7 + log_3(k) = -3 $
    \myAnswer{\hfill\tiny $k=81$ }
}
{
    Find the \myEmph{exact} solution of this equation, and then \myEmph{approximate} it using a calculator.\\[0.5\onelineskip]
    $ 3 \, ln(10-m) - 4 = -1 $
    \myAnswer{\hfill\tiny $m = -e+10 \approx 7.282$}
}
{2.75in}
