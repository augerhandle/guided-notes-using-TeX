\section{Parent Functions}

\vfil
{
    \begin{minipage}{0.2\textwidth}
        \begin{center}
            {\bfseries\itshape Quadratics}\\
            \begin{myTikzpictureGrid}{0.25} {6}{6}
                \tkzFct[ dashed,  thick, samples=100, domain=-10:10,]{x**2}
            \end{myTikzpictureGrid}
        \end{center}\vspace{-1\onelineskip}
        {$f(x) = x^2$}

        {$g(x) = \bm{a}(x-\bm{h})^2 + \bm{k}$}
    \end{minipage}
}
\hfill 
{
    \begin{minipage}{0.2\textwidth}
        \begin{center}
            {\bfseries\itshape Reciprocals}\\
            \begin{myTikzpictureGrid}{0.25} {6}{6}
                \tkzFct[ dashed,  thick, samples=100, domain=0.01:6,]{1/x}
                \tkzFct[ dashed,  thick, samples=100, domain=-6:-0.01,]{1/x}
            \end{myTikzpictureGrid}
        \end{center}\vspace{-1\onelineskip}
        {$f(x) = \frac{1}{x}$}

        {$g(x) = \frac{\bm{a}}{x-\bm{h}} + \bm{k}$}
    \end{minipage}
}
\hfill
{
    \begin{minipage}{0.2\textwidth}
        \begin{center}
            {\bfseries\itshape Absolute Value}\\
            \begin{myTikzpictureGrid}{0.25} {6}{6}
                \tkzFct[ dashed,  thick, samples=100, domain=-10:10,]{abs(x)}
            \end{myTikzpictureGrid}
        \end{center}\vspace{-1\onelineskip}
        {$f(x) = |x|$}

        {$g(x) = \bm{a}|x-\bm{h}| + \bm{k}$}
    \end{minipage}
}
\hfill 
{
    \begin{minipage}{0.2\textwidth}
        \begin{center}
            {\bfseries\itshape Cube Root}\\
            \begin{myTikzpictureGrid}{0.25} {6}{6}
                \tkzFct[dashed, thick,samples=3500,domain =0:6]{   ( 10.0**(log10(\x)/3))   }
                \tkzFct[dashed, thick,samples=3500,domain =-6:0]  {  -( 10**(log10(-\x)/3))   }
        \end{myTikzpictureGrid}
    \end{center}\vspace{-1\onelineskip}
    {$f(x) = \myRoot[3]{x}$}

        {$g(x) = \bm{a}\myRoot[3]{x-\bm{h}} + \bm{k}$}
    \end{minipage}
}


{
    \begin{minipage}{0.3\textwidth}
        \begin{center}
            {\bfseries\itshape Cubics}\\
            \begin{myTikzpictureGrid}{0.25} {6}{6}
                \tkzFct[ dashed,  thick, samples=200, domain=-6:6,]{\x**3}
            \end{myTikzpictureGrid}
        \end{center}\vspace{-1\onelineskip}
        {$f(x) = x^3$}

        {$g(x) = \bm{a} (x-\bm{h})^3 + \bm{k}$}
    \end{minipage}
}
\hfill 
{
    \begin{minipage}{0.3\textwidth}
        \begin{center}
            {\bfseries\itshape Exponential Growth}\\
            \begin{myTikzpictureGrid}{0.25} {6}{6}
                \tkzFct[ dashed,  thick, samples=200, domain=-6:6,]{1.1**x}
                \tkzFct[ dashed,  thick, samples=200, domain=-6:6,]{1.2**x}
                \tkzFct[ dashed,  thick, samples=200, domain=-6:5,]{1.6**x}
                %
                \tkzFct[ solid, ultra thick, samples=200, domain=-6:3,]{2**x}
                %
                \tkzFct[ dashed,  thick, samples=200, domain=-6:3,]{6**x}
            \end{myTikzpictureGrid}
        \end{center}\vspace{-1\onelineskip}
        {$f(x) = b^x$} {\small($b>1$ for growth)}

        {$g(x) = \bm{a}\,b^{x-\bm{h}} + \bm{k}$}
    \end{minipage}
}
\hfill 
{
    \begin{minipage}{0.3\textwidth}
        \begin{center}
            {\bfseries\itshape Exponential Decay}\\
            \begin{myTikzpictureGrid}{0.25} {6}{6}
                \tkzFct[ dashed,  thick, samples=200, domain=-3:6,]{6**-x}
                %
                \tkzFct[ solid, ultra thick, samples=200, domain=-3:6,]{0.5**x}
                %
                \tkzFct[ dashed,  thick, samples=200, domain=-6:6,]{1.1**-x}
                \tkzFct[ dashed,  thick, samples=200, domain=-6:6,]{1.2**-x}
                \tkzFct[ dashed,  thick, samples=200, domain=-6:5,]{1.6**-x}
            \end{myTikzpictureGrid}
        \end{center}\vspace{-1\onelineskip}
        {$f(x) = b^x$} {\small($0<b<1$ for decay)}

        {$g(x) = \bm{a}\,b^{x-\bm{h}} + \bm{k}$}
    \end{minipage}
}

\begin{tcolorbox}[center,width=5.5in,colback=white,]
    \small
    Remember that the base, $b$, is a \gap{multipler} 
    that defines how fast or slowly the {\myEmph exponential functions }
    grow or decay.\\[0.5\onelineskip]
    So this is really a \gap{bunch} of parent functions, 
    one for each value of $b$.
\end{tcolorbox}


