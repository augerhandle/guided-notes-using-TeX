\section{Exponential Form}
\vspace{-1\onelineskip}
\myProblemsWithContent[Fill in the blanks.]
{
    $2^5 = $ \gap{32}\whenTEACHER{\tiny 2*2*2*2*2 ... doubling 5 times}
    \vspace{1\onelineskip}
}
{
    $3^4 = $ \gap{32}\whenTEACHER{\tiny 3*3*3*3 ... tripling 4 times}
}

\whenTEACHER{
    \tiny
    There are TWO WAYS to explain this...
    FIRST, the obvious is to multiply the base by itself a bunch of times.
    The kids will "get" this. But it doesn't explain ZERO or NEGATIVE exponents, 
    since what does it mean to multiply anything by itself "no times"? 
    And what on earth does it mean to multiply anything by itself a NEGATIVE number of 
    times!?\\[0.5\onelineskip]
    %
    Another way to explain this (which I think we should do on the whiteboard) 
    is to draw a table of powers of b.
    %
    \begin{center}
    \begin{tabular}{lr} 
        $x$ & $2^x$ ($b=2$) \\ 
        \hline 
        1 & 2 \\ 
        2 & 2*2=4 \\ 
        3 & 2*2*2=8 \\
        4 & 16 \\ 
        5 & 32 \\ 
        etc
    \end{tabular}
    \end{center}
    % 
    Point out that as you go down the table, you're MULTIPLYING BY 2 (the base).
    And point out that this is the same as the EXPONENTS CHEAT SHEET.
    Finally, ask them what happens when to GO UP row-by-row in the table 
    (we DIVIDE) by 2 (the base). \\[0.5\onelineskip]
    %
    Start at the bottom and go up, dividing by 2 each time. 
    When you get to the x=1 row, ask what would happen if we KEPT GOING. 
    Divide again by 2 which gives us 2/2 WHICH IS ONE! 
    THIS IS THE ONLY REASON $b^0 = 1$ --- so that as we go UP in the table 
    when we get to x=0, we just keep dividing. \\[0.5\onelineskip]
    %
    Keep going up to x=-1. We DIVIDE AGAIN: 1/2 $ = \frac{1}{2}$.
    So 2 (the base) raised to a negative power is a reciprocal. 
    I confess to my students that this IS NOT multiplying 2 (the base) 
    by itself a negative numbedr of times. 
    I tell them that mathematicians define exponents this way  
    so that this table "works" when we keep going up by dividing.
}

This is known as \gap{exponential} \gap{form}. 
\begin{tcolorbox}[center,colback=white,width=4.5in,title={\itshape Exponential Form},colbacktitle=white,coltitle=black]
    \vspace{-0.5\onelineskip}
    \Large
    \[ b^x = A \]
    %
    \small
    ``What do I get when I raise $b$ to the $x$ power?''
\end{tcolorbox}

\myProblemsWithContent[Fill in the blanks for these exponential forms.]
{
    $6^3 =$ \gap{216}
    \vspace{1\onelineskip}
}
{
    $2^7 = $ \gap{128}
}