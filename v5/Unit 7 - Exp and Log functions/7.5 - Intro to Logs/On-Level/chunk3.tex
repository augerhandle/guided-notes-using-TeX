\section{Logarithmic Form}

\gap{Logarithms} are a \myEmph{different way} to think about exponential functions.
\begin{tcolorbox}[center,colback=white,width=4.5in,title={\itshape Logarithmic Form},colbacktitle=white,coltitle=black]
    \vspace{-0.5\onelineskip}
    \Large
    \[ log_b(A) = x \]
    %
    \small
    ``What power do I raise $b$ to to get $A$?''
\end{tcolorbox}

\whenTEACHER{
    I tell the students that this log notation is confusing, 
    but that if they think of it as just a different way of thinking 
    about $b^x=A$, it's easier. \\[0.5\onelineskip]
    %
    So exponential form $b^x=A$ answers this question:
    What do I get if I multiply b times itself x times? 
    \\[0.5\onelineskip] 
    %
    Logarithmic form just turns that a LITTLE SIDEWAYS. 
    $log_b(A)=x$ answers this question:
    What power do I raise b to in order to get A?
    \\[0.5\onelineskip]
    %
    It's all still about b and x and A 
    and the same relationships between them. 
}

\myProblemsWithContent[Fill in the blanks for these logarithmic forms.]
{
    $log_2(32) = $\gap{$5$} 
    \myAnswer{\hfill\tiny b/c $32 = 2^5$}
    \vspace{1\onelineskip}
}
{
    $log_3(81) = $\gap{$4$}  
    \myAnswer{\hfill\tiny b/c $81 = 3^4$}
}
\whenTEACHER{
    Encourage them to use the cheat sheet to do these.
    Alternatively, they may use calculators and just multiply 2 or 3 
    by themselves a bunch of times until they get A.
}