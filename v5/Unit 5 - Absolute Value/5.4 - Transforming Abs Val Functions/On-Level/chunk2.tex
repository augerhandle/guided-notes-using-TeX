\section{Describing Transformations}
We will focus on \myEmph{three} kinds of transformations
of the absolute value parent function:
\begin{itemize}[nosep]
    \item horizontal shifts \gap{left} \gap{or} \gap{right}
    \item vertical shifts \gap{up} \gap{or} \gap{down}
    \item \gap{reflections} across the $x$-axis
\end{itemize}

\begin{myConceptSteps}{To decide which of these transformations is involved for a particular function\dots}
    \myStep{graph}{Look at the graph the function using a \gap{TI-84} calculator.}
    \myStep{shifts}{Use the location of the \gap{vertex} to decide what kinds of shifts are involved.}
    \myStep{reflection}{Decide if there is a reflection based on whether the graph opens \gap{up} or opens \gap{down}.}
\end{myConceptSteps}

\myProblemsWithContent[%
    For these problems, do the following things. (Use the TI-84 to help you.)
    \begin{itemize}[nosep]
        \item Sketch the graph of the absolute value parent function.
        \item Sketch the graph of $g(x)$.
        \item Describe what transformations are involved in $g(x)$.
    \end{itemize}
    ]
{
    $g(x) = |x+2| - 5$
    \tcblower 
    \begin{myTikzpictureGrid}{0.4} {6}{6}
        \whenTEACHER{
            \tkzFct[ dashed, thick, samples=100, domain=-6:6,]{abs(x)}
            \draw[draw=red, fill=red] (0,0) circle (2.5mm);    
            \tkzFct[ solid, ultra thick, samples=100, domain=-6:6,]{abs(x+2)-5}
            \draw[draw=red, fill=red] (-2,-5) circle (2.5mm);    
            \tkzText(0,-5){\tiny$(-2,-5)$}
        }
    \end{myTikzpictureGrid}
    \whenTEACHER{
        \tiny 
        \begin{itemize}[nosep]
            \item shift \gap{LEFT} by \gap{2} 
            \item shift \gap{DOWN} by \gap{5}
        \end{itemize}
    }
}
{
    $g(x) = -|x| +3 $
    \tcblower 
    \begin{myTikzpictureGrid}{0.4} {6}{6}
        \whenTEACHER{
            \tkzFct[ dashed, thick, samples=100, domain=-6:6,]{abs(x)}
            \draw[draw=red, fill=red] (0,0) circle (2.5mm);    
            \tkzFct[ solid, ultra thick, samples=100, domain=-6:6,]{-abs(x)+3}
            \draw[draw=red, fill=red] (0,3) circle (2.5mm);    
            \tkzText(1.5,3.5){\tiny$(0,4)$}
        }
    \end{myTikzpictureGrid}
    \whenTEACHER{
        \tiny 
        \begin{itemize}[nosep]
            \item \gap{REFLECT} across the $x$-axis 
            \item shift \gap{UP} by \gap{3}
        \end{itemize}
    }
}

\whenTEACHER{
    \tiny 
    Make sure to explain that it's not sufficient to say ``left'' or ``right''.
    They must tell us BY HOW MUCH.
}