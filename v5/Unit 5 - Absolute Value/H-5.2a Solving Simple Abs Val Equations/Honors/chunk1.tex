\section{Introduction}

\begin{tcbraster}[
    raster equal height, 
    raster columns = 3,
    raster column skip = 0.5in,
]
    \begin{tcolorbox}[colback=white,boxrule=0.5pt,]
        \raggedright
        {\small solving \myEmph{linear} equations \phantom{xxxxxxx}}
        \begin{center}
        $5x +3 = 13$\\[0.25\baselineskip]
        {\HUGE$\Downarrow$}\\
        $x =$ \gap{$2$}
        \end{center}
    \end{tcolorbox}
    \begin{tcolorbox}[colback=white,boxrule=0.5pt,]
        \raggedright
        {\small solving \myEmph{quadratic} equations}
        \begin{center}
        $x^2 + 5x + 6 = 0$\\[0.25\baselineskip]
        {\HUGE$\Downarrow$}\\[0.2\baselineskip]
        $x =$ \gap{$-2$}, \gap{$-3$}
        \end{center}
    \end{tcolorbox}
    \begin{tcolorbox}[colback=white,boxrule=0.5pt,]
        \raggedright
        {\small Now, solving \myEmph{absolute value} equations}
        \begin{center}
        $|4x -2 | = 10$\\[0.25\baselineskip]
        {\HUGE$\Downarrow$}\\[0.2\baselineskip]
        $x =$ \gap{$-2$}, \gap{$3$}
        \end{center}
    \end{tcolorbox}
\end{tcbraster}

\myWideProblemWithContent[Solve this absolute value equation]
{
    $|x| = 4$
    \tcblower
    The key is to think about \gap{distance}\,:
    \hfill
    \begin{tikzpicture}[scale=0.75]
        \myDrawNumberlineCircle{0}{white}
        \whenTEACHER{
            \myDrawNumberlineCircle{4}{black}
            \myDrawNumberlineCircle{-4}{black}
        }
        \myDrawNumberline{5}
    \end{tikzpicture}\\
    %
    \begin{tcbraster}[
        raster columns = 2,
        raster before skip = 0.5\baselineskip,
        ]
        \begin{tcolorbox}[colback=white,boxrule=0pt,colframe=white,]
            \vspace{0.9in}
        \end{tcolorbox}
        \begin{tcolorbox}[colback=white,]
            When you solve absolute value equations,
            you will \myEmph{always} \gap{split} the equation in two like this.
        \end{tcolorbox}
    \end{tcbraster}
}
