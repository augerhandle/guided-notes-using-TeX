\section{Evaluating Absolute Value Expressions}

A number and its negative are the \gap{same} \gap{distance} from the origin.
\whenTEACHER{
    \tiny 
    The point of the diagrams below is to remind the students of the fact 
    that there are TWO POINTS that are BOTH a given distance from the origin.
    This is our transition to the equations having TWO SOLUTIONS. So we want to "hammer"
    on this point. :-)
}
\begin{tcbraster}[
    raster columns=2,
    raster before skip = 1em, raster after skip = 1em,
    ]
    \begin{tcolorbox}[colback=white,]
        \vspace{2\baselineskip}
        \begin{tikzpicture}[scale=0.75]
            \myDrawNumberlineCircle{0}{white}
            \myDrawNumberlineCircle{2}{black}
            \myDrawNumberlineCircle{-2}{black}
            \myDrawNumberline{5}
        \end{tikzpicture}
        \begin{align*}
            |2| &= 2\\
            |-2| &= 2
        \end{align*}
    \end{tcolorbox}
    \begin{tcolorbox}[colback=white,]
        \vspace{2\baselineskip}
        \begin{tikzpicture}[scale=0.75]
            \myDrawNumberlineCircle{0}{white}
            \myDrawNumberlineCircle{4}{black}
            \myDrawNumberlineCircle{-4}{black}
            \myDrawNumberline{5}
        \end{tikzpicture}
        \begin{align*}
            |4| &= 4\\
            |-4| &= 4
        \end{align*}
    \end{tcolorbox}
\end{tcbraster}
\whenTEACHER{
    \tiny
    Look at those equations. 
    They suggest that you DROP the negative of a number 
    when you take its absolute value.
    So...
}

\begin{myConcept}{To find the absolute value of a number\dots}
    \begin{itemize}
        \item \gap{Drop} the negative sign (if any) in front of the number.
        \item The result will \gap{never} be negative.
    \end{itemize}
    \begin{tcolorbox}[center,width=5.5in,colback=white,]
        \centering
        This only works with numbers, \gap{not} with expressions or variables.
    \end{tcolorbox}
\end{myConcept}

\myProblemsWithContent[Find the following absolute values.]
{
    $|12|=$   \gap{$12$}
}
{
    $|-8|=$   \gap{$8$}
}

\par
\begin{myConceptSteps}{To evaluate an absolute value expression\dots}
    \myStep{substitute}{Plug variable values into the expression.}
    \myStep{simplify inside}{%
        The vertical bars are {\sffamily GEMDAS} \gap{grouping} \gap{operators}.
        Simplify \gap{inside} 
        the absolute value first.
        You will get the absolute value of a (positive or negative) \gap{number}.
    }
    \myStep{evaluate}{%
        Simplify the absolute value of the number by dropping any inside negative 
        and removing the vertical bars.
    }
    \myStep{simplify the rest}{Now that there are no vertical bars, simplify the rest of the expression.}
\end{myConceptSteps}

\begin{myWarningBox}[colback=white,width=5in,center,]
    \centering
    \myEmph{Do not} \gap{distribute} across the vertical bars!
\end{myWarningBox}

\myProblems[Evaluate these expressions at the given value of $x$.]
{
    $|2x|$ \quad at $x=-3$ \hfill \myAnswer{$6$}
}
{
    $3|-5x|$ \quad at $x=2$ \hfill \myAnswer{$30$}
}
{6\onelineskip}

\myProblems
{
    $|x - 8|$ \quad at $x=6$ \hfill \myAnswer{$2$}
}
{
    $1 - 2\left|x + 3\right|$ \quad at $x=-5$ \hfill \myAnswer{$-3$}
    % 1 - (2)|-5 + 3| 
    % 1 - (2)|-2|
    % 1 - (2)(2) 
    % 1 - 4 = -3
}
{10\onelineskip}

