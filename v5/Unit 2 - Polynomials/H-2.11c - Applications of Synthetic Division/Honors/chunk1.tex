If you divide one polynomial (the {\scshape Dividend}) 
by another (the {\scshape Divisor}),
the result is written like this. 
{
    \begin{center}
    $ 
    \frac
        {\text{\scshape Dividend\rule[-0.2em]{0em}{0.2em}}}
        {\text{\scshape Divisor\rule{0em}{0.75em}}} 
    =
    \text{\scshape Quotient} 
        + 
        \frac
            {\text{\scshape Remainder\rule[-0.2em]{0em}{0.2em}}}
            {\text{\scshape Divisor\rule{0em}{0.75em}}} 
    $
    \end{center}
}

These are polynomial \gap{functions}. 
Let's give them names.
Let the divisor be \gap{$(x-a)$}.
\begin{center}
    \begin{tabular}{rc}
        {\itshape name} & {\itshape function} \\ 
        \midrule
        {\scshape Dividend} & $D(x)$ \\ 
        {\scshape Divisor} & $(x-a)$ \\ 
        {\scshape Quotient} & $Q(x)$ \\ 
        {\scshape Remainder} & $R(x)$ \\ 
    \end{tabular}
\end{center}

So we can rewrite the equation above as
{
\    \begin{center}
    $ 
    \frac
        {D(x)}
        {(x-a)} 
    =
    Q(x) 
        + 
        \frac
            {R(x)}
            {(x-a)} 
    $
    \end{center}
}
or
\begin{tcolorbox}[center,width=4in]
    {
        \begin{center}
        $ 
            D(x) = Q(x)\:(x-a)  +  R(x)
        $
        \end{center}
    }           
\end{tcolorbox}

This \gap{division} \gap{equation} is at the heart of what we're about to do.