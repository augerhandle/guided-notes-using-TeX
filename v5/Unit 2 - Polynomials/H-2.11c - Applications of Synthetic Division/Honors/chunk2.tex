\section{Is $(x-a)$ a factor of a polynomial?}

Suppose we are given:
\begin{itemize}
    \item a polynomial, $D(x)$, and 
    \item a linear binomial, $(x-a)$.
\end{itemize}

Start with
$ 
D(x) = Q(x)\:(x-a)  +  R
$
and notice what happens to the division equation if the remainder is \gap{zero}:
\begin{align*}
    D(x) &= (x-a)\:Q(x)  +  R \\
    D(x) &= (x-a)\:Q(x)  +  0 \\
    D(x) &= (x-a)\:Q(x)   \\
\end{align*}
%
which means that $D(x)$ is the \gap{product} of two things.
One of those is $(x-a)$.
Wait! That means $(x-a)$ \myEmph{is a factor} of $D(x)$.

\begin{myConceptSteps}{To use synthetic division to decide if $(x-a)$ is a factor of $D(x)$\dots}
    \myStep{Divide}{Use \gap{synthetic} \gap{division} for $\frac{D(x)}{(x-a)}$.}
    \myStep{Remainder}{Look at the remainder, which will be a \gap{number}.}
    \myStep{Factor}[?]{$(x-a)$ is a factor if the remainder is \gap{zero}. }
\end{myConceptSteps}

\myProblems[Use synthetic division to decide if $(x-a)$ is a factor of the given polynomial.]
{
    Is 
    $(v-4)$ 
    a factor of 
    $(8v^3 - 30v^2 - 2v - 21)$
    ?
}
{
    Is 
    $(1+b)$ 
    a factor of 
    $(-4b^3 - 7 - 10b - 7b^2)$
    ?
}
{2in}
