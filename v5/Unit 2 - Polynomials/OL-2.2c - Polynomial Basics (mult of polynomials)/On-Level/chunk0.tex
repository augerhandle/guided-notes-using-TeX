\section{The Box Method}

We will \gap{multiply} polynomials using the \gap{box} method.
It is useful for
\begin{itemize}[nosep]
    \item multiplying \gap{long} polynomials,
    \item collecting \gap{like} terms, and 
    \item \gap{factoring} (which we will do later).
\end{itemize}
\vspace{1\baselineskip}

\begin{myConceptSteps}{To multiply polynomials using the \myEmph{box method}\dots}
    \myStep{table}{%
        Create a table with a bunch of \gap{boxes} in it.
        \begin{itemize}
            \item Write the polynomials along the 
            \gap{top} and \gap{left} side of a rectangle.
            \item Make a \gap{row} and \gap{column} for each term.
        \end{itemize}
    }
    \myStep{boxes}{Fill each {\bfseries\itshape row/column} \gap{box}
        with the \gap{product} of two monomials.
    }
    \myStep{combine}{\gap{Add} all the boxes together, 
        \gap{combining} like terms.
        }
\end{myConceptSteps}

Here is the idea. To multiply $(3x-2)(x^2 + 2x + 3)$, 
create a 2 $\times$ 3 table and fill it out like this.

\begin{center}
    \Large
    \renewcommand{\arraystretch}{1.5}
    \begin{tabular}{r||p{0.75in}|p{0.75in}|p{0.75in}|}
        & $x^2$ & $2x$ & $3$ \\
        \hline\hline
        $3x$ & & & \\ 
        \hline
        $-2$ & & & \\
        \hline
    \end{tabular}
\end{center}
