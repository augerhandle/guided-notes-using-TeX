\section{Cube Root Functions}

The \myEmph{cube root parent function} is \tcbox[on line]{$f(x) = \myRoot[3]{x}$.}
\vspace{1\onelineskip}

\myWideProblemWithContent[Fill in the table, plot the points, and sketch the graph.]
{
\begin{minipage}{0.57\textwidth}
    \centering 
    \small
    \renewcommand{\arraystretch}{1.5}
    \begin{tabular}{r|p{1.5in}|c}
        $x$  & $y = f(x) = \myRoot[3]{x}$ & $(x,y)$\\ 
        \hline\hline
        $-8$ & $f(-8)=$ & $(\hspace{0.25in},\hspace{0.25in})$ \\
        $-3.5$ & $f(-3.5)=$ & $(\hspace{0.25in},\hspace{0.25in})$ \\
        $-1$ & $f(-1)=$ & $(\hspace{0.25in},\hspace{0.25in})$ \\
        $0$ & $f(0)=$ & $(\hspace{0.25in},\hspace{0.25in})$ \\
        $1$ & $f(1)=$ & $(\hspace{0.25in},\hspace{0.25in})$ \\
        $3.5$ & $f(3.5)=$ & $(\hspace{0.25in},\hspace{0.25in})$ \\
        $8$ & $f(8)=$ & $(\hspace{0.25in},\hspace{0.25in})$ \\
    \end{tabular}
\end{minipage}
\begin{minipage}{0.425\textwidth}
    \centering
    \begin{myTikzpictureGrid}{0.3} {9}{9}
        % \tkzText(13,8){$x - y = 0$}
        \whenTEACHER{
            \tkzFct[ solid, ultra thick, samples=100, domain=0:9,]{x**(1.0/3.0)}
            \tkzFct[ solid, ultra thick, samples=100, domain=-9:0,]{-(-x)**(1.0/3.0)}
            \draw[black,thick,fill=red] (-8,-2) circle (0.2 cm);
            \draw[black,thick,fill=red] (-3.5,-1.52) circle (0.2 cm);
            \draw[black,thick,fill=red] (-1,-1) circle (0.2 cm);
            \draw[black,thick,fill=red] (0,0) circle (0.2 cm);
            \draw[black,thick,fill=red] (1,1) circle (0.2 cm);
            \draw[black,thick,fill=red] (3.5,1.52) circle (0.2 cm);
            \draw[black,thick,fill=red] (8,2) circle (0.2 cm);
        }
    \end{myTikzpictureGrid}
\end{minipage}
}

\myWideProblem{Describe how this curve is \myEmph{similar to} the graph of the cubic parent function?}{0.5in}

\myWideProblemWithContent[Do these things:
    \begin{itemize}[nosep]
        \item Sketch a transformed parent function that has been
            \myEmph{reflected} across the $x$-axis and
            shifted \myEmph{up} by 5. 
        \item Put a BIG DOT at the inflection point.
        \item Then fill in the blanks.
    \end{itemize}
    ]
{
\begin{minipage}{0.3\textwidth}
    \centering
    \begin{myTikzpictureGrid}{0.25} {9}{9}
        % \tkzText(13,8){$x - y = 0$}
        \whenTEACHER{
            \tkzFct[ solid, ultra thick, samples=100, domain=0:9,]{-x**(1.0/3.0)+5}
            \tkzFct[ solid, ultra thick, samples=100, domain=-9:0,]{--(-x)**(1.0/3.0)+5}
            \draw[fill,black,fill=red] (0,5) circle (3mm);
        }
    \end{myTikzpictureGrid}
\end{minipage}
\begin{minipage}{0.64\textwidth}
    \centering 
    \small
    \renewcommand{\arraystretch}{1.5}
    \begin{tabular}{lcl}
        coords of the inflection pt & : & (\gap{$0$},\gap{$5$}) \\
        $a$, $h$, and $k$           & : & $a=$ \gap{$-1$}, $h=$ \gap{$0$}, $k=$ \gap{$5$} \\
        transformed function        & : & $g(x)=$ \gap{$-x^3 + 5$}
    \end{tabular}
\end{minipage}
}

\myWideProblemWithContent[Write the equations for these parent functions.]
{
    \renewcommand{\arraystretch}{1.25}
    \begin{tabular}{p{2in}cp{2.5in}}
        \myEmph{cubic} parent function     & : & $f(x) = $ \gap{$x^3$}\\
        \myEmph{cube root} parent function & : & $f(x) = $ \gap{$\myRoot[3]{x}$}\\
        \myEmph{linear} parent function    & : & $f(x) = $ \gap{$x$}\\
        \myEmph{quadratic} parent function & : & $f(x) = $ \gap{$x^2$}\\
    \end{tabular}
}