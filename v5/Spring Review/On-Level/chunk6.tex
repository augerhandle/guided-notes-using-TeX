\section{Reciprocal Functions}

The \myEmph{reciprocal parent function} is \tcbox[on line]{$f(x) = \frac{1}{x}$.}
\vspace{1\onelineskip}

\myWideProblemWithContent[Fill in the table, plot the points, and sketch the graph.]
{
\begin{minipage}{0.57\textwidth}
    \centering 
    \small
    \renewcommand{\arraystretch}{1.5}
    \begin{tabular}{r|p{1.5in}|c}
        $x$  & $y = f(x) = \frac{1}{x}$ & $(x,y)$\\ 
        \hline\hline
        $-3$ & $f(-3)=$ & $(\hspace{0.25in},\hspace{0.25in})$ \\
        $-2$ & $f(-2)=$ & $(\hspace{0.25in},\hspace{0.25in})$ \\
        $-1$ & $f(-1)=$ & $(\hspace{0.25in},\hspace{0.25in})$ \\
        $-0.5$ & $f(-0.5)=$ & $(\hspace{0.25in},\hspace{0.25in})$ \\
        $-0.25$ & $f(-0.25)=$ & $(\hspace{0.25in},\hspace{0.25in})$ \\
        $-0.1$ & $f(-0.11)=$ & $(\hspace{0.25in},\hspace{0.25in})$ \\
        %
    \end{tabular}
\end{minipage}
\begin{minipage}{0.425\textwidth}
    \centering
    \begin{myTikzpictureGrid}{0.25} {10}{10}
        \whenTEACHER{
            \tkzFct[ solid, ultra thick, samples=100, domain=0.01:10,]{1/x}
            \tkzFct[ solid, ultra thick, samples=100, domain=-10:-0.01,]{1/x}
            %
            \draw[black,thick,fill=red] (-3,-0.333) circle (0.2 cm);
            \draw[black,thick,fill=red] (-2,-0.5) circle (0.2 cm);
            \draw[black,thick,fill=red] (-1,-1) circle (0.2 cm);
            \draw[black,thick,fill=red] (-0.5,-2) circle (0.2 cm);
            \draw[black,thick,fill=red] (-0.25,-4) circle (0.2 cm);
            \draw[black,thick,fill=red] (-0.1,-10) circle (0.2 cm);
            %
            \draw[black,thick,fill=red] (3,0.333) circle (0.2 cm);
            \draw[black,thick,fill=red] (2,0.5) circle (0.2 cm);
            \draw[black,thick,fill=red] (1,1) circle (0.2 cm);
            \draw[black,thick,fill=red] (0.5,2) circle (0.2 cm);
            \draw[black,thick,fill=red] (0.25,4) circle (0.2 cm);
            \draw[black,thick,fill=red] (0.1,10) circle (0.2 cm);
        }
    \end{myTikzpictureGrid}
\end{minipage}
}

\myWideProblem{
    If you evaluate this function at $x=0$ on a calculator, 
    you get an error.
    Why?
}{0.25in}

\myWideProblemWithContent[Do these things:
    \begin{itemize}[nosep]
        \item Sketch a transformed function that has been
            shifted \myEmph{down} by 3. 
        \item Put a BIG DOT at the center.
        \item Draw the horizontal and vertical asymptotes 
            that cross at the center (your DOT) as DASHED LINES.
        \item Then fill in the blanks.
    \end{itemize}
    ]
{
    \begin{minipage}{0.3\textwidth}
        \centering
        \begin{myTikzpictureGrid}{0.35}{6}{6}
            \whenTEACHER{
                \tkzFct[ solid, ultra thick, samples=100, domain=-9:-0.01,]{(1/x)-3}
                \tkzFct[ solid, ultra thick, samples=100, domain=0.01:9,]{(1/x)-3}
                \draw[fill,black,fill=red] (0,-3) circle (3mm);
                \draw [red, ultra thick, dashed] (-0.1,-7) -- (-0.1,7);
                \draw [red, ultra thick, dashed] (-7,-3.1) -- (7,-3.1);
            }
        \end{myTikzpictureGrid}
    \end{minipage}
    \begin{minipage}{0.64\textwidth}
        \centering 
        \small
        \renewcommand{\arraystretch}{1.5}
        \begin{tabular}{lcl}
            coords of the vertex & : & (\gap{$0$},\gap{$-3$}) \\
            $a$, $h$, and $k$           & : & $a=$ \gap{$1$}, $h=$ \gap{$0$}, $k=$ \gap{$-3$} \\
            transformed function        & : & $g(x)=$ \gap{$\frac{1}{x} - 3 $}
        \end{tabular}
    \end{minipage}
}

\myWideProblem
{
   The word ``reciprocal'' means to \myEmph{turn upside down}.
   The reciprocal of $5$ is $\frac{1}{5}$. The reciprocal of $10$ is $\frac{1}{10}$.
   Explain why you think this function is called 
   the \myEmph{reciprocal} parent function.
}
{0.5in}

