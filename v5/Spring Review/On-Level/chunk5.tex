\section{Absolute Value Functions}

The \myEmph{absolute value parent function} is \tcbox[on line]{$f(x) = |x|$.}
\vspace{1\onelineskip}

\myWideProblemWithContent[Fill in the table, plot the points, and sketch the graph.]
{
\begin{minipage}{0.57\textwidth}
    \centering 
    \small
    \renewcommand{\arraystretch}{1.5}
    \begin{tabular}{r|p{1.5in}|c}
        $x$  & $y = f(x) = |x|$ & $(x,y)$\\ 
        \hline\hline
        $-8$ & $f(-5.5)=$ & $(\hspace{0.25in},\hspace{0.25in})$ \\
        $-3$ & $f(-3.25)=$ & $(\hspace{0.25in},\hspace{0.25in})$ \\
        $-1$ & $f(-1)=$ & $(\hspace{0.25in},\hspace{0.25in})$ \\
        $0$ & $f(0)=$ & $(\hspace{0.25in},\hspace{0.25in})$ \\
        $1$ & $f(1)=$ & $(\hspace{0.25in},\hspace{0.25in})$ \\
        $3$ & $f(3.25)=$ & $(\hspace{0.25in},\hspace{0.25in})$ \\
        $8$ & $f(8.5)=$ & $(\hspace{0.25in},\hspace{0.25in})$ \\
    \end{tabular}
\end{minipage}
\begin{minipage}{0.425\textwidth}
    \centering
    \begin{myTikzpictureGrid}{0.28} {9}{9}
        % \tkzText(13,8){$x - y = 0$}
        \whenTEACHER{
            \tkzFct[ solid, ultra thick, samples=100, domain=-9:9,]{abs(x)}
            \draw[black,thick,fill=red] (-5.5,5.5) circle (0.2 cm);
            \draw[black,thick,fill=red] (-3.25,3.25) circle (0.2 cm);
            \draw[black,thick,fill=red] (-1,1) circle (0.2 cm);
            \draw[black,thick,fill=red] (0,0) circle (0.2 cm);
            \draw[black,thick,fill=red] (1,1) circle (0.2 cm);
            \draw[black,thick,fill=red] (3.25,3.25) circle (0.2 cm);
            \draw[black,thick,fill=red] (5.5,5.5) circle (0.2 cm);
        }
    \end{myTikzpictureGrid}
\end{minipage}
}

\myWideProblem{It's possible to think of this graph as two ``half lines''. Describe what you think this means.}{0.5in}

\myWideProblemWithContent[Do these things:
    \begin{itemize}[nosep]
        \item Sketch a transformed function that has been
            \myEmph{reflected} across the $x$-axis
            and shifted \myEmph{right} by 5. 
        \item Put a BIG DOT at the vertex.
        \item Put another BIG DOT at the $y$-intercept.
        \item Then fill in the blanks.
    \end{itemize}
    ]
{
\begin{minipage}{0.3\textwidth}
    \centering
    \begin{myTikzpictureGrid}{0.25} {9}{9}
        % \tkzText(13,8){$x - y = 0$}
        \whenTEACHER{
            \tkzFct[ solid, ultra thick, samples=100, domain=-9:9,]{-abs(x-5)}
            \draw[fill,black,fill=red] (5,0) circle (3mm);
            \draw[fill,black,fill=red] (0,-5) circle (3mm);
        }
    \end{myTikzpictureGrid}
\end{minipage}
\begin{minipage}{0.64\textwidth}
    \centering 
    \small
    \renewcommand{\arraystretch}{1.5}
    \begin{tabular}{lcl}
        coords of the vertex & : & (\gap{$5$},\gap{$0$}) \\
        $a$, $h$, and $k$           & : & $a=$ \gap{$-1$}, $h=$ \gap{$5$}, $k=$ \gap{$0$} \\
        transformed function        & : & $g(x)=$ \gap{$-|x-5| $}
    \end{tabular}
\end{minipage}
}

\myWideProblem
{
    Use a TI-84 to graph both $y=-|x-5|$ and $y=-(x-5)^2$.
    (Use \myKey{Y1} and \myKey{Y2}.)
    What are \myEmph{two ways} that the graphs are \myEmph{similar}?
    {
        \footnotesize
        (For absolute value, 
        use \myKey{MATH} \fbox{\tiny\sffamily NUM} \myKey{1}.)
    }
}
{0.5in}