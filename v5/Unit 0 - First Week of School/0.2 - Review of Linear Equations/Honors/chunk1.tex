\section{Linear Equations with Two Variables}

\begin{myConceptSteps}{To convert a linear equation with {two} variables to {slope-intercept} form\dots}
    \myStep{simplify}{\gap{Distribute} where needed, and then \gap{combine} like terms.}
    \myStep{isolate}{Get $y$ by \gap{itself} using inverse operations.}
    \myStep{simplify}{\gap{Combine} like terms again to get \gap{$y=mx+b$}.}
\end{myConceptSteps}

\myProblems[Convert these equations to slope-intercept form.]
{
    $2x - 3y + 8 = 20$
    % 2x - 3y = 12 
    % -3y = -2x + 12 
    % y = -2/3 x + 4
}
{
    $\frac{1}{3}y + 2(x-1) = 3$
    % 1/3 y + 2x - 2 = 3
    % 1/3 y + 2x = 5
    % 1/3 y = -2x + 5
    % y = -6x + 15
    % 
}
{1.5in}





\begin{myConceptSteps}{To sketch the graph of a linear equation with {two} variables\dots}
    \myStep{rewrite}{Convert the equation to \gap{slope-intercept} form.}
    \myStep{$m$ and $b$}{Write down the slope \gap{$m$} and $y$-intercept \gap{$b$}. }
    \myStep{1st point}{Plot the $y$-intercept as a point on the \gap{$y$-axis}.}
    \myStep{2nd point}{Plot another point using the slope: (\gap{rise} \gap{over} \gap{run}).}
    \myStep{line}{Draw a line through the two points.}
\end{myConceptSteps}

\myProblems[Sketch the graphs of these equations.]
{
    $3x + 2y = 4$
    % 2y = -3x + 4 
    % y = -3/2 x + 2
}
{
    $3x + 10 - y = 8$
    % 3x - y = -2 
    % -y = -3x - 2 
    % y = 3x + 2
}
{1.5in}
