\section{Tranformed Quadratic Functions}

We have been exploring \gap{transformed} quadratic functions.

\begin{tcbraster}[
    raster columns = 2,
    raster equal height,
    raster left skip = 0.5in, raster right skip = 0.5in, raster column skip = 0.25in,
    raster before skip = 0.25in, raster after skip = 0.25in,
    colback=white,
]
    \begin{tcolorbox}[]
        \centering
        {\itshape parent function}\\[1\baselineskip]
        \Large
        $ f(x) = x^2 $
    \end{tcolorbox}
    \begin{tcolorbox}[]
        \centering
        {\itshape transformed function}\\[1\baselineskip]
        \Large
        $g(x) = \bm{a}(x-\bm{h})^2 + \bm{k}$
    \end{tcolorbox}
\end{tcbraster}
%

\begin{minipage}[t]{0.5\textwidth}
    \centering
    \begin{myTikzpictureGrid}{0.3} {6}{6}
        \tkzFct[ solid, ultra thick, samples=100, domain=-6:6,]{x**2}
        % \tkzText(5,7){$f(x) = x^2$}
    \end{myTikzpictureGrid}
    \\[1.5ex]
    \large
    $f(x) = x^2$
\end{minipage}
\hfill{}
\begin{minipage}[t]{0.5\textwidth}
    \centering
    \begin{myTikzpictureGrid}{0.3} {6}{6}
        \tkzFct[ solid, ultra thick, samples=100, domain=-6:6,]{-4*(x-3)**2 + 5}
        % \tkzText(2,1){$f(x) = x^2$}
    \end{myTikzpictureGrid}
    \\[1.5ex]
    \large
    $g(x) = -4(x-3)^2+5$
    {
        \small
        \begin{align*} 
            a &= -4\\
            h &= 3\\
            k &= 5
        \end{align*} 
    }
\end{minipage}

\begin{myWarningBox}
    $h$ is the \myEmph{opposite} of what you see:
    \begin{center}
    $
    g(x) = -4(x-3)^2+5 
    \qquad
    \text{\Large $\bm{\Rightarrow}$}
    \qquad
    h = +3
    $
    \end{center}
\end{myWarningBox}

