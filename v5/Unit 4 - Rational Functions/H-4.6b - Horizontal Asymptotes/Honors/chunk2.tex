\section{Finding Horizontal Asymptotes }

\begin{myConceptSteps}{To find the horizontal asymptote of a rational function\dots}
    \myStep{rewrite}{Rewrite the numerator and denominator in \gap{descending} \gap{order}}
    \myStep{degree and coefficients}{%
        Find these \gap{4} numbers:
        \begin{itemize}
            \item the \myEmph{degree} of the numerator \gap{$n$} and denominator \gap{$d$}.
            \item the \myEmph{leading coefficient} of the numerator \gap{$a$} and denominator \gap{$b$}.
        \end{itemize}
    }
    \myStep{horizontal asymptote}{%
        \begin{itemize}
            \item The horizontal asymptote is:
                \begin{itemize}
                    \item when \gap{$ \bm{n < d} $}, horizontal asymptote: \gap{$y=0$}
                    \item when \gap{$ \bm{n = d} $}, horizontal asymptote: \gap{$y = \frac{a}{b}$}
                    \item when \gap{$ \bm{n > d} $}, there is \myEmph{no} horizontal asymptote
                \end{itemize}
        \end{itemize}
    }
\end{myConceptSteps}

The key to understanding \myEmph{why} this works is to \gap{divide} 
the polynomials by \gap{$x^d$}.
