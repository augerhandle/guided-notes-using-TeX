\section{Simplifying rational expressions}

\begin{myConceptSteps}{To \myEmph{simplify} a rational expression\dots}
    \myStep{factor}{Completely \gap{factor} the numerator and denominator.}
    \myStep{domain}{%
        \begin{itemize}
            \item The \gap{excluded} \gap{values} are 
            the \gap{zeros} of the \gap{denominator}.
            \item The domain as \gap{$x \ne \text{excluded values}$}.
        \end{itemize}
    }
    \myStep{simplified function}{\gap{Cancel} common factors.}
    \myStep{result}{%
        The result has \gap{two} parts.
        \begin{itemize}
            \item the \gap{simplified} \gap{function}, and 
            \item the \gap{domain}        
        \end{itemize}
    }
\end{myConceptSteps}

\myProblems[Simplify these rational expressions. State the domain as well as the simplified function.]
    {
        \large
        $\frac{2x+4}{x^2+10x+16}$
    }
    {
        \large
        $\frac{x^2 + 8x + 15}{2x^2 + 10x}$
    }
{3.5in}
[\myAnswer{sf: $\frac{2}{x+8}$, dom: $x \ne -2, -8$}]
[\myAnswer{sf: $\frac{x+3}{2x}$, dom: $x \ne 0, -5$}]
