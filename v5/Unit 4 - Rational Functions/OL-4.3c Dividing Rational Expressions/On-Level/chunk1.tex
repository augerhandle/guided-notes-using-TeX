\section{Dividing rational expressions}


\begin{myConceptSteps}{To divide two rational expressions\dots}
    \myStep{factor}{%
        Factor the numerators and denominators.
    }
    \myStep{domain}{%
        The domain is \gap{$x \ne \text{excluded values}$}.
        The excluded values are 
        \begin{itemize}[nosep]
            \item the \gap{zeros} of the first \gap{denominator}, 
            \item the \gap{zeros} of the second \gap{denominator}, 
            \item the \gap{zeros} of the second \gap{numerator}. 
        \end{itemize}
        }
    \myStep{flip and multiply}{%
        Multiply the first expression by the \gap{reciprocal} of the second expression.
        \begin{itemize}
            \item Write the factors next to each other. \gap{Do not} distribute.
        \end{itemize}
    }
    \myStep{cancel}{%
        Cancel common factors.
    }
    \myStep{result}{%
        The result has \gap{two} parts.
        \begin{itemize}
            \item the \gap{simplified} function, and 
            \item the \gap{domain}        
        \end{itemize}
    }
\end{myConceptSteps}

\begin{tcolorbox}[center,colback=white,width=5in,]
    The domain for division problems is \gap{ tricky}
    because there is division happening in \gap{three} places.
    So be careful!
\end{tcolorbox}


\myWideProblem[Divide these rational expressions.]
        {
            $
            \frac
            {3(x+4)}
            {(x+2)(x+3)}
            \div
            \frac
            {(x+4)}
            {(x+2)}
            $
        }
{3.25in}
[\myAnswer{$\frac{3}{x+3}\quad;\quad x \ne -4, -3, -2$}]

\myWideProblem
{
    \large
    $
    \frac
    {
        \frac
        {(x-1)}
        {(x-5)}
    }
    {
        \quad
        \frac
        {(x+1)}
        {(x^2 + 2x -35)}
        \quad
    }
    $
}
{3.25in}
[\myAnswer{$\frac{(x-1)(x+7)}{(x+1)}$ ; $x\ne -7, -1, 5$}]