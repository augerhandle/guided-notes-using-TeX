\section{Back Substitution}

Once an augmented matrix is in row echelon form,
it is easy to solve the system, because the matrix is \gap{triangular}.

\begin{myConceptSteps}{To solve a system with an augmented matrix in row echelon form\dots}
  \myStep{start}{%
    Start solving at the \gap{bottom} of the matrix.
    }
  \myStep{substitute and solve}{%
      Substitute the variable values
      into the \gap{row above}.
      Solve for the next variable.
      }
  \myStep{keep going}{%
      Follow these steps to the \gap{top}.
    }
\end{myConceptSteps}

\begin{tcolorbox}[center,width=5in]
  Because we start at the \gap{bottom} and work \gap{up},
  this is called \gap{back} \gap{substitution}.
\end{tcolorbox}

\myWideProblem[
    Solve the system represented by this augmented matrix in
    row echelon form.
]
{
  \(
    \begin{bmatrix}[ccc|c]
      1 &  4 & 6 & -7\\
      0 & 1  & 2 & -3\\
      0 &  0 & 1 & 2
    \end{bmatrix}
\)
}{3in}[%
  \begin{flushright}
    \large
    $x = $ \gap{9}\\[0.5\baselineskip]
    $y = $ \gap{-7}\\[0.5\baselineskip]
    $z = $ \gap{2}
  \end{flushright}
  ]