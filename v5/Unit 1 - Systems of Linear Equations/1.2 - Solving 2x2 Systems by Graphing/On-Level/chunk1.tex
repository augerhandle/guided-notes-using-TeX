\section{Introduction}

We can look at systems
\gap{algebraically} or \gap{geometrically}:

\noindent
\begin{minipage}{0.45\textwidth}
    \begin{center}
        \large
        \begin{align*}
            x - y &= 0 \\
            x + 2y &= 6 
            % 2y = -x + 6
            % y = -0.5x + 3
        \end{align*}
    \end{center}
\end{minipage}
%
\hfil 
%
\begin{minipage}{0.45\textwidth}
    \begin{center}
        \begin{myTikzpictureGrid}{0.25} {5}{5}
            \tkzFct[ solid, ultra thick, samples=100, domain=-10:10,]{x}
            % \tkzText(13,8){$x - y = 0$}
            \tkzFct[ dotted, ultra thick, samples=100, domain=-10:10,]{-x/2 + 3}
            % \tkzText(-13,7.25){$x + 2y = 6$}
            \end{myTikzpictureGrid}
    \end{center}
\end{minipage}

\noindent 
The geometric way of looking at things can tell you \gap{how} \gap{many} solutions 
a system has. 
For example, here are three different systems.

\vfil

% \noindent
\begin{minipage}[t]{0.3\textwidth}
    \begin{center}
        {\bfseries\itshape one solution}
        \begin{align*} y &= x \\ y &= -\frac{1}{2}x + 3 \end{align*}
        \begin{myTikzpictureGrid}{0.25} {5}{5}
            \tkzFct[ solid, ultra thick, samples=100, domain=-10:10,]{x}
            \tkzFct[ dotted, ultra thick, samples=100, domain=-10:10,]{-x/2 + 3}
        \end{myTikzpictureGrid}
    \end{center}
\end{minipage}
%
\hfil 
%
\begin{minipage}[t]{0.3\textwidth}
    \begin{center}
        {\bfseries\itshape no solutions}
        \begin{align*} y &= x - 2 \\ y &= x+3 \end{align*} 
        \begin{myTikzpictureGrid}{0.25} {5}{5}
            \tkzFct[ solid, ultra thick, samples=100, domain=-10:10,]{x-2}
            \tkzFct[ dotted, ultra thick, samples=100, domain=-10:10,]{x+3}
        \end{myTikzpictureGrid}
    \end{center}
\end{minipage}
%
\hfil 
%
\begin{minipage}[t]{0.3\textwidth}
    \begin{center}
        {\bfseries\itshape infinitely many solutions}
        \begin{align*} x &= y \\ 5x &= 5y \end{align*}
        \begin{myTikzpictureGrid}{0.25} {5}{5}
            \tkzFct[ solid, ultra thick, samples=100, domain=-10:10,]{x+0.1}
            \tkzFct[ dotted, ultra thick, samples=100, domain=-10:10,]{x-0.1}
        \end{myTikzpictureGrid}
    \end{center}
\end{minipage}

\vfil

\begin{tcolorbox}[center,width=6in]
    In this lesson,
    you will learn to solve $2\times2$ systems \gap{geometrically}.
\end{tcolorbox}
