\section{Solving Systems of 2 Equations by Graphing}

\begin{myConceptSteps}{To solve a system of 2 equations by graphing\dots}
    \myStep{slope-intercept}{Rewrite both equations in {\bfseries\itshape slope-intercept} form.}
    \myStep{graph}{Sketch the graph of both lines.}
    \myStep{solution}{
            \begin{itemize}
                \item If the lines intersect at a point, 
                    the solution is the \gap{coordinates} of the point.
                \item If the lines do not intersect (same slope, different $y$-intercept), 
                    there are \gap{no} solutions.
                \item If the lines are the same,
                    there are \gap{infinitely} many solutions.
            \end{itemize}
        }
    \end{myConceptSteps}


\myProblemsWithContent[Solve these equations by graphing.]
{
    \vspace{-\onelineskip}
    \begin{align*}
        y &= 3x - 4 \\
        y &= -3x +2 
    \end{align*}  
    \begin{myTikzpictureGrid}{0.4} {5}{5}
    \end{myTikzpictureGrid}
    \whenTEACHER{
        \hfil 
        $(1,-1)$
    }
}
{
    \vspace{-\onelineskip}
    \begin{align*}
        y = 4x + 2 \\
        y + 3 = 4x  
    \end{align*}  
    \begin{myTikzpictureGrid}{0.4} {5}{5}
    \end{myTikzpictureGrid}
    \whenTEACHER{ \hfil no solution }
}[\normalsize]

\myWideProblemWithContent
{
    \vspace{-\onelineskip}
    \begin{align*}
        y = -\frac{1}{2}x + 1 \\
        2y + x = -2
    \end{align*}  
    \begin{myTikzpictureGrid}{0.4} {5}{5}
    \end{myTikzpictureGrid}
    \whenTEACHER{ \hfil inf. many solutions}
}[\normalsize]
