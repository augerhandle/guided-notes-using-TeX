\section{Reduced Row-Echelon Form}

Gaussian elimination is useful because the resulting \gap{augmented} \gap{matrix} 
is easy to solve using \gap{back}-\gap{substitution}.
But we can do much better. 
We can create augmented matrices that are \gap{diagonal}. 
These are \gap{crazy} \gap{easy} to solve.

\myProblems[Solve the linear equations represented by these augmented matrices.]
{
    $
    \begin{bmatrix}[ccc|c]
        \boldsymbol{1} & 0 & 0 & 100 \\
        0 & \boldsymbol{1} & 0 & 200 \\
        0 & 0 & \boldsymbol{1} & 300
    \end{bmatrix}
    $
}
{
    $
    \begin{bmatrix}[ccc|c]
        \boldsymbol{1} & 0 & 0 & 0.51 \\
        0 & \boldsymbol{1} & 0 & 1.234 \\
        0 & 0 & \boldsymbol{1} & \pi
    \end{bmatrix}
    $
}
{2in}