\section*{Sketch a line from slope-intercept form}

In Algebra 1, you learned how to sketch the graph
of a linear equation in \emph{slope--intercept} form.

\begin{center}
    \begin{tcolorbox}[width=4in]
        An equation in slope-intercept form looks like this:
        \[ y=mx+b \]
        where the $m$ is a number (the \emph{slope}) 
        and $b$ is a number (the \emph{$y$-intercept}).
    \end{tcolorbox}
\end{center}

\begin{myConceptSteps}{To sketch the graph of an equation from slope-intercept form, $y=mx+b$\dots}
    \myStep{plot}{Plot the $y$-intercept as a dot at $(0,b)$ on the $y$-axis.}
    \myStep{plot}{Plot another point as a dot using ``rise over run'' from the slope, $m$.}
    \myStep{connect}{Connect the points with a straight line.}
\end{myConceptSteps}




\myBlankExample{2in}{
    Sketch the graph of the following linear equation.
    \[
        y = x + 3 
    \]   
}


\myBlankExample{2.5in}{
    Sketch the graph of the following linear equation.
    \[
        y = 3x + 2 
    \]   
}


\begin{taggedblock}{on-level}
    \myBlankExample{2.5in}{
        Sketch the graph of the following linear equation.
        \[
            y = -2x + 4 
        \]   
    }
\end{taggedblock}
\begin{taggedblock}{pre-AP}
    \myBlankExample{2.5in}{
        Sketch the graph of the following linear equation.
        \[
            y = -2x - 4 
        \]   
    }
\end{taggedblock}


\begin{taggedblock}{on-level}
\myBlankExample{3in}{
    Sketch the graph of the following linear equation.
    \[
        y = \frac{1}{2}x + 1
    \]   
}
\end{taggedblock}
\begin{taggedblock}{pre-AP}
\myBlankExample{3in}{
    Sketch the graph of the following linear equation.
    \[
        y = \frac{1}{2}x - \frac{5}{2}
    \]   
}
\end{taggedblock}

