\section*{Introduction}

In lesson 1.9 you learned how to find the inverse $f^{-1}$ 
of a function $f$.
It was a lot of \emph{algebra}.
The key part of those steps was {\bfseries\itshape swapping} 
the $x$ and $y$ variables. 
But, what does that mean \emph{geometrically}? 
In this lesson, we will graph some inverse functions to find out.

You will learn two things:

\begin{center}
\begin{tcolorbox}[width=4in]
    Graphing the inverse of $f$ is really just a matter 
    of finding $f^{-1}$ first. 
    Then you just sketch $y = f^{-1}(x)$.
\end{tcolorbox}
\end{center}

\begin{center}
\begin{tcolorbox}[width=4in]
    If you visually compare the graph of a function
    to the graph of its inverse,
    you will see that they are \emph{mirror images} (reflections)
    across the diagonal line, $y=x$.
\end{tcolorbox}
\end{center}
