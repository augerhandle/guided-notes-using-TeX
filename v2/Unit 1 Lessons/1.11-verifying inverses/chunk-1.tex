\section*{Introduction}

Inverse functions \emph{undo} each other.
That means that if you pass $x$ as the input into one function,
and then you pass that output to the inverse function,
the output of that inverse function is the $x$.
In other words,
when you \emph{compose} the two functions,
you get back what you started with.
(If you start with $x$, you get $x$ back.)

\begin{center}
\begin{tcolorbox}[width=4in]
    When two functions are inverses of each other,
    if you compose them 
    in {\bfseries\itshape both directions},
    you get back the original input (usually $x$) as an output.
\end{tcolorbox}
\end{center}
%
This is true 
\emph{no matter which order} you pass the original input $x$
to $f$ and $g$.
We can draw it like this:
\[
    x 
    \Longrightarrow 
    \fbox{g} 
    \Longrightarrow 
    \fbox{f}
    \Longrightarrow 
    x
\]
\[
    x 
    \Longrightarrow 
    \fbox{f} 
    \Longrightarrow 
    \fbox{g} 
    \Longrightarrow 
    x
\]
Algebraically, this means:
\[
    (f \circ g)(x) = x
\]
\[
    (g \circ f)(x) = x,
\]
which can be rewritten like this:
\[
    f(g(x)) = x
\]
\[
    g(f(x)) = x.
\]


\begin{center}
    \begin{tcolorbox}[width=3in]
        When two functions are inverses of each other,
        \[
            f(g(x)) = x
        \]
        \[
            g(f(x)) = x.
        \]
        
    \end{tcolorbox}
    \end{center}
    