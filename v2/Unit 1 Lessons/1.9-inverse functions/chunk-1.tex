\section*{Inverse functions}

Remember that one way a relation can be represented
is as a set of ordered pairs that tell you
which value of $x$ is associated with 
which value of $y$.
Relations represented as a bunch of disconnected ordered pairs 
are called \emph{discrete functions}.

We can think of that as a \emph{mapping}
from the inputs ($x$'s) to the outputs ($y$'s).
For example the following relation 
\[  
    \{ (1,5), (2,-3), (3,3) \}
\]
can be represented by the following mappings
(one for each ordered pair)
\begin{align*}
    1 &\Longrightarrow 5
    \text{
        \small\itshape\qquad
        \dots With 1 as input we get 5 as output.
        This is from the $(1,5)$ ordered pair.
    }\\
    %
    2 &\Longrightarrow -3
    \text{
        \small\itshape\quad
        \dots With 2 as input we get -3 as output.
        This is from the $(2,-3)$ ordered pair.
    } \\
    %
    3 &\Longrightarrow 3
    \text{
        \small\itshape\qquad
        \dots With 3 as input we get 3 as output.
        This is from the $(3,3)$ ordered pair.
    }
\end{align*}

Inverse relations answer the question, 
``How do we get back to the inputs if we know the outputs?''
They {\bfseries\itshape turn the mapping around}.
They ``undo'' what the relation did.
They take you back to where you started.

In the example above,
the inverse mapping would be
\begin{align*} 
    5 &\Longrightarrow 1
    \text{
        \small\itshape\quad
        \dots With 5 we get back to 1.
    }\\
    %
    -3 &\Longrightarrow 2
    \text{
        \small\itshape\quad
        \dots With -3 we get back to 2.
    } \\
    %
    3 &\Longrightarrow 3
    \text{
        \small\itshape\quad
        \dots With 3 we get back to 3.
    }
\end{align*}
And this is a new relation with the order of the numbers \emph{reversed}:
\[  
    \{ (5,1), (-3,2), (3,3) \}
\]

So, this is what you should remember.
\begin{center}
    \begin{tcolorbox}[width=4in]
        Given a relation (or a function),
        you get the inverse by reversing (swapping, switching) 
        the order of the inputs and outputs.
        When the variables are $x$ and $y$, 
        this is {\bfseries\itshape swapping $x$ and $y$}.
    \end{tcolorbox}
\end{center}