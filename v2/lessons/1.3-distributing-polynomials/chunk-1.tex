When we ``distribute'' polynomials, we are \emph{multiplying} them together.
The only challenge in doing this
is {\bfseries\itshape being neat} so you don't lose or forget some of the terms.
The key to success is giving yourself lots of space on your paper.

The basic idea is this.
\begin{center}
\begin{tcolorbox}[width=4in]
    Multiply each of the terms in the first polynomial 
    into each of the terms in the second polynomial.
    Then combine like terms (if there are any).
\end{tcolorbox}
\end{center}

Remember that when you multiply two terms together,
you multiply the coefficients, and you add the exponents on the variables.


\begin{myConceptSteps}{To multiply a monomial and a binomial\dots}
    \myStep{multiply}{Multiply the monomial into each of the two terms in the binomial.}
    \myStep{combine}{Combine like terms if there are any.}
\end{myConceptSteps}

\myBlankExample{1.75in}{
    Multiply these two polynomials: $-3x ( 1-x)$.
}

\myBlankExample{1.75in}{
    Multiply these two polynomials: $5x^3 (2x-3)$.
}