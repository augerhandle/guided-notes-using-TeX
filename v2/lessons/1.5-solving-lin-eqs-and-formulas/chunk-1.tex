The key to success when solving equations is to get the unknown variable \emph{by itself}
on one side of the equation.

\begin{center}
\begin{tcolorbox}[width=4in]
    It is not really important whether the variable is on the left or right.
    In this class, I will assume that you understand that
    $x=55$ means the exact same thing as $55=x$.
\end{tcolorbox}
\end{center}


\begin{myConceptSteps}{To solve single-step equations\dots}
    \myStep{which operation}{Find the one operation in the equation.}
    \myStep{inverse operation}{Find the corresponding inverse operation.}
    \myStep{apply}{Apply that inverse operation to \emph{both sides of the equation}.}
    \myStep{solve}{Solve the equation for the variable. (Get the variable by itself.)}
\end{myConceptSteps}

One-step equations are crazy-simple.
We will not be spending much time on them.
Here are two examples.

\myBlankExample{1.25in}{
    Solve this equation:
    \(
        p - 6 = 4
    \)
}

\myBlankExample{1.25in}{
    Solve this equation:
    \(
        -3p = 5
    \)
}

