\section*{Sketch horizontal and vertical lines from their equations}

Many students get confused by the equations of vertical and horiztonal lines.
They struggle sketching them.


By the way, the numbers $h$ and $k$ can be positive or negative.
{\bfseries\itshape They can also be zero.}
For example, $x=0$ is the equation of a vertical line that goes through the origin.

\begin{center}
    \begin{tcolorbox}[width=5in]
        {\bfseries\itshape Horizontal} lines run left and right.
        All the points on the line have the same $y$ coordinates.
        So the equation of a horizontal line looks like
        \[ y = k \]
        where $k$ is a number.
    \end{tcolorbox}
\end{center}

\begin{myConceptSteps}{To sketch horizontal lines\dots}
    \myStep{dot}{Put a dot on the $y$-axis at $(0,k)$.}
    \myStep{draw}{Draw a horizontal line left and right through that point.}
\end{myConceptSteps}


\myBlankExample{2in}{
    Sketch the graph of $x=3$.
}

\myBlankExample{2in}{
    Sketch the graph of $x=-4$.
}

\myBlankExample{2in}{
    Sketch the graph of $x=0$.
}




Sketching vertical lines is very similar.
\begin{center}
    \begin{tcolorbox}[width=5in]
        {\bfseries\itshape Vertical} lines run up and down.
        All the points on the line have the same $x$ coordinates.
        So the equation of a horizontal line looks like
        \[ x = h \]
        where $h$ is a number.
    \end{tcolorbox}
\end{center}

\begin{myConceptSteps}{To sketch horizontal lines\dots}
    \myStep{dot}{Put a dot on the $x$-axis at $(h,0)$.}
    \myStep{draw}{Draw a vertical line up and down from that point.}
\end{myConceptSteps}


\myBlankExample{2in}{
    Sketch the graph of $y=-5$.
}

\myBlankExample{2in}{
    Sketch the graph of $y=0$.
}


