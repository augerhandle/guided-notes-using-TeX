You learned the \emph{order of operations} in Algebra 1.
It's often called PEMDAS or GEMDAS.

\begin{myConceptSteps}{To apply the GEMDAS order of operations\dots}
    \myStep{\Large\slshape G}{
        simplify {\bfseries\itshape grouping} symbols (), [], \{\}\footnote{
        There are two other grouping symbols.
        Whenever you have a {\bfseries\itshape radical}, you should write parentheses around everything inside, grouping them together:
        $\sqrt{x^2+9} \mapsto \sqrt{(x^2+9)} $.
        Whenever you have a {\bfseries\itshape fraction with a long, horizontal line}, 
        you should write parentheses around the numberator and denominator, grouping them together:
        $\frac{x^2 + 4x + 4}{x^2-4} \mapsto \frac{(x^2 + 4x + 4)}{(x^2-4)}$.
        The parentheses make it obvious that there is grouping going on.
        }
    }
    \myStep{\Large\slshape E}{
        simplify terms that have {\bfseries\itshape exponents}
    }
    \myStep{\Large\slshape MD}{
        apply {\bfseries\itshape multiplication and division} going from \emph{left to right}
    }
    \myStep{\Large\slshape AS}{
        apply {\bfseries\itshape addition and subtraction} going from \emph{left to right}
    }
\end{myConceptSteps}

\myBlankExample{1.25in}{
    Evaluate this expression:
    \(
        2 + 3\cdot5^2
    \)
}

\myBlankExample{2in}{
    Evaluate this expression:
    \(
        16 + 3(8-3)^2 + 9
    \)
}

\begin{taggedblock}{on-level}
    \myBlankExample{2in}{
        Evaluate this expression:
        \(
            1 - 3(3-5)^2 
        \)
    }
\end{taggedblock}
\begin{taggedblock}{pre-AP}
    \myBlankExample{3in}{
        Evaluate this expression:
        \(
            16 - 3(8-3)^2 + 9(1-2)^2
        \)
    }
\end{taggedblock}
            
