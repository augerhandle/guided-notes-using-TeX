\begin{myConceptSteps}{To find the slope and $y$-intercept of a line from the slope-intercept form of its equation\dots}
    \myStep{slope-intercept form}{
        Rewrite the equation in slope-intercept form,
        $y=mx+b$,
        where $m$ and $b$ are numbers (not variables).
        (Get $y$ by itself.)
    }
    \myStep{slope}{
        The slope is the coefficient, $m$, in front of $x$.
        (Th)
    }
    \myStep{$y$-intercept}{
        The $y$-intercept is the number, $b$, by itself in that equation.
    }
\end{myConceptSteps}

It is possible that you know this so well from Algebra 1
that you are wondering why we're talking about it.
If you feel that way, 
make sure that you understand all of the following (yes, simple) examples.
Some of these can be confusing.


\myBlankExample{1.5in}{
    Find the slope and $y$-intercept of the line corresponding to this function:
    $y = -2x + 4$.
}

\myBlankExample{2.5in}{
    Find the slope and $y$-intercept of the line corresponding to this function:
    $y + 2x - 1 = 5x + 2$.
}

\myBlankExample{1.5in}{
    Find the slope and $y$-intercept of the line corresponding to this function:
    $y = -\frac{3}{7}x - \frac{8}{7}$.
}

\myBlankExample{1.5in}{
    Find the slope and $y$-intercept of the line corresponding to this function:
    $y = 6.02x$.
}

\myBlankExample{1.5in}{
    Find the slope and $y$-intercept of the line corresponding to this function:
    $y = x$.
}

\myBlankExample{1.5in}{
    Find the slope and $y$-intercept of the line corresponding to this function:
    $y = 8$.
}

