\begin{taggedblock}{pre-AP}


\section*{Solving ``special case'' quadratic equations by factoring}

Some quadratic expressions are special cases 
and are easy to factor:
\begin{itemize}[itemsep=0in]
    \item {\bfseries\itshape differences of perfect squares}
    \item {\bfseries\itshape perfect square trinomials}
\end{itemize}

\begin{center}
    \begin{tcolorbox}[width=5.5in]
        The \gap{difference} of two perfect \gap{squares} can be rewritten as:
        \Large
        \[
            a^2 - b^2 = (a + b)(a-b)
        \]
    \end{tcolorbox}
\end{center}

\begin{center}
    \begin{tcolorbox}[width=5.5in]
        A \gap{perfect square} trinomial can be rewritten as the square of a binomial:
        \Large
        \begin{align*}
            a^2 + 2ab + b^2 &= (a + b)^2 \\
            a^2 - 2ab + b^2 &= (a -b)^2  
        \end{align*}
        \vspace{-1em}
    \end{tcolorbox}
\end{center}



\begin{myConceptSteps}{
    To solve special case quadratic expressions that equal zero\dots
}
\myStep{rewrite}{
    Rewrite the special case expression in factored form.
}
\myStep{solve}{
    Solve that equation using the Zero Product Property.
}
\end{myConceptSteps}

\myBlankExample{2in}{
    Solve this factored equation using the zero product property:
    \[
        4x^2 - 25  = 0
    \]
}

\myBlankExample{2in}{
    Solve this factored equation using the zero product property:
    \[
        x^2 + 6x + 9  = 0
    \]
}

\myBlankExample{2in}{
    Solve this factored equation using the zero product property:
    \[
        x^2 - 8x + 16  = 0
    \]
}

\end{taggedblock}