
\begin{taggedblock}{on-level}

\section*{Converting to vertex form}

\begin{myConceptSteps}{
    To convert a quadratic function in standard form into vertex form\dots
    }
    \myStep{$\mathbf a$, $\mathbf b$, $\mathbf c$}{
        Find $\mathbf a$, $\mathbf b$, and $\mathbf c$ from the standard form equation.
    }
    \myStep{vertex}{
        Find the coordinates of the vertex.
    }
    \myStep{$\mathbf a$, $\mathbf h$, $\mathbf k$}{
        The constant $\mathbf a$ is the {\bfseries\itshape same} in both forms.
        The constants $\mathbf h$ and $\mathbf k$ are the coordinates of the vertex.
    }
    \myStep{write the equation}{
        Knowing $\mathbf a$, $\mathbf h$, $\mathbf k$,
        you can write the vertex form as
        $y = a(x-h)^2 + k$.
    }
\end{myConceptSteps}

\myBlankExample{2.25in}{
    Convert this quadratic function in vertex form:
    \[y = 2x^2 - 12x + 22\]
}



\end{taggedblock}