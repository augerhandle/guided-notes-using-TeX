
\begin{taggedblock}{pre-AP}
\section*{Squaring a binomial}

For any binomial, it is easy to calculate what the {\bfseries\itshape square} 
of that binomial by distributing (aka, FOIL). 

\begin{center}
    \begin{tcolorbox}[width=5in]
        The square of the binomial $(a+b)$ is the sum of the following three parts:
        \vskip1em
        \begin{itemize}[itemsep=0in]
            \item the square of the first term: $a^2$,
            \item twice the product of the two terms: $2ab$, and
            \item the square of the second term: $b^2$.
        \end{itemize}
        \vskip1em
        \[ (a+b)^2 = a^2 + 2ab + b^2\]
    \end{tcolorbox}
\end{center}

\noindent
This rule is so easy, that
you should be able to apply it quickly {\bfseries\itshape without} 
explicitly distributing.

\myBlankExample{0.5in}{
    Find the square of this binomial:
    $(x + 5)$
}

\myBlankExample{0.5in}{
    Find the square of this binomial:
    $(2z - \frac{1}{3})$
}
\end{taggedblock}

