\begin{taggedblock}{pre-AP}

\section*{Completing the square}

Few trinomials are perfect squares.
But we can write them as ``almost'' perfect squares.
This is known as {\bfseries\itshape completing the square}.

\begin{myConceptSteps}{
    To complete the square for a given trinomial of the form $(x^2+bx+c)$\dots
    }
    \myStep{$x$ coefficient}{
        Take half of the coefficient of $x$ and square it.
    }
    \myStep{add and subtract}{
        Add and subtract the quantity from the previous step 
        so that the right-hand side of value of the expression does not change.
    }
    \myStep{perfect square trinomial}{
        Factor the resulting perfect square trinomial 
        and combine constant terms.
    }
\end{myConceptSteps}

\begin{center}
    \begin{tcolorbox}[width=4in]
        This procedure only works when the coefficient on $x^2$ is 1.
    \end{tcolorbox}
\end{center}

\myBlankExample{2.5in}{
    Complete the square for this trinomial:
    \[ x^2 +4x + 7 \]
}
    
\myBlankExample{2.5in}{
    Complete the square for this trinomial:
    \[ x^2 -8x -9 \]
}

    
\end{taggedblock}