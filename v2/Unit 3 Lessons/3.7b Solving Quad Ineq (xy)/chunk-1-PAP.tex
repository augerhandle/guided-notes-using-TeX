\begin{taggedblock}{pre-AP}

\section*{Thinking about boundaries in one-dimension}

You now how to sketch the graphs of simple inequalities like this.

\begin{center}
\begin{minipage}{0.4\textwidth}
    \begin{itemize}[itemsep=0in]
        \item $x <    c$
        \item $x \leq c$
    \end{itemize}
\end{minipage}
\begin{minipage}{0.4\textwidth}
    \begin{itemize}[itemsep=0in]
        \item $x >    c$
        \item $x \geq c$
    \end{itemize}
\end{minipage}
\end{center}

\noindent
You do two things:
\begin{enumerate}
    \item Draw a circle at the point $x=c$.
    \item Shade to the left or right of the circle.
\end{enumerate}

\begin{center}
    \begin{tcolorbox}[width=6in]
        \begin{itemize}
            \item The point at $x=c$ in these problems is called the \gap{boundary}.
            \item The boundary might or might not be part of the solution.
                \begin{itemize}
                    \item[$\circ$] \gap{part} when the circle is \gap{filled-in}
                    \item[$\circ$] \gap{not part} when the circle is \gap{empty}
                \end{itemize}
            \item The boundary \gap{partitions} the universe (our number line) into two parts:
                \begin{itemize}
                    \item[$\circ$] a greater-than part (to the \gap{right} of the boundary)
                    \item[$\circ$] a less-than part (to the \gap{left} of the boundary)
                \end{itemize}
            \item We shade the part that satisfies the the inequality.
        \end{itemize}
    \end{tcolorbox}
\end{center}


\myBlankExample{2in}{
    Sketch the graph of this inequality on a number line:
    \[
        x > 3
    \]
}

\end{taggedblock}