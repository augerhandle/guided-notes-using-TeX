

\section*{Vertical Shifts}

\begin{myConceptSteps}{
    To understand {\bfseries\itshape vertical shifts}\dots
    }
    \myStep{Find $\mathbf k$}{Find $k$ from the equation of the transformed function.}
    \myStep{graphs}{
        Sketch the graphs of the parent function $y=x^2$ and the transformed function.
        (Use Desmos or a TI-84 to help you).
    }
    \myStep{attributes}{
        Find the attributes of the transformed function by looking at its graph
        (domain, range, vertex, axis of symmetry).
    }
    \myStep{describe}{
        Describe what changed when the parent function was transformed.
    }
\end{myConceptSteps}

\begin{myExample}{
        For the transformed function
        {
            \LARGE 
            $y = x^2 + 3$
        }
        \begin{itemize}[itemsep=0in]
            \item {\bfseries\itshape sketch} graphs of the parent and transformed function on the same graph,
            \item {\bfseries\itshape find} the specified attributes,
            \item {\bfseries\itshape describe} how the transformed graph is different
        \end{itemize}
    }
    \begin{minipage}{0.35\textwidth}
        \vspace{2.5in}
    \end{minipage}
    \hfill
    \begin{minipage}{0.55\textwidth}
        \Large
        \begin{taggedblock}{pre-AP}
            domain:           \hfill\underline{\hspace{2in}}\\
            range:            \hfill\underline{\hspace{2in}}\\
            minimum           \hfill\underline{\hspace{2in}}\\
            maximum           \hfill\underline{\hspace{2in}}\\
        \end{taggedblock}
        vertex:           \hfill\underline{\hspace{2in}}\\
        axis of symmetry: \hfill\underline{\hspace{2in}}
        \vskip2em
        description:\\
        \underline{\hspace{3.3in}}\\
        \underline{\hspace{3.3in}}\\
        \underline{\hspace{3.3in}}\\
        \underline{\hspace{3.3in}}\\
        \underline{\hspace{3.3in}}
    \end{minipage}
\end{myExample}
