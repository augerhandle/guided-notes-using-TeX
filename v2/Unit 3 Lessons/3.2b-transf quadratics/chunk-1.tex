

\section*{Introduction}

\begin{center}
\begin{tcolorbox}[width=5.5in]
    {
        A quadratic function may be written in 
        {\bfseries\itshape vertex form}
        as follows.
        \vskip1em
        {
            \Huge
            \begin{center}
        $y = {\mathbf a} (x-{\mathbf h})^2 + {\mathbf k}$
            \end{center}
        }
        \vskip1.75em
    }
\end{tcolorbox}
\end{center}

\noindent The constants {$\mathbf a$}, {$\mathbf h$}, and {$\mathbf k$} result in 
different kinds of {\bfseries\itshape transformations} 
of the parent function:
\begin{itemize}[itemsep=0in]
    \item vertical shifts
    \item horizontal shifts
    \item vertical stretches and compressions
    \item reflections across the $x$-axis
\end{itemize}