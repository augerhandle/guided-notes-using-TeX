\begin{taggedblock}{pre-AP}

\section*{Powers of $i$}

We know that $i = \sqrt{-1}$ and $i^2 = -1$,
but what about {\bfseries\itshape other} powers of $i$?
It turns that for increasing powers of $i$
a \gap{cyclic pattern} emerges.
%
\begin{center}
\Large
\begin{tabular}{|l|l|l|}
    \toprule
    $i^0 =$ \underline{\hspace{2em}} & 
        $i^4$ = $i^3\cdot i = -i\cdot i = -i^2 = -(-1) = 1$ & 
        $i^8 = 1$ \\
    $i^1 = $ \underline{\hspace{2em}} & 
        $i^5 = i^4\cdot i=1\cdot i = i$ & 
        $i^9 = i$ \\
    $i^2 =$ \underline{\hspace{2em}}& 
        $i^6 = i^5\cdot i= i\cdot i= i^2 = -1$ & 
        $i^{10}=-1$ \\
    $i^3 = i^2\cdot i = (-1)\cdot i =$ \underline{\hspace{2em}} & 
        $i^7 = i^6\cdot i=-1\cdot i = -i$ & 
        $i^{11}=-i$ \\    
    \bottomrule
\end{tabular}
\end{center}

\begin{center}
    \begin{tcolorbox}[width=4.5in]
        When raising $i$ to any positive integer power,
        the result is always:
        \vskip1em
        \Large
        \centering $i$, $-1$, $-i$, or $1$.
    \end{tcolorbox}
\end{center}

\begin{myConceptSteps}{
    To simplify powers of $i$\dots
    }
    \myStep{divide}{
        Divide the exponent by 4.
    }
    \myStep{remainder}{
        Find the remainder:
        $R = 0, 1, 2, \text{or\,} 3$.
    }
    \myStep{$i^R$}{
        The result is $i^R$.
        \begin{itemize}[itemsep=0in]
            \item If the remainder is 0, the answer is $i^0 = 1$.
            \item If the remainder is 1, the answer is $i^1 = i$.
            \item If the remainder is 2, the answer is $i^2 = -1$.
            \item If the remainder is 3, the answer is $i^3 = -i$.
        \end{itemize}
    }
\end{myConceptSteps}
    
\myBlankExample{2in}{
    Simplify \Large $i^{13}$.
}
    
\myBlankExample{2in}{
    Simplify \Large $i^{68}$.
}

    
\end{taggedblock}