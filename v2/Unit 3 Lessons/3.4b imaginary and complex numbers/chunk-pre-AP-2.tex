\begin{taggedblock}{pre-AP}

\section*{Complex numbers as vectors}

Complex numbers can be interpreted as 2-D {\bfseries\itshape vectors}
which can be drawn in the {\bfseries\itshape complex plane},
where the $x$-axis is real numbers 
and the $y$-axis is imaginary numbers. 

\begin{center}
\begin{tcolorbox}[width=5.5in]
    Complex numbers $a + bi$ can be 
    drawn as a points $(a,b)$ 
    or arrows from the origin to $(a,b)$.
\end{tcolorbox}
\end{center}

\begin{myExample}{
    Draw the vectors corresponding to these complex numbers:
    \begin{itemize}[itemsep=0in]
        \item $(4 + 2i)$
        \item $(-2 + 3i)$
        \item $(3 -i)$
    \end{itemize}
}
\begin{center}
\resizebox{3.5in}{!}{%
\begin{tikzpicture}
    \begin{scope}[thick,font=\small]
    % Axes:
    % Are simply drawn using line with the `->` option to make them arrows:
    % The main labels of the axes can be places using `node`s:
    \draw [->] (-5,0) -- (5,0) node [above right]  {\large $\Re\{z\}$};
    \draw [->] (0,-5) -- (0,5) node [above right] {\large $\Im\{z\}$};

    % Axes labels:
    % Are drawn using small lines and labeled with `node`s. The placement can be set using options
    \iffalse% Single
    % If you only want a single label per axis side:
    \draw (1,-3pt) -- (1,3pt)   node [above] {$1$};
    \draw (-1,-3pt) -- (-1,3pt) node [above] {$-1$};
    \draw (-3pt,1) -- (3pt,1)   node [right] {$i$};
    \draw (-3pt,-1) -- (3pt,-1) node [right] {$-i$};
    \else% Multiple
    % If you want labels at every unit step:
    \foreach \n in {-4,...,-1,1,2,...,4}{%
        \draw (\n,-3pt) -- (\n,3pt)   node [above] {$\n$};
        \draw (-3pt,\n) -- (3pt,\n)   node [right] {$\n i$};
    }
    \fi
    \end{scope}
\end{tikzpicture}
}
\end{center}
\end{myExample}

\begin{myExample}{
        Find the complex numbers corresponding to these points/vectors drawn below.
    }
    \begin{center}
        \resizebox{3.5in}{!}{%
        \begin{tikzpicture}
            \begin{scope}[thick,font=\small]
            % Axes:
            % Are simply drawn using line with the `->` option to make them arrows:
            % The main labels of the axes can be places using `node`s:
            \draw [->] (-5,0) -- (5,0) node [above right]  {\large $\Re\{z\}$};
            \draw [->] (0,-5) -- (0,5) node [above right] {\large $\Im\{z\}$};
            %
            % Axes labels:
            % Are drawn using small lines and labeled with `node`s. The placement can be set using options
            \iffalse% Single
            % If you only want a single label per axis side:
            \draw (1,-3pt) -- (1,3pt)   node [above] {$1$};
            \draw (-1,-3pt) -- (-1,3pt) node [above] {$-1$};
            \draw (-3pt,1) -- (3pt,1)   node [right] {$i$};
            \draw (-3pt,-1) -- (3pt,-1) node [right] {$-i$};
            \else% Multiple
            % If you want labels at every unit step:
            \foreach \n in {-4,...,-1,1,2,...,4}{%
                \draw (\n,-3pt) -- (\n,3pt)   node [above] {$\n$};
                \draw (-3pt,\n) -- (3pt,\n)   node [right] {$\n i$};
            }
            \fi
            %
            \node at (3,3)[circle,fill,inner sep=2pt]{};
            \draw [-{>[scale=2.5,length=3,width=3]},line width=1.5pt] (0,0) -- (3,3);
            %
            \node at (-2,4)[circle,fill,inner sep=2pt]{};
            \draw [-{>[scale=2.5,length=3,width=3]},line width=1.5pt] (0,0) -- (-2,4);
            %
            \node at (-1,-4)[circle,fill,inner sep=2pt]{};
            \draw [-{>[scale=2.5,length=3,width=3]},line width=1.5pt] (0,0) -- (-1,-4);
            \end{scope}
        \end{tikzpicture}
        }
        \end{center}
    \end{myExample}


\end{taggedblock}