\newcommand{\myEmph}{\bfseries\itshape}
\newcommand{\myClassName}{{\tagged{pre-AP}{pre-AP }}Algebra 2}

% So I can save/restore \fboxsep
\newlength{\mySavedFboxsep}
\newcommand{\mySaveFboxsep}{\setlength{\mySavedFboxsep}{\fboxsep}}
\newcommand{\mySaveAndSetFboxsep}[1]{
    \setlength{\mySavedFboxsep}{\fboxsep}
    \setlength{\fboxsep}{#1}
}
\newcommand{\myRestoreFboxsep}{\setlength{\fboxsep}{\mySavedFboxsep}}

% A centered tcolorbox
%
% #1 - options to pass to tcolorbox
%
\NewDocumentEnvironment{myCenteredBox}{m}{%
    \begin{center}
    \begin{tcolorbox}[#1]
}{
    \end{tcolorbox}
    \end{center}
}


% A centered system of equations
%
\NewDocumentCommand{\myCenteredSysteme}{m}{%
    \begin{center}\systeme{#1}\end{center}
}

%
% This specialized command is my way of typesetting a table for
% students to use when solving systems of equations using matrices.
%
% - I make it really wide, because I need horizontal space. The increase in margin width 
%   is adjustable, but frankly, there are a lot of hard-coded dimensions in the table, so
%   I'm not positive that generality works well.
%
% - I put the content in a tikz picture with an OPAQUE background, since I 
%   plan to overlay this on top of Examples, which have dotted boxes around 
%   them at the "normal" margins.
%
% - The table uses the multirow package so that I can have the "Solution" box span two cells.
%
\NewDocumentCommand{\myWideMatrixTable}{O{-0.7in}}{
    \begin{adjustwidth}{#1}{#1}
        \begin{tikzpicture}
            \node
            [
                text width=1.25\textwidth, %I dinked with the multiplier to get balanced margins
                fill=white!30, 
                fill opacity=1,
                text opacity=1,
                inner sep=0pt,
            ]
            {%
                \begin{tabular}{|l|m{1.8in}|m{1.8in}|m{2.1in}|}
                    \hline
                    {\bfseries\scshape Variables:} & {\bfseries\scshape Equations:} & {\bfseries\scshape System:} & {\bfseries\scshape Matrices:} \\
                    % & & & \\
                    & & & 
                    \(
                        A = 
                        \begin{bmatrix}
                            \phantom{99} & \phantom{99} & \phantom{99} \\
                            \phantom{99} & \phantom{99} & \phantom{99} \\
                            \phantom{99} & \phantom{99} & \phantom{99} \\
                            \phantom{99} & \phantom{99} & \phantom{99} \\
                        \end{bmatrix}
                    \)
                    \(
                        B = 
                        \begin{bmatrix}
                            \phantom{999}\\
                            \phantom{999}\\
                            \phantom{999}\\
                            \phantom{999}\\
                    \end{bmatrix}
                    \)
                    \\
                    & & &
                    \(
                        A^{-1}\cdot B = 
                        \begin{bmatrix}
                            \phantom{9999}\\
                            \phantom{999}\\
                            \phantom{999}\\
                            \phantom{999}\\
                    \end{bmatrix}
                    \)
                    \\
                    & & & \\ \cline{3-4}
                    & & 
                    \multicolumn{2}{l|}{\bfseries\scshape Solution:}
                    \\ 
                    & & 
                    \multicolumn{2}{l|}{\,}
                    \\ 
                    \hline
                \end{tabular}
            };
        \end{tikzpicture}
    \end{adjustwidth}
}
