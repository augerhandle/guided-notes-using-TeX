%-------- LESSONS --------------

\newcounter{myLessonCounter}[chapter] 
\newcommand{\myCurrentLessonTitle}{}
\newcommand{\myFullLessonNumber}{}  % for example 2.3b

% #1 - lesson title
% #2 - optional lesson number (default increments from previous)
% #3 - optional lesson number suffix (eg, 'a' or 'b')
%
\NewDocumentCommand{\myLesson}{ m o O{} }{
    \cleartorecto
    \renewcommand{\myCurrentLessonTitle}{#1}
    \IfValueTF{#2}
        {\setcounter{myLessonCounter}{#2}}
        {\stepcounter{myLessonCounter}}
    \renewcommand{\myFullLessonNumber}{\thechapter.\themyLessonCounter#3}
    \noindent
    \begin{minipage}[b]{0.25\textwidth}
        {
            \HUGE\bfseries\sffamily\myFullLessonNumber
        }
    \end{minipage}
    \begin{minipage}[b]{0.74\textwidth}
        \begin{flushright}
            {\sffamily\chaptername\,\thechapter\,\,\myCurrentChapterTitle}\\
            \vspace{0.75\baselineskip}
            {\Huge\bfseries\sffamily#1}
        \end{flushright}
    \end{minipage}
    
    \noindent\rule{\textwidth}{1.25pt}
    \par
}

%-------- ANNOTATIONS (in boxes) --------------
%
% I use the term "annotations" to capture the common
% infrastructure I use to define Objectives, Vobabulary & Concepts.
%

% font and styling commands for
% Objectives, Voculary, Key Concepts, etc...
\newcommand{\myAnnotationStyling}{\bfseries\large}

% #1 : name of the kind of annotation (Objectives, ...)
% #2 : title text to go with the annotation
\NewDocumentEnvironment{myAnnotate}{ m m }{
    \vspace{1.25em}
    \begin{tcolorbox}[
        colbacktitle=blue!15!white,
        colback=white,
        coltitle=black,
        fonttitle={\myAnnotationStyling},
        title={#1: },
        after title={\normalfont\itshape#2},
        ]
}{
    \end{tcolorbox}
}

% #1 : name of the kind of annotation 
% #2 : title 
\NewDocumentEnvironment{myTabularAnnotate}{ m m }{
    \begin{myAnnotate}{#1}{#2}
    \begin{tabular}{r|l}
}{
    \end{tabular}
    \end{myAnnotate}
}
% #1 - column 1 text
% #2 - column 2 text
\NewDocumentCommand{\myRow}{mm}{{\bfseries\itshape \textcolor{blue}{#1}}&#2\\[1ex]}

% #1 : name of the kind of annotation 
% #2 : title 
% #3 : text before the list starts
\NewDocumentEnvironment{myListAnnotate}{ m m o }{
    \begin{myAnnotate}{#1}{#2}
    \IfValueT{#3}{#3}
    \begin{enumerate}[itemsep=0pt,]
}{
    \end{enumerate}
    \end{myAnnotate}
}
% #1 - column 1 text
% #2 - column 2 text
\NewDocumentCommand{\myItem}{mm}{\item{\bfseries\itshape \textcolor{blue}{#1}} #2}



%----------- OBJECTIVES , VOCABULARY, CONCEPTS ---------------
\newenvironment{myObjectives}{
    \begin{myListAnnotate}{Objectives}{}
}{
    \end{myListAnnotate}
}
\newcommand{\myObjective}[2]{
    \myItem{#1}{#2}
}

\newenvironment{myVocabulary}{
    \begin{myTabularAnnotate}{Vocabulary}{}
}{
    \end{myTabularAnnotate}
}
\newcommand{\myDefinition}[2]{
    \myRow{#1}{#2}
}

\NewDocumentEnvironment{myConcept}{m}{
    \begin{myAnnotate}{Key Concept}{#1}
}{
    \end{myAnnotate}
}

\NewDocumentEnvironment{myConceptSteps}{m}{
    \begin{myListAnnotate}{Key Concept}{#1}[Follow these steps.]
}{
    \end{myListAnnotate}
}
\newcommand{\myStep}[2]{
    \myItem{#1.}{#2}
}

%----------- EXAMPLEs ---------------
\newcounter{myExampleCounter}[myLessonCounter] 

% #1 A statement of the example problem.
%
\NewDocumentEnvironment{myExample}{m}{%
    \vspace{1.25em}
    \begin{tcolorbox}[
        enhanced,
        sharp corners, 
        colback=white,
        boxrule=0pt,
        borderline={0.5pt}{0pt}{black,dashed},
        ]
        \stepcounter{myExampleCounter}
        {\myAnnotationStyling Example \themyExampleCounter:} #1
        \tcblower
}{
    \end{tcolorbox}
}

% #1 amount of vspace for the blank space
% #2 A statement of the example problem
\NewDocumentCommand{\myBlankExample}{mm}{
    \begin{myExample}{#2}
        \vspace{#1}
    \end{myExample}
}