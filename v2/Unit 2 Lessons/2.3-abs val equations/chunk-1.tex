

\section*{Solving Absolute Value Equations}

An {\bfseries\itshape absolute value equation} is an equation that contains absolute value expressions. 
For us there will only be one such expression, although there will generally be other numbers in the equation.
Here are some examples.
\begin{itemize}[itemsep=0in]
    \item $|x| = 2$
    \item $|x+1| = 3$
    \item $|5x-3| = 4$
    \item $|6x+5| -3 = 5$
    \item $2|4x+5| -3 = 5$
\end{itemize}

In this lesson, you will learn to solve these equations.
The solution will often have {\bfseries\itshape two} answers, 
but sometimes it will have {\bfseries\itshape one} or {\bfseries\itshape none}.

\begin{myConceptSteps}{To solve absolute value equations\dots}
    \myStep{isolate}{
        Get the absolute value by itself. (Move everything else to the other side of the equation.)
    }
    \myStep{check}{
        If the absolute value equals a negative number, there is {\bfseries\itshape no solution.}
        Make sure to check for this before you go further, otherwise you will get a bogus (extraneous) solution
        that will be incorrect.
    }
    \myStep{split}{
        Rewrite the absolute value equation as {\bfseries\itshape two} equations.
        If $E$ is an algebraic expression (like $5x-3$) and $c$ is a number, 
        you split into two equations like this:
        \[
            |E| = c
            \qquad\Longrightarrow\qquad
            \begin{cases}
                E = c \\
                -E = c
            \end{cases}
        \]
    }
    \myStep{solve twice}{Solve each of the two equations you got in the previous step.}
\end{myConceptSteps}

\begin{center}
    \begin{tcolorbox}[width=4in]
        After you split the absolute value equation,
        there are {\bfseries\itshape no} absolute values left.
    \end{tcolorbox}
\end{center}





\subsection*{Answers with two solutions}

In most cases, there will be two solutions to absolute value equations.

\myBlankExample{1in}{
    Solve this equation:
    $|x| = 4$
}

\myBlankExample{1in}{
    Solve this equation:
    $|2s| = 4$
}

\myBlankExample{1.5in}{
    Solve this equation:
    $|x - 1| = 4$
}


\myBlankExample{1.5in}{
    Solve this equation:
    $|2t + 3| = 13$
}


\myBlankExample{2.5in}{
    Solve this equation:
    $-4|3x-1| = -11$
}


\begin{taggedblock}{pre-AP}
    \myBlankExample{2.5in}{
        Solve this equation:
        $ 2 - 5|5z - 5| = -73$
    }
\end{taggedblock}

\begin{center}
    \begin{tcolorbox}[width=4in]
        Absolute value equations usually have {\bfseries\itshape two solutions}
        because you split the original equation into two equations,
        each of which has a solution.
    \end{tcolorbox}
\end{center}



\subsection*{Answers with one solution}

In some cases, when split into two equations, those two equations will be the \emph{same thing}.
In those cases, there is really only one equation to solve, so the solution will only have a single solution.
Here are some examples.

\begin{taggedblock}{on-level}
    \myBlankExample{1in}{
        Solve this equation:
        $|x| = 0$
    }
\end{taggedblock}
\begin{taggedblock}{pre-AP}
    \myBlankExample{2in}{
        Solve this equation:
        \large
        $\left|   \frac{3}{5}x   \right| = 0$
    }
\end{taggedblock}

\myBlankExample{2in}{
    Solve this equation:
    $| 5x - 10| + 8 = 8$
}

\begin{taggedblock}{pre-AP}
    \myBlankExample{2in}{
        Solve this equation:
        $7 + |x-8| = 7$
    }
\end{taggedblock}

\begin{center}
    \begin{tcolorbox}[width=4in]
        Notice that you only get one answer when the number on the right 
        of the equation is {\bfseries\itshape zero}
        after you have isolated the absolute value expression.
    \end{tcolorbox}
\end{center}







\subsection*{Answers with no solutions}

In the second step of the procedure
(after you isolate the absolute value expression),
I told you to check if the number on the other side of the equation is negative.
This can never happen.
No value of the variable will make the absolute value expression negative,
because absolute values are {\bfseries\itshape always positive}. 
In this case, you say that there is {\bfseries\itshape no solution}. 
Here are some examples of that.

\myBlankExample{1in}{
    Solve this equation:
    $ |x| = -3$
}

\myBlankExample{1in}{
    Solve this equation:
    $ |6x+234| = -3$
}

\myBlankExample{1.25in}{
    Solve this equation:
    $ -2|x+3| = 6$
}

\myBlankExample{1.5in}{
    Solve this equation:
    $ 2|3x+3| + 10 = 5$
}

\begin{taggedblock}{pre-AP}
    \myBlankExample{2in}{
        Solve this equation:
        \large
        $ \frac{5}{2}\left|3x-\frac{1}{2}\right| + 10 = 5$
    }
\end{taggedblock}

\begin{center}
    \begin{tcolorbox}[width=4in]
        An absolute value equation will have {\bfseries\itshape no solutions}
        whenever the \emph{isolated} absolute value expression
        is equal to a negative number.
    \end{tcolorbox}
\end{center}

