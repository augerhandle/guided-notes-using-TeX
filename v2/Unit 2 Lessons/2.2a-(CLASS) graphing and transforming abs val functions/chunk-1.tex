

\section*{The Absolute Value Parent Function}

In the last lesson, you learned how to evaluate absolute values.
In this lesson, you will begin studying 
{\bfseries\itshape absolute value functions},
functions that use absolute values.
Here are some examples of absolute value functions:
\begin{itemize}
    \item $f(x) = |x +1|$
    \item $g(x) = |x - 3|$
    \item $h(x) = 2|x-5|$
    \item $q(t) = |x+2| - 4$
\end{itemize}
%
The {\bfseries\itshape absolute value parent function} 
is the simplest function that involves an absolute value.

\begin{center}
    \begin{tcolorbox}
        The {\bfseries\itshape absolute value parent function} is written like this:
        \[
            f(x) = |x|
        \]
        where the expression on the right is the absolute value that we talked about 
        in the last lesson.
    \end{tcolorbox}
\end{center}



\begin{myExample}{
    Using an $x$-$y$ table,
    sketch the graph of the absolute value parent function,
    $f(x) = |x|$.
}
    \huge
    \begin{minipage}{0.49\textwidth}
        \centering
        \begin{tabular}{r|l}
            $x$ & \quad $y=|x|$ \\
            \midrule
            3 & \\ 
            2 & \\
            1 & \\
            0 & \\
            -1 & \\
            -2 & \\
            -3 & 
        \end{tabular}
    \end{minipage}
    %
    %
    \begin{minipage}{0.49\textwidth}
        % \begin{tikzpicture}
        %     \begin{axis}[
        %         width=3in,
        %         grid=both,
        %         axis x line = middle,axis y line = middle,
        %         axis equal image,
        %         % xtick distance = 2, ytick distance = 2,
        %         xmin = -4, xmax = 4,
        %         ymin = -4, ymax = 4,
        %         minor tick num = 1,
        %         yticklabels={,,},
        %         xticklabels={,,},
        %         ]
        %         % \addplot[
        %         %     no marks,
        %         %     mark size = 0.1cm,
        %         %     ] expression { abs(x) };
        %     \end{axis}
        % \end{tikzpicture}
        \end{minipage}
\end{myExample}

\begin{center}
    \begin{tcolorbox}[width=5in]
        The graph of the absolute value parent function looks like a letter {\sffamily\bfseries V}.
        \vskip1em
        You should be able to quickly sketch this graph in homework or on a test.
    \end{tcolorbox}
\end{center}


\begin{myConceptSteps}{To sketch the graph of the absolute value parent function\dots}
    \myStep{vertex}{Draw a dot at the coordinates $(0,0)$. (This is the \emph{vertex} of the graph.)}
    \myStep{right branch}{
        Starting from the vertex, draw a line with 
        rise:\,{\bfseries\itshape up 1}, run:\,{\bfseries\itshape right 1}. 
    }
    \myStep{left branch}{
        Starting from the vertex again, draw a line with 
        rise:\,{\bfseries\itshape up 1}, run:\,{\bfseries\itshape left 1}. 
    }
\end{myConceptSteps}


\myBlankExample{2.5in}{
    Sketch the graph of the absolute value parent function,
    $f(x) = |x|$.
}
