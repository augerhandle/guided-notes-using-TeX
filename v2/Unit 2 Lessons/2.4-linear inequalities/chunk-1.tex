

\begin{myConcept}{Solving linear inequalities and sketch the solution on a number line.}
    Solving linear inequalities
    is very similar to solving linear equations.
    The basic idea is to get the variable by itself {\bfseries\itshape on the left}
    using inverse operations.

    However, there are a few differences:
    \begin{itemize}
        \item {\bfseries\itshape variable on the left.}
        With equations, we don't care whether the variable is by itself on the left or on the right. 
        But for inequalities, you will get confused unless you solve for the variable on the left.
        \item {\bfseries\itshape flipping the inequality.}
        If the variable does end up on the right side (where you don't want it),
        flip the inequality (including the inequality symbol).
        \item {\bfseries\itshape multiply/divide by negatives?}
        If you multiply or divide an inequality by a negative number,
        you must reverse the inequality symbol (make it point the other way).
        \item {\bfseries\itshape circles and arrows.}
        The graph of the solution will generally be a circle (filled-in or empty) 
        with an arrow (pointing either right or left).
        \begin{itemize}
            \item The circle is filled in ({$\bullet$}) if the inequality is $\le$ or $\ge$.
            \item The circle is empty ({$\circ$}) if the inequality is $<$ or $>$.
            \item The arrow points in the same direction as the inequality symbol.
        \end{itemize}
    \end{itemize}
\end{myConcept}

\begin{center}
\begin{tcolorbox}[width=4in]
    Inequality symbols and number lines are related as follows.
    \begin{center}
        \huge
        \begin{tabular}{c|l}
            $>$     &   \quad\,\, $\circ\kern-0.5em\longrightarrow$   \\
            $\ge$   &   \quad\,\, $\bullet\kern-0.5em\longrightarrow$   \\
            $<$     &   $\longleftarrow\kern-0.5em\circ$  \\
            $\le$   &   $\longleftarrow\kern-0.5em\bullet$
        \end{tabular}
    \end{center}
    But this only works if the variable is {\bfseries\itshape on the left}!
\end{tcolorbox}
\end{center}


\subsection*{Single-step inequalities}

\myBlankExample{1in}{
    Solve this inequality and graph the solution on a number line:
    \[ 3x < 6 \]
}

\myBlankExample{1in}{
    Solve this inequality and graph the solution on a number line:
    \[ x + 5 > 6 \]
}

\myBlankExample{1in}{
    Solve this inequality and graph the solution on a number line:
    \[ x + 5 \ge 6 \]
}

\myBlankExample{1in}{
    Solve this inequality and graph the solution on a number line:
    \[ 5x \le -15 \]
}

\begin{taggedblock}{on-line}
\myBlankExample{1in}{
    Solve this inequality and graph the solution on a number line:
    \[ 20 \le 4x \]
}
\end{taggedblock}
\begin{taggedblock}{pre-AP}
    \myBlankExample{1in}{
        Solve this inequality and graph the solution on a number line:
        \[ \frac{1}{3} \le \frac{1}{4}z \]
    }
\end{taggedblock}
    








