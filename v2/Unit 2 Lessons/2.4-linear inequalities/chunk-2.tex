\section*{Solving Inequalities}

\begin{myConcept}{Solving linear inequalities and sketch the solution on a number line.}
    Solving linear inequalities
    is very similar to solving linear equations.
    The basic idea is to get the variable by itself {\bfseries\itshape on the left}
    using inverse operations.

    However, there are a few differences:
    \begin{itemize}
        \item {\bfseries\itshape variable on the left.}
        Always solve for the variable on the left side of the inequality.
        \item {\bfseries\itshape flip the inequality.}
        If the variable does end up on the right (where you don't want it),
        flip the inequality (including the inequality symbol).
        \item {\bfseries\itshape multiply/divide by negatives?}
        If you multiply or divide by a {\bfseries\itshape negative number},
        you must reverse the inequality symbol (make it point the other way).
    \end{itemize}
\end{myConcept}


\subsection*{Single-step inequalities}

\begin{taggedblock}{on-level}
\myBlankExample{0.75in}{
    Solve this inequality and graph the solution on a number line:
    \[ 3x < 6 \]
}
\end{taggedblock}

\myBlankExample{0.75in}{
    Solve this inequality and graph the solution on a number line:
    \[ 6 \leq x + 5 \]
}


\myBlankExample{1in}{
    Solve this inequality and graph the solution on a number line:
    \[ -5x \geq 15 \]
}

\begin{taggedblock}{on-line}
\myBlankExample{1in}{
    Solve this inequality and graph the solution on a number line:
    \[ 20 \le 4x \]
}
\end{taggedblock}
\begin{taggedblock}{pre-AP}
    \myBlankExample{1.5in}{
        Solve this inequality and graph the solution on a number line:
        \[ \frac{1}{3} \le \frac{1}{6}z \]
    }
\end{taggedblock}
    








