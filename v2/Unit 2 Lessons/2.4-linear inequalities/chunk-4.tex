
\subsection*{What do you do when the variable cancels out?}

If the variable cancels out entirely, you will be left with an inequality that is just numbers. 
What do you do then?
\begin{center}
    \begin{tcolorbox}[width=4in]
        If the variable cancels out
        and the remaining inequality is a {\bfseries\itshape true statement}
        (like $5 \le 23$),
        then the answer is 
        \vskip1em
        \centering\LARGE{\bfseries\itshape all real numbers}.
    \end{tcolorbox}
\end{center}

\begin{center}
    \begin{tcolorbox}[width=4in]
        If the variable cancels out
        and the remaining inequality is a {\bfseries\itshape false statement}
        (like $100 < 8$),
        then the answer is 
        \vskip1em
        \centering\LARGE{\bfseries\itshape no solution}.
    \end{tcolorbox}
\end{center}

\myBlankExample{3in}{
    Solve this inequality and graph the solution on a number line:
    \[
        4x < x + 7 + \frac{1}{3}(9x - 3)
    \]
}






\myBlankExample{3in}{
    Solve this inequality and graph the solution on a number line:
    \[
        3x \ge x + 5 + 2(x+3)
    \]
}

