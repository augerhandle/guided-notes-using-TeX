

\subsection*{What do you do when the variable cancels out?}

If the variable cancels out entirely, you will be left with an inequality that is just numbers. 
What do you do then?
\begin{center}
    \begin{tcolorbox}[width=4in]
        If the variable cancels out
        and the remaining inequality is a {\bfseries\itshape true statement}
        (like $5 \le 23$),
        then the answer is 
        \vskip1em
        \centering\LARGE{\bfseries\itshape all real numbers}.
    \end{tcolorbox}
\end{center}

\myBlankExample{1in}{
    Solve this inequality and graph the solution on a number line:
    \[
        x < x + 5
    \]
}

\begin{taggedblock}{on-level}
    \myBlankExample{2in}{
        Solve this inequality and graph the solution on a number line:
        \[
            5(z-2) <  5z + 18
        \]
    }
\end{taggedblock}
\begin{taggedblock}{pre-AP}
    \myBlankExample{4in}{
        Solve this inequality and graph the solution on a number line:
        \[
            -2(2-2q) - 4(q+5) \le -24
        \]
    }
\end{taggedblock}




\begin{center}
    \begin{tcolorbox}[width=4in]
        If the variable cancels out
        and the remaining inequality is a {\bfseries\itshape false statement}
        (like $100 < 8$),
        then the answer is 
        \vskip1em
        \centering\LARGE{\bfseries\itshape no solution}.
    \end{tcolorbox}
\end{center}


\myBlankExample{1in}{
    Solve this inequality and graph the solution on a number line:
    \[
        x \ge x + 5
    \]
}

\begin{taggedblock}{on-level}
    \myBlankExample{2in}{
        Solve this inequality and graph the solution on a number line:
        \[
            3(s-2) \ge  3s + 33
        \]
    }
\end{taggedblock}
\begin{taggedblock}{pre-AP}
    \myBlankExample{4in}{
        Solve this inequality and graph the solution on a number line:
        \[
            -6(1+7k) + 7(1+6k) \le -2
        \]
    }
\end{taggedblock}
