

\section*{The Four Transformations}

You know what \emph{transformation} means from Geometry, 
where you studied ``translations'' and ``reflections''.
We will study four general kinds of transformations this year:
\begin{itemize}[itemsep=0in]
    \item horizontal shifts (or translations)
    \item vertical shifts (or translations)
    \item reflections across the $x$-axis
    \item vertical stretches/compressions
\end{itemize}

We will use these to transform (morph) \emph{parent functions} into \emph{transformed functions}.
Today I will illustrate these four transformations with the absolute value parent function.
But we will use these same transformations over and over for the rest of the year.

\myBlankExample{1in}{
    What are the four general kinds of transformations that we will study this year?
}




\newpage
\subsection*{Horizontal Shifts}

\begin{center}
    \begin{tcolorbox}[width=4in]
        A {\bfseries\itshape horizontal shift} 
        moves all the points on the graph of the parent function
        to the {\bfseries\itshape left or right}.
    \end{tcolorbox}
\end{center}

\myBlankExample{1in}{Explain what a ``horizontal shift'' is.}

\myBlankExample{3in}{
    Sketch the following:
    \begin{itemize}[itemsep=0in]
        \item the graph of the absolute value parent function
        \item the graph after the parent has been horizontally shifted right by 3
        \item the graph after the parent has been horizontally shifted left by 4
    \end{itemize}
    Your answer should have three graphs on it.
}