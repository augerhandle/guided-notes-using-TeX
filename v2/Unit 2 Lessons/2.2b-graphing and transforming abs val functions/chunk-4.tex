\newpage
\subsection*{Vertical stretches/compressions}

\begin{center}
    \begin{tcolorbox}[width=4in]
        A {\bfseries\itshape vertical stretch or compression} 
        multiplies the $y$-coordinates of all the points on the graph of the parent function
        by a number.
        (It leaves the $x$-coordinates alone.)
        \vskip1em
        When the number is greater than 1, the points \emph{get stretched away from the $x$-axis}.
        \vskip1em
        When the number is less than 1, the points \emph{get compressed toward from the $x$-axis}.
    \end{tcolorbox}
\end{center}

\myBlankExample{0.5in}{Explain what a ``vertical stretch'' is.}
\myBlankExample{0.5in}{Explain what a ``vertical compression'' is.}

\myBlankExample{2.5in}{
    Sketch the following:
    \begin{itemize}[itemsep=0in]
        \item the graph of the absolute value parent function
        \item the graph after the parent has been vertically stretched by 2.
        \item the graph after the parent has been vertically compressed by $\frac{1}{2}$.
    \end{itemize}
    Your answer should have three graphs on it.
}
