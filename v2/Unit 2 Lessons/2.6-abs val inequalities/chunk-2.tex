

\section*{Absolute value inequalities of the form $| ax + b | > c$}

\begin{center}
    \begin{tcolorbox}[width=3.5in]
        Everything we discuss in this section applies for $>$ as well as $\geq$.
    \end{tcolorbox}
\end{center}

Remember that $|E|$ is the distance of $E$ from the origin.
Then $| ax + b |$ is the distance of $ax+b$ from the origin.
And thus, the inequality $|ax+b| > c$ says that the distance of $ax+b$ from the origin is greater than $c$.
But a number can be far away from the origin by being far to the left,
{\bfseries\itshape or} it can be far away from the origin by being far to the right.
The word {\bfseries\itshape or} there is important.
It allows us to rewrite the absolute value inequality as a compound inequality using the word ``or''.


\begin{myConceptSteps}{To solve inequalities of the form $| ax + b | > c$\dots}
    \myStep{rewrite}{
        Rewrite 
        \fbox{
            $| ax + b | > c$
        }
        as  
        \fbox{
                \(
                ax + b > c
                \quad\text{\bfseries\itshape or}\quad
                ax + b < -c
            \)
        }
    }
    \myStep{solve}{
        Solve that compound inequality. 
        The solution to that compound inequality is the solution to the absolute value inequality.
    }
\end{myConceptSteps}

\begin{center}
    \begin{tcolorbox}[width=4in]
        In more difficult problems,
        you might have to {\bfseries\itshape isolate} the absolute value expression
        before you can apply those steps.
        \vskip1em
        Get the absolute value by itself first!
    \end{tcolorbox}
\end{center}


\myBlankExample{1.5in}{
    Solve this absolute value inequality and sketch its graph on a number line.
    \[
        |x+3| \geq 4
    \]
}


\begin{taggedblock}{on-level}
    \myBlankExample{2in}{
        Solve this absolute value inequality and sketch its graph on a number line.
        \[
            |-6x| \geq 12
        \]
    }
\end{taggedblock}
\begin{taggedblock}{pre-AP}
    \myBlankExample{2.5in}{
        Solve this absolute value inequality and sketch its graph on a number line.
        \[
            |5-7p| > 75
        \]
    }
\end{taggedblock}


\begin{taggedblock}{on-level}
    \myBlankExample{2in}{
        Solve this absolute value inequality and sketch its graph on a number line.
        \[
            |8 + 3n| > 23
        \]
    }
\end{taggedblock}
\begin{taggedblock}{pre-AP}
    \myBlankExample{2.75in}{
        Solve this absolute value inequality and sketch its graph on a number line.
        \[
            |4-7b| + 1 > 18
        \]
    }
\end{taggedblock}


\begin{taggedblock}{on-level}
    \myBlankExample{4in}{
        Solve this absolute value inequality and sketch its graph on a number line.
        \[
            |4-7b| + 1 > 18
        \]
    }
\end{taggedblock}
\begin{taggedblock}{pre-AP}
    \myBlankExample{3.25in}{
        Solve this absolute value inequality and sketch its graph on a number line.
        \[
            5 - 7|8-3r| \geq -93
        \]
    }
\end{taggedblock}