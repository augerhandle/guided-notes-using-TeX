

\section*{Introduction}


In this lesson you will learn to solve \emph{absolute value inequalities}
This brings together three different things you have already learned:
\begin{itemize}[itemsep=0in]
    \item linear inequalities,
    \item compound inequalities, and
    \item absolute values.
\end{itemize}

To understand how to solve absolute value inequalities,
it helps to remember that absolute value (which is always postive) 
is the distance of a point from the origin of the number line.
With this in mind,
we can rewrite absolute value inequalities as compound inequalities (without absolute values).
There are two kinds of absolute value inequalities that we will solve:
\begin{itemize}
    \item $| ax + b | > c$ \quad(also $\geq$)
    \item $| ax + b | < c$ \quad(also $\leq$)
\end{itemize}
where $a$, $b$, $c$ are numbers.

