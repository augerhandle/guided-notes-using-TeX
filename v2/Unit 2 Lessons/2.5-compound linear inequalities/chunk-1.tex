

\section*{Introduction}

In previous lesson, we solved linear inequalities. 
Our solutions to those problems generally came in two parts:
\begin{itemize}
    \item an inequality with the variable (for example, $x \geq 5$), and
    \item a graph of that inequality on the number line (which involves circled and arrows).
\end{itemize}

\myEmptyExampleBox[4in]{0.75in}{
    \small\itshape
    For example, the solution to $2x+5 \leq 25$ would look like this.
}


In this lesson we will be working with more complicated inequalities called
{\bfseries\itshape compound inequalities}.
Their solutions will also come in two parts (inequalities and number lines).
But since the problems are more complicated, the inequalities and number lines will also be more complicated.

\myEmptyExampleBox[4in]{0.75in}{
    \small\itshape
    For example, you will learn that the graph of this compound inequality
    \[
        x \geq -1 
        \qquad\text{and}\qquad
        x < 3
    \]
    on a number line will look like this.
}


\begin{myConceptSteps}{To solve a compound inequality\dots}
    \myStep{solve each}{Solve each inequality separately.}
    \myStep{inequality solution}{
        The solution written as inequalities 
        is the two separate solutions connected by ``and'' or ``or'' 
        (from the original problem).
    }
    \myStep{number line solution}{
        The solution sketched as a graph
        is either the {\bfseries\itshape overlap} or {\bfseries\itshape combination}
        of the two separate solution number lines
        depending on whether the original problem had an ``and'' or an ``or''. 
    }
    \myStep{collapse}{Put both number lines together onto a single number line}
\end{myConceptSteps}

This is best shown with some examples.


