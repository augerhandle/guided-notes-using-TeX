\documentclass[10pt,letterpaper]{memoir}
% memoir commands to define the text block geometry
\setulmarginsandblock{1in}{*}{*}
\setlrmarginsandblock{1in}{*}{*}

\usepackage{xparse}
\usepackage{blindtext}
\usepackage{xwatermark}
\usepackage{enumitem}
\usepackage{graphicx}
\usepackage{amsmath}

\usepackage{tcolorbox}
    \tcbuselibrary{skins}
\usepackage{pgfplots}
    \pgfplotsset{compat=newest}
\usepackage{tagging}

\newcommand{\myHeadFootStyle}{\footnotesize\sffamily}
\copypagestyle{myPagestyle}{empty}
%
% FIXME
% The following header definitions do NOT work right in all cases.
% I have found that the chapter title sometimes gets picked up from
% a chapter that begins on the NEXT PAGE. Not sure what's going on.
% So I abandoned embedding the info in the header and instead updated
% \myLesson to print it out, and that seems to work find.
%
% \makeoddhead{myPagestyle}
%     {\,}
%     {\,}
%     {\myHeadFootStyle\chaptername\,\thechapter\,\,\myCurrentChapterTitle}
% \makeevenhead{myPagestyle} 
%     {\,}
%     {\,}
%     {\myHeadFootStyle\chaptername\,\thechapter\,\,\myCurrentChapterTitle}
\makeoddhead{myPagestyle}
    {\myHeadFootStyle\myClassName}
    {\,}
    {\,}
\makeevenhead{myPagestyle} 
    {\,}
    {\,}
    {\myHeadFootStyle\myClassName}
\makeoddfoot{myPagestyle}
    {\myHeadFootStyle\myClassName}
    {\myHeadFootStyle\thepage}
    {\myHeadFootStyle\myCurrentLessonTitle}
\makeevenfoot{myPagestyle}
    {\myHeadFootStyle\myCurrentLessonTitle}
    {\myHeadFootStyle\thepage}
    {\myHeadFootStyle\myClassName}


\setlength{\parindent}{0em}
\setlength{\parskip}{0.75em}

\begin{document}
\pagestyle{plain}
\checkandfixthelayout
% \raggedbottom
\dashundergapssetup{teacher-mode=false,}

% ---------------------------------------------------------------------------
% a header to put at the top of a homework assignment
% ---------------------------------------------------------------------------
% #1 class name 
% #2 assignment number (eg, 2.5 CW)
% #3 assignment title (eg, Solve Quadratic Equations)
%
\NewDocumentCommand{\myAssignmentHeader}{ O{Algebra~2} m m }{
    \noindent
    \whenHONORS{Honors~}#1
    \hfill 
    Name : \fbox{\phantom{X\hspace{2in}}}\par
    \vspace{0.5em}
    \noindent
    {%
        \LARGE\sffamily
        \whenHONORS{H-}#2 -- #3
    }
    \hfill
    Period : \fbox{\phantom{\large 9999}}
    \hrule\hspace{\baselineskip}
    \vspace{0.5\baselineskip}
}


% ---------------------------------------------------------------------------
% I'm SO tired of writing these two explicitly!
% ---------------------------------------------------------------------------
\newcommand{\myEmph}[1]{{\bfseries\itshape#1}}

% ---------------------------------------------------------------------------
% For writing TI-84 instructions it's useful to have a visually
% distinct way to format the keys.
% ---------------------------------------------------------------------------
\newcommand{\myKey}[1]{{\ttfamily [#1]}}

\newcommand{\myDesmos}{{\scshape Desmos~}}
\newcommand{\myTi}{{\scshape TI-84~}}

% ---------------------------------------------------------------------------
% Teachers vs. Students ("tstu")
% ---------------------------------------------------------------------------
\newif\iftstu

% Typically, I will set one of these ONCE at the top of the doc.
\newcommand{\forTEACHER}{%
    \tstutrue%
    \dashundergapssetup{teacher-mode=true,}%
}
\newcommand{\forSTUDENT}{%
    \tstufalse%
    \dashundergapssetup{teacher-mode=false,}%
}

% These will conditionally generate output (the argument) based on 
% whether this document is "for" HONORS or ONLEVEL according to the previous commands.
%
% #1 the content 
% #2 text color
\NewDocumentCommand{\whenTEACHER}{ O{red} m }{\iftstu{\ECFAugie\color{#1}#2}\else{}\fi}
\newcommand{\whenSTUDENT}[1]{\iftstu{}\else{#1}\fi}



% ---------------------------------------------------------------------------
% command for creating a 1- or 2-question warmup problem
%
% #1 date
% #2 optional general directions
%
% ---------------------------------------------------------------------------
\NewDocumentEnvironment{myWarmupProblems}{ 
        O{
            \begin{itemize}
                \item Show {\itshape all} your work.
                \item Put a \fbox{box} around your answer.
            \end{itemize}
            \vspace{0.5em}
        } 
    }
{
    \newpage
    {
        \sffamily
        \begin{center}
            {
                \huge\bfseries%
                \whenHONORS{Honors~}Algebra 2 Warmup
            }%
            \\[0.5em]
            {\Large\itshape\printdate}
        \end{center}
        \noindent{\Large#1}
    }
    \large
}{
}


% ---------------------------------------------------------------------------
% A centered tcolorbox
% ---------------------------------------------------------------------------
%
% #1 - optional options to pass to tcolorbox
% #2 - the contents to put in the box
%
\NewDocumentEnvironment{myCenteredBox}{ O{} m }{%
    \begin{center}
        \begin{tcolorbox}[#1]#2\end{tcolorbox}
    \end{center}
}

% ---------------------------------------------------------------------------
% Split the output into two (roughly) equally sized side-by-side minipages.
% ---------------------------------------------------------------------------
%
% #1 - content of first minipage
% #2 - content of second minipage
%
\NewDocumentCommand{\myTwoMinipages}{mm}{
    \begin{minipage}{0.49\textwidth}
        #1
    \end{minipage}
    % \hfill 
    \begin{minipage}{0.49\textwidth}
        #2
    \end{minipage}
}


% ---------------------------------------------------------------------------
% A version of \sqrt 
% ---------------------------------------------------------------------------
%
% #1 index of the root
% #2 uproot amount
% #3 radicand
%
\NewDocumentCommand{\myRoot}{ o O{2} m  }{%
    \IfNoValueTF{#1}{\sqrt{#3\,}}{\sqrt[\uproot{#2}#1]{#3\,}}
}


% ---------------------------------------------------------------------------
% changes to part/chapter/section
% I am piggybacking Units and Lessons on Latex Parts and Chapters, respectively. 
% ---------------------------------------------------------------------------

% Here are some of the macros that capture "my settings".
\newcommand{\myPartName}{Unit}      % what I'll replace "Unit" with.
\newcommand{\myChapterName}{Lesson} % what I'll replace "Chapter" with.
\newcommand{\myLessonSuffix}{}      % a suffix after lesson (chapter) numbers (eg, 'a' in Lesson 5.2a)

% Redefine the Latex part and chapter names to use "my settings".
\renewcommand{\partname}{\myPartName}       % "Unit 2" instead of "Part II"
\renewcommand{\thepart}{\arabic{part}}
\renewcommand{\chaptername}{\myChapterName} % "Lesson 2" instead of "Chapter 2"

% A \chapter-based command to generate a lesson.
% This is a thin wrapper around \chapter that allows me
% to explicitly specify the lesson number. 
%
% #1 lesson name
% #2 optional lesson number (eg, the '2' in Lesson 5.2)
% #3 optional lesson suffix (eg, the 'a' in Lesson 5.2a)
%
\NewDocumentCommand{\myLesson}{ m O{1} O{} }
{
    \setcounter{chapter}{#2-1}
    \renewcommand{\myLessonSuffix}{#3}
    \chapter{#1}
}

% remove chapter numbers from section numbering
\makeatletter
\renewcommand\thesection{\@arabic\c@section}
\makeatother

% simulate a part without typesetting one
% 
% #1 - the part (unit) number
% #2 - the part (unit) title
\newcommand{\dummypart}[2]{%
    \setcounter{part}{#1}
    \partmark{#2}
}

% ---------------------------------------------------------------------------
% These are "annotations", the foundation of the blocked-in-text that I use.
%
% I use the term "annotations" to capture the common
% infrastructure I use to define Objectives, Vobabulary & Concepts.
% ---------------------------------------------------------------------------

% font and styling commands for
% Objectives, Voculary, Key Concepts, etc...
\newcommand{\myAnnotationStyling}{\bfseries\large}

% #1 : name of the kind of annotation (Objectives, ...)
% #2 : title text to go with the annotation
% #3 : extra tcolorbox options
%
\NewDocumentEnvironment{myAnnotate}{ m m O{}}{
    \begin{tcolorbox}[
        colbacktitle=blue!10!white,
        colback=white,
        coltitle=black,
        fonttitle={\myAnnotationStyling},
        title={#1:~},
        after title={\normalfont\itshape#2},
        #3,
        ]
}{
    \end{tcolorbox}
}

% #1 : name of the kind of annotation 
% #2 : title 
%
\NewDocumentEnvironment{myTabularAnnotate}{ m m }{
    \begin{myAnnotate}{#1}{#2}
    \begin{tabular}{r|l}
}{
    \end{tabular}
    \end{myAnnotate}
}
% #1 - column 1 text
% #2 - column 2 text
%
\NewDocumentCommand{\myRow}{mm}{{\bfseries\itshape \textcolor{blue}{#1}}&#2\\[1ex]}

% #1 : name of the kind of annotation 
% #2 : title 
% #3 : text before the list starts
%
\NewDocumentEnvironment{myListAnnotate}{ m m o }{
    \begin{myAnnotate}{#1}{#2}
    \IfValueT{#3}{#3}
    \begin{enumerate}[itemsep=0pt,fullwidth,]
}{
    \end{enumerate}
    \end{myAnnotate}
}
% #1 - column 1 text
% #2 - column 2 text
%
\NewDocumentCommand{\myItem}{mm}{\item{\bfseries\itshape \textcolor{blue}{#1}} #2}


% ---------------------------------------------------------------------------
% A table for systems of equations word problems.
%
% #1 scale factor
% ---------------------------------------------------------------------------
\NewDocumentCommand{\mySystemTable}{O{8}}{
    \begin{center}
        \begin{tabular}{|m{5em}|m{2in}|m{2in}|m{1.5in}|}
            \hline
            {\bfseries\itshape variables} & {\bfseries\itshape equations} & {\bfseries\itshape system} & {\bfseries\itshape augmented matrix} \\
            \hline\hline
            \scalebox{#1}{\fontsize{32pt}{0pt}\selectfont \phantom{\textbf{I}}} & \phantom{X} & \phantom{X} & \phantom{X}\\
            \hline
        \end{tabular}
    \end{center}
}

\NewDocumentCommand{\myBetterSystemTable}{O{4}}{
    \begin{center}
        \begin{tabular}{|m{3.25in}|m{3.25in}|}
            \hline
            \underline{\bfseries\itshape variables:} & \underline{\bfseries\itshape system of equations:}  \\
            \scalebox{#1}{\fontsize{32pt}{0pt}\selectfont \phantom{\textbf{I}}} & \phantom{X} \\
            \hline
            \underline{\bfseries\itshape augmented matrix:} & \underline{\bfseries\itshape RREF matrix:} \\
            \scalebox{#1}{\fontsize{32pt}{0pt}\selectfont \phantom{\textbf{I}}} & \phantom{X} \\
            \hline
        \end{tabular}
    \end{center}
}

{\small Pre-AP Algebra 2}\hfill Name: \rule{2in}{0.15mm}

{\bfseries\Large PAP HW 4.4 (DAY 2) A Sneak Peak at Linear Regression}\hfill Period: \rule{0.5in}{0.15mm}



\begin{tcolorbox}
    \itshape\small
    Work this problem if you have finished yesterday's sheet of problems.
    This is a ``warm up'' to the topic of 
    {\bfseries\itshape linear regression}, which we will discuss tomorrow.
\end{tcolorbox}




\section*{Four equations \& four unknown variables} 
Here are four $x$-$y$ points:
\begin{center}\begin{tabular}{cc}
    \toprule
    $x$ & $y$ \\
    \midrule
    -5 & 5 \\
    0 & -2 \\
    3 & 2 \\
    5 & 0 \\
    \bottomrule
\end{tabular}\end{center}
Using what you learned a few days ago,
find the equation of the cubic  
\[
    y = a + bx + cx^2 + dx^3
\]
that goes through
those four points.
Remember,
\begin{itemize}[itemsep=0in]
    \item You have four unknown variables, $a$, $b$, $c$, $d$.
    \item By evaluating the cubic at the four points, you get four equations
    involving the four unknow variables.
    \item Using the {\scshape Desmos Matrix Calculator}, 
    you can calculate $A^{-1}B$ to find the unknown variables.
\end{itemize}
(You'll probably need some scratch paper.)\vspace{1em}

\begin{minipage}{2in}
    {\Large
    $a$ = \gap{\hspace{1em}}\\[1em]
    $b$ = \gap{\hspace{1em}}\\
    }
\end{minipage}
\begin{minipage}{2in}
    {\Large
    $c$ = \gap{\hspace{1em}}\\[1em]
    $d$ = \gap{\hspace{1em}}\\
    }
\end{minipage}
\hrule





\section*{Graph it!}
Enter that function into the {\scshape Desmos Graphing Calculator}.
Does the curve seem to pass exactly through the four points in the table?

Answer yes or no: \gap{\hspace{3em}}

Does the point $(2,1)$ seem to be ``near'' the graph of the cubic function?

Answer yes or no: \gap{\hspace{3em}}\vspace{1em}
\hrule





\section*{Five equations \& four unknown variables?} 
So let's consider {\bfseries\itshape five} $x$-$y$ points.
\begin{center}\begin{tabular}{cc}
    \toprule
    $x$ & $y$ \\
    \midrule
    -5 & 5 \\
    0 & -2 \\
    3 & 2 \\
    5 & 0 \\
    2 & 1 \\
    \bottomrule
\end{tabular}\end{center}

Using all {\bfseries\itshape five} of those points,
formulate {\bfseries\itshape five} equations 
and write out the new $A$ and $B$ matrices below:
\par
{\Large
    \vspace{3em}
    \hfill $A = $\hspace{1.5in} \hfill $B = $\hspace{1.5in} \hfill
    \vspace{3em}
}

Use the {\scshape Desmos Graphing Calculator}
to calculate $A^{-1}B$. 
What happens? Explain what you think is going on.
\vspace{2.5in}
\hrule


\section*{Sneak peak at linear regression}

In a web browser (Safari, Chrome... on your phone), go to this URL:\\
\url{https://www.desmos.com/calculator/nzz8zxtguy}.

The graph there has a table of the same five data points that you used, above.
And it uses something called 
{\bfseries\itshape linear regression}
to find a cubic equation passing through those points.

This is something the $A^{-1}B$ matrix math was unable to do
(as you found above).

Linear regression uses more complicated matrix math 
to do this. 
So in fact it {\bfseries\itshape is} possible. 
It's just that the curve won't go through all five points exactly.
Linear regression is a method for finding the ``best'' curve that 
passes through all the points.
To see this, 
play with the slider on the graph. It changes the
$y$-coordinate of that fifth point. 

As you slide the slider,
you should be able to see that the curve no longer passes through 
all the points exactly.
Yet, it is possible to prove that the curve is the best one
in the sense of being {\itshape as close as possible to all the points
at the same time.}

We will talk more about linear regression tomorrow.
\end{document}
