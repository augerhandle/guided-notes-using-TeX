% ---------------------------------------------------------------------------
% a header to put at the top of a homework assignment
% ---------------------------------------------------------------------------
% #1 class name 
% #2 assignment number (eg, 2.5 CW)
% #3 assignment title (eg, Solve Quadratic Equations)
%
\NewDocumentCommand{\myAssignmentHeader}{ O{Algebra~2} m m }{
    \noindent
    \whenHONORS{Honors~}#1
    \hfill 
    Name : \fbox{\phantom{X\hspace{2in}}}\par
    \vspace{0.5em}
    \noindent
    {%
        \LARGE\sffamily
        \whenHONORS{H-}#2 -- #3
    }
    \hfill
    Period : \fbox{\phantom{\large 9999}}
    \hrule\hspace{\baselineskip}
    \vspace{0.5\baselineskip}
}


% ---------------------------------------------------------------------------
% I'm SO tired of writing these two explicitly!
% ---------------------------------------------------------------------------
\newcommand{\myEmph}[1]{{\bfseries\itshape#1}}

% ---------------------------------------------------------------------------
% For writing TI-84 instructions it's useful to have a visually
% distinct way to format the keys.
% ---------------------------------------------------------------------------
\newcommand{\myKey}[1]{{\ttfamily [#1]}}

\newcommand{\myDesmos}{{\scshape Desmos~}}
\newcommand{\myTi}{{\scshape TI-84~}}

% ---------------------------------------------------------------------------
% Teachers vs. Students ("tstu")
% ---------------------------------------------------------------------------
\newif\iftstu

% Typically, I will set one of these ONCE at the top of the doc.
\newcommand{\forTEACHER}{%
    \tstutrue%
    \dashundergapssetup{teacher-mode=true,}%
}
\newcommand{\forSTUDENT}{%
    \tstufalse%
    \dashundergapssetup{teacher-mode=false,}%
}

% These will conditionally generate output (the argument) based on 
% whether this document is "for" HONORS or ONLEVEL according to the previous commands.
%
% #1 the content 
% #2 text color
\NewDocumentCommand{\whenTEACHER}{ O{red} m }{\iftstu{\ECFAugie\color{#1}#2}\else{}\fi}
\newcommand{\whenSTUDENT}[1]{\iftstu{}\else{#1}\fi}



% ---------------------------------------------------------------------------
% command for creating a 1- or 2-question warmup problem
%
% #1 date
% #2 optional general directions
%
% ---------------------------------------------------------------------------
\NewDocumentEnvironment{myWarmupProblems}{ 
        O{
            \begin{itemize}
                \item Show {\itshape all} your work.
                \item Put a \fbox{box} around your answer.
            \end{itemize}
            \vspace{0.5em}
        } 
    }
{
    \newpage
    {
        \sffamily
        \begin{center}
            {
                \huge\bfseries%
                \whenHONORS{Honors~}Algebra 2 Warmup
            }%
            \\[0.5em]
            {\Large\itshape\printdate}
        \end{center}
        \noindent{\Large#1}
    }
    \large
}{
}


% ---------------------------------------------------------------------------
% A centered tcolorbox
% ---------------------------------------------------------------------------
%
% #1 - optional options to pass to tcolorbox
% #2 - the contents to put in the box
%
\NewDocumentEnvironment{myCenteredBox}{ O{} m }{%
    \begin{center}
        \begin{tcolorbox}[#1]#2\end{tcolorbox}
    \end{center}
}

% ---------------------------------------------------------------------------
% Split the output into two (roughly) equally sized side-by-side minipages.
% ---------------------------------------------------------------------------
%
% #1 - content of first minipage
% #2 - content of second minipage
%
\NewDocumentCommand{\myTwoMinipages}{mm}{
    \begin{minipage}{0.49\textwidth}
        #1
    \end{minipage}
    % \hfill 
    \begin{minipage}{0.49\textwidth}
        #2
    \end{minipage}
}


% ---------------------------------------------------------------------------
% A version of \sqrt 
% ---------------------------------------------------------------------------
%
% #1 index of the root
% #2 uproot amount
% #3 radicand
%
\NewDocumentCommand{\myRoot}{ o O{2} m  }{%
    \IfNoValueTF{#1}{\sqrt{#3\,}}{\sqrt[\uproot{#2}#1]{#3\,}}
}


% ---------------------------------------------------------------------------
% changes to part/chapter/section
% I am piggybacking Units and Lessons on Latex Parts and Chapters, respectively. 
% ---------------------------------------------------------------------------

% Here are some of the macros that capture "my settings".
\newcommand{\myPartName}{Unit}      % what I'll replace "Unit" with.
\newcommand{\myChapterName}{Lesson} % what I'll replace "Chapter" with.
\newcommand{\myLessonSuffix}{}      % a suffix after lesson (chapter) numbers (eg, 'a' in Lesson 5.2a)

% Redefine the Latex part and chapter names to use "my settings".
\renewcommand{\partname}{\myPartName}       % "Unit 2" instead of "Part II"
\renewcommand{\thepart}{\arabic{part}}
\renewcommand{\chaptername}{\myChapterName} % "Lesson 2" instead of "Chapter 2"

% A \chapter-based command to generate a lesson.
% This is a thin wrapper around \chapter that allows me
% to explicitly specify the lesson number. 
%
% #1 lesson name
% #2 optional lesson number (eg, the '2' in Lesson 5.2)
% #3 optional lesson suffix (eg, the 'a' in Lesson 5.2a)
%
\NewDocumentCommand{\myLesson}{ m O{1} O{} }
{
    \setcounter{chapter}{#2-1}
    \renewcommand{\myLessonSuffix}{#3}
    \chapter{#1}
}

% remove chapter numbers from section numbering
\makeatletter
\renewcommand\thesection{\@arabic\c@section}
\makeatother

% simulate a part without typesetting one
% 
% #1 - the part (unit) number
% #2 - the part (unit) title
\newcommand{\dummypart}[2]{%
    \setcounter{part}{#1}
    \partmark{#2}
}

% ---------------------------------------------------------------------------
% These are "annotations", the foundation of the blocked-in-text that I use.
%
% I use the term "annotations" to capture the common
% infrastructure I use to define Objectives, Vobabulary & Concepts.
% ---------------------------------------------------------------------------

% font and styling commands for
% Objectives, Voculary, Key Concepts, etc...
\newcommand{\myAnnotationStyling}{\bfseries\large}

% #1 : name of the kind of annotation (Objectives, ...)
% #2 : title text to go with the annotation
% #3 : extra tcolorbox options
%
\NewDocumentEnvironment{myAnnotate}{ m m O{}}{
    \begin{tcolorbox}[
        colbacktitle=blue!10!white,
        colback=white,
        coltitle=black,
        fonttitle={\myAnnotationStyling},
        title={#1:~},
        after title={\normalfont\itshape#2},
        #3,
        ]
}{
    \end{tcolorbox}
}

% #1 : name of the kind of annotation 
% #2 : title 
%
\NewDocumentEnvironment{myTabularAnnotate}{ m m }{
    \begin{myAnnotate}{#1}{#2}
    \begin{tabular}{r|l}
}{
    \end{tabular}
    \end{myAnnotate}
}
% #1 - column 1 text
% #2 - column 2 text
%
\NewDocumentCommand{\myRow}{mm}{{\bfseries\itshape \textcolor{blue}{#1}}&#2\\[1ex]}

% #1 : name of the kind of annotation 
% #2 : title 
% #3 : text before the list starts
%
\NewDocumentEnvironment{myListAnnotate}{ m m o }{
    \begin{myAnnotate}{#1}{#2}
    \IfValueT{#3}{#3}
    \begin{enumerate}[itemsep=0pt,fullwidth,]
}{
    \end{enumerate}
    \end{myAnnotate}
}
% #1 - column 1 text
% #2 - column 2 text
%
\NewDocumentCommand{\myItem}{mm}{\item{\bfseries\itshape \textcolor{blue}{#1}} #2}


% ---------------------------------------------------------------------------
% A table for systems of equations word problems.
%
% #1 scale factor
% ---------------------------------------------------------------------------
\NewDocumentCommand{\mySystemTable}{O{8}}{
    \begin{center}
        \begin{tabular}{|m{5em}|m{2in}|m{2in}|m{1.5in}|}
            \hline
            {\bfseries\itshape variables} & {\bfseries\itshape equations} & {\bfseries\itshape system} & {\bfseries\itshape augmented matrix} \\
            \hline\hline
            \scalebox{#1}{\fontsize{32pt}{0pt}\selectfont \phantom{\textbf{I}}} & \phantom{X} & \phantom{X} & \phantom{X}\\
            \hline
        \end{tabular}
    \end{center}
}

\NewDocumentCommand{\myBetterSystemTable}{O{4}}{
    \begin{center}
        \begin{tabular}{|m{3.25in}|m{3.25in}|}
            \hline
            \underline{\bfseries\itshape variables:} & \underline{\bfseries\itshape system of equations:}  \\
            \scalebox{#1}{\fontsize{32pt}{0pt}\selectfont \phantom{\textbf{I}}} & \phantom{X} \\
            \hline
            \underline{\bfseries\itshape augmented matrix:} & \underline{\bfseries\itshape RREF matrix:} \\
            \scalebox{#1}{\fontsize{32pt}{0pt}\selectfont \phantom{\textbf{I}}} & \phantom{X} \\
            \hline
        \end{tabular}
    \end{center}
}
%
% I am piggy-backing off of the standard memoir document
% divisions:
%
% book    -- is for the name of my class (eg, Alg2)
% part    -- is for the semester 
% chapter -- is for units
%
% To do this, I need to change the default value of various
% division-related commands. That's what the following is for.

%----- CLASS ---------
\newcommand{\myCurrentBookTitle}{}
\let\myOldBook\book

\RenewDocumentCommand{\book}{m}{
    \renewcommand{\myCurrentBookTitle}{#1}
    \myOldBook{#1}
}
% I don't want roman numerals.
\makeatletter
\renewcommand*{\thebook}{\@arabic\c@book}
\makeatother
% I just want the title (not "Class 1 <title>")
\renewcommand{\printbookname}{}
\renewcommand{\booknamenum}{}
\renewcommand{\printbooknum}{}


%----- SEMESTER -------
% I don't want roman numerals.
\makeatletter
\renewcommand*{\thepart}{\@arabic\c@part}
\makeatother


%----- UNIT -----------
\renewcommand{\chaptername}{Unit}
\newcommand{\myCurrentChapterTitle}{}
\let\myOldChapter\chapter
\renewcommand{\chapter}[1]{
    \renewcommand{\myCurrentChapterTitle}{#1}
    \myOldChapter{#1}
}
%-------- LESSONS --------------

\newcounter{myLessonCounter}[chapter] 
\newcommand{\myCurrentLessonTitle}{}
\newcommand{\myFullLessonNumber}{}  % for example 2.3b

% #1 - lesson title
% #2 - optional lesson number (default increments from previous)
% #3 - optional lesson number suffix (eg, 'a' or 'b')
%
\NewDocumentCommand{\myLesson}{ m o O{} }{
    \cleartorecto
    \renewcommand{\myCurrentLessonTitle}{#1}
    \IfValueTF{#2}
        {\setcounter{myLessonCounter}{#2}}
        {\stepcounter{myLessonCounter}}
    \renewcommand{\myFullLessonNumber}{\thechapter.\themyLessonCounter#3}
    \noindent
    \begin{minipage}[b]{0.25\textwidth}
        {
            \HUGE\bfseries\sffamily\myFullLessonNumber
        }
    \end{minipage}
    \begin{minipage}[b]{0.74\textwidth}
        \begin{flushright}
            {\sffamily\chaptername\,\thechapter\,\,\myCurrentChapterTitle}\\
            \vspace{0.75\baselineskip}
            {\Huge\bfseries\sffamily#1}
        \end{flushright}
    \end{minipage}
    
    \noindent\rule{\textwidth}{1.25pt}
    \par
}

%-------- ANNOTATIONS (in boxes) --------------
%
% I use the term "annotations" to capture the common
% infrastructure I use to define Objectives, Vobabulary & Concepts.
%

% font and styling commands for
% Objectives, Voculary, Key Concepts, etc...
\newcommand{\myAnnotationStyling}{\bfseries\large}

% #1 : name of the kind of annotation (Objectives, ...)
% #2 : title text to go with the annotation
% #3 : extra tcolorbox options
%
\NewDocumentEnvironment{myAnnotate}{ m m O{}}{
    \begin{tcolorbox}[
        colbacktitle=blue!10!white,
        colback=white,
        coltitle=black,
        fonttitle={\myAnnotationStyling},
        title={#1: },
        after title={\normalfont\itshape#2},
        #3,
        ]
}{
    \end{tcolorbox}
}

% #1 : name of the kind of annotation 
% #2 : title 
%
\NewDocumentEnvironment{myTabularAnnotate}{ m m }{
    \begin{myAnnotate}{#1}{#2}
    \begin{tabular}{r|l}
}{
    \end{tabular}
    \end{myAnnotate}
}
% #1 - column 1 text
% #2 - column 2 text
%
\NewDocumentCommand{\myRow}{mm}{{\bfseries\itshape \textcolor{blue}{#1}}&#2\\[1ex]}

% #1 : name of the kind of annotation 
% #2 : title 
% #3 : text before the list starts
%
\NewDocumentEnvironment{myListAnnotate}{ m m o }{
    \begin{myAnnotate}{#1}{#2}
    \IfValueT{#3}{#3}
    \begin{enumerate}[itemsep=0pt,]
}{
    \end{enumerate}
    \end{myAnnotate}
}
% #1 - column 1 text
% #2 - column 2 text
%
\NewDocumentCommand{\myItem}{mm}{\item{\bfseries\itshape \textcolor{blue}{#1}} #2}



%----------- OBJECTIVES , VOCABULARY, CONCEPTS ---------------
\newenvironment{myObjectives}{
    \begin{myListAnnotate}{Objectives}{}
}{
    \end{myListAnnotate}
}
\newcommand{\myObjective}[2]{
    \myItem{#1}{#2}
}

\newenvironment{myVocabulary}{
    \begin{myTabularAnnotate}{Vocabulary}{}
}{
    \end{myTabularAnnotate}
}
\newcommand{\myDefinition}[2]{
    \myRow{#1}{#2}
}

\NewDocumentEnvironment{myConcept}{m}{
    \begin{myAnnotate}{Key Concept}{\,#1}
}{
    \end{myAnnotate}
}

\NewDocumentEnvironment{myConceptSteps}{m}{
    \begin{myListAnnotate}{Key Concept}{#1}[Follow these steps.]
}{
    \end{myListAnnotate}
}
\newcommand{\myStep}[2]{
    \myItem{#1.}{#2}
}



\newcounter{myCweRowCounter}

% #1 width of column 1
% #2 text in column 1
% #3 text in column 2
\newcommand{\myCweRow}[3]{
    \begin{tcolorbox}[%
        equal height group = \arabic{myCweRowCounter},%
        toprule=0pt, leftrule=0pt,%
        rightrule=1pt, bottomrule=1pt,%
        title=row1col1, notitle,%
        sharp corners,%
        colback=white,%
        width=#1,%
        ]%
        #2%
    \end{tcolorbox}%
    \begin{tcolorbox}[%
        equal height group = \arabic{myCweRowCounter},%
        toprule=0pt,leftrule=0pt,rightrule=0pt,%
        bottomrule=1pt,%
        title=row1col1, notitle,%
        sharp corners,%
        colback=white,%
        ]%
        #3%
    \end{tcolorbox}%
    \stepcounter{myCweRowCounter}%
}

% #1 width of column 1
% #2 text in column 1
% #3 text in column 2
\newcommand{\myCweBottomRow}[3]{
    \begin{tcolorbox}[%
        equal height group = \arabic{myCweRowCounter},%
        bottomrule=0pt, leftrule=0pt, toprule=0pt,%
        rightrule=1pt,%
        title=row1col1, notitle,%
        sharp corners,%
        colback=white,%
        width=#1,%
        ]%
        #2%
    \end{tcolorbox}%
    \begin{tcolorbox}[%
        equal height group = \arabic{myCweRowCounter},%
        bottomrule=0pt, leftrule=0pt, toprule=0pt, rightrule=0pt,%
        title=row1col1, notitle,%
        sharp corners,%
        colback=white,%
        ]%
        #3%
    \end{tcolorbox}%
    \stepcounter{myCweRowCounter}%
}

\newcolumntype{z}[2]{>{\columncolor{#1}\raggedleft\arraybackslash}p{#2}}
% #1 title (eg, "Key Concepts")
% #2 description of the concepts
% #3 width of concept column
% #4 row 1 concept
% #5 row 1 example
% #6 row 2 concept
% #7 row 2 example
%
\NewDocumentEnvironment{myConceptWithExamples2}{ O{Key Concepts} mm mmmm}{
    \begin{myAnnotate}{\hspace{5mm}#1}{\,#2}[%
        left=0em,right=0em,%
        top=0em,bottom=0em,%
        boxsep=0em,%
        toptitle=1mm, bottomtitle=1mm, % since we set boxsep to zero
        ]%
    \noindent
    \begin{tabular}{>{\columncolor[gray]{0.95}}c!{\vrule width 1pt}c}
        \begin{tcolorbox}[%
            width=#3,
            leftrule=0em,rightrule=0em,toprule=0em,bottomrule=0em,
            colback=black!5, colframe=white,
            ]%
            #4
        \end{tcolorbox}
        &
        \begin{tcolorbox}[%
            width=\linewidth-#3-10mm,
            leftrule=0em,rightrule=0em,toprule=0em,bottomrule=0em,
            colback=white, colframe=white,
            ]%
            #5
        \end{tcolorbox}
        \\
        \hhline{=:=}
        \begin{tcolorbox}[%
            width=#3,
            leftrule=0em,rightrule=0em,toprule=0em,bottomrule=0em,
            colback=black!5, colframe=white,
            ]%
            #6
        \end{tcolorbox}
        &
        \begin{tcolorbox}[%
            width=\linewidth-#3-10mm,
            leftrule=0em,rightrule=0em,toprule=0em,bottomrule=0em,
            colback=white, colframe=white,
            ]%
            #7
        \end{tcolorbox}
        \\
    \end{tabular}
    }{
    \end{myAnnotate}
}


% Units are based on \chapter. But we're not doing chapters
% in this one-lesson file. So we "fake out" the relevant 
% chapter (unit) number and title.
\renewcommand{\thechapter}{4} 
\renewcommand{\myCurrentChapterTitle}{Systems of Equations}

\myLesson{Solving Systems of Equations with Matrices}[3]

\begin{myObjectives}
    \myObjective{rewrite}{systems of equations using \gap{matrix} notation }
    \myObjective{solve}{systems of equations using \gap{matrices} }
\end{myObjectives}

\begin{myVocabulary}
    \myDefinition{system of equations}{several equations with the same variables}
    \myDefinition{matrix}{a table of numbers arranged in rows and columns}
    \myDefinition{matrices}{the plural of matrix, more than one matrix}
    \myDefinition{Desmos matrix calculator}{
        Go to {\ttfamily https://desmos.com/matrix} in your {\bfseries\itshape browser}
    }
\end{myVocabulary}

In the last lessons,
you learned how to solve systems of equation by {\bfseries\itshape substition}.
Today you will learn how to solve systems of equations using \gap{matrix} 
notation and the Desmos matrix \gap{calculator}.\footnote{
    The Desmos matrix calculator {\bfseries\itshape is not} a phone app.
    To use it, use your browser (Chrome, Safari, ...) to go the the calculator
    web site: {\ttfamily https://desmos.com/matrix}.
}

\section*{The Quadratic Formula}

\begin{center}
\begin{tcolorbox}[width=5.5in]
    Given a quadratic equation in standard form 
    {\LARGE
        \[
            a x^2 + bx + c = 0
        \]
    }
    the solution for $x$ can be calculated from
    {\bfseries\itshape the Quadratic Formula}:
    {\LARGE
        \[
            x = \frac
                {-b \pm \sqrt{b^2 - 4ac}}
                {2a}
        \]
    }
\end{tcolorbox}
\end{center}

