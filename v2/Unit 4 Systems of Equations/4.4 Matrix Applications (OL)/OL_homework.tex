\documentclass[10pt,letterpaper]{memoir}
% memoir commands to define the text block geometry
\setulmarginsandblock{0.75in}{*}{*}
\setlrmarginsandblock{0.75in}{*}{*}

\usepackage{xparse}
\usepackage{blindtext}
\usepackage{xwatermark}
\usepackage{enumitem}
\usepackage{graphicx}
\usepackage{amsmath}

\usepackage{tcolorbox}
    \tcbuselibrary{skins}
\usepackage{pgfplots}
    \pgfplotsset{compat=newest}
\usepackage{tagging}

\newcommand{\myHeadFootStyle}{\footnotesize\sffamily}
\copypagestyle{myPagestyle}{empty}
%
% FIXME
% The following header definitions do NOT work right in all cases.
% I have found that the chapter title sometimes gets picked up from
% a chapter that begins on the NEXT PAGE. Not sure what's going on.
% So I abandoned embedding the info in the header and instead updated
% \myLesson to print it out, and that seems to work find.
%
% \makeoddhead{myPagestyle}
%     {\,}
%     {\,}
%     {\myHeadFootStyle\chaptername\,\thechapter\,\,\myCurrentChapterTitle}
% \makeevenhead{myPagestyle} 
%     {\,}
%     {\,}
%     {\myHeadFootStyle\chaptername\,\thechapter\,\,\myCurrentChapterTitle}
\makeoddhead{myPagestyle}
    {\myHeadFootStyle\myClassName}
    {\,}
    {\,}
\makeevenhead{myPagestyle} 
    {\,}
    {\,}
    {\myHeadFootStyle\myClassName}
\makeoddfoot{myPagestyle}
    {\myHeadFootStyle\myClassName}
    {\myHeadFootStyle\thepage}
    {\myHeadFootStyle\myCurrentLessonTitle}
\makeevenfoot{myPagestyle}
    {\myHeadFootStyle\myCurrentLessonTitle}
    {\myHeadFootStyle\thepage}
    {\myHeadFootStyle\myClassName}


\setlength{\parindent}{0em}
\setlength{\parskip}{0.75em}

\begin{document}
\pagestyle{plain}
\checkandfixthelayout
\raggedbottom

% ---------------------------------------------------------------------------
% a header to put at the top of a homework assignment
% ---------------------------------------------------------------------------
% #1 class name 
% #2 assignment number (eg, 2.5 CW)
% #3 assignment title (eg, Solve Quadratic Equations)
%
\NewDocumentCommand{\myAssignmentHeader}{ O{Algebra~2} m m }{
    \noindent
    \whenHONORS{Honors~}#1
    \hfill 
    Name : \fbox{\phantom{X\hspace{2in}}}\par
    \vspace{0.5em}
    \noindent
    {%
        \LARGE\sffamily
        \whenHONORS{H-}#2 -- #3
    }
    \hfill
    Period : \fbox{\phantom{\large 9999}}
    \hrule\hspace{\baselineskip}
    \vspace{0.5\baselineskip}
}


% ---------------------------------------------------------------------------
% I'm SO tired of writing these two explicitly!
% ---------------------------------------------------------------------------
\newcommand{\myEmph}[1]{{\bfseries\itshape#1}}

% ---------------------------------------------------------------------------
% For writing TI-84 instructions it's useful to have a visually
% distinct way to format the keys.
% ---------------------------------------------------------------------------
\newcommand{\myKey}[1]{{\ttfamily [#1]}}

\newcommand{\myDesmos}{{\scshape Desmos~}}
\newcommand{\myTi}{{\scshape TI-84~}}

% ---------------------------------------------------------------------------
% Teachers vs. Students ("tstu")
% ---------------------------------------------------------------------------
\newif\iftstu

% Typically, I will set one of these ONCE at the top of the doc.
\newcommand{\forTEACHER}{%
    \tstutrue%
    \dashundergapssetup{teacher-mode=true,}%
}
\newcommand{\forSTUDENT}{%
    \tstufalse%
    \dashundergapssetup{teacher-mode=false,}%
}

% These will conditionally generate output (the argument) based on 
% whether this document is "for" HONORS or ONLEVEL according to the previous commands.
%
% #1 the content 
% #2 text color
\NewDocumentCommand{\whenTEACHER}{ O{red} m }{\iftstu{\ECFAugie\color{#1}#2}\else{}\fi}
\newcommand{\whenSTUDENT}[1]{\iftstu{}\else{#1}\fi}



% ---------------------------------------------------------------------------
% command for creating a 1- or 2-question warmup problem
%
% #1 date
% #2 optional general directions
%
% ---------------------------------------------------------------------------
\NewDocumentEnvironment{myWarmupProblems}{ 
        O{
            \begin{itemize}
                \item Show {\itshape all} your work.
                \item Put a \fbox{box} around your answer.
            \end{itemize}
            \vspace{0.5em}
        } 
    }
{
    \newpage
    {
        \sffamily
        \begin{center}
            {
                \huge\bfseries%
                \whenHONORS{Honors~}Algebra 2 Warmup
            }%
            \\[0.5em]
            {\Large\itshape\printdate}
        \end{center}
        \noindent{\Large#1}
    }
    \large
}{
}


% ---------------------------------------------------------------------------
% A centered tcolorbox
% ---------------------------------------------------------------------------
%
% #1 - optional options to pass to tcolorbox
% #2 - the contents to put in the box
%
\NewDocumentEnvironment{myCenteredBox}{ O{} m }{%
    \begin{center}
        \begin{tcolorbox}[#1]#2\end{tcolorbox}
    \end{center}
}

% ---------------------------------------------------------------------------
% Split the output into two (roughly) equally sized side-by-side minipages.
% ---------------------------------------------------------------------------
%
% #1 - content of first minipage
% #2 - content of second minipage
%
\NewDocumentCommand{\myTwoMinipages}{mm}{
    \begin{minipage}{0.49\textwidth}
        #1
    \end{minipage}
    % \hfill 
    \begin{minipage}{0.49\textwidth}
        #2
    \end{minipage}
}


% ---------------------------------------------------------------------------
% A version of \sqrt 
% ---------------------------------------------------------------------------
%
% #1 index of the root
% #2 uproot amount
% #3 radicand
%
\NewDocumentCommand{\myRoot}{ o O{2} m  }{%
    \IfNoValueTF{#1}{\sqrt{#3\,}}{\sqrt[\uproot{#2}#1]{#3\,}}
}


% ---------------------------------------------------------------------------
% changes to part/chapter/section
% I am piggybacking Units and Lessons on Latex Parts and Chapters, respectively. 
% ---------------------------------------------------------------------------

% Here are some of the macros that capture "my settings".
\newcommand{\myPartName}{Unit}      % what I'll replace "Unit" with.
\newcommand{\myChapterName}{Lesson} % what I'll replace "Chapter" with.
\newcommand{\myLessonSuffix}{}      % a suffix after lesson (chapter) numbers (eg, 'a' in Lesson 5.2a)

% Redefine the Latex part and chapter names to use "my settings".
\renewcommand{\partname}{\myPartName}       % "Unit 2" instead of "Part II"
\renewcommand{\thepart}{\arabic{part}}
\renewcommand{\chaptername}{\myChapterName} % "Lesson 2" instead of "Chapter 2"

% A \chapter-based command to generate a lesson.
% This is a thin wrapper around \chapter that allows me
% to explicitly specify the lesson number. 
%
% #1 lesson name
% #2 optional lesson number (eg, the '2' in Lesson 5.2)
% #3 optional lesson suffix (eg, the 'a' in Lesson 5.2a)
%
\NewDocumentCommand{\myLesson}{ m O{1} O{} }
{
    \setcounter{chapter}{#2-1}
    \renewcommand{\myLessonSuffix}{#3}
    \chapter{#1}
}

% remove chapter numbers from section numbering
\makeatletter
\renewcommand\thesection{\@arabic\c@section}
\makeatother

% simulate a part without typesetting one
% 
% #1 - the part (unit) number
% #2 - the part (unit) title
\newcommand{\dummypart}[2]{%
    \setcounter{part}{#1}
    \partmark{#2}
}

% ---------------------------------------------------------------------------
% These are "annotations", the foundation of the blocked-in-text that I use.
%
% I use the term "annotations" to capture the common
% infrastructure I use to define Objectives, Vobabulary & Concepts.
% ---------------------------------------------------------------------------

% font and styling commands for
% Objectives, Voculary, Key Concepts, etc...
\newcommand{\myAnnotationStyling}{\bfseries\large}

% #1 : name of the kind of annotation (Objectives, ...)
% #2 : title text to go with the annotation
% #3 : extra tcolorbox options
%
\NewDocumentEnvironment{myAnnotate}{ m m O{}}{
    \begin{tcolorbox}[
        colbacktitle=blue!10!white,
        colback=white,
        coltitle=black,
        fonttitle={\myAnnotationStyling},
        title={#1:~},
        after title={\normalfont\itshape#2},
        #3,
        ]
}{
    \end{tcolorbox}
}

% #1 : name of the kind of annotation 
% #2 : title 
%
\NewDocumentEnvironment{myTabularAnnotate}{ m m }{
    \begin{myAnnotate}{#1}{#2}
    \begin{tabular}{r|l}
}{
    \end{tabular}
    \end{myAnnotate}
}
% #1 - column 1 text
% #2 - column 2 text
%
\NewDocumentCommand{\myRow}{mm}{{\bfseries\itshape \textcolor{blue}{#1}}&#2\\[1ex]}

% #1 : name of the kind of annotation 
% #2 : title 
% #3 : text before the list starts
%
\NewDocumentEnvironment{myListAnnotate}{ m m o }{
    \begin{myAnnotate}{#1}{#2}
    \IfValueT{#3}{#3}
    \begin{enumerate}[itemsep=0pt,fullwidth,]
}{
    \end{enumerate}
    \end{myAnnotate}
}
% #1 - column 1 text
% #2 - column 2 text
%
\NewDocumentCommand{\myItem}{mm}{\item{\bfseries\itshape \textcolor{blue}{#1}} #2}


% ---------------------------------------------------------------------------
% A table for systems of equations word problems.
%
% #1 scale factor
% ---------------------------------------------------------------------------
\NewDocumentCommand{\mySystemTable}{O{8}}{
    \begin{center}
        \begin{tabular}{|m{5em}|m{2in}|m{2in}|m{1.5in}|}
            \hline
            {\bfseries\itshape variables} & {\bfseries\itshape equations} & {\bfseries\itshape system} & {\bfseries\itshape augmented matrix} \\
            \hline\hline
            \scalebox{#1}{\fontsize{32pt}{0pt}\selectfont \phantom{\textbf{I}}} & \phantom{X} & \phantom{X} & \phantom{X}\\
            \hline
        \end{tabular}
    \end{center}
}

\NewDocumentCommand{\myBetterSystemTable}{O{4}}{
    \begin{center}
        \begin{tabular}{|m{3.25in}|m{3.25in}|}
            \hline
            \underline{\bfseries\itshape variables:} & \underline{\bfseries\itshape system of equations:}  \\
            \scalebox{#1}{\fontsize{32pt}{0pt}\selectfont \phantom{\textbf{I}}} & \phantom{X} \\
            \hline
            \underline{\bfseries\itshape augmented matrix:} & \underline{\bfseries\itshape RREF matrix:} \\
            \scalebox{#1}{\fontsize{32pt}{0pt}\selectfont \phantom{\textbf{I}}} & \phantom{X} \\
            \hline
        \end{tabular}
    \end{center}
}

{\small Algebra 2}\hfill Name: \rule{2in}{0.15mm}

{\bfseries\Large In-Class 4.4 Matrix Applications}\hfill Period: \rule{0.5in}{0.15mm}

{\itshape
For the following word problems, formulate a system of linear equations,
and solve them using the Desmos matrix calculator at
} 
{\bfseries\ttfamily desmos.com/matrix}.
\vspace{1.5em}




{\bfseries\large 1)} 
You find a jar of dimes and quarters.
There is a total of 70 coins in the jar.
When you take them to the bank (to the coin counting machine),
you find that they added up to \$13.00.
How many dimes and how many quarters were in the jar?

\myWideMatrixTable[-0.1in]
\vspace{0.5in}
\hfill{\itshape (Ans: 30 dimes, 40 quarters)}
\vspace{2em}




{\bfseries\large 2)} 
You find another jar of 120 pennies, nickles and dimes.
You just happen to know that 
the number of nickles is equal to the number of dimes minus 10.
When you take them to the bank (to the coin counting machine),
you find that they added up to \$6.00.
How many pennies, nickles, and dimes were in the jar?

\myWideMatrixTable[-0.1in]

\vspace{0.5in}
\hfill{\itshape (Ans: 50 pennies, 30 nickles, 40 dimes)}
\vspace{2em}



\newpage
{\bfseries\large 3)} 
Arly's Game Arcade uses three different kinds of tokens for their game machines.
There are gold tokens.
There are silver tokens.
There are bronze tokens.
For \$20.00, you can buy any of the following mixes of tokens:
\begin{itemize}[itemsep=0in]
    \item 14 gold, 20 silver, 24 bronze
    \item 20 gold, 15 silver, 19 bronze
    \item 30 gold, 5 silver, 13 bronze.
\end{itemize}
How much does each kind of token cost?

\myWideMatrixTable[-0.1in]

\vspace{0.5in}
\hfill{\itshape (Ans: gold: \$0.50, silver: \$0.35, bronze: \$0.25)}
\vspace{2em}





{\bfseries\large 4)} 
Last weekend, the Nostream Movie Theater sold a total of 8500 movie tickets.
They earned \$64,600.00.
Tickets are sold in one of three ways:
\begin{itemize}
    \item A matinee (early show) ticket costs \$5.00.
    \item A student all-day ticket costs \$6.00.
    \item Regular tickets cost \$8.50.
\end{itemize}
They sold twice as many student tickets as matinee tickets.
How many of teach kind of ticket did they sell?

\myWideMatrixTable[-0.1in]

\vspace{0.5in}
\hfill{\itshape (Ans: 900 matinee, 1800 student, 5800 regular)}
\vspace{2em}




\newpage
{\bfseries\large 5)} 
Rudy's Ripoff Tienda sells hats, T-shirts, and jackets. 
You know the following three facts:
\begin{itemize}[itemsep=0in]
    \item 3 hats, 2 T-shirts, and 1 jacket cost \$140.00.
    \item 2 hats, 2 T-shirts, and 2 jackets cost \$170.00.
    \item 1 hat, 3 T-shirts, and 2 jackets cost \$180.00.
\end{itemize}
What is the cost of each of these three items?

\myWideMatrixTable[-0.1in]

\vspace{0.5in}
\hfill{\itshape (Ans: hat: \$15.00, T-shirt: \$25.00, jacket: \$45.00)}
\vspace{2em}



{\bfseries\large 6)} 
Guadalupe, Jennifer, and Karla work at Notear Jeans Company.
One day, the three of them sold a combined \$1480.00 worth of jeans.
Guadalupe sold \$120.00 more than Jennifer.
Jennifer and Karla together sold \$280.00 more than Guadalupe.
How much did each of them sell?

\myWideMatrixTable[-0.1in]

\vspace{0.5in}
\hfill{\itshape (Ans: Guadalupe: \$600.00, Jennifer: \$480.00, Karla: \$400.00)}
\vspace{2em}



\newpage
{\bfseries\large 7)} 
Max, Jonathan, and Kevin work at Topnotch Boot Company.
One day, the three of them sold a combined \$1550.00 worth of boots.
Guadalupe sold \$150.00 more than Jonathan.
Kevin sold \$250.00 less than Max.
How much did each of them sell?

\myWideMatrixTable[-0.1in]

\vspace{0.5in}
\hfill{\itshape (Ans: Max: \$650.00, Jonathan: \$500.00, Karla: \$400.00)}
\vspace{2em}





{\bfseries\large 8)} 
Juanita is buying fruit at the Fructose Fruit Market.
One week, she pays \$3.70 for 1 orange, 2 grapefruit, and 3 pears.
Another week, she pays \$8.09 for 5 oranges, 6 grapefruit, and 2 pears.
On the last week, she pays \$2.12 for 2 oranges, NO grapefruit, and 2 pears.
Find the price per orange, price per grapefruit, and price per pear.


\myWideMatrixTable[-0.1in]

\vspace{0.5in}
\hfill{\itshape (Ans: oranges: \$0.49 each, grapefruit: \$0.75 each, pears: \$0.57 each)}
\vspace{2em}


\end{document}